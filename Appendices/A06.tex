\chapter{On the Universal Seesaw Mechanism}\label{app:universal_seesaw}

The masses of the SM fermions are generated not through direct Yukawa couplings to the Higgs, but via a universal seesaw mechanism by mixing with new vector-like fermions $\chi_f$. The most general gauge-invariant and renormalizable Lagrangian for the quark sector is given by:
\begin{align}
\mathcal{L}_{\text{mass}}^{\text{quark}} \supset & \, \bar{Q}_{L}^{i} (Y_{u L})_{ij} \chi_{u R}^{j} \tilde{H} + \bar{Q}_{L}^{i} (Y_{d L})_{ij} \chi_{d R}^{j} H \nonumber \\
& + \bar{\chi}_{u L}^{i} (Y_{u R})_{ij} u_{R}^{j} \phi^{*} + \bar{\chi}_{d L}^{i} (Y_{d R})_{ij} d_{R}^{j} \phi \nonumber \\
& + \bar{\chi}_{u L}^{i} (m_{\chi u})_{ij} \chi_{u R}^{j} + \bar{\chi}_{d L}^{i} (m_{\chi d})_{ij} \chi_{d R}^{j} + \text{h.c.}, \label{eq:L_Yuk_general_detail}
\end{align}
where $i, j = 1,2,3$ are flavor indices. An entirely analogous set of terms exists for the lepton sector. The Yukawa matrices $Y_{uL}, Y_{dL}, Y_{uR}, Y_{dR}$ and the vector-like mass matrices $m_{\chi u}, m_{\chi d}$ are general complex $3 \times 3$ matrices, making the flavor structure highly non-trivial.

\section{Diagonalization and Field Redefinitions}
To make physical predictions, we must diagonalize these matrices. We express them in terms of their singular value decompositions (i.e., their diagonal forms) and the associated unitary mixing matrices:
\begin{align*}
Y_{u L} &= U_{L L}^{\dagger} Y_{u L}^{d} U_{L R}, & Y_{u R} &= U_{R L}^{\dagger} Y_{u R}^{d} U_{R R}, \\
Y_{d L} &= V_{L L}^{\dagger} Y_{d L}^{d} V_{L R}, & Y_{d R} &= V_{R L}^{\dagger} Y_{d R}^{d} V_{R R}, \\
m_{\chi u} &= W_{u L}^{\dagger} m_{\chi u}^{d} W_{u R}, & m_{\chi d} &= W_{d L}^{\dagger} m_{\chi d}^{d} W_{d R}.
\end{align*}
Here, the matrices $Y^{d}$ and $m^{d}$ are real, diagonal, and non-negative. The unitary matrices $U, V, W$ are not physical by themselves but encode the mixing between flavor states.

We now perform a series of field redefinitions to absorb the maximal number of these unitary matrices into the definitions of the fermion fields. The goal is to make as many mass parameters diagonal as possible. The redefinitions are:
\begin{align*}
Q_L &\to U_{LL} Q_L, & \chi_{uR} &\to W_{uR} \chi_{uR}, & \chi_{uL} &\to W_{uL} \chi_{uL}, & u_R &\to U_{RR} u_R, \\
\chi_{dR} &\to W_{dR} \chi_{dR}, & \chi_{dL} &\to W_{dL} \chi_{dL}, & d_R &\to V_{RR} d_R.
\end{align*}
Applying these transformations to the Lagrangian \eqref{eq:L_Yuk_general_detail} and using the definitions above, we obtain the simplified form:
\begin{align*}
\mathcal{L}_{\text{Yuk}} = & \, \bar{Q}_{L} Y_{u L}^{d} (U_{L R} W_{u R}^{\dagger}) \chi_{u R} \tilde{H} + \bar{Q}_{L} (U_{LL}V_{LL}^\dagger) Y_{d L}^{d} (V_{L R} W_{d R}^{\dagger}) \chi_{d R} H \\
& + \bar{\chi}_{u L} (W_{u L} U_{R L}^{\dagger}) Y_{u R}^{d} u_{R} \phi^{*} + \bar{\chi}_{d L} (W_{d L} V_{R L}^{\dagger}) Y_{d R}^{d} d_{R} \phi \\
& + \bar{\chi}_{u L} m_{\chi u}^{d} \chi_{u R} + \bar{\chi}_{d L} m_{\chi d}^{d} \chi_{d R} + \text{h.c.}
\end{align*}
The matrix $\tilde{V}_{\text{CKM}} \equiv U_{LL}V_{LL}^\dagger$ is identified as the unitary matrix that will yield the observed Cabibbo-Kobayashi-Maskawa (CKM) quark mixing. For simplicity, and to focus on the essential mass generation mechanism, we adopt a \textit{flavor-aligned} scenario. This assumes that all other unitary matrices ($U_{LR}$, $W_{uR}$, $W_{uL}$, $U_{RL}$, etc.) are equal to the identity matrix. This is a strong assumption that minimizes new sources of flavor violation beyond the SM. Under this assumption, the Lagrangian simplifies dramatically to:
\begin{align*}
\mathcal{L}_{\text{Yuk}} = & \, \bar{Q}_{L} Y_{u L}^{d} \chi_{u R} \tilde{H} + \bar{Q}_{L} \tilde{V}_{\text{CKM}} Y_{d L}^{d} \chi_{d R} H \\
& + \bar{\chi}_{u L} Y_{u R}^{d} u_{R} \phi^{*} + \bar{\chi}_{d L} Y_{d R}^{d} d_{R} \phi \\
& + \bar{\chi}_{u L} m_{\chi u}^{d} \chi_{u R} + \bar{\chi}_{d L} m_{\chi d}^{d} \chi_{d R} + \text{h.c.}
\end{align*}
All matrices $Y^{d}$ and $m^{d}$ are now diagonal. The only remaining off-diagonal flavor structure is in $\tilde{V}_{\text{CKM}}$.

\section{Symmetry Breaking and the Mass Matrix}
After the electroweak symmetry breaking ($\langle H \rangle = v_h / \sqrt{2}$) and the $U(1)_{T_R^3}$ breaking ($\langle \phi \rangle = v_\phi / \sqrt{2}$), the mass terms for the up-type quarks (and analogously for down-type and leptons) are generated. For a single generation, the mass terms in the basis $(\bar{u}_L, \bar{\chi}_{u L})$, $(u_R, \chi_{u R})^T$ form a $2 \times 2$ matrix:
\begin{equation}
\mathcal{L}_{\text{mass}} = - \begin{pmatrix} \bar{u}_L & \bar{\chi}_{u L} \end{pmatrix}
\begin{pmatrix}
0 & m_L \\
m_R & m_{\chi}
\end{pmatrix}
\begin{pmatrix} u_R \\ \chi_{u R} \end{pmatrix} + \text{h.c.}, \label{eq:mass_matrix_detail}
\end{equation}
where the Dirac masses are:
\begin{align*}
m_L &= \frac{v_h}{\sqrt{2}} Y_{uL}, \\
m_R &= \frac{v_\phi}{\sqrt{2}} Y_{uR}.
\end{align*}
The entry $m_{\chi}$ is the vector-like mass. For three generations, this generalizes to a $6 \times 6$ matrix:
\begin{equation}
M_f = \begin{pmatrix}
0 & m_L \\
m_R & m_{\chi}
\end{pmatrix},
\end{equation}
where each entry is now a $3 \times 3$ matrix: $m_L = \frac{v_h}{\sqrt{2}} Y_{fL}^{d}$, $m_R = \frac{v_\phi}{\sqrt{2}} Y_{fR}^{d}$, and $m_{\chi} = m_{\chi f}^{d}$.

\section{Bi-Unitary Transformation and Mass Eigenvalues}
The general mass matrix $M_f$ is diagonalized by a bi-unitary transformation:
\begin{equation}
U_{R}^{\dagger} M_f U_{L} = M_f^d = \text{diag}(m_{f_1}, m_{f_2}, m_{f_3}, m_{F_1}, m_{F_2}, m_{F_3}), \label{eq:bi_unitary_detail}
\end{equation}
where $U_{L}$ and $U_{R}$ are $6 \times 6$ unitary matrices. The physical masses are found by solving the eigenvalues of the Hermitian matrices $H_L = M_f M_f^\dagger$ and $H_R = M_f^\dagger M_f$, as $U_L$ diagonalizes $H_L$ and $U_R$ diagonalizes $H_R$.

For the one-generation case, these matrices are:
\begin{align*}
H_L &= M_f M_f^\dagger = \begin{pmatrix}
m_L m_L^\dagger & m_L m_{\chi}^\dagger \\
m_{\chi} m_L^\dagger & m_R m_R^\dagger + m_{\chi} m_{\chi}^\dagger
\end{pmatrix} = \begin{pmatrix}
|m_L|^2 & m_L m_{\chi}^* \\
m_{\chi} m_L^* & |m_R|^2 + |m_{\chi}|^2
\end{pmatrix}, \\
H_R &= M_f^\dagger M_f = \begin{pmatrix}
m_R m_R^\dagger & m_R m_{\chi}^\dagger \\
m_{\chi} m_R^\dagger & m_L m_L^\dagger + m_{\chi} m_{\chi}^\dagger
\end{pmatrix} = \begin{pmatrix}
|m_R|^2 & m_R m_{\chi}^* \\
m_{\chi} m_R^* & |m_L|^2 + |m_{\chi}|^2
\end{pmatrix}.
\end{align*}
The eigenvalues $\lambda$ of $H_L$ (and $H_R$) are found from the characteristic equation $\det(H_L - \lambda \mathbb{I}) = 0$:
\begin{align*}
&\left||m_L|^2 - \lambda \right| \left| |m_R|^2 + |m_{\chi}|^2 - \lambda \right| - |m_L|^2 |m_{\chi}|^2 = 0 \\
&\Rightarrow \lambda^2 - \lambda (|m_L|^2 + |m_R|^2 + |m_{\chi}|^2) + |m_L|^2 |m_R|^2 = 0.
\end{align*}
The solutions to this quadratic equation are the squared masses of the two mass eigenstates:
\begin{align}
m_f^2 &= \frac{1}{2} \left( m_{\chi}^2 + m_L^2 + m_R^2 - \sqrt{ (m_{\chi}^2 + m_L^2 + m_R^2)^2 - 4 m_L^2 m_R^2 } \right), \label{eq:m_f_detail} \\
m_F^2 &= \frac{1}{2} \left( m_{\chi}^2 + m_L^2 + m_R^2 + \sqrt{ (m_{\chi}^2 + m_L^2 + m_R^2)^2 - 4 m_L^2 m_R^2 } \right), \label{eq:m_F_detail}
\end{align}
where we have now assumed all parameters are real for clarity. $m_f$ is the light SM-like fermion mass, and $m_F$ is the heavy vector-like partner mass.

\section{Yukawa Coupling Enhancement and Perturbativity}
Equation \eqref{eq:m_f_detail} is fundamental. It shows that the light mass $m_f$ is not simply proportional to $m_L$ (the SM Higgs VEV). We can solve Eq. \eqref{eq:m_f_detail} for $m_L^2$:
\begin{align*}
m_f^2 (m_{\chi}^2 + m_L^2 + m_R^2 - m_f^2) &= m_L^2 m_R^2 \quad \text{(from the exact seesaw relation)} \\
m_L^2 (m_R^2 - m_f^2) &= m_f^2 (m_{\chi}^2 + m_R^2 - m_f^2) \\
m_L^2 &= m_f^2 \left( \frac{m_{\chi}^2 + m_R^2 - m_f^2}{m_R^2 - m_f^2} \right) = m_f^2 \left( 1 + \frac{m_{\chi}^2}{m_R^2 - m_f^2} \right). \label{eq:m_L_solution_detail}
\end{align*}
Expressing this in terms of the original Yukawa couplings, where $m_L = \frac{v_h}{\sqrt{2}} Y_{fL}$ and the SM Yukawa is defined by $m_f = \frac{v_h}{\sqrt{2}} Y_f^{\text{SM}}$, we find:
\begin{equation}
Y_{fL}^2 = (Y_f^{\text{SM}})^2 \left( 1 + \frac{m_{\chi}^2}{m_R^2 - m_f^2} \right). \label{eq:Y_L_enhanced_detail}
\end{equation}
This relation reveals the core of the universal seesaw mechanism: the Yukawa coupling $Y_{fL}$ that couples the SM fermions to the Higgs is \textit{enhanced} compared to the standard model value $Y_f^{\text{SM}}$. The enhancement factor is $\sqrt{1 + m_{\chi}^2/(m_R^2 - m_f^2)}$.

This has profound implications:
\begin{itemize}
    \item \textbf{Light Fermions (e.g., electron, u-quark):} Here, $Y_f^{\text{SM}} \ll 1$. A large hierarchy $m_{\chi}^2 \gg m_R^2 \gg m_f^2$ can generate this tiny mass from a more ``natural'' $Y_{fL} \sim \mathcal{O}(0.1-1)$.
    \item \textbf{Top Quark:} Here, $Y_t^{\text{SM}} \approx 1$ is already large. An enhancement could easily push $Y_{tL}$ into the non-perturbative regime ($Y_{tL}^2 / 4\pi > 1$). To avoid this, we must require the enhancement factor to be $\mathcal{O}(1)$, which implies $m_{\chi}^2 \lesssim m_R^2 - m_t^2$. Since $m_R = \frac{v_\phi}{\sqrt{2}} Y_{fR}$, this suggests $v_\phi > m_{\chi}$ is a natural condition.
\end{itemize}

\section{Exact Diagonalization and Mixing Angles}
The bi-unitary transformation \eqref{eq:bi_unitary_detail} is performed by matrices that can be parameterized by a mixing angle. For one generation, the left-handed mixing matrix is:
\begin{equation}
U_L = \begin{pmatrix}
\cos\theta_L & \sin\theta_L \\
-\sin\theta_L & \cos\theta_L
\end{pmatrix}.
\end{equation}
The angle $\theta_L$ quantifies the mixing between the SM fermion and its vector-like partner. The exact expressions for the fundamental parameters $m_L, m_R, m_\chi$ in terms of the physical masses $m_f, m_F$ and the mixing angle $\theta_L$ can be found by equating $U_L^\dagger H_L U_L = \text{diag}(m_f^2, m_F^2)$. This yields the system of equations:
\begin{align*}
m_L^2 &= m_f^2 \cos^2\theta_L + m_F^2 \sin^2\theta_L, \\
m_R^2 + m_\chi^2 &= m_f^2 \sin^2\theta_L + m_F^2 \cos^2\theta_L, \\
m_L m_\chi &= (m_F^2 - m_f^2) \sin\theta_L \cos\theta_L.
\end{align*}
Solving this system (and a similar one from $H_R$ for $\theta_R$) gives:
\begin{align}
m_L^2 &= \frac{1}{2} \left( m_f^2 + m_F^2 - (m_F^2 - m_f^2) \cos 2\theta_L \right), \label{eq:m_L_exact_detail} \\
m_R^2 &= \frac{m_f^2 m_F^2}{m_L^2} = \frac{2 m_f^2 m_F^2}{m_f^2 + m_F^2 - (m_F^2 - m_f^2) \cos 2\theta_L}, \label{eq:m_R_exact_detail} \\
m_{\chi}^2 &= m_R^2 + m_F^2 + m_f^2 - m_L^2 - \frac{m_f^2 m_F^2}{m_L^2} = \frac{(m_F^2 - m_f^2)^2 \sin^2 2\theta_L}{4 m_L^2}. \label{eq:m_chi_exact_detail}
\end{align}
Substituting Eq. \eqref{eq:m_L_exact_detail} into the expression for $m_{\chi}^2$ yields the form shown in the original text.

The critical constraint to keep the top Yukawa perturbative is $m_{\chi}^2 < m_R^2$. Using Eqs. \eqref{eq:m_R_exact_detail} and \eqref{eq:m_chi_exact_detail}, the ratio is:
\begin{equation}
\frac{m_{\chi}^2}{m_R^2} = \frac{(m_F^2 - m_f^2)^2 \sin^2 2\theta_L}{4 m_f^2 m_F^2} < 1. \label{eq:constraint_detail}
\end{equation}
For the top quark with $m_f = m_t \approx 173$ GeV and assuming a heavy partner $m_F \gg m_t$, this simplifies to:
\begin{equation}
\frac{m_F^4 \sin^2 2\theta_L}{4 m_t^2 m_F^2} \approx \frac{m_F^2}{4 m_t^2} \sin^2 2\theta_L < 1 \quad \Rightarrow \quad \sin^2 2\theta_L < \frac{4 m_t^2}{m_F^2}. \label{eq:sin_constraint_detail}
\end{equation}
This is a very strong constraint. For example, if $m_F = 1$ TeV, then $\sin^2 2\theta_L < 0.12$, meaning $\theta_L < 10^\circ$. In the small $\theta_L$ limit, $\cos 2\theta_L \approx 1 - 2\theta_L^2$ and $\sin^2 2\theta_L \approx 4\theta_L^2$. Substituting this into Eq. \eqref{eq:m_L_exact_detail}:
\begin{align*}
m_L^2 &\approx \frac{1}{2} \left( m_t^2 + m_F^2 - (m_F^2 - m_t^2)(1 - 2\theta_L^2) \right) \\
&= \frac{1}{2} \left( m_t^2 + m_F^2 - m_F^2 + m_t^2 + 2(m_F^2 - m_t^2)\theta_L^2 \right) \\
&= \frac{1}{2} \left( 2m_t^2 + 2(m_F^2 - m_t^2)\theta_L^2 \right) = m_t^2 + (m_F^2 - m_t^2)\theta_L^2.
\end{align*}
From the constraint \eqref{eq:sin_constraint_detail}, $\theta_L^2 < m_t^2 / m_F^2$. Therefore:
\begin{equation}
m_L^2 < m_t^2 + (m_F^2 - m_t^2) \frac{m_t^2}{m_F^2} = m_t^2 + m_t^2 - \frac{m_t^4}{m_F^2} = 2m_t^2 - \frac{m_t^4}{m_F^2}.
\end{equation}
Converting back to Yukawa couplings:
\begin{equation}
Y_{tL}^2 \lesssim (Y_t^{\text{SM}})^2 \left( 2 - \frac{m_t^2}{m_F^2} \right). \label{eq:Y_final_detail}
\end{equation}
This shows that the maximum enhancement for the top Yukawa is less than a factor of $\sqrt{2}$, which is perfectly perturbative.

The generalization to three generations involves the diagonalization of the full $6 \times 6$ matrices. The matrix $\tilde{V}_{\text{CKM}}$ introduced during field redefinition will manifest in the charged current weak interactions of the mass eigenstates. After diagonalization, the SM $W$ boson will couple not only to the three light quarks but also to the heavy vector-like quarks, with couplings suppressed by the mixing angles $\theta_L^i$. The observed $3 \times 3$ CKM matrix emerges as the effective mixing matrix among the three light quarks when the heavy states are integrated out.

The lepton sector follows an identical procedure for the charged leptons. The neutrino sector, however, offers further richness. The right-handed neutrinos $\nu_R$ can possess both Dirac masses ($m_R$) from coupling to $\phi$ and Majorana mass terms $M_R \bar{\nu}_R^c \nu_R$, which are allowed by the gauge symmetry. The vector-like neutrinos $\chi_\nu$ can also have Majorana masses. This combination of Dirac and Majorana masses for both $\nu_R$ and $\chi_\nu$ can generate a double or triple seesaw mechanism, providing a natural explanation for the tiny masses of the observed light neutrinos. The diagonalization of this extended neutrino mass matrix also generates the Pontecorvo-Maki-Nakagawa-Sakata (PMNS) mixing matrix.
