\chapter{The $4321$ Model}
\label{sec:4321}
%%%%%%%%%%%%%%%%%%%%%%%%%%%%%%%%%%%%%%%%%%%%

This appendix summarises the main features of the 4321 model, based on the construction presented in \cite{DiLuzio:2017vat, DiLuzio2018}. The model is built upon the extended gauge group
\[
\mathcal{G}_{\rm 4321} \equiv SU(4) \times SU(3)' \times SU(2)_L \times U(1)'.
\]
The Standard Model (SM) gauge group, $\mathcal{G}_{\rm 321} \equiv SU(3)_c \times SU(2)_L \times U(1)_Y$, is embedded into $\mathcal{G}_{\rm 4321}$ through two key identifications.

First, the SM strong force is identified with the diagonal subgroup of the two $SU(3)$ factors:
\begin{equation}
SU(3)_c = \left( SU(3)_{[4]} \times SU(3)' \right)_{\rm diag},
\end{equation}
where $SU(3)_{[4]} \subset SU(4)$. Second, and more crucially, the SM hypercharge is a linear combination of charges from the $SU(4)$ and $U(1)'$ sectors:
\begin{equation}
Y = \frac{1}{2}Q_{B-L} + 2Q_{T^3_R}.
\end{equation}
Here, the baryon minus lepton number ($Q_{B-L}$) is generated by a diagonal $SU(4)$ generator, $Q_{B-L} = 2\sqrt{6} \, T^{15}$, while the $U(1)'$ charge is identified with twice the third component of right-handed isospin, $Y' \equiv 2Q_{T^3_R}$. This specific embedding reveals the model's left-right symmetric foundation; the SM electric charge operator can now be expressed in the manifestly left-right symmetric form:
\begin{equation}
Q_{\text{EM}} = Q_{T^3_L} + Q_{T^3_R} + \frac{Q_{B-L}}{4}.
\end{equation}

The spontaneous breaking of the full $\mathcal{G}_{\rm 4321}$ symmetry down to the SM $\mathcal{G}_{\rm 321}$ gives mass to the gauge bosons associated with the broken generators. The spectrum of these new massive vectors and their quantum numbers under the SM group are:
\begin{itemize}
    \item A vector leptoquark, $U \sim (\mathbf{3},\mathbf{1},2/3)$,
    \item A coloron, $g' \sim (\mathbf{8},\mathbf{1},0)$,
    \item A massive neutral boson, $Z' \sim (\mathbf{1},\mathbf{1},0)$.
\end{itemize}
Heuristically, each of these bosons originates from a distinct part of the symmetry breaking pattern: the leptoquark ($U$) emerges from the breaking $SU(4)\to SU(3)_{[4]}\times U(1)_{B-L}$, the coloron ($g'$) from the breaking $SU(3)_{[4]}\times SU(3)'\to SU(3)_c$, and the $Z'$ from the subsequent breaking $U(1)_{B-L}\times U(1)_{T_R^3}\to U(1)_Y$.

