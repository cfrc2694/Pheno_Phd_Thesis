\chapter{The $4321$ Model}
\label{sec:4321}
%%%%%%%%%%%%%%%%%%%%%%%%%%%%%%%%%%%%%%%%%%%%

This appendix summarises the main features of the 4321 model, based on the construction presented in \cite{DiLuzio:2017vat, DiLuzio2018}. The model is built upon the extended gauge group
\[
\mathcal{G}_{\rm 4321} \equiv SU(4) \times SU(3)' \times SU(2)_L \times U(1)'.
\]
The Standard Model (SM) gauge group, $\mathcal{G}_{\rm 321} \equiv SU(3)_c \times SU(2)_L \times U(1)_Y$, is embedded into $\mathcal{G}_{\rm 4321}$ through two key identifications.

First, the SM strong force is identified with the diagonal subgroup of the two $SU(3)$ factors:
\begin{equation}
SU(3)_c = \left( SU(3)_{[4]} \times SU(3)' \right)_{\rm diag},
\end{equation}
where $SU(3)_{[4]} \subset SU(4)$. Second, and more crucially, the SM hypercharge is a linear combination of charges from the $SU(4)$ and $U(1)'$ sectors:
\begin{equation}
Y = \frac{1}{2}Q_{B-L} + 2Q_{T^3_R}.
\end{equation}
Here, the baryon minus lepton number ($Q_{B-L}$) is generated by a diagonal $SU(4)$ generator, $Q_{B-L} = 2\sqrt{6} \, T^{15}$, while the $U(1)'$ charge is identified with twice the third component of right-handed isospin, $Y' \equiv 2Q_{T^3_R}$. This specific embedding reveals the model's left-right symmetric foundation; the SM electric charge operator can now be expressed in the manifestly left-right symmetric form:
\begin{equation}
Q_{\text{EM}} = Q_{T^3_L} + Q_{T^3_R} + \frac{Q_{B-L}}{4}.
\end{equation}

The spontaneous breaking of the full $\mathcal{G}_{\rm 4321}$ symmetry down to the SM $\mathcal{G}_{\rm 321}$ gives mass to the gauge bosons associated with the broken generators. The spectrum of these new massive vectors and their quantum numbers under the SM group are:
\begin{itemize}
    \item A vector leptoquark, $U \sim (\mathbf{3},\mathbf{1},2/3)$,
    \item A coloron, $g' \sim (\mathbf{8},\mathbf{1},0)$,
    \item A massive neutral boson, $Z' \sim (\mathbf{1},\mathbf{1},0)$.
\end{itemize}
Heuristically, each of these bosons originates from a distinct part of the symmetry breaking pattern: the leptoquark ($U$) emerges from the breaking $SU(4)\to SU(3)_{[4]}\times U(1)_{B-L}$, the coloron ($g'$) from the breaking $SU(3)_{[4]}\times SU(3)'\to SU(3)_c$, and the $Z'$ from the subsequent breaking $U(1)_{B-L}\times U(1)_{T_R^3}\to U(1)_Y$.

The spontaneous breaking of the $\mathcal{G}_{\rm 4321}$ symmetry down to the Standard Model $\mathcal{G}_{\rm 321}$ and subsequently to electromagnetism is achieved through a scalar sector comprising four multiplets. The primary breaking $\mathcal{G}_{\rm 4321} \to \mathcal{G}_{\rm 321}$ is induced by the vacuum expectation values (VEVs) of three scalar fields:
\begin{itemize}
    \item $\Omega_1 \sim \left( \mathbf{\bar 4}, \mathbf{1}, \mathbf{1}, -1/2 \right)$,
    \item $\Omega_3 \sim \left( \mathbf{\bar 4}, \mathbf{3}, \mathbf{1}, 1/6 \right)$,
    \item $\Omega_{15} \sim \left( \mathbf{15}, \mathbf{1}, \mathbf{1}, 0 \right)$ (taken to be a real field).
\end{itemize}
The final electroweak symmetry breaking, $\mathcal{G}_{\rm 321} \to U(1)_{\rm EM}$, is triggered by the Higgs doublet $H \sim (\mathbf{1},\mathbf{1},\mathbf{2},1/2)$.

A suitable scalar potential (analysed in detail in Section~\ref{scalpot}) allows for a VEV configuration that ensures this breaking pattern. Phenomenological constraints suggest a clear hierarchy between these scales:
\begin{equation}
    \langle \Omega_{3} \rangle > \langle \Omega_{1} \rangle \gg \langle \Omega_{15} \rangle \gg \langle H \rangle.
\end{equation}
Given this hierarchy, we simplify the analysis by first considering the $\Omega_{3}$ and $\Omega_{1}$ system in isolation to understand the primary TeV-scale breaking. The effects of incorporating the smaller VEVs of $\Omega_{15}$ and $H$ will be discussed subsequently.

To analyze the $\Omega_3$--$\Omega_1$ subsystem, we represent these fields as a $4 \times 3$ matrix and a $4$-vector, transforming as $\Omega_3 \to U^*_4 \Omega_3 U_{3'}^T$ and $\Omega_1 \to U^*_4 \Omega_1$ under $SU(4) \times SU(3)'$, respectively. The desired vacuum configuration that breaks $\mathcal{G}_{\rm 4321}$ to $\mathcal{G}_{\rm 321}$ is:
\begin{equation}
\label{vevconf}
\langle \Omega_3 \rangle = 
\tfrac{1}{\sqrt{2}}
\left(
\begin{array}{ccc}
v_3 & 0 & 0 \\
0 & v_3 & 0 \\ 
0 & 0 & v_3 \\
0 & 0 & 0
\end{array}
\right) \, , \qquad
\langle \Omega_1 \rangle = 
\tfrac{1}{\sqrt{2}}
\left(
\begin{array}{c}
0 \\ 
0 \\ 
0 \\
v_1
\end{array}
\right).
\end{equation}
The most general renormalizable scalar potential that admits this vacuum as a stationary point, and in the limit where the bare masses vanish ($\mu_3 = \mu_1 = 0$) and the cubic coupling is absent ($\lambda_6 = 0$), can be written as:
{\small
\begin{equation}\label{eqscalpot}
\begin{aligned}
V_{\Omega_{3},\Omega_{1}}
= & \mu_1^2 \abs{\Omega_1}^2 + \mu_3^2 \, \Tr (\Omega_3^\dag \Omega_3) 
\\&+ \lambda_1 \left( \Tr (\Omega_3^\dag \Omega_3) - \tfrac{3}{2} v_3^2 \right)^2 
+ \lambda_2 \Tr \left( \Omega_3^\dag \Omega_3 - \tfrac{1}{2} v_3^2 \mathbb{1}_3 \right)^2 \\
& 
+\lambda_3 \left( \abs{\Omega_1}^2 - \tfrac{1}{2} v_1^2 \right)^2 
+ \lambda_4 \left( \Tr (\Omega_3^\dag \Omega_3) - \tfrac{3}{2} v_3^2 \right) \left( \abs{\Omega_1}^2 - \tfrac{1}{2} v_1^2 \right)  \\
& + \lambda_5 \Omega_1^\dag \Omega_3 \Omega_3^\dag \Omega_1 + \lambda_6 \left( \left[ \Omega_3 \Omega_3 \Omega_3 \Omega_1 \right]_1 + \text{h.c.} \right).
\end{aligned}  
\end{equation}
}

\noindent
Here, $\mathbb{1}_3$ denotes the $3\times 3$ identity matrix. We have used a relative rephasing between the fields $\Omega_{1}$ and $\Omega_{3}$ to remove the phase of $\lambda_6$. The unique quartic term,
\begin{equation}
  \left[ \Omega_3 \Omega_3 \Omega_3 \Omega_1 \right]_1 \equiv 
\epsilon_{\alpha\beta\gamma\delta} \epsilon^{abc} (\Omega_3)^\alpha_a (\Omega_3)^\beta_b (\Omega_3)^\gamma_c (\Omega_1)^\delta,
\end{equation}
is required to avoid accidental global symmetries in the scalar potential that would lead to unwanted massless Goldstone bosons.

The inclusion of the other two representations, $\Omega_{15}$ and $H$, in the scalar potential can be safely considered as a perturbation. They are assumed to take the VEVs $\langle \Omega_{15} \rangle = T_{15} v_{15}$ and $\langle H \rangle = \tfrac{1}{\sqrt{2}} (0, v)^T$, with $v = 246$ GeV. This treatment is justified because their VEVs are subleading for phenomenological reasons and they do not alter the pattern of global symmetries of the $\Omega_3$--$\Omega_1$ potential.
Finally, the decomposition of $\Omega_{15}$ under $\mathcal{G}_{321}$ is $\Omega_{15} \to (\mathbf{1},\mathbf{1},0) \oplus (\mathbf{3},\mathbf{1},2/3) \oplus (\mathbf{\bar 3},\mathbf{1},-2/3) \oplus (\mathbf{8}, \mathbf{1}, 0)$. The mixing of these states with those contained in $\Omega_{3,1}$ is parametrically suppressed by the ratio $v^2_{15} / v^2_{3,1}$, hence they play a subleading role in phenomenology.


Given the extended gauge group $\mathcal{G}_{\rm 4321}$,
we denote the gauge fields by $H^\alpha_\mu$, $G'^a_\mu$, $W^i_\mu$, $B'_\mu$; the gauge couplings by $g_4$, $g_3$, $g_2$, $g_1$; and the generators by $T^\alpha$, $T^a$, $T^i$, $Y'$
(with indices $\alpha = 1, \dots, 15$, $a = 1, \dots, 8$, $i=1,2,3$). 

To determine the gauge boson spectrum, we start from the covariant derivatives acting on the scalar fields $\Omega_{3,1,15}$:
\begin{align*}
D_\mu \Omega_1 &= \left(\partial_\mu + i g_4 H_\mu^\alpha T^{\alpha \ast} - \tfrac{1}{2} i g_1 B'_\mu \right)\Omega_1,  \\
D_\mu \Omega_3 &= \left(\partial_\mu + i g_4 H_\mu^\alpha T^{\alpha\ast} - i g_3 G'^a_\mu T^a  + \tfrac{1}{6} i g_1 B'_\mu \right) \Omega_3,  \\
D_\mu \Omega_{15} &= \partial_\mu \Omega_{15} - i g_4 \left[ T^\alpha, \Omega_{15} \right] H^\alpha_\mu.
\end{align*}
We define the index $A=9,\ldots,14$ to span the $SU(4) / (SU(3)_4 \times U(1)_4)$ coset. Neglecting electroweak symmetry breaking effects, the gauge boson masses are extracted from the canonically normalized kinetic terms of the scalar fields:
\begin{equation}
  \begin{aligned}
  \mathcal L \supset &  
    +\frac{1}{2}\left( g_4^2 v_1^2 + g_4^2 v_3^2 + \frac{4}{3} g_4^2 v_{15}^2\right)H_\mu^A H^{\mu A}  
  \\&
    +\frac{v_3^2}{4}
    \begin{pmatrix}
      H^a_\mu & G'^a_\mu
    \end{pmatrix}
    \begin{pmatrix}
      g_4^2   & - g_4 g_3 \\
    - g_4 g_3 & g_3^2
    \end{pmatrix}
    \begin{pmatrix}
      H^{b\mu} \\ 
      G'^{b\mu}    
    \end{pmatrix}
  \\&
    +\frac{3v_1^2+v_3^2}{4}
    \begin{pmatrix}
      H^{15}_\mu & B'_\mu
    \end{pmatrix}
    \begin{pmatrix}
      \dfrac{g_4^2}{4}   & - \dfrac{g_4 g_1}{2 \sqrt{6}} \\
      - \dfrac{g_4 g_1}{2 \sqrt{6}} & \dfrac{g_1^2}{6}
    \end{pmatrix}
    \begin{pmatrix}
      H^{15\mu} \\ 
      B'^\mu
    \end{pmatrix}.
  \end{aligned}
\end{equation}
Diagonalizing these mass matrices, we obtain the massive gauge boson spectrum:
\begin{align}
  U_\mu^{1,2,3} 
    &= \frac{1}{\sqrt{2}} \left( H^{9,11,13}_\mu \!\!\!- i H^{10,12,14}_\mu \right), 
  &
  M^2_{U} 
    &= \frac{1}{4} g_4^2 \left(v_1^2 + v_3^2 + \frac{4}{3} v_{15}^2\right), \label{defU} \\
  g'^a_\mu 
    &= \frac{g_4 H^a_\mu - g_3 G'^a_\mu}{\sqrt{g_4^2 + g_3^2}},
  &
  M^2_{g'} 
    &= \frac{1}{2}  (g_4^2 + g_3^2) v_3^2,\label{gptransf}\\
  Z'_\mu 
    &= \frac{g_4 H^{15}_\mu - \sqrt{\frac{2}{3}} g_1 B'_\mu}{\sqrt{g_4^2 + \frac{2}{3} g_1^2}},
  &
  M^2_{Z'} 
    &= \frac{1}{4} \left( g_4^2 + \frac{2}{3} g_1^2 \right) \left(v_1^2 + \frac{1}{3} v_3^2 \right). \label{Zptransf}
\end{align}
The combinations orthogonal to \eqref{gptransf} and \eqref{Zptransf}
correspond to the massless $SU(3)_c \times U(1)_Y$ gauge bosons of $\mathcal{G}_{\rm 321}$ 
prior to electroweak symmetry breaking:
\begin{align}
\label{gtransf} 
g^a_\mu &= \frac{g_3 H^a_\mu + g_4 G'^a_\mu}{\sqrt{g_4^2 + g_3^2}} \, , \\
\label{Btransf} 
B_\mu &= \frac{\sqrt{\frac{2}{3}} g_1 H^{15}_\mu + g_4 B'_\mu}{\sqrt{g_4^2 + \frac{2}{3} g_1^2}} \, .
\end{align}
The matching between the fundamental couplings $g_4$, $g_3$, $g_1$ and the SM couplings $g_s$, $g_Y$ is readily obtained by acting with the covariant derivative on a field which transforms trivially under $SU(4)$. This yields:
\begin{align}
\label{matchinggsgs}
g_s &= \frac{g_4 g_3}{\sqrt{g_4^2 + g_3^2}}, 
\\ 
\label{matchinggsgY}
g_Y &= \frac{g_4 g_1}{\sqrt{g_4^2 + \frac{2}{3} g_1^2}}.
\end{align}
Evolving the SM gauge couplings up to $\mu=2$ TeV, we obtain 
$g_s = 1.02$ and $g_Y = 0.363$. 
Since $g_s \leq g_{4,3}$ and $g_Y \leq \sqrt{\tfrac{3}{2}} g_{4}, g_{1}$,  
the hierarchy $g_s \gg g_Y$ also implies $g_{4,3} \gg g_Y \simeq g_1$. 
In the limit $v_3 \gg v_1 \gg v_{15}$, the mass spectrum simplifies. For example, if the gauge couplings also satisfy $g_4 \sim g_3$, one finds $M_{g'} \simeq \sqrt{2} M_U$ and $M_{Z'} \simeq \tfrac{1}{\sqrt{2}} M_U$.


