The Standard Model (SM) of particle physics is the most successful framework for describing the subatomic world. It is continuously tested in experiments worldwide, with the Large Hadron Collider (LHC) being the flagship project in this endeavor. One of the primary goals of the LHC is to precisely measure SM parameters and search for deviations that could signal new physics.  

In recent years, reported anomalies, such as those in $B$-meson decays from LHCb, BaBar, and Belle experiments, along with the potential discrepancy in the muon’s magnetic moment ($g-2$) from Fermilab, suggest a violation of lepton flavor universality (LFU). These observations provide a compelling window into physics beyond the SM. Among the proposed SM extensions to explain LFU violation, many introduce new particles with preferential couplings to third and second-generation fermions. Popular candidates include heavy states such as $Z'$ bosons, $\phi'$ scalars, and leptoquarks (LQs), among others.

This work presents two phenomenological studies aiming to probe different models, such as the $4321$ \cite{DiLuzio2018}, $U(1)_{T^3_R}$ \cite{Qureshi:2024naw}, THDM \cite{Wang_2022}, and $\tilde S_1$ \cite{Bhaskar:2023ftn} models, that extend the SM particle content. The studies use benchmark scenarios where the new particle fields have preferential couplings to second and third generation SM-fermions, depending on the model. The hypothetical signal and background samples are generated using Monte Carlo simulations, emulating the current running conditions of the LHC and the performance of the CMS detector. The expected sensitivity for the different signal models under study is obtained by performing a detailed analysis of the available (non-excluded) experimental phase-space, boosted by  machine learning (ML) techniques to optimize the discovery potential for these exotic states.
