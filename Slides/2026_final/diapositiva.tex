\documentclass{../bredelebeamer}
\usepackage{multirow}
\usepackage{pdfpages}
\usepackage{braket,bigstrut}
\usepackage{palatino}
\usepackage{multicol,bigstrut}
\usepackage{listings}
\usepackage{tikz}
\usepackage{pgfplots}
\pgfplotsset{compat=1.17}
\usepackage{booktabs}
\usepackage{amsmath,amssymb,amsfonts,cancel,physics,siunitx}
\usetikzlibrary{positioning,shadows,backgrounds,calc}%
\setbeamercolor{footnote mark}{fg=black}
\setbeamercolor{footnote}{fg=black}
\usepackage{appendixnumberbeamer}


\renewcommand{\baselinestretch}{0.9}

\usepackage[backend=bibtex8,style=authortitle,autocite=footnote]{biblatex}
\addbibresource{referencia.bib}

\renewbibmacro*{cite:title}{%
	\printtext[bibhyperref]{%
		\printfield[citetitle]{labeltitle}%
		\setunit{\space}%
		\printtext[parens]{\printdate}%
	}%
}
\renewcommand{\figurename}{{\bf Fig.}}
\usefonttheme{serif} % default family is serif

% Define arXiv command for stylized formatting
\newcommand{\arxiv}{ar$\chi$iv:}

\renewcommand{\baselinestretch}{0.9}

\title[Final Dissertation]{Machine Learning-Enhanced Feasibility Studies on the Production of New Particles with Preferential Couplings to Third Generation Fermions at the LHC}
\subtitle{}
\author[C. Rodríguez]{%
    \vspace{2em}\\
    PhD(c). Cristian Fernando Rodríguez Cruz\inst{1}\\
    \href{mailto:c.rodriguez45@uniandes.edu.co}{c.rodriguez45@uniandes.edu.co}\\
    \vspace{1em}
    Research advisors:\\
    \vspace{0.5em}
    Prof. Andrés Florez\inst{1} ( \href{mailto:ca.florez@uniandes.edu.co}{ca.florez@uniandes.edu.co} )\\
    \vspace{0.5em}
    Prof. J. Jones-Pérez\inst{2} ( \href{mailto:jones.j@pucp.edu.pe}{jones.j@pucp.edu.pe} )\\
    \vspace{1em}
}

\institute[ML-BSM3G@LHC]{\inst{1} Universidad de los Andes\and
\inst{2} Pontificia Universidad Católica del Perú
}
\date{\today}
\lstset{language=C++,
  basicstyle=\ttfamily,
  keywordstyle=\color{blue}\ttfamily,
  stringstyle=\color{red}\ttfamily,
  commentstyle=\color{green}\ttfamily,
  morecomment=[l][\color{magenta}]{\#}
}

\newcounter{finalframe}

\begin{document}
\frame{\titlepage}

\begin{frame}
    \frametitle{Outline}
    \tableofcontents
\end{frame}

\section{Introduction}
\subsection{Standar Model of Particle Physics}
\begin{frame}
    
\end{frame}
\subsection{Deficiencies of the SM}
\begin{frame}

\end{frame}

\subsection{LHC and Beyond the SM Physics}
\begin{frame}

\end{frame}

\section{Phenomenological Framework}
\subsection{Madgraph-Pythia-Delphes}
\begin{frame}

\end{frame}

\subsection{Hypothesis Testing and Significance}
\begin{frame}

\end{frame}

\subsection{Machine Learning in High Energy Physics}
\begin{frame}

\end{frame}

\section{$U(1)_{T^3_R}$ Model}
\begin{frame}
    
\end{frame}


\section{$U_1$ Leptoquark Model}
\begin{frame}
    
\end{frame}
\section{$Z'$ Interferences}
\begin{frame}{The need of a $Z'$ boson in gauge $U_1$ models}
    If $U_1$ has a gauge origin,  we could rewrite the interaction term in the covariant derivative as
	$$
	\psi_{L}^{\mathrm{SM}}=\begin{pmatrix}
		q_{Lr}\\ q_{Lg}\\ q_{Lb}\\ \ell_L
	\end{pmatrix}
	\Longrightarrow\;\;
	\mathcal{L}_{\text{int}}
	\sim U_{1\alpha}^\mu \bar{\psi}_{L}^{\mathrm{SM}}\, \gamma_{\mu} T_{+}^{\alpha } \psi_{L}^{\mathrm{SM}} + \text{h.c.},
	\quad
	T_{+}^{\alpha}=\begin{pmatrix}
		0 & 0 & 0 & \delta_{r\alpha}\\
		0 & 0 & 0 & \delta_{g\alpha}\\
		0 & 0 & 0 & \delta_{b\alpha}\\
		0 & 0 & 0 & 0
	\end{pmatrix},
	$$
	we have six generators $T_{\pm}^{\alpha }$ with closure relation and projecting into a color singlet operator:
	$$
	\sum_{\alpha} \left[{T_{+}^{\alpha }},{T_{-}^{\alpha}}\right] =
	3 T_{B-L}=\begin{pmatrix}
		1 & 0 & 0 & 0\\
		0 & 1 & 0 & 0\\
		0 & 0 & 1 & 0\\
		0 & 0 & 0 & -3
	\end{pmatrix}.
	$$
	So, the gauge group with this leptoquark must include a $U(1)_{B-L}$ symmetry (The right-handed currents also have a similar interaction term).
    
    \pause
    The interaction terms for the $Z'$ boson have the form
	\begin{align*}
		\mathcal{L}_{\text{int}} &\sim Z'_\mu\left(\bar{\psi}_{L}^{\mathrm{SM}}\gamma^\mu (3T_{B-L}) \psi_{L}^{\mathrm{SM}}\right)\\
		&\sim Z'_\mu \left(\bar q_{L} \gamma^\mu q_{L} - 3 \bar \ell_L \gamma^\mu \ell_L\right).
	\end{align*}
\end{frame}
\begin{frame}{Interference with a $Z'$ vector boson}
\begin{minipage}{.48\linewidth}
	Non-Resonant Production (leptoquarks)
	\begin{center}
		\includegraphics[width=.9\linewidth,height=.5\linewidth]{../Images/non-res_vector.pdf}
	\end{center}
	\begin{equation}
		\mathcal{M}_{U_1} \sim \frac{1}{t-m_{U_1}^2 + i m_{U_1} \Gamma_{U_1}},
	\end{equation}
\end{minipage}
\hfill
\begin{minipage}{.48\linewidth}
	Resonant Production (neutral bosons)
	\begin{center}
		\includegraphics[width=.9\linewidth,height=.5\linewidth]{../Images/DY.pdf}
	\end{center}
	\begin{equation}
		\mathcal{M}_{Z'} \sim \frac{1}{s-m_{Z'}^2 + i m_{Z'} \Gamma_{Z'}},
	\end{equation}
\end{minipage}
so the interference has the form
	\begin{equation*}
		% \mathrm{RKI}
		% 			=\frac{1}{\sigma_{ LQ+Z'}}
					\dv{m}\left[
							\sigma_{ LQ+Z'}-\left(\sigma_{ LQ} +\sigma_{Z'}
							\vphantom{\frac{1}2}
							\right)
							\right]
							\sim \frac{g_{z'}g_{U}}{s}\frac{ m_{LQ}m_{Z'}\Gamma_{LQ}\Gamma_{Z'} - (t-m_{LQ}^2)(s-m_{Z'}^2)}{\left[
					(t-m_{LQ}^2)^2 + m_{LQ}^2\Gamma_{LQ}^2
			\right]
			\left[
					(s-m_{Z'}^2)^2 + m_{Z'}^2\Gamma_{Z'}^2
			\right]}.
	\end{equation*}
\end{frame}

\begin{frame}{Interference with a $Z'$ vector boson}


	\begin{center}
		\includegraphics[width=.45\linewidth]{../Images/Kinematic_Interference_gu_1.0_gzp_1.0_zp_upper_limit_woRHC.pdf}
		\hfill
		\includegraphics[width=.53\linewidth]{../Images/xs_interference.png}
	\end{center}
\end{frame}

\begin{frame}{Effects on the Sensitivity reach}
	
	\begin{center}
		\includegraphics[width=.48\linewidth]{../Images/reach_wo_interference.png}
		\hfill
		\includegraphics[width=.48\linewidth]{../Images/reach_w_interference.png}
	\end{center}
	
\end{frame}

\section{Summary and Conclusions}
\begin{frame}

\end{frame}

\appendix
\section*{Two body scattering}
\begin{frame}{Two body scattering}{CM-Frame}
    Consider the process
    \begin{equation}
        A(\va*{p}_1) + B(\va*{p}_2) \longrightarrow C(\va*{p}_3) + D(\va*{p}_4),
    \end{equation}
    \begin{center}
        \includegraphics[width=0.6\textwidth]{../Images/scatter.png}
    \end{center}

    From the Golden Rule, the cross section is given by
    \begin{equation}
        \sigma = \frac{S (2\pi)^4}{4\sqrt{(\va*{p}_1\cdot\va*{p}_2)^2 - (m_1m_2)^2}}\int \abs{\mathcal{M}}^2\delta^{(4)}(p_1 + p_2 - p_3 - p_4)\frac{d^3\va*{p}_3}{(2\pi)^3 2E_3}\frac{d^3\va*{p}_4}{(2\pi)^3 2E_4}.
    \end{equation}
    But, in the CM frame, $\va*{p}_1 + \va*{p}_2 = 0$, where
    \begin{gather}
        \sqrt{(\va*{p}_1\cdot\va*{p}_2)^2 - (m_1m_2)^2} = E_1E_2 \abs{\va*{p}_1},
        \\
        \delta^{(4)}(p_1 + p_2 - p_3 - p_4) = \delta\left(
        E_1 + E_2 - E_3 - E_4
        \right)
        \delta^{(3)}(\va*{p}_3 + \va*{p}_4).
    \end{gather}
    Thus
    \begin{equation}
        \sigma = \left(\frac{1}{8\pi}\right)^2 \frac{S}{(E_1E_2) \abs{\va*{p}_1}}\int \abs{\mathcal{M}}^2
        \frac{\delta\left(
            E_1 + E_2 - \sqrt{\va p_3^2 + m_3^2} - \sqrt{\va p_3^2 + m_4^2}
            \right)}{\sqrt{\va p_3^2 + m_3^2}\sqrt{\va p_3^2 + m_4^2}} \dd \va p_3
    \end{equation}
\end{frame}

\begin{frame}%{Two body scattering}{CM-Frame}
    Integrating over the radial part $\abs{\va p_3}$, we get
    \begin{equation}
        \sigma = \left(\frac{1}{8\pi}\right)^2 \frac{S |\va*{p}_3|}{(E_1 +E_2)^2 \abs{\va*{p}_1}} \int \abs{\mathcal{M}}^2 \dd\Omega,
    \end{equation}
    with
    \begin{equation}
        |\va*{p}_3| = \frac{1}{2} \frac{\sqrt{((E_1+E_2)^2 - m_3^2 -m_4^2)^2- 4m^2_3m^2_4} }{E_1+E_2},
    \end{equation}
    the outgoing momentum in the CM frame.

    $$ $$

    We prefer work with differential cross section as
    \begin{equation}
        \frac{\dd \sigma}{\dd \Omega} = \frac{1}{64\pi^2} \frac{S }{(E_1 +E_2)^2 } \frac{|\va*{p}_3|}{\abs{\va*{p}_1}} \abs{\mathcal{M}}^2.
    \end{equation}

    Defining $\sqrt s = E_1 + E_2$, we have
    \begin{equation}
        |\va*{p}_3| = \frac{1}{2} \frac{\sqrt{(s - m_3^2 -m_4^2)^2- 4m^2_3m^2_4} }{\sqrt s}, \quad  |\va*{p}_1|=\frac{1}{2} \frac{\sqrt{(s - m_1^2 -m_2^2)^2- 4m^2_1m^2_2} }{\sqrt s}.
    \end{equation}
    so the differential cross section is
    \begin{equation}
        \frac{\dd \sigma}{\dd \Omega} = \frac{1}{64\pi^2} \frac{n! }{s }
        \sqrt{\frac{(s - (m_3 + m_4)^2)(s - (m_3 - m_4)^2)}{(s - (m_1 + m_2)^2)(s - (m_1 - m_2)^2)}}
        \abs{\mathcal{M}}^2.
    \end{equation}

    Note that at this point, we don't need to know the explicit form of the matrix element $\mathcal{M}$, so it is a generic result.
\end{frame}

\begin{frame}{Writting $t$ in terms of $s$ and $\theta$}
    

    In general, there are three Lorentz-invariant useful kinematical variables to describe the scattering process, known as Mandelstam variables:
    \begin{align}
        \hat s & = (p_1 + p_2)^2 = (p_3 + p_4)^2 = m_1^2 + m_2^2 + 2p_1^\mu p_{2\mu }= m_3^2 + m_4^2 + 2p_3^\mu p_{4\mu}, \\
        \hat t & = (p_1 - p_3)^2 = (p_2 - p_4)^2= m_1^2 + m_3^2 - 2p_1^\mu p_{3\mu }= m_2^2 + m_4^2 - 2p_2^\mu p_{4\mu},  \\
        \hat u & = (p_1 - p_4)^2 = (p_2 - p_3)^2= m_1^2 + m_4^2 - 2p_1^\mu p_{4\mu }= m_2^2 + m_3^2 - 2p_2^\mu p_{3\mu}.
    \end{align}
    $$ $$
    In the CM-frame, $\hat s = s = (E_1+E_2)^2$. If, $m_3=m_4$ and $m_1=m_2$ with $E_1=E_2=E_3=E_4=E$, we have
    \begin{equation}
        t = -\left(\va p_1 - \va p_3\right)^2 = - \va p_1^2 - \va p_3^2 + 2\va p_1\va p_3 
    \end{equation}
    where $\va p_1^2 = E^2 - m_1^2 $  and $\va p_3^2 = E^2 - m_3^2 $.
    $$ $$
    So, in terms of $s$, $t$ could be written as
    \begin{equation}
        t= -2s + (m_1^2 + m_3^2) + 2\sqrt{(s/4 - m_1^2)(s/4 - m_3^2)} \cos\theta,
    \end{equation}
    with $\theta$ the scattering angle in the CM-frame.
\end{frame}



\end{document}
