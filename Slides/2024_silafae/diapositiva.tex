\documentclass{../bredelebeamer}
\usepackage{multirow}
\usepackage{pdfpages}
\usepackage{braket,bigstrut}
\usepackage{palatino}
\usepackage{multicol,bigstrut}
\usepackage{listings}
\usepackage{tikz}
\usepackage{pgfplots}
\pgfplotsset{compat=1.17}
\usepackage{booktabs}
\usepackage{amsmath,amssymb,amsfonts,cancel,physics,siunitx}
\usetikzlibrary{positioning,shadows,backgrounds,calc}%
\setbeamercolor{footnote mark}{fg=black}
\setbeamercolor{footnote}{fg=black}


\renewcommand{\baselinestretch}{0.9}

\usepackage[backend=bibtex8,style=authortitle,autocite=footnote]{biblatex}
%\addbibresource{../../references.bib}

\renewbibmacro*{cite:title}{%
	\printtext[bibhyperref]{%
		\printfield[citetitle]{labeltitle}%
		\setunit{\space}%
		\printtext[parens]{\printdate}%
	}%
}

\renewcommand{\figurename}{{\bf Fig.}}
\usefonttheme{serif} % default family is serif

\renewcommand{\baselinestretch}{0.9}

\title[Ditaus - Silafae 2024]{On the Effects of Interference in BSM Production and Detection of diTaus at the LHC}
\subtitle{}
\author[Cristian F. Rodríguez C.]{%
	$ $\\
    $ $\\
	Cristian Fernando Rodríguez Cruz\\
	$ $\\
	$ $\\
    $ $\\
	Authors:\\
	A. Flórez\inst{1}, \textcolor{Framableu}{\textbf{C. Rodriguez}}\inst{1}, J. Reyes-Vega\inst{1},\\
	J. Jones-Pérez\inst{2}. \\
	$ $\\
}

\institute[Uniandes]{%
    \inst{1} Universidad de los Andes\and
    \inst{2} Pontificia Universidad Católica del Perú
}
\date{\today}
\lstset{language=C++,
  basicstyle=\ttfamily,
  keywordstyle=\color{blue}\ttfamily,
  stringstyle=\color{red}\ttfamily,
  commentstyle=\color{green}\ttfamily,
  morecomment=[l][\color{magenta}]{\#}
}

\begin{document}
\frame{\titlepage}

\begin{frame}
    \frametitle{Outline}
    \tableofcontents
\end{frame}

\section{Introduction}
\subsection{Motivation}
\begin{frame}{The Standard Model of Particle Physics}
    \begin{minipage}[t]{0.43\linewidth}
        Weak bosons mix the different generations of quarks via the CKM matrix, but this does not happen for leptons. This property of the model is known as \textbf{lepton flavor universality (LFU)}.

        \begin{center}
            \includegraphics[width=.9\linewidth]{SM}
        \end{center}

    \end{minipage}
    \hfill
    \pause
    \begin{minipage}[t]{0.53\linewidth}
        %\centering

        However, recent measurements of the $R(D)$ and $R(D^*)$ ratios show a deviation from the SM predictions. This could be a hint of \textbf{lepton flavor violation (LFV)} and then  \textbf{new physics beyond the SM}.

        $$ $$

        \includegraphics[width=1.05\linewidth]{../2023_paper/RDRDst_hflav.pdf}    
    
    \end{minipage}
\end{frame}

\subsection{BSM Signatures}
\begin{frame}{BSM Signatures on the Di-Tau Channel at the LHC}{$\tau$ lepton as window}
	%Assuming that the LFV can be explained models which have a preferential coupling to the third generation of fermions, we can explore the structure of that models by studying the di-tau channel at the LHC.
	\vfill
	In the different models, we can have different production mechanisms, like
	\begin{minipage}{.48\linewidth}
		\begin{center}
			\includegraphics[width=.9\linewidth]{DY.pdf}
		\end{center}
	\end{minipage}
	\hfill
	\begin{minipage}{.48\linewidth}
		\begin{center}
			\includegraphics[width=.9\linewidth]{non-res.pdf}
		\end{center}
	\end{minipage}

	\begin{minipage}{.48\linewidth}
		\begin{center}
			\includegraphics[width=.9\linewidth]{DY_scalar.pdf}
		\end{center}
	\end{minipage}
	\hfill
	\begin{minipage}{.48\linewidth}
		\begin{center}
			\includegraphics[width=.9\linewidth]{DY.pdf}
		\end{center}
	\end{minipage}
\end{frame}

\section{Interference Phenomena}
\subsection{Generalities of the Interference Phenomena}
\begin{frame}{Generalities of the Interference Phenomena}
    
\end{frame}

\subsection{Interference Phenomena in the SM}
\begin{frame}{Interference Phenomena in the SM}
    
\end{frame}








\begin{frame}{Montecarlo Generators}
	

	\begin{center}
		\includegraphics[scale=0.5]{../2023_paper/A1}
	\end{center}

	Useful to predict what we expect to see under certain conditions:
	\begin{itemize}
		\item To perform studies before having the data
		\item To compute event selection efficiency/acceptance
		\item  To predict the ammount and composition of background events
		\item To distinguish different signals. 
	\end{itemize}
	
\end{frame}

\begin{frame}{Feasibility Studies Workflow}
	\includegraphics[width=1.0\linewidth]{../2023_paper/Workflow.png}
\end{frame}


\section{Example: The $4321$-Model}
\subsection{The model}

\begin{frame}{Leptoquarks and $R(D)$/ $R(D^*)$ anomalies}
	\begin{multicols}{2}
		\includegraphics[width=.99\linewidth]{../2023_paper/B_SM_1.png}
		$$ $$
		\includegraphics[width=.99\linewidth]{../2023_paper/B_SM_2.png}
		\pause
		\includegraphics[width=.99\linewidth]{../2023_paper/B_LQ_1.png}
		$$ $$
		\includegraphics[width=.99\linewidth]{../2023_paper/B_LQ_2.png}
	\end{multicols}
	\pause
	\begin{center}
		{\large $ $\\
		How can we test this hypothesis?
		}
	\end{center}
\end{frame}

\begin{frame}{Example: The Vector Leptoquark Model}
    \begin{multicols}{2}
        A leptoquark is defined as a particle with a vertex that mix vectors and quarks.
        \begin{center}
        \includegraphics[width=.25\textwidth]{../2023_paper/LQ_vertex.png}
        \end{center}
        If $U_1$ is a vector leptoquark that preserves the chirality on the vertex, we expect an interaction term like
        $$
        \sim U_1^\mu\bar{q}_{L} \gamma_{\mu} \ell_{L},
        $$
        and these allows a similar interaction term for the right handed currents 
        $$
        \sim U_1^\mu\bar{d}_{R} \gamma_{\mu} e_{R}.
        $$
    
        \end{multicols}
        Where the SM charges for the leptoquark, in the $Y=2(Q-T_3)$ convention, are
        \begin{center}
            \begin{tabular}{|c|c|c|c|c|}
                \hline & $\bar{q}_{L}$ & $\ell_{L}^{j}$ & $\bar{q}_{L}\gamma_{\mu} \ell_{L}$ & $U_{1}^{\mu}$ \\
                \hline$U(1)$ & $-1 / 3$ & $-1 $ & $-4 / 3$ & $+4 / 3$ \\
                \hline $\mathrm{SU}(2)$ & $\overline{\mathbf 2}$ & $\mathbf{2}$ & $\mathbf{1}$ & $\mathbf{1}$ \\
                \hline $\mathrm{SU}(3)$ & $\overline{\mathbf 3}$ & $\mathbf{1}$ & $\overline{\mathbf3}$ & $\mathbf{3}$ \\
                \hline
            \end{tabular}	
        \end{center}
        Then, the leptoquark $U_1 \sim \left(\mathbf{3}_{C}, \mathbf{1}_{I}, 4 / 3_{Y}\right)$, and its covariant derivative is
        $$
        \mathcal{D}_\mu U_\nu = \left(\partial_\mu+i g_s T^a G_\mu^a+ i \frac{2}{3} g' B_\mu \right)U_\nu.
        $$
\end{frame}

\begin{frame}{The Vector Leptoquark Lagrangian}
	
	The full Lagrangian for the vector leptoquark is
	\begin{align*}
		\mathcal{L}_{U}=&-\frac{1}{2} U_{\mu \nu}^{\dagger} U^{\mu \nu}+M_{U}^{2} U_{\mu}^{\dagger} U^{\mu}\\&-i g_{s}\left(1-\kappa_{c}\right) U_{\mu}^{\dagger} T^{a} U_{\nu} G^{\mu \nu}_a -\frac{2 i}{3} g'\left(1-\kappa_{Y}\right) U_{\mu}^{\dagger} U_{\nu} B^{\mu \nu}\\
		&+\frac{g_{U}}{\sqrt{2}}\left[U_{1}^{\mu}\left(\beta_{L}^{i j} \bar{q}_{L}^{i} \gamma_{\mu} e_{L}^{j}+\beta_{R}^{i j} \bar{d}_{R}^{i} \gamma_{\mu} e_{R}^{j}
		%+\beta_{N}^{i} \bar{u}_{R}^{i} \gamma_{\mu} N_{R}
		\right)+\text { h.c. }\right]
	\end{align*}
	where $U_{\mu \nu}=\mathcal D_{\mu} U_{\nu}-\mathcal D_{\nu} U_{\mu}$, $\mathcal D_{\mu}=\partial_{\mu}-i g_{s} G_{\mu}^{a} T^{a}-i \frac{2}{3} g_{Y} B_{\mu}$, and the couplings $\beta_{L}$ and $\beta_{R}$ are complex $3 \times 3$ matrices in flavor space. If $U_1$ has a gauge origin $\kappa_{c}=\kappa_{Y}=0$.
	\pause
	Defining 
	$$
	\psi_{L}^{\mathrm{SM}}=\begin{pmatrix}
		q_{Lr}\\ q_{Lg}\\ q_{Lb}\\ \ell_L
	\end{pmatrix}
	\Longrightarrow\;\;
	\mathcal{L}_{\text{int}}
	\sim U_{1\alpha}^\mu \bar{\psi}_{L}^{\mathrm{SM}}\, \gamma_{\mu} T_{+}^{\alpha } \psi_{L}^{\mathrm{SM}} + \text{h.c.},
	\quad
	T_{+}^{\alpha}=\begin{pmatrix}
		0 & 0 & 0 & \delta_{r\alpha}\\
		0 & 0 & 0 & \delta_{g\alpha}\\
		0 & 0 & 0 & \delta_{b\alpha}\\
		0 & 0 & 0 & 0
	\end{pmatrix},
	$$
	we have six generators $T_{\pm}^{\alpha }$ with closure relation,
	$$
	\sum_{\alpha} \left[{T_{+}^{\alpha }},{T_{-}^{\alpha}}\right] =
	3 T_{B-L}=\begin{pmatrix}
		1 & 0 & 0 & 0\\
		0 & 1 & 0 & 0\\
		0 & 0 & 1 & 0\\
		0 & 0 & 0 & -3
	\end{pmatrix}.
	$$
	So, the gauge group with this leptoquark must include a $U(1)_{B-L}$ symmetry. The right-handed currents also have a similar interaction term.
\end{frame}

\subsection{Sensitivity Reach of the $U_1$ Leptoquark}
\begin{frame}{Sensitivity Reach of the $U_1$ Leptoquark}
	\begin{minipage}{.32\linewidth}
		\includegraphics[width=\linewidth]{../2023_paper/non_resonant.png}
	\end{minipage}
	\begin{minipage}{.66\linewidth}
		\includegraphics[width=\linewidth]{../2023_paper/Significance_Heatmap_13TeV_L137_non-res_combined_woRHC.pdf}
	\end{minipage}

	{\large
		  Non-resonant production is highly dependent on the couplings, so it dominates the regions of high coupling constants at all masses.
	}
\end{frame}


\begin{frame}{Take care, you could need a $Z'$ boson}
	The generator $T_{B-L}$ is associated with the $U(1)_{B-L}$ symmetry with a $Z'$ boson. The interaction terms for the $Z'$ boson have the form
	\begin{align*}
		\mathcal{L}_{\text{int}} &\sim Z'_\mu\left(\bar{\psi}_{L}^{\mathrm{SM}}\gamma^\mu (3T_{B-L}) \psi_{L}^{\mathrm{SM}}\right)\\
		&\sim Z'_\mu \left(\bar q_{L} \gamma^\mu q_{L} - 3 \bar \ell_L \gamma^\mu \ell_L\right).
	\end{align*}\pause
	so, the full Lagrangian for the $Z'$ boson is
	\begin{equation}
		\begin{aligned}
		\mathcal{L}_{Z^{\prime}}= & -\frac{1}{4} Z_{\mu \nu}^{\prime} Z^{\prime \mu \nu} +\frac{1}{2} M_{Z^{\prime}}^2 Z_\mu^{\prime} Z^{\prime \mu} \\
		& +\frac{g_{Z^{\prime}} }{2 \sqrt{6}} Z^{\prime \mu}\left(\zeta_q^{i j} \bar{q}_L^i \gamma_\mu q_L^j+\zeta_u^{i j} \bar{u}_R^i \gamma_\mu u_R^j+\zeta_d^{i j} \bar{d}_R^i \gamma_\mu d_R^j-3 \zeta_{\ell}^{i j} \bar{\ell}_L^i \gamma_\mu \ell_L^j-3 \zeta_e^{i j} \bar{e}_R^i \gamma_\mu e_R^j\right),
		\end{aligned}
	\end{equation}
	where the couplings $\zeta$ are complex $3\times 3$ matrices in flavor space.\pause

	\vfill
	We assume that both, the $Z'$ and the vector leptoquark $U_1$, have preferential couplings to third generation fermions, so $\beta^{33}\gg \beta^{ij}$ and $\zeta^{33}\gg \zeta^{ij}$.
\end{frame}



\subsection{Interference with a $Z'$ vector boson}
\begin{frame}{Interference with a $Z'$ vector boson}
\begin{minipage}{.48\linewidth}
	Non-Resonant Production (leptoquarks)
	\begin{center}
		\includegraphics[width=.9\linewidth,height=.5\linewidth]{non-res.pdf}
	\end{center}
	
\end{minipage}
\hfill
\begin{minipage}{.48\linewidth}
	Resonant Production (neutral bosons)
	\begin{center}
		\includegraphics[width=.9\linewidth,height=.5\linewidth]{DY.pdf}
	\end{center}
\end{minipage}

\end{frame}


\section{Final Remarks}
\subsection{Conclusions}
\begin{frame}{Conclusions}

\end{frame}

\subsection{Future Work}
\begin{frame}{Future Work}

\end{frame}



\end{document}