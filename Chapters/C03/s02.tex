\section{The Minimal $U(1)_{T_R^3}$ Model}\label{sec:model}

% \subsection{Scalar Potential}

The model extends the SM by an Abelian gauge symmetry $U(1)_{T^3_R}$, under which only the right-handed fermions are charged. The symmetry breaking is achieved via two independent Higgs mechanisms: one with the SM Higgs doublet $H$ for electroweak symmetry breaking, and another with a Higgs singlet $\phi$ for $U(1)_{T^3_R}$ breaking. These scalars acquire independent vacuum expectation values (VEVs), $\langle H \rangle = v_h / \sqrt{2}$ and $\langle \phi \rangle = v_\phi / \sqrt{2}$. In the Kibble parametrization, the fields are written as:
\begin{align}
    H & = \begin{pmatrix}
        G_{+} \\
        \frac{1}{\sqrt{2}}\left(v_h + \rho_0 + i G_{0}\right)
    \end{pmatrix}, \label{eq:higgskibblepara1} \\
    \phi & = \frac{1}{\sqrt{2}}\left(v_\phi + \rho_\phi + i G_{\phi}\right). \label{eq:higgskibblepara2}
\end{align}
In Eqs.~\eqref{eq:higgskibblepara1} and~\eqref{eq:higgskibblepara2}, $G_\pm$, $G_0$, and $G_\phi$ are the Goldstone bosons absorbed by the SM $W^\pm$ and $Z$ bosons and the dark photon $A'$ (associated with $U(1)_{T^3_R}$) to acquire mass. The fields $\rho_0$ and $\rho_\phi$ mix to form the physical mass eigenstates, the SM-like Higgs boson $h$ and a dark Higgs $\phi'$:
\begin{equation}
    \begin{pmatrix}
        h \\
        \phi'
    \end{pmatrix}
    =
    \begin{pmatrix}
        \cos\alpha & -\sin\alpha \\
        \sin\alpha & \cos\alpha
    \end{pmatrix}
    \begin{pmatrix}
        \rho_0 \\
        \rho_\phi
    \end{pmatrix}.
\end{equation}
This mixing arises from diagonalizing the mass matrix derived from the gauge-invariant scalar potential:
\begin{equation}
    \begin{aligned}
        V(H, \phi) &= \mu_H^2 H^\dagger H + \mu_\phi^2 \phi^* \phi \\
                    &\quad + \lambda (H^\dagger H)(\phi^* \phi) + \lambda_H (H^\dagger H)^2 + \lambda_\phi (\phi^* \phi)^2.
    \end{aligned}
\end{equation}
Minimizing the potential yields the tadpole equations:
\begin{align}
    \frac{\partial V}{\partial H} &= \frac{v_h}{\sqrt{2}} \left( \mu_H^2 + \lambda_H v_h^2 + \frac{1}{2} \lambda v_\phi^2 \right) = 0, \\
    \frac{\partial V}{\partial \phi} &= \frac{v_\phi}{\sqrt{2}} \left( \mu_\phi^2 + \lambda_\phi v_\phi^2 + \frac{1}{2} \lambda v_h^2 \right) = 0.
\end{align}
\textcolor{red}{The physical scalar masses are given by:....Af: Sugiero expendir un poco más esta parte. Considero apropiado explicar al lector que se hace luego de encontrar los tadpole equations, mostrar explicitamente la forma de la matriz no diagonal de masas y luego explicar que al diagonalizar se encuentran los valores físicos de masas escalares de la 3.8. Tambien explicar un poco más de donde surge la tangente.}
\begin{equation}
    m_{h,\phi'}^2 = \frac{1}{2} \left( \lambda_H v_h^2 + \lambda_\phi v_\phi^2 \right) \pm \sqrt{ \lambda^2 v_h^2 v_\phi^2 + \left( \lambda_H v_h^2 - \lambda_\phi v_\phi^2 \right)^2 },
\end{equation}
and the mixing angle $\alpha$ satisfies:
\begin{equation}
    \tan 2\alpha = \frac{-\lambda v_h v_\phi}{ \lambda_\phi v_\phi^2 - \lambda_H v_h^2}.
\end{equation}
The quartic couplings can be expressed in terms of the physical parameters:
\begin{align}
  \lambda_H &= \frac{m_{\phi'}^2 + m_h^2 + (m_{\phi'}^2 - m_h^2)\cos 2\alpha}{4 v_h^2}, \\
  \lambda_\phi &= \frac{m_{\phi'}^2 + m_h^2 + (m_{\phi'}^2 - m_h^2)\cos 2\alpha}{4 v_\phi^2}, \\
  \lambda &= \frac{m_{\phi'}^2 - m_h^2}{2 v_h v_\phi} \sin 2\alpha.
\end{align}
Thus, the scalar sector has four free parameters: the masses $m_h$ and $m_{\phi'}$, the mixing angle $\alpha$, and the dark Higgs VEV $v_\phi$. Similar to how $v_h$ is fixed by the electroweak gauge boson masses, $v_\phi$ is related to the dark photon mass by $m_{A'}^2 = g_{T^3_R}^2 v_\phi^2$, where $g_{T^3_R}$ is the $U(1)_{T^3_R}$ gauge coupling. Depending on the value of $g_{T^3_R}$, this gauge boson can behave as a heavy $Z'$ or a light dark photon. In this chapter, we assume $g_{T^3_R}$ is sufficiently small such that $A'$ can be treated as a dark photon.

\subsection{The Universal Seesaw Mechanism}

In this model, the masses of the SM fermions are generated through a universal seesaw mechanism by mixing with vector-like fermions $\chi_f$. The relevant mass terms in the Lagrangian are:
\begin{equation}
    -\mathcal{L} \supset Y_{f_L} \bar{f}_L' \chi_{fR}' H + Y_{f_R} \bar{\chi}_{fL}' f_R' \phi^* + m_{\chi_f'} \bar{\chi}_{f L}' \chi_{f R}' + \text{h.c.}
\end{equation}
This leads to the mass matrix:
\begin{equation}
    M_f = \begin{pmatrix}
        0 & Y_{f_L} v_h / \sqrt{2} \\
        Y_{f_R} v_\phi / \sqrt{2} & m_{\chi_f'}
    \end{pmatrix}.
\end{equation}
The mass eigenstates $(f, \chi_f)$ are obtained by rotating the gauge eigenstates:
\begin{equation}
    \begin{pmatrix}
        f_{L,R} \\
        \chi_{f_{L,R}}
    \end{pmatrix}
    =
    \begin{pmatrix}
        \pm \cos\theta_{f_{L,R}} & \mp \sin \theta_{f_{L,R}} \\
        \sin \theta_{f_{L,R}} & \cos\theta_{f_{L,R}}
    \end{pmatrix}
    \begin{pmatrix}
        f_{L,R}' \\
        \chi_{f_{L,R}}'
    \end{pmatrix},
\end{equation}
such that $\mathcal{R}(\theta_{f_L}) M_f \mathcal{R}^{-1}(\theta_{f_R}) = \text{diag}(m_f, m_{\chi_f})$. For real parameters, the physical masses and mixing angles are given by:
\begin{gather}
    m_f m_{\chi_f} = \frac{ Y_{f_L} v_h Y_{f_R} v_\phi }{2}, \label{eq:prodmass} \\
    m_f^2 + m_{\chi_f}^2 = m_{\chi_f'}^2 + \frac{1}{2} \left( Y_{f_L}^2 v_h^2 + Y_{f_R}^2 v_\phi^2 \right), \label{eq:summass} \\
    \tan \theta_{f_{L,R}} = \frac{\sqrt{2}}{m_{\chi_f'}} \left( \frac{Y_{f_{L,R}} v_{h,\phi}}{2} - \frac{m_f^2}{Y_{f_{L,R}} v_{h,\phi}} \right).
\end{gather}
The Yukawa interactions of the physical fermions with the scalars $h$ and $\phi'$ are:
\begin{equation}
    -\mathcal{L}_{\text{yuk}} = h \, \bar{\psi}_{f_L} \, \mathcal{Y}_{h} \, \psi_{f_R} + \phi' \, \bar{\psi}_{f_L} \, \mathcal{Y}_{\phi} \, \psi_{f_R},
\end{equation}
where $\psi_f = (f, \chi_f)^T$. The Yukawa matrices are:
\begin{align}
    \mathcal{Y}_{h} &= \frac{1}{\sqrt{2}} \mathcal{R}(\theta_{f_L}) \left( Y_{f_L} \sigma_+ \cos\alpha - Y_{f_R} \sigma_- \sin\alpha \right) \mathcal{R}^{-1}(\theta_{f_R}), \label{eq:YukawaL} \\
    \mathcal{Y}_{\phi} &= \frac{1}{\sqrt{2}} \mathcal{R}(\theta_{f_L}) \left( Y_{f_L} \sigma_+ \sin\alpha + Y_{f_R} \sigma_- \cos\alpha \right) \mathcal{R}^{-1}(\theta_{f_R}), \label{eq:YukawaR}
\end{align}
with $\sigma_{\pm} = (\sigma_1 \pm i \sigma_2)/2$ being the ladder Pauli matrices.

The expressions above provide a simplified, one-generation view. The complete model involves a non-trivial flavor structure where the mass matrices are general $3 \times 3$ matrices. The diagonalization of the full $6 \times 6$ mass matrices, the procedure for absorbing unphysical unitary rotations, and the emergence of the CKM matrix are detailed in Appendix~\ref{app:universal_seesaw}. Furthermore, the appendix contains a rigorous treatment of the mass eigenvalue problem, deriving the exact relationship between the fundamental parameters $(m_L, m_R, m_\chi)$ and the physical observables $(m_f, m_F, \theta_L)$, which leads to critical constraints on the model's parameter space to ensure perturbativity.



\subsection{Minimal UV-complete Theory}

To generate non-zero masses for all SM fermions and ensure gauge anomaly cancellation, the model must include at least one full generation of vector-like fermions $\{\chi_\mathrm{u}, \chi_\mathrm{d}, \chi_\mathrm{\ell}, \chi_\mathrm{\nu}\}$ and the right-handed neutrinos $\nu_R$ for each SM generation. Their quantum numbers are listed in Tab.~\ref{tab:QMnumbers}. The Yukawa interactions in the UV-complete theory are:
\begin{equation}
    \begin{aligned}
        -\mathcal{L} \supset&\,
        Y_{L u}^{ij} \bar{q}_L^{\prime i} \chi_{u R}^{\prime j} \widetilde{H}
        + Y_{R u}^{ij} \bar{\chi}_{u L}^{\prime i} u_R^{\prime j} \phi^*
        + m_{\chi_\mathrm{u}}^{ij} \bar{\chi}_{u L}^{\prime i} \chi_{u R}^{\prime j} \\
        &+ Y_{L d}^{ij} \bar{q}_L^{\prime i} \chi_{d R}^{\prime j} H
        + Y_{R d}^{ij} \bar{\chi}_{d L}^{\prime i} d_R^{\prime j} \phi
        + m_{\chi_d}^{ij} \bar{\chi}_{d L}^{\prime i} \chi_{d R}^{\prime j} \\
        &+ Y_{L \ell}^{ij} \bar{\ell}_L^{\prime i} \chi_{\ell R}^{\prime j} H
        + Y_{R \ell}^{ij} \bar{\chi}_{\ell L}^{\prime i} \ell_R^{\prime j} \phi
        + m_{\chi_\ell}^{ij} \bar{\chi}_{\ell L}^{\prime i} \chi_{\ell R}^{\prime j} \\
        &+ Y_{L \nu}^{ij} \bar{\ell}_L^{\prime i} \chi_{\nu R}^{\prime j} \widetilde{H}
        + Y_{R \nu}^{ij} \bar{\chi}_{\nu L}^{\prime i} \nu_R^{\prime j} \phi^*
        + m_{\chi_\nu}^{ij} \bar{\chi}_{\nu L}^{\prime i} \chi_{\nu R}^{\prime j}
        + \text{h.c.}
    \end{aligned}
\end{equation}
Here, $i, j = 1,2,3$ are generation indices. The diagonalization of the mass matrices for each fermion type follows the structure outlined in Eqs.~\eqref{eq:prodmass} and~\eqref{eq:summass}, while the Yukawa matrices generalize the structure of Eqs.~\eqref{eq:YukawaL} and~\eqref{eq:YukawaR}, now encoding the CKM and PMNS mixing matrices. The neutrino sector has a more complex structure due to the possibility of a Majorana mass term for the vector-like neutrinos $\chi_\nu'$.

\begin{table}[]
    \centering
    \begin{tabular}{ccccc}
        \hline
        \hline
        Field & $SU(3)_C$  & $SU(2)_L$ & $U(1)_Y$ & $U(1)_{T^3_R}$ \\
        \hline\hline
        $q_L'$                    & \textbf{3} & \textbf{2} & 1/6 & 0\\
        $\ell_L'$                 & \textbf{1} & \textbf{2} & -1/2 & 0\\
        $H$                       & \textbf{1} & \textbf{2} & 1/2 & 0\\
        \hline
        $u_R^{\prime c}$          & \textbf{3} & \textbf{1} & -2/3 & -2\\
        $d_R^{\prime c}$          & \textbf{3} & \textbf{1} & 1/3 & 2\\
        $\ell_R^{\prime c}$       & \textbf{1} & \textbf{1} & 1 & 2\\
        $\nu_R^{\prime c}$        & \textbf{1} & \textbf{1} & 0 & -2\\
        $\phi$                    & \textbf{1} & \textbf{1} & 0 & 2\\
        \hline
        $\chi_{u_L}'$             & \textbf{3} & \textbf{1} & 2/3 & 0\\
        $\chi_{u_R}^{\prime c}$   & \textbf{3} & \textbf{1} & -2/3 & 0\\
        $\chi_{d_L}'$             & \textbf{3} & \textbf{1} & -1/3 & 0\\
        $\chi_{d_R}^{\prime c}$   & \textbf{3} & \textbf{1} & 1/3 & 0\\
        $\chi_{\ell_L}'$          & \textbf{1} & \textbf{1} & -1 & 0\\
        $\chi_{\ell_R}^{\prime c}$& \textbf{1} & \textbf{1} & 1 & 0\\
        $\chi_{\nu_L}'$           & \textbf{1} & \textbf{1} & 0 & 0\\
        $\chi_{\nu_R}^{\prime c}$ & \textbf{1} & \textbf{1} & 0 & 0\\
        \hline
        \hline
    \end{tabular}
    \caption{Minimal field content of the model and their representations under the SM and $U(1)_{T^3_R}$ gauge groups.}
    \label{tab:QMnumbers}
\end{table}
