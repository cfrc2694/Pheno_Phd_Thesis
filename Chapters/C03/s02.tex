
\section{\boldmath The Minimal $U(1)_{T_R^3}$ Model}\label{sec:model}
\subsection{Scalar Potential}
In this model, the SM is extended by the Abelian gauge symmetry $U(1)_{T^3_R}$, where only right-handed fermions are charged. We assume two independent Higgs mechanisms, one with a Higgs doublet $\mathrm{H}$ for electroweak symmetry breaking and the other with a Higgs singlet $\phi$ for the $U(1)_{T^3_R}$ symmetry breaking. Both scalars have independent vacuum expectation values (VEVs), $\expval{H}=v_h/\sqrt2$ and $\expval\phi =v_\phi/\sqrt2$, allowing us to express the doublet and singlet Higgs fields, following a Kibble parametrization, as 
\begin{align}
    H & = \begin{pmatrix}
        G_{+} \\
        \frac{1}{\sqrt{2}}\left(v_h+\rho_0+i G_{0}\right)
    \end{pmatrix}\label{eq:higgskibblepara1}
    \\
    \phi & =\frac{1}{\sqrt{2}}\left(v_\phi + \rho_\phi+i G_{\phi}\right). \label{eq:higgskibblepara2}
\end{align}
In Eqs.\ref{eq:higgskibblepara1} and Eq.\ref{eq:higgskibblepara2}, $G_\pm$, $G_0$, and $G_\phi$ are the Goldstone bosons that allow the SM $\textrm{W}^\pm$ and $\textrm{Z}$ bosons and the dark photon $A'$, associated with the $U(1)_{T^3_R}$ symmetry, to acquire mass. The $\rho_h$ and $\rho_\phi$ are an orthogonal mixture of the SM Higgs boson and the dark Higgs
\begin{equation}
    \begin{pmatrix}
        h
        \\
        \phi'
    \end{pmatrix}
    =
    \begin{pmatrix}
        \cos\alpha & -\sin\alpha
        \\
        \sin\alpha & \cos\alpha
    \end{pmatrix}
    \begin{pmatrix}
        \rho_0
        \\
        \rho_\phi
    \end{pmatrix},
\end{equation}
that results from the diagonalization of the mass matrices arising from the gauge invariant potential
\begin{equation}
    \begin{aligned}
        \mathcal V(\phi,H)
    &= \mu_H^2 H^{\dagger} H 
    +\mu_\phi^2 \phi^* \phi
    \\
    &+\lambda\left(H^{\dagger} H\right)\left(\phi^* \phi\right)
    +\lambda_H\left(H^{\dagger} H\right)^2
    +\lambda_\phi\left(\phi^* \phi\right)^2.
    \end{aligned}
\end{equation}
The tadpole equations are given from the minimization of the potential as
\begin{align}
    \pdv{\mathcal V}{H} 
     &= \frac{v_h}{\sqrt2} \left( \mu_H^2 +\lambda_Hv_h^2 + \frac{1}{2} \lambda v_\phi^2 \right) = 0,
    \\
    \pdv{\mathcal V}{\phi}
    &= \frac{v_\phi}{\sqrt2} \left( \mu_\phi^2 +\lambda_\phi v_\phi^2 + \frac{1}{2} \lambda v_h^2 \right) = 0.
\end{align}
The masses of the scalar bosons can be written as
\begin{equation}
    \begin{aligned}
        m_{h,\phi'}^2 &= \frac{1}{2}\left( 
    \lambda_H v_h^2 + \lambda_\phi v_\phi^2
    \right)
    \pm 
    \sqrt{
        \lambda^2 v_h^2 v_\phi^2
        +
        \left(
        \lambda_H v_h^2 - \lambda_\phi v_\phi^2
        \right)^2
    },
    \end{aligned}
\end{equation}
and the mixing angle $\alpha$ as
\begin{equation}
    \tan 2\alpha = \frac{-\lambda v_h v_\phi}{ \lambda_\phi v_\phi^2-\lambda_H v_h^2}.
\end{equation}

If we invert this relations, we can express the quartic couplings in terms of the masses and mixing angle as
\begin{align}
  \lambda_H &= \frac{m_{\phi'}^2+m_h^2+(m_{\phi'}^2 -m_h^2)\cos(2\alpha)}{4 v_h^2},\\
  \lambda_\phi &= \frac{m_{\phi'}^2+m_h^2+(m_{\phi'}^2 -m_h^2)\cos(2\alpha)}{4 v_\phi^2},\\
  \lambda &= \frac{m_{\phi'}^2-m_h^2}{2 v_h v_\phi}\sin(2\alpha).
\end{align}

So, we have four free parameters in the scalar potential, the masses of the physical scalar bosons $m_h$ and $m_{\phi'}$, the mixing angle $\alpha$, and the VEV of the dark Higgs $v_\phi$. In a similar way that the $v_h$ is fixed by de mass of the electroweak gauge bosons, the $v_\phi$ can be fixed by the mass of the dark photon $A'$, which is given by $m_{A'}^2 = g_{T^3_R}^2 v_\phi^2$, where $g_{T^3_R}$ is the gauge coupling of the $U(1)_{T^3_R}$ group. Depending on the range of values of $g_{T^3_R}$ this gauge bosson correspond to a $Z'$ boson or a dark photon. In this chapter, we will assume that the $g_{T^3_R}$ is small enough so that the $A'$ boson can be considered as a dark photon.

\subsection{The Universal Seesaw Mechanism}
In the model, each electrically charged SM fermion $f$ has a mass protected by both VEVs. In turn, they  acquire mass from the mixture with a vector-like fermion $\chi_f$, which is charged as the right-handed component of the respective SM fermion, in a UV complete theory. The terms in the Lagrangian density that contribute to the mass of physical fermions are,
\begin{equation}
    \begin{aligned}
        -\mathcal{L}&\supset 
    Y_{f_L} \bar{f}_L' \chi_{fR}' H 
    +Y_{f_R} \bar\chi_{fL}' f'_R  \phi^* 
    + m_{\chi_f'} \bar{\chi}_{f L}' \chi_{f R}'\\
&+\text { h.c.}
    \end{aligned}
\end{equation}
Therefore, in the vacuum, the mass matrix is
\begin{equation}
    M_f=
    \begin{pmatrix}
    0 & Y_{f_L} v_h /\sqrt2\\
    Y_{f_R} v_\phi /\sqrt2 & m_{\chi_f'}    
    \end{pmatrix}.
\end{equation}
The left- and right-handed components of the physical fermions $(f,\,\chi_f)$ are given by two rotations $\mathcal R(\theta_{f_{L,R}})$ as, 
\begin{equation}
    \begin{pmatrix}
        f_{L,R}
        \\
        \chi_{f_{L,R}}
    \end{pmatrix}
    =
    \begin{pmatrix}
        \pm\cos\theta_{f_{L,R}} & \mp \sin \theta_{f_{L,R}}
        \\
        \sin \theta_{f_{L,R}} & \cos\theta_{f_{L,R}}
    \end{pmatrix}
    \begin{pmatrix}
        f_{L,R}'
        \\
        \chi_{f_{L,R}}'
    \end{pmatrix},
\end{equation}
in a way that $\mathcal{R}(\theta_{f_L})M_f\mathcal{R}^{-1}(\theta_{f_R})=\text{diag}(m_f,m_{\chi_f})$ up to a phase. Assuming real parameters, the physical masses and the mixing angles are given by
\begin{gather}
    m_f m_{\chi_f}=\frac{ \left(Y_{f_{L}} v_h\right) \left(Y_{f_R} v_\phi\right)}{2}, \label{eq:prodmass}
     \\ 
    m_f^2 + m_{\chi_f}^2 = m_{\chi_f'}^2 + \frac{1}{2}\left(Y_{f_L}^2v_h^2+Y_{f_R}^2v_\phi^2\right),\label{eq:summass}
    \\
    \tan \theta_{f_{L,R}} =  \frac{\sqrt 2}{m_{\chi_f'}}\left(\frac{Y_{f_{L,R}}v_{h,\phi}}{2} - 
    \frac{m_f^2}{Y_{f_{L,R}}v_{h,\phi}} \right).
\end{gather}

\noindent The Yukawa interactions of the physical fermions with the scalar bosons have the form
\begin{equation}
    -\mathcal{L}_{\text{yuk}} 
    = h \bar\psi_{f_L} \mathcal{Y}_{h}\psi_{f_R} + \phi' \bar\psi_{f_L} \mathcal{Y}_{\phi}\psi_{f_R},
\end{equation} 
with $\psi_{f} = (f,\chi_{f})^T$, and the matrices $\mathcal{Y}_{f_{L,R}}$ given by
\begin{align}
    \mathcal{Y}_{h} &= \frac{1}{\sqrt{2}}
    \mathcal{R}(\theta_{f_L})
    \left(
        Y_{f_L}\sigma_+ \cos\alpha 
    - 
    Y_{f_R}\sigma_-\sin\alpha
    \right)
    \mathcal{R}^{-1}(\theta_{f_R})\label{eq:YukawaL}
    \\
    \mathcal{Y}_{\phi} &= \frac{1}{\sqrt{2}}
    \mathcal{R}(\theta_{f_L})
    \left(
    Y_{f_L}\sigma_+ \sin\alpha
    +
    Y_{f_R}\sigma_-\cos\alpha
    \right)
    \mathcal{R}^{-1}(\theta_{f_R}),\label{eq:YukawaR}
\end{align}
where $\sigma_{\pm}=(\sigma_1\pm i\sigma_2)/2$ are the ladder Pauli matrices.
\begin{center}
    \begin{tabular}{ccccc}
    \hline
    \hline
        Field & $SU(3)_C$  & $SU(2)_L$ & $U(1)_Y$ & $U(1)_{T^3_R}$ \\
    \hline\hline
        $q_L'$                    & \bf{3} & \bf{2} & 1/6 & 0\\
        $\ell_L'$                 & \bf{1} & \bf{2} & -1/2 & 0\\
        $H$                         & \bf{1} & \bf{2} & 1/2 & 0\\
        \hline
        $u_R^{\prime c}$          & \bf{3} & \bf{1} & -2/3 & -2\\
        $d_R^{\prime c}$          & \bf{3} & \bf{1} & 1/3 & 2\\
        $\ell_R^{\prime c}$       & \bf{1} & \bf{1} & 1 & 2\\
        $\nu_R^{\prime c}$        & \bf{1} & \bf{1} & 0 & -2\\
        $\phi$                      & \bf{1} & \bf{1} & 0 & 2\\
        \hline
        $\chi_{u_L}'$               & \bf{3} & \bf{1} & 2/3 & 0\\
        $\chi_{u_R}^{\prime c}$     & \bf{3} & \bf{1} & -2/3 & 0\\
        $\chi_{d_L}'$               & \bf{3} & \bf{1} & -1/3 & 0\\
        $\chi_{d_R}^{\prime c}$     & \bf{3} & \bf{1} & 1/3 & 0\\
        $\chi_{\ell_L}'$            & \bf{1} & \bf{1} & -1 & 0\\
        $\chi_{\ell_R}^{\prime c}$  & \bf{1} & \bf{1} & 1 & 0\\
        $\chi_{\nu_L}'$             & \bf{1} & \bf{1} & 0 & 0\\
        $\chi_{\nu_R}^{\prime c}$   & \bf{1} & \bf{1} & 0 & 0\\
    \hline
    \hline
    \end{tabular}
    \captionof{figure}{Minimal field content of the model and their representations under the SM and $U(1)_{T^3_R}$ gauge groups.}
    \label{tab:QMnumbers}
\end{center}


\subsection{Minimal UV-complete theory}

The model must provide non-zero masses for all the SM fermions and be free of gauge anomalies. So, we must have at least one full generation of vector-like fermions $\{\chi_\mathrm{u}$, $\chi_\mathrm{d}$, $\chi_\mathrm{\ell}$, $\chi_\mathrm{\nu}\}$ and the right-handed component of the SM neutrinos, $\nu_R$, charged as shown in Table~\ref{tab:QMnumbers} \textit{for each SM generation}. Therefore, the Yukawa interactions in the UV-complete theory must be of the form
\begin{equation}
    \begin{aligned}
        -\mathcal{L}
        \supset&\quad
        Y_{L u}^{ij} \bar{q}_L^{\prime i} \chi_{u R}^{\prime j} \widetilde{H}
        + Y_{R u}^{ij} \bar{\chi}_{u L}^{\prime i} u_R^{\prime j} \phi^*  
        + m_{\chi_\mathrm{u}}^{ij} \bar{\chi}_{u L}^{\prime i} \chi_{u R}^{\prime j}
        \\&
        +Y_{L d}^{ij} \bar{q}_L^{\prime i} \chi_{d R}^{\prime j} H 
        +Y_{R d}^{ij} \bar{\chi}_{d L}^{\prime i} d_R^{\prime j} \phi
        +m_{\chi_d}^{ij} \bar{\chi}_{d L}^{\prime i} \chi_{d R}^{\prime j}
        \\&
        +Y_{L \ell}^{ij} \bar{\ell}_L^{\prime i} \chi_{\ell R}^{\prime j} H
        +Y_{R \ell}^{ij} \bar{\chi}_{\ell L}^{\prime i} \ell_R^{\prime j} \phi
        +m_{\chi_\ell}^{ij} \bar{\chi}_{\ell L}^{\prime i} \chi_{\ell R}^{\prime j}
        \\
        &
        +Y_{L \nu}^{ij} \bar{\ell}_L^{\prime i} \chi_{\nu R}^{\prime j} \widetilde{H}
        +Y_{R \nu}^{ij} \bar{\chi}_{\nu L}^{\prime i} \nu_R^{\prime j} \phi^*
        +m_{\chi_\nu}^{ij} \bar{\chi}_{\nu L}^{\prime i} \chi_{\nu R}^{\prime j} \\
        &+\text { h.c., }
    \end{aligned}
\end{equation}
where the $i$ index runs over the three generations of fermions. The simultaneous diagonalization of the mass matrices of each fermion sector will have a similar structure to the one presented in Eqs.~\ref{eq:prodmass} and~\ref{eq:summass} and the Yukawa matrices will have a similar structure of Eqs.~\ref{eq:YukawaL} and~\ref{eq:YukawaR} but codifying the $CKM$ matrix. For the neutrino sector, the structure of the mass matrix will be more complicated due to the presence of the additional Majorana mass term for the vector-like neutrino $\chi_\nu'$.

