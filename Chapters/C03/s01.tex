\section{Current Exclusion Limits on Vector-Like Quarks}\label{sec:exp}

The ATLAS and CMS collaborations at CERN have conducted various searches for heavy vector-like quarks. These searches utilized $\mathrm{pp}$ collisions at center-of-mass energies of $\sqrt{s} = 8$ and $13$ \textrm{TeV}. The studies primarily focused on $\chi_u$ production through gluon-mediated QCD processes, either in pair production from quark-antiquark annihilation (Fig.~\ref{fig:qcd_T_prod}) or in single-$\chi_u$ production from electroweak processes involving associated quarks (Fig.~\ref{fig:qed_T_prod}). 

\begin{figure}[h!]
    \centering
    \begin{fmffile}{feyngraphs/feyngraph_qcd_T_prod}
    \begin{fmfgraph*}(80,60)
        \fmfleft{i2,i1}
        \fmfright{o1,o2}
        
        \fmf{fermion}{i1,v1}
        \fmf{fermion}{v1,i2}
        \fmf{fermion}{o1,v2}
        \fmf{fermion}{v2,o2}
        \fmf{gluon,label=$g$}{v1,v2}

        \fmflabel{$q$}{i2}
        \fmflabel{$\bar{q}$}{i1}
        \fmflabel{$\bar \chi_u$}{o1}
        \fmflabel{${\chi}_u$}{o2}
    \end{fmfgraph*}
    \end{fmffile}
    \caption{Representative Feynman diagram for $\chi_u$ pair production via gluon-mediated QCD processes ($q\bar{q} \to g \to \chi_u\bar{\chi}_u$).}
    \label{fig:qcd_T_prod}
\end{figure}

\begin{figure}[h!]
    \centering
    \begin{fmffile}{feyngraphs/feyngraph_qed_T_prod}
    \begin{fmfgraph*}(80,60)
        \fmfleft{i2,i1}
        \fmfright{o1,o2}
        
        \fmf{fermion}{i1,v1}
        \fmf{fermion}{v1,i2}
        \fmf{fermion}{o1,v2}
        \fmf{fermion}{v2,o2}
        \fmf{photon,label=$W$}{v1,v2}

        \fmflabel{$q$}{i2}
        \fmflabel{$\bar{q}'$}{i1}
        \fmflabel{$\bar{b}$}{o1}
        \fmflabel{$\chi_u$}{o2}
    \end{fmfgraph*}
    \end{fmffile}
    \caption{Representative Feynman diagram for single $\chi_u$ production via electroweak processes ($q\bar{q}' \to W \to \chi_u \bar{b}$).}
    \label{fig:qed_T_prod}
\end{figure}



In those studies, $\chi_u$ decays into $\mathrm{bW}$, $\mathrm{tZ}$, or $\mathrm{tH}$ have been considered. In the context of $\chi_u$ pair production, $\chi_u\bar{\chi}_u$, via QCD processes, the cross sections are well-known and solely depend on the mass of the vector-like quark.  Assuming a narrow $\chi_u$ decay width ($\Gamma / m(\chi_u) < 0.05$ or $0.1$) and a $100$\% branching fraction to $\textrm{bW}$, $\textrm{tZ}$, or $\textrm{tH}$, these searches have set stringent bounds on $m(\chi_u)$, excluding masses below almost $1.5$ \textrm{TeV} at $95$\% confidence level~\parencite{CMS:2024bni,CMS:2024qdd,ATLAS:2022ozf,ATLAS:2023bfh,ATLAS:2022hnn,ATLAS:2022tla,ATLAS:2023pja,ATLAS:2024fdw}. The most recent analysis from the CMS collaboration probes $\chi_u$-quark production via $\mathrm{pp} \to \chi_u\textrm{qb}$, in final states with $\chi_u \to \textrm{tZ}$ or $\chi_u \to \textrm{tH}$, considering scenarios with preferential couplings to third-generation fermions. The analysis sets $95$\% confidence level upper limits of $68-1260$ \textrm{fb} on the production cross section, for $\chi_u$ masses ranging from 600-1200 \textrm{GeV}~\parencite{CMS:2024qdd}. The latest studies from ATLAS probe vector-like quarks using the single-$\chi_u$ production mode with the $\chi_u \to \textrm{tH}$ decay channel leading to a fully hadronic final state~\parencite{ATLAS:2022ozf}, the single-$\chi_u$ production mode with the $\chi_u \to \textrm{tZ}$ decay channel leading to a multileptonic final state~\parencite{ATLAS:2023bfh}, the $\chi_u\chi_u$ pair production mode with various $\chi_u$ decay channels leading to multileptonic final states~\parencite{ATLAS:2022hnn}, and the $\chi_u\chi_u$ pair production mode with various $\chi_u$ decay channels leading to a single lepton plus missing momentum final state~\parencite{ATLAS:2022tla,ATLAS:2023pja}. 
The multilepton search offers the greatest sensitivity in most of the phase space, but the missing transverse energy based search has better sensitivity for low branching fraction $\mathfrak{B}(\chi_u\to \textrm{Wb})$ and high $\mathfrak{B}(\chi_u\to \textrm{Ht})$. These searches have similar sensitivities for the singlet and doublet models, resulting in exclusion bounds for masses below about $1.25$ \textrm{TeV} and $1.41$ \textrm{TeV}, respectively. 


A key consideration in the model interpretations summarized above is that the $\chi_u$ branching fractions depend on the chosen model. The excluded mass range is less restrictive for specific branching fraction scenarios, such as $\{\mathfrak{B}(\chi_u \to \textrm{tZ})$, $\mathfrak{B}(\chi_u \to bW)$, $\mathfrak{B}(\chi_u \to \textrm{tH})\}= \{0.2, 0.6, 0.2\}$, setting bounds on masses below about $0.95$ \textrm{TeV}. Moreover, if the $\chi_u \to \phi't $ decay is allowed, or if the branching fractions $\mathfrak{B}(\chi_u \to \textrm{tH/bW})$ are lower, the limits previously quoted must be re-evaluated. The authors of Ref.~\parencite{Cacciapaglia:2019zmj} emphasize that bounds on $m(\chi_u)$ can be around $500$ \textrm{GeV} when $\chi_u \to \mathrm{t}\phi'$ decays are permitted. Therefore, to facilitate a comprehensive study, benchmark scenarios in this paper are considered down to $m(\chi_\mathrm{u}) = 500$ \textrm{GeV}.
