\section{Discussion}\label{sec:discussion}

The LHC will continue to run with pp collisions at $\sqrt{s} = 13.6$~\textrm{TeV} for the next decade. Given the increase in the integrated luminosity expected from the high-luminosity program, it is important to consider unexplored new physics phase space that diverges from the conventional assumptions made in many BSM theories, and which could have remained hidden in processes that have not yet been thoroughly examined. It is additionally crucial to explore advanced analysis techniques, in particular the use of artificial intelligence algorithms, to enhance the probability of detecting these rare corners where production cross sections are lower and discrimination from SM backgrounds is difficult. 

In this work, we examine a model based on a $U(1)_{T^3_R}$ extension of the SM, which can address various conceptual and experimental issues with the SM, including the mass hierarchy between generations of fermions, the thermal dark matter abundance, and the muon $g - 2$, $R_{(D)}$, and $R_{(D^*)}$ anomalies. This model contains a light scalar boson $\phi'$, with potential masses below the electroweak scale, and~\textrm{TeV}-scale vector-like quarks $\chi_\mathrm{u}$. We consider the scenario where the scalar $\phi'$ has family non-universal fermion couplings and $m(\phi') \ge 1$~\textrm{GeV}, as was suggested in Ref.~\cite{Dutta2020}, and thus the $\phi^{\prime}$ can primarily decay to a pair of muons. Previous works in Refs.~\cite{Dutta2023, Banerjee_2016} considered scenarios motivating a search methodology with a merged diphoton system from $\phi' \to \gamma\gamma$ decays. The authors of Ref~\cite{Dutta2023}, in which $m(\phi') < 1$~\textrm{GeV},  indeed pointed out that if the $\phi'$ is heavier than about 1~\textrm{GeV}, then decays to $\mu^+ \mu^-$ can become the preferable mode for discovery, which is the basis for the work presented in this paper. We further note that the final state topology studied in this paper would represent the most important mode for discovery at $m(\phi') < 2 m_{\mathrm{t}}$ where the $\phi' \to \mathrm{t\bar{t}}$ decay is kinematically forbidden. 

The main result of this paper is that we have shown that the LHC can probe the visible decays of new bosons with masses below the electroweak scale, down to the~\textrm{GeV}-scale, by considering the simultaneous production of heavy QCD-coupled particles, which then decay to the SM particles that contain large momentum values and can be observed in the central regions of the CMS and ATLAS detectors. The boosted system combined with innovative machine learning algorithms allows for the signal extraction above the lower-energy SM background. The LHC search strategy described here can be used to discover the prompt decay of new light particles.  An important conclusion from this paper is that the detection prospects for low-mass particles are enhanced when it is kinematically possible to simultaneously access the heavy degrees of freedom which arise in the UV completion of the low-energy model.  This specific scenario in which the couplings of the light scalars are generationally dependent, with important coupling values to the top quark, is an ideal example which would be difficult to directly probe at low energy beam experiments.

The proposed data analysis represents a competitive alternative 
to complement searches already being conducted at the LHC, allowing us to probe $\phi'$ masses from 1 to 325 \textrm{GeV}, for $m(\chi_{\mathrm{u}})$ values up to almost 2~\textrm{TeV}, at the HL-LHC. Therefore, we strongly encourage the ATLAS and CMS Collaborations to consider the proposed analysis strategy in future new physics searches. 