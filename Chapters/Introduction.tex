\chapter*{Introduction}\addcontentsline{toc}{chapter}{\numberline{}\spacedlowsmallcaps{Introduction}}
%\lipsum
$ $ 

The pursuit of a fundamental description of nature's building blocks and their interactions is a central endeavor of modern physics. This quest has led to the development of the Standard Model (SM) of particle physics, a quantum field theory that encapsulates our current understanding of the subatomic world. With breathtaking precision, the SM describes the electromagnetic, weak, and strong nuclear forces, and classifies all known elementary particles. Its triumphs are undeniable, crowned by the landmark discovery of the Higgs boson at the Large Hadron Collider (LHC) in 2012, which confirmed the mechanism for generating the masses of elementary particles, and represented the final piece of the SM puzzle.

Yet, for all its success, the SM is universally acknowledged to be an incomplete theory. It offers no candidate for dark matter, cannot account for the matter--antimatter asymmetry in the universe, does not incorporate gravity, and leaves the mass of the Higgs boson itself unnaturally unstable under quantum corrections—a problem known as the hierarchy problem. These profound theoretical shortcomings provide a clear motivation for physics beyond the SM (BSM). However, the most compelling guide for this search has always come from experimental data itself.

The primary mission of the LHC is not only to consolidate the SM but to probe its boundaries and search for new physics. While no direct evidence of new particles has been found so far, a series of subtle but persistent discrepancies—termed ``anomalies''—have emerged from experiments worldwide, suggesting a potential crack in the SM's foundation.

A particularly intriguing set of these anomalies points towards a violation of Lepton Flavor Universality (LFU). In the SM, the electroweak force couples with identical strength to the three charged leptons (electrons, muons, and taus), a fundamental principle known as LFU. The most significant and long-standing hints of LFU violation come from measurements of semileptonic $B$-meson decays. The ratios $R(D^{(*)}) = \mathcal{B}(B \to D^{(*)} \tau \nu_\tau) / \mathcal{B}(B \to D^{(*)} \ell \nu_\ell)$, where $\ell$ is a muon or electron, have been measured by the BaBar, Belle, and LHCb collaborations to consistently exceed the SM predictions by a combined significance of approximately $3\sigma$--$4\sigma$. This deviation suggests that $B$ mesons are more likely to decay to a final state containing a tau lepton than the SM allows, providing a compelling hint of new physics that couples preferentially to the third generation fermions. Furthermore, the longstanding discrepancy in the muon's anomalous magnetic moment ($g-2$), recently confirmed with increased precision by the Fermilab experiment, adds another layer of intrigue, as it also hints at new physics potentially coupled preferentially to the second generation.

While each anomaly individually requires careful scrutiny, their collective persistence has generated significant excitement, as they seem to point towards new physics that breaks lepton flavor universality, potentially involving enhanced couplings to heavier fermions.


The pattern of these LFU-violating anomalies has inspired a vast landscape of theoretical models extending the SM. A common thread among the most promising explanations is the introduction of new heavy particles that mediate interactions with non-universal couplings to the different generations of fermions. This generational hierarchy is crucial to evade tight constraints from precision measurements on electrons (first generation) while affecting processes involving muons and taus.

In this thesis, we contextualize and present two of our phenomenological studies that propose different strategies to probe new physics models, such as the $4321$~\cite{Florez2023} and $U(1)_{T^3_R}$~\cite{Qureshi:2024naw} models, which extend the SM particle content to explain the observed LFU violation. These models introduce new particles with preferential couplings to second and third-generation fermions, making them prime candidates for explaining the experimental anomalies.

The experimental challenge lies in probing these models at the LHC. The proposed new particles are often heavy, leading to low production rates, and their decay signatures are complex and overwhelmed by enormous  backgrounds from SM processes. Given the immense number of theoretical possibilities and the finite resources available to experimental collaborations, it is impossible to pursue every potential signature with equal vigor. This is where \textbf{phenomenological feasibility studies} become critical. They provide a vital bridge between theory and experiment by performing a detailed \textit{a priori} assessment of the discovery potential for a given signal model. By using Monte Carlo simulations to emulate the detector response and analysis chain, these studies can identify the most promising signatures, optimize event selection criteria, and estimate the sensitivity achievable with the available data. This process is essential for prioritizing the experimental program, justifying the dedication of significant computing and human resources to a particular search, and ultimately guiding the LHC experiments towards the most well-motivated and detectable signals of new physics.

This thesis contributes to this effort by presenting two dedicated phenomenological studies that propose and develop novel strategies to probe the $4321$ and $U(1)_{T^3_R}$ models at the LHC. The work is situated at the intersection of theoretical model-building and experimental high-energy physics, with the explicit goal of assessing the feasibility of these searches.

The core methodology of this research involves:
\begin{enumerate}
    \item Defining \textbf{benchmark scenarios} within each model, selecting specific mass points and coupling structures that explain the LFU anomalies while remaining experimentally viable.
    \item Using \textbf{Monte Carlo simulation} to accurately generate the hypothetical signal processes alongside the dominant SM background processes, emulating the run conditions of the LHC and the performance of the CMS detector.
    \item Performing a detailed analysis of the available experimental phase-space. Given the high-dimensionality of the final states (e.g., involving multiple jets, leptons, and missing energy) and the complex, overlapping kinematical distributions of signal and background, traditional ``cut-and-count'' analyses are often sub-optimal. To overcome this, we employ advanced \textbf{Machine Learning (ML) techniques}, specifically supervised learning algorithms such as Boosted Decision Trees (BDTs) or Deep Neural Networks (DNNs). These algorithms are trained to learn the complex, non-linear correlations between many kinematic variables (e.g., invariant masses, angular separations, transverse momenta) to construct powerful discriminators that optimally separate the rare signal events from the large and diverse SM backgrounds. This ML-enhanced approach significantly boosts the analysis sensitivity, allowing for the detection of weaker signals or the setting of more stringent limits than would otherwise be possible.
    \item Deriving the \textbf{expected sensitivity} for each model, establishing the exclusion limits or discovery potential that the LHC experiments could achieve with the current dataset. This final step is the ultimate quantitative measure of the search's feasibility.
\end{enumerate}

The structure of this thesis is as follows. We begin by establishing the theoretical foundation with a review of the SM in Chapter \ref{ch:sm}. Then, Chapter \ref{ch:pheno} details the experimental context, describing the LHC and the CMS detector, and introduces the general analysis techniques employed, including a discussion on the application of Machine Learning in high-energy physics. The original phenomenological work of this thesis is presented in the subsequent chapters: Chapter \ref{ch:U1T3R} details a search for new physics in the process $pp \to t\bar{t}\mu^+\mu^-$, while Chapter \ref{ch:vector_lq} presents a search for vector leptoquarks in the process $pp \to \tau^+\tau^- + b\text{-jets}$. Finally, Chapter \label{ch:discussion} concludes by summarizing our findings and discussing their implications for the field, along with an outlook on future prospects.

