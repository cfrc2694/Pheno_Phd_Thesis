\chapter{Standard Model of Particle Physics}

The standard model (SM) of particle physics is a quantum field theory (QFT) in which fundamental particles are excitations of interacting relativistic fields in the quantum vacuum \cite{greiner2000relativistic}. In this context, matter in nature is formed by particles that have a fermionic character, and their interactions are described by the gauge principle where integer spin particles are defined as vector bosons, from the adjoint representation of a symmetry group (\textit{gauge group}), are the messengers of the interaction \cite{pokorski2000gauge}.

%TO DO -> summary of the chapter

\section{Fields and Symmetries}
Relativistic quantum fields are the degrees of freedom in QFT. Formally, they are \textit{operator valued functions of the spacetime that transform under a representation of the Lorentz group within an invariant subspace} \cite{Tong1995,CRodriguezUPTC}. The different representations of the Lorentz group are mainly characterized by their spin and their fields obey a different equation of motion (see table \ref{tab-repLorentz2}). 

In classical field theory, a variational principle is established which generates the equations that govern the dynamics of the different fields in a theory, \textit{the equations of motion}. Hamilton's principle, or principle of minimal action, indicates that all possible physical configurations for a set of fields $\varphi^I$, with $I=1,2,3,\cdots,n$, are those where the integral of the action $S$ is a minimal \cite{Goldstein,jose1998classical}:
\begin{equation}\label{eq-action}
	S=\int \mathcal{L}(\varphi^I,\partial_\mu\varphi^I) d^4x,
\end{equation}
here, $d^4x=dx^0dx^1 dx^2dx^3$ and $x\equiv(ct,x^1,x^2,x^3)\equiv(x^0,x^1,x^2,x^3)\in\mathcal{M}^4$ are the space-time coordinates in the minkowskian spacetime $\mathcal M^4$, and the function $\lag(\varphi^I,\partial_\mu\varphi^I)$ is called \textit{the Lagrangian density} of a theory \cite{greiner2000relativistic,Goldstein}. The problem in classical field dynamics is to find the functions $\varphi^I(x)$ in a space-time $\mathcal{M}^4$, fixing their boundary conditions. The solution to this classical problem is given by the Euler-Lagrange equations:
\begin{equation}\label{eq_EulerLag}
	\dpr{\mathcal{L}}{\varphi^I}-\dpr{}{x^\mu}\dpr{\lag}{\fac{\partial_\mu \varphi^I}}=0,
\end{equation}
and they are used to obtain the equations of motion of the set of fields $\varphi^I$ \cite{jose1998classical}. 

In quantum field theory, the situation is more complicated: if we adopt the approach of quantization by path integrals \cite{martinez2002,Weinberg}, the idea of an equation of motion vanishes and we go on to searching correlations between free particle states. However, the notion of action is still the cornerstone in the description of these observables.
Explicitly, the correlation functions are calculated through the LSZ formula from the path integral \cite{greiner1996qft,peskin}:
\begin{equation}
	\begin{aligned}
		Z[J]&=\langle\text { out, } 0| 0, \text { in }\rangle
		\\&=\mathcal{N}\int \mathcal{D}(\varphi, \bar{\varphi})  e^{i S[\varphi]} e^{i \int J_I\varphi^I  d^{4} x}
		\\&=\mathcal{N}\int \mathcal{D}(\varphi, \bar{\varphi})  e^{i \int d^{4} x \mathcal{L}} e^{i \int J_I\varphi^I  d^{4} x},
	\end{aligned}
\end{equation}
taken over the space of fields $\varphi$ with an appropriate measure $\mathcal{D}(\varphi, \bar{\varphi})$ and normalized by $\mathcal{N}$. The quantity $Z$ is known as the partition function for the theory and gives the transition amplitude from the initial vacuum $\mid 0$, in $\rangle$ to the final vacuum $\mid 0$, out $\rangle$ in the presence of a source $J(x)$ which is producing particles \cite{birrell75900}. Therefore the dynamics, at both the classical and quantum levels, in a theory are entirely determined by the Lagrangian density. Table \ref{tab-repLorentz2} records the Lagrangian density for different types of free fields, i.e. non-interacting fields.

\begin{center}
	\begin{tabular}{|l|c|c|l|l|}\hline\bigstrut
		Name							& Field				& Spin & Dimensions & Free-Lagrangian	\\\hline\hline\bigstrut
		Klein-Gordón				&	$\phi$					& 0			&[mass]					&	$\lag=\fac{\partial^\mu\bar \phi\partial_\mu \phi-m^2 \bar \phi\phi}$						\\\hline\bigstrut
		Dirac								& $\chi$			& $1/2$	&[mass]$^{3/2}$	&$\lag=\bar\chi\fac{i\pmb\gamma^\mu \partial_\mu -m\pmb 1}\chi$\\\hline\bigstrut
		Maxwell	& $A^\mu$ 		& $1$		&[mass]					&$\lag=-\frac{1}{4} F^{\mu v} F_{\mu v} $\\\hline
	\end{tabular}
	\captionof{table}{Some relevant representations of the Lorentz group in  $4$-dimensional space-time. In this notation $\eta_{\mu\nu}=\diag(1,-1-,-1-,1)$, $\pmb \gamma^\mu$ are the Dirac matrices, $F_{\mu \nu}^{A}=\partial_{\mu} A_{\nu}^{A}-\partial_{\nu} A_{\mu}^{A}+g f_{B C}{ }^{A} A_{\mu}^{B} A_{\nu}^{C}$ is the stregth field and the array of real numbers $f_{A B}^{C}$ are structure constants of the gauge group algebra \cite{freedman2012supergravity}, equations are written in natural units with $c=\hbar=1$.}\label{tab-repLorentz2}
\end{center}

In this paradigm, our task is to propose a Lagrangian density for a set of fields that correctly models the propagation and interactions of fundamental particles. With the formal development of QFT, a set of "rules" have been introduced, allowing the systematic construction of these Lagrangian densities.  For example, if the theory is relativistic, the equations of motion must be equal in all inertial frames, which implies that the action has to be invariant under Poincaré transformations \cite{pall}, \textit{i.e.} the Lagrangian density must be a Lorentz scalar and transform under translations at most as a total derivative \cite{jose1998classical}. Besides, $\mathcal{L}$ must be Hermitian in a way that allows the construction of physical observables \cite{pall,peskin}; in turn, $\mathcal{L}$ has units of energy density (dimensions of [mass]$^4$ in natural units). If the theory is required to be renormalizable (that is, we will be able to make perturbative calculations), fields can have maximum spin 1 and the associated coeficients of expansion must have, in natural units, dimensions of [mass]$^n$ where $n\geq0$ \cite{peskin,Weinberg}. These restrictions drastically reduce the number of terms that are allowed in a Lagrangian density. In particular, the terms of interaction between allowed fields are the Yukawa vertex, the scalar potential, and gauge couplings to vectorial bosons. 

At this point, we seem to have total freedom to mix these terms as possible interactions. However, the concept of symmetry has proven to be our most powerful ally for the construction of terms of interaction between fields. 
The procedure turns out to be simple, once a set of spin 0 and spin 1/2 fields has been established as part of the theory, these are organized to transform under a representation of a unitary gauge group $G$ such that the Lagrangian density must be a global-scalar of $G$. Then, once the global Lagrangian density is known, it is sought to ``promote'' symmetry to a local symmetry by a slight modification of associated kinematic terms \cite{pokorski2000gauge,freedman2012supergravity, Gallego2016,VanProeyen1999,Martin2012}.
This ``promotion'' is described in more detail below.

Given a Lagrangian density $\lag(\varphi_i,\partial_\mu \varphi_i)$, a gived field $\varphi$ is said to be \textit{globally simmetric} under unitary transformations, $\varphi_i\mapsto \mathcal{U}_G(\varphi_i)$, if the action is invariant under the variations of the fileds $\phi^{I}$ which are given, at infinitesimal level, by:
\begin{equation}
	\delta_G(\theta) \varphi^I\approx i\theta^A(T_A)_{J}^I\varphi^J,
\end{equation}
where $\theta^{A}$ are the parameters of the transformation $\mathcal{U}$ and $T_{A}$ are the representations of the generators of a unitary continuos group $G$. This considers an expansion of $U$ at first order in $\theta^{A}$. This group, $G$, support unitary representations of the shape:
\begin{equation}
	\mathcal{U}_G\dot=U(\theta)=\exp\fac{i\theta^A T_A}.
\end{equation}
The operators $T_A$ satisfy a commutation relation according to the Lie algebra:
\begin{equation}
	\cor{ T_A,T_B }= if_{ab}^{\;\;C}T_C,
\end{equation}
where $f_{AB}^{\;\;C}$ are the structure constants of $G$.

If invariance under local symmetry is desired, it is required to replace all the space-time derivatives $\partial_\mu$ that appear in $\lag$ by a new type known as \textit{covariant derivatives} $\mathcal{D}_\mu$, which implicitly bring the coupling of the given fields with new fields $B_\mu$, known as \textit{gauge fields}:
\begin{equation}
	\partial_\mu\rightarrow \mathcal{D}_\mu=\partial_\mu-
	\delta_G(B_\mu)\quad
	\Longrightarrow
	\quad\lag(\varphi_i,\partial_\mu\varphi_i)\rightarrow 
	\lag(\varphi_i,\mathcal{D}_\mu
	\varphi_i;B_\mu).  
\end{equation}
Term $\delta_G(B_\mu)$ is called \textit{connection} and it introduces a \textit{gauge} field $B_\mu^A$ for each generator $T_A$ of $G$ (note that 
$\delta_T(B_\mu)\equiv iB_\mu^AT_A$).
The covariant derivative is defined such that its transformation is of the form
\begin{equation}
	\Dcov'_\mu=U\Dcov_\mu U^\dagger,
	\entonces \mathcal{D}_\mu (\varphi) \rightarrow U \mathcal{D}_\mu (\varphi).
\end{equation} 
For this, it is enough that ${B}_\mu^C$ transforms as
\begin{equation}
	\delta_G(\theta){B}_\mu^C=\theta^Af_{AB}^{\;\;\;\;C} B^B_\mu
	+\partial_\mu \theta^C.\label{eq2}
\end{equation}

Since additional fields have been introduced, and, in order to implement local symmetry, it is necessary to construct a kinetic Lagrangian for such fields. Following the ideas of Yang and Mills based on the antisymmetric curvature tensor which is defined as
\begin{equation}
	F_{\mu\nu}^CT_C
		=F_{\mu\nu}
		=-\cor{\Dcov_\mu,\Dcov_\nu}
		=\fac{
			\partial_\mu B^C_\nu-\partial_\nu B^C_\mu+f_{AB}^{\;\;\;\;C}B^A_\mu B^B_\nu
		}T_C,
\end{equation}
and with it the kinetic Lagrangian for gauge fields is generalized as:
$$
\lag=-\frac{\delta_{AB}}{4 g^2} F^A_{\nu\mu}F^{\nu\mu B},
$$
where $g$ is known as gauge coupling constant which indicates the strength of the interaction. Usually, the gauge fields are rescaled so that the coefficient of the kinetic term is $1 / 4$ and $g$ appears in the covariant derivative.

As a way of illustration let us consider a renormalizable theory with a real scalar $\phi$ and a  Dirac spinor $\psi$ so that both are non-interacting, and suppose that this theory is globally invariant under phase transformations, i.e. the fields $\varphi\in\{\phi,\psi\}$ transforms as $\varphi\mapsto e^{i\theta \hat Q}\varphi $ such that $\hat Q \psi = q \psi$ and $\hat Q \phi=0\phi=0$. The Lagrangian turns out to be: 
\begin{equation}
	\mathcal L_{\text{free}}=\frac{1}{2} \partial^{\mu} \phi \partial_{\mu} \phi-\frac{1}{2}\mu^2\phi^2+\bar{\psi}(i \gamma_\mu  \partial^\mu-m) \psi
\end{equation}
If we want to add globally symmetric interaction terms, the scalar potential must be an expansion in the fields of order four at maximum, so that it remains renormalizable. The linear term of the potential does not contribute to the action, and the quadratic term of the potential is contained by the mass term. Whereas, a fermionic potential is not allowed since the only term renormalizable is precisely the term of mass. A cross term is allowed $\sim \phi\bar\psi\psi$, which is called \textit{Yukawa coupling}, then the globally invariant Lagrangian is
\begin{equation}
	\begin{aligned}
		\mathcal L_{\text{global}}&=\frac{1}{2} \partial^{\mu} \phi \partial_{\mu} \phi-V(\phi)+\bar{\psi}(i \gamma_\mu  \partial^\mu-m) \psi + k_1 \phi\bar\psi\psi,
		\\
		V(\phi)&=\frac{\mu^2}{2!}\phi^2 +\frac{\alpha}{3!}\phi^3+\frac{\lambda}{4!}\phi^4.
	\end{aligned}
\end{equation}
Promoting to local, 
\begin{multline}
	\mathcal L_{\text{local}}=\frac{1}{2} \mathcal D^{\mu} \phi \mathcal D_{\mu} \phi-V(\phi)\\
	+\bar{\psi}(i \gamma_\mu  \mathcal D^{\mu}-m) \psi 
	+ k_1 \phi\bar\psi\psi-\frac1{4g^2} F_{\mu\nu}F^{\mu\nu},
\end{multline}
where, 
\begin{equation}
	\mathcal D_\mu\varphi=\fac{\partial_{\mu}-ig A_\mu\hat Q }\varphi
	\Longrightarrow
	\begin{cases}
		\mathcal D_\mu\phi=\partial_\mu \phi,\\
		\mathcal D_\mu\psi=\partial_\mu \psi-ig q A_\mu \psi.
	\end{cases}
\end{equation}
With these ingredients, we are ready to approach the standard model Lagrangian. 

%TO DO -> ADD Feynman Diagrams for this interaction.