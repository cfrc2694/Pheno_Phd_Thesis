\chapter{Phenomenological Framework for LHC Observables, Searches and Analysis}

Since its formulation, the Standard Model (SM) has proven remarkably successful in describing the fundamental particles and interactions, and its parameters have been measured with increasing precision over several decades. However, as discussed in the previous chapter, various theoretical and experimental observations suggest that the SM is incomplete. As outlined previously, this is motivated by theoretical shortcomings—such as the hierarchy problem, the absence of a dark matter candidate, and non-zero neutrino masses—as well as by experimental anomalies. These limitations motivate the search for new physics (NP) beyond the SM (BSM).

The search for BSM physics proceeds along two main axes: the construction of theoretical extensions to the SM, and the development of experimental methods to probe them. A necessary condition for any viable BSM model is consistency with existing experimental data, which places strong constraints on its parameter space. These constraints include lower limits on the masses of new particles from direct searches at high-energy colliders, and upper bounds on couplings and mixing angles from precision measurements at both high and low energies, which are sensitive to virtual corrections.

Phenomenology connects theoretical models to experimental observables by calculating cross sections, decay rates, and other signatures for given model parameters. A critical function of this field is to assess the experimental feasibility of BSM scenarios—evaluating whether predicted signals would be observable above background processes given the capabilities of current and future detectors. This involves estimating production rates, modeling detector acceptance and efficiency, and developing discrimination variables to maximize signal-to-background ratios. This feasibility assessment is essential for designing analysis strategies, particularly at the Large Hadron Collider (LHC), where signals of new physics must be discriminated from large Standard Model backgrounds.

This feasibility assessment is essential for designing analysis strategies at the Large Hadron Collider (LHC), a proton-proton ($pp$) collider operating since 2009. The LHC has provided data at center-of-mass energies from $7~\mathrm{TeV}$ to $13.6~\mathrm{TeV}$. During Run~I (2010–2013), operations at $7$–$8~\mathrm{TeV}$ led to the discovery of the Higgs boson. Run~II (2015–2018) collected data at $13~\mathrm{TeV}$, reaching instantaneous luminosities up to $1.5 \times 10^{34}~\mathrm{cm}^{-2}\mathrm{s}^{-1}$. Run~III (2022–2025) is currently underway at $13.6~\mathrm{TeV}$ with increased luminosity, and the High-Luminosity LHC (HL-LHC) is anticipated to start around 2029. In this high-luminosity regime, signals of new physics must be discriminated from large and complex SM backgrounds, making sophisticated phenomenological tools increasingly important.

\section{Detectors and Subsystems}\label{sec:detectors}

When two particle bunches from colliding beams cross each other, they generate individual interactions known as events~\cite{CMS:2008xjf,ATLAS:2008xda}. At the LHC, the beam intensity is so high that multiple interactions can take place in a single event; this phenomenon is referred to as in-time pile-up~\cite{Apollinari2017_HLLHC,Bertolini:2014}. In other words, the probability that several proton-proton interactions occur within the same bunch crossing is non-negligible, leading to multiple overlapping events in a single detector readout~\cite{Cacciari:2011ma,Cacciari:2008gp}. In addition, particles from other bunch crossings with respect to the primary collision of interest can be detected. This latter experimental feature is known as out-of-time pile-up. The sum of these two effects, in-time and out-of-time pile-up, is commonly referred to as PU. 


The particle collisions at the LHC, $pp$ and heavy-ions,  occur at four main interaction points, each hosting a large particle detector designed to record and analyze the outcomes~\cite{CMS:2008xjf,ATLAS:2008xda}. The two largest and most comprehensive experiments~\cite{CMS:2008xjf,ATLAS:2008xda} are  the Compact Muon Solenoid (CMS) and A Toroidal LHC ApparatuS (ATLAS). Both are multipurpose detectors with broad physics programs, capable of exploring a wide range of phenomena~\cite{CMS:2008xjf,ATLAS:2008xda}. They perform precision measurements within the electroweak sector of the SM~\cite{1674-1137-40-10-100001}, probe the dynamics of quarks and gluons (including through heavy-ion collisions)~\cite{deFavereau:2013fsa}, and conduct extensive searches for BSM physics using $pp$ collision data~\cite{CMS:2012gu,ATLAS:2012yve}. While CMS and ATLAS differ in their detector designs and reconstruction strategies, their physics goals are largely overlapping, and their results are complementary~\cite{CMS:2008xjf,ATLAS:2008xda}.


\begin{figure}[h!]
  \centering
    \includegraphics[width=0.9\textwidth]{Images/coordinatechart.png}
    \caption{Coordinate system employed by the CMS experiment (retrieved from~\parencite{cmsplots}).}\label{fig_coordinates}
\end{figure}

Throughout this work, phenomenological studies and comparisons are primarily developed in the context of CMS, although several results from ATLAS are also referenced, given the close alignment in sensitivity and scope~\cite{CMS:2008xjf,ATLAS:2008xda}. Measurements performed at CMS adopt a right-handed coordinate system with its origin at the nominal collision point~\cite{CMS:2008xjf}. The $z$-axis is defined along the beam direction, the $x$-axis points radially inward toward the center of the LHC ring, and the $y$-axis points vertically upward~\cite{CMS:2008xjf}. The azimuthal angle $\phi$ is measured in the transverse ($xy$) plane from the $x$-axis, while the polar angle $\theta$ is measured from the $z$-axis, as shown in Fig.~\ref{fig_coordinates}~\cite{CMS:2008xjf}. Moreover, for kinematic analysis at hadron colliders, the Cartesian coordinate system is often reparameterized into quantities that are more physically meaningful and experimentally convenient as shown in Fig.~\ref{fig_cms_coor}~\cite{CMS:PF2017}:

\begin{description}
    \item[Pseudo-rapidity $(\eta)$] The polar angle is not a Lorentz invariant quantity. In addition, since the vast majority of particles are detected in the forward region of the detector, known as the endcap region, the distribution of the particle multiplicity as a function of $\theta$ is not uniform. This non-uniformity  makes it difficult to study the agreement between the observed data and the background prediction in the central part of the detector. Therefore, the CMS experiment uses a variable known as pseudo-rapidity~\cite{CMS:2008xjf,1674-1137-40-10-100001}, $\eta$, defined in terms of the polar angle as: 
    \begin{equation}
      \eta=-\ln \left(\tan \frac{\theta}{2}\right).
    \end{equation}
    
    Therefore, the main advantages of using $\eta$ instead of $\theta$ are that it provides more  uniform distributions than those over the polar angle~\cite{CMS:PF2017}. And, furthermore, the difference in $\eta$ is a Lorentz boosts invariant quantity, along the beam direction~\cite{1674-1137-40-10-100001}.
    
    \item[Transverse Momentum ($p_T$)] It refers to the component of momentum which is perpendicular to the beam line~\cite{CMS:PF2017}. This quantity is preferred over the total momentum because the longitudinal momentum component (along the beam axis) is dominated by the remnants of the colliding protons, which carry unknown momentum fractions. In contrast, the transverse momentum is directly associated with the hard scattering process at the interaction vertex, making it a more robust observable for characterizing the collision dynamics~\cite{CMS:PF2017}.
    
    \item[Azimuthal Angle ($\phi$)] it measures the angle in the transverse plane relative to the $x$-axis, providing the directional component perpendicular to the beam line~\cite{CMS:2008xjf}.
\end{description}

\begin{center}
  		% CMS detector - left perspective
		\tdplotsetmaincoords{75}{50} % to reset previous setting
		\begin{tikzpicture}[scale=2.6,tdplot_main_coords,rotate around x=90]
			
			% VARIABLES
			\def\rvec{\L/2/cos(\thetavec)}
			\def\thetavec{18}
			\def\phivec{60}
			\def\L{3.3}    % detector length
			\def\R{0.75}   % detector cylinder radius
			\def\l{4.3}    % beam pipe length
			\def\r{0.04}   % beam pipe radius
			\def\rt{0.042} % beam pipe radius + line thickness
			\def\xmax{1}   % maximum x axis
			\def\ymax{1}   % maximum y axis
			\def\zmin{-\l/2-0.2} % minimum z axis
			\def\zmax{\l/2+0.3}  % maximum z axis
			\def\w{0.3}
			\coordinate (O) at (0,0,0);
			\coordinate (Z) at (0,0,\L/2);
			\tdplotsetcoord{O'}{0.022}{\thetavec}{\phivec} % slightly shifted origin
			\tdplotsetcoord{O''}{0.018}{90}{\phivec} % slightly shifted origin
			\tdplotsetcoord{P}{\rvec}{\thetavec}{\phivec}
			
			% CYLINDER behind
			\def\ang{19} % rotate lines to simulate cylinder
			\fill[top color=red!50!black!4,bottom color=red!60!black!2,rotate around z=\ang]
			(0,\R,\L/2) --++ (0,0,-\L) arc(90:270:\R) --++ (0,0,\L) arc(270:90:\R) -- cycle;
			\fill[detector surface] % transverse plane at z=L/2
			(0,0,\L/2) --++ (0,\R,0) arc(90:270:\R) -- cycle;
			\fill[detector surface] % transverse plane at z=-L/2
			(0,0,-\L/2) --++ (0,\R,0) arc(90:270:\R) -- cycle;
			\tdplotdrawarc[detector]{(0,0,\L/2)}{\R}{0}{360}{}{}
			\tdplotdrawarc[detector,thin]{(0,0,-\L/2)}{\R}{0}{360}{}{}
			%\draw[detector,canvas is yx plane at z=-\L/2] (0,0,0) circle(\R);
			\draw[detector,thin, dashed] % transverse plane at z=0
			(90-\ang:\R) arc (90-\ang:270:\R);
			\draw[detector] (0,0,-\L/2)++(90:\R) --++ (0,0,\L); % top horizontal
			\draw[detector] (0,0,-\L/2)++(-90:\R) --++ (0,0,\L); % bottom horizontal
			
			% BEAM PIPE
			\tdplotdrawarc[beam pipe]{(0,0,\l/2)}{\r}{0}{360}{}{}
			%\tdplotdrawarc[beam pipe]{(0,0,-\l/2)}{\r}{\ang-90}{90}{}{}
			%\draw[beam pipe] % cylindric beam pipe
			%  (0,\r,-\l/2) --++ (0,0,\l) arc(90:-90:\r)
			%  --++ (0,0,-\l) arc(-90:90:\r);
			\draw[beam pipe] % beam pipe, thinner in middle
			(0,\r,-\l/2) -- (0,\r,-0.2*\l) -- (90:0.5*\r)
			-- (0,\r,0.2*\l) -- (0,\r,0.5*\l) arc(90:-90:\r)
			-- (0,-\r,0.2*\l) -- (-90:0.5*\r) --
			(0,-\r,-0.2*\l) -- (0,-\r,-\l/2) arc(-90:90:\r);
			\draw[beam pipe] (0,0,\l/2) circle(\r);
			
			% AXES
			%\draw[thick,->] (0,0,0) -- (0,0,1) node[below right]{$z$}; % short
			\draw[axis,-] (0,0,\zmin) -- (0,0,0); % long
			\fill[CMScol] (O) circle(0.5pt) node[right=1,below=1] {IP};
			\draw[axis] (0,0,0.020) -- (0,0,\zmax) node[right=3,above=0.1]{$z$}; % long
			\draw[axis] (0,0.019,0) -- (0,\ymax,0) node[below left]{$y$};
			\draw[axis] (0.022,0,0) -- (\xmax,0,0) node[below=1,right=-2]{$x$};
			
			% LABELS
			\node[mydarkred,above] at (0,\ymax,0) {$\eta=0$};
			\node[mydarkred,above=0.6, left] at (0,\R,0.3*\L) {$\eta>0$};
			\node[mydarkred,above=0.7, right] at (0,\R,-0.2*\L) {$\eta<0$};
			\node[mydarkred,below=1,left] at (0,0,\zmax) {$\eta=\infty$};
			\node[mydarkred,above=1,right] at (0,0,\zmin) {$\eta=-\infty$};
			
			% VECTORS
			%\fill[radius=0.4,red] (P) circle;
			\draw[dashed,myred] (P)  -- (Pxy);
			\draw[dashed,myred] (Py) -- (Pxy);
			\draw[dashed,myred] (P) -- (Pz);
			
			
			\draw[->,miverde,line cap=round,draw opacity=0.9] (O') -- (P) node[anchor=-30] {\contour{white}{$\va*{p}$}};
			\draw[->,miverde,line cap=round] (O') -- (P) node[anchor=-30] {$\va*{p}$};
			
			\draw[->,azulF,line cap=round,draw opacity=0.9] (O') -- (Pxy) node[right, anchor=-100] {\contour{white}{$\va*{p}_T$}};
			% \draw[->,azulF,line cap=round] (O') -- (Pxy) node[right , anchor=-100] {$\va*{p}_T$};
			
			
			% CYLINDER front
			\draw[beam pipe,fill=none] (0,\r,-\l/2) arc(90:-90:\r);
			\fill[detector surface] % transverse plane at z=L/2
			(0,\rt,\L/2) --++ (0,\R-\rt,0) arc(90:-90:\R) --++ (0,\R-\rt,0) arc(-90:90:\rt);
			\fill[detector surface] % transverse plane at z=-L/2
			(0,\rt,-\L/2) --++ (0,\R-\rt,0) arc(90:-90:\R) --++ (0,\R-\rt,0) arc(-90:90:\rt);
			\tdplotdrawarc[detector]{(0,0,\L/2)}{\R}{-90}{90}{}{} % transverse plane at z=L/2
			\tdplotdrawarc[detector]{(0,0,-\L/2)}{\R}{-90}{90}{}{} % transverse plane at z=-L/2
			\draw[beam pipe,fill=none] (0,\r,\l/2) arc(90:-90:\r);
			\draw[detector,very thin, dashed] % transverse plane at z=0
			(90-\ang:\R) arc (90-\ang:-90:\R);
			
			% ANGLES
			\tdplotdrawarc[thick,red!57!black!3] % contour
			{(O)}{0.2}{4}{0.7*\phivec}{}{}

			% white to contour
			\tdplotdrawarc[draw=azulF, line width=0.6pt, draw opacity=0.9]{(O)}{0.2}{0}{\phivec}{above=2,right=0.75,anchor=-30,text=black}{\contour{white}{$\phi$}}
			\tdplotdrawarc[->, azulF]{(O)}{0.2}{0}{\phivec}{above=2,right=0.75,anchor=-30}{$\phi$}


			\tdplotdrawarc[->,rotate around z=\phivec-90,rotate around y=-90]
			{(O)}{0.88}{0}{\thetavec}{anchor=mid east}{$\theta$}
			\tdplotdrawarc[thick,red!58!black!4,rotate around z=\phivec-90,rotate around y=-90] % contour
			{(O)}{0.3}{88}{0.5*(90+\thetavec)}{}{}
			\tdplotdrawarc[-{>[flex'=1]},rotate around z=\phivec-90,rotate around y=-90,line cap=round]
			{(O)}{0.3}{90}{\thetavec}{above=4.5,right=0.5,anchor=mid east}{$\eta$}
			\draw[mydarkred] (0,0,\L/2) --++ (\R,0,0);
			\tdplotdrawarc[thick,red!60!black!6] % contour
			{(Z)}{0.2}{4}{0.7*\phivec}{}{}
			\tdplotdrawarc[draw=none,opacity=0.8]{(Z)}{0.2}{0}{\phivec}{above=2,right=0.7,anchor=-30}{\contour{red!60!black!6}{$\phi$}}
			\tdplotdrawarc[->]{(Z)}{0.2}{0}{\phivec}{above=2,right=0.7,anchor=-30}{$\phi$}
			
			% COMPASS - CMS-ATLAS axis has a ~12° declination (http://googlecompass.com)
			\begin{scope}[shift={(1.1*\R,-\R,0.2*\L)},rotate around y=12]
				\draw[<->,black!50] (-\w,0,0) -- (\w,0,0);
				\draw[<->,black!50] (0,0,-\w) -- (0,0,\w);
				\node[left,black!50,scale=0.6] at (-\w,0,0) {N};
				\node[below=3,left=-2,green!20!black!50,scale=0.6] at (0,0,\w) {Jura};
				%\node[below=1,right,black!50,scale=0.6,align=center] at (\w,0,0) {center of\\the LHC};
				%\node[below=1,right,blue!30!black!50,scale=0.6] at (\w,0,0) {ATLAS};
			\end{scope}
			\draw[->,thick,orange!30!black] (1.4*\w,-\R,-0.1*\L) --++ (2*\w,0,0)
			node[right,scale=0.8,align=center] {center of\\[-1pt]the LHC};
			
		\end{tikzpicture}
  \captionof{figure}{Detailed reparametrization of the coordinate system employed by the CMS experiment (retrieved from~\parencite{cmsplots}).}\label{fig_cms_coor}
\end{center}

Together, the triplet $(p_T, \phi, \eta)$ forms a natural coordinate system that fully describes a particle's three-momentum vector at a hadron collider~\cite{CMS:PF2017,1674-1137-40-10-100001}. The full four-momentum $(E, p_x, p_y, p_z)$ can be reconstructed from these quantities, typically supplemented by either the particle's mass hypothesis (for identified particles like electrons or muons) or the energy deposited in the calorimeters (for neutral objects like photons or jets)~\cite{CMS_EGM_17001,CMS_MUON_17001,Cacciari:2011ma}. This $(p_T, \phi, \eta)$ system serves as the fundamental framework for defining physical objects, calculating event variables, and performing analyses at the LHC, providing both experimental convenience and physical insight into the collision dynamics~\cite{CMS:PF2017,Cacciari:2011ma}.

\begin{figure}[h!]
  \centering
    \includegraphics[width=\textwidth]{Images/Layers.pdf}
    \caption{Illustration of high-energy particles being identified by consecutive types of subdetectors in a typical collider experiment. The curvature of the tracks in the magnetic field is not shown for simplicity. Representation of which particles and kinds of detectors are used in a multipurpose detector such as CMS or ATLAS. (retrieved from \cite{Rubbia:2022hry})}\label{fig_layers}
\end{figure}

A key challenge is isolating the primary hard interaction from the additional concurrent PU interactions~\cite{Apollinari2017_HLLHC,Bertolini:2014}. This is accomplished by reconstructing distinct interaction vertices along the beam direction and associating charged particles to their point of origin using the CMS tracking and vertexing algorithms~\cite{CMS:TRK2014,CMS:PF2017}. The ultimate aim of the reconstruction chain is to identify all stable particles produced in the collision and measure their four-momenta, thereby enabling the identification of the underlying fundamental process~\cite{CMS:PF2017}.

However, the reconstruction is complicated by several factors~\cite{CMS:2008xjf,CMS:PF2017}. The initial state of the colliding protons is not fully known, as they are composite particles made up of quarks and gluons (collectively referred to as partons)~\cite{Collins:1989,NNPDF:2014otw}. The fraction of the proton's momentum carried by each parton is described by parton distribution functions (PDFs), which are determined experimentally. Among the available groups of PDFs, LHC analyses using Run II/III data have mainly implemented the PDF4LHC~\cite{Butterworth:2015oua,NNPDF:2014otw} set. 
As a result, the longitudinal momentum is not constrained on an event-by-event basis, and it is not possible to reconstruct the four-momentum of the colliding protons~\cite{Collins:1989}. 
Furthermore, not all particles are stable enough to reach the detector; some decay before being detected, and only their decay products are observed~\cite{1674-1137-40-10-100001}. The design of a collider experiment, illustrated in Fig.~\ref{fig_layers}, is optimized for the identification and energy measurement of the particles produced in high-energy collisions~\cite{CMS:2008xjf,deFavereau:2013fsa}. Using the information from the different particle sub-detectors, it is possible to differentiate signatures from various particle types. This information is utilized by software algorithms, optimized for particle reconstruction and identification, to calculate the likelihood that a detector signature was  created by a specific type of particle. 

Finally, some hypothetical particles, such as those comprising dark matter, along with known neutrinos, interact very weakly with matter and escape direct detection~\cite{Bertone2005_DM_review,1674-1137-40-10-100001}. Therefore, a hermetic detector design is crucial to infer their presence by accurately measuring the imbalance of energy and momentum in the transverse plane, referred to as missing transverse momentum~\cite{CMS:2019ctu,CMS:PF2017}.

In this way, a typical collider experiment comprises several main detector subsystems that are used jointly to detect and measure the properties of particles produced in the collision~\cite{CMS:2008xjf,ATLAS:2008xda,deFavereau:2013fsa,CMS:PF2017}. A \textit{schematic representation} of such a generic multipurpose detector is shown in Fig.~\ref{fig_detector}~\cite{CMS:2008xjf,deFavereau:2013fsa,Lee:2018pag}. The detector features an "onion-like" design of several concentric layers, each optimized to identify different types of particles and measure their properties~\cite{CMS:2008xjf,CMS:PF2017}.

\begin{figure}[h!]
  \centering
    \includegraphics[width=0.85\textwidth]{Images/transversal_detector.pdf}
    \caption{Schematic representation of a transverse section of a generic multipurpose detector. The inner detector (ID) is used to measure the trajectories of charged particles, the electromagnetic calorimeter (ECAL) measures the energy of photons and electrons, the hadronic calorimeter (HCAL) measures the energy of hadrons, and the muon system (MS) identifies and measures muons. The missing transverse momentum (MET) is inferred from the momentum imbalance in the transverse plane (retrieved from \cite{Lee:2018pag})}\label{fig_detector}
\end{figure}

The innermost subsystem, the inner detector (ID) or tracker, is immersed in a strong axial magnetic field (typically 1--4 T)~\cite{CMS:2008xjf,CMS:TRK2014}. It is designed to reconstruct the trajectories of charged particles, which are bent by the magnetic field~\cite{CMS:TRK2014,CMS:PF2017}. The direction and curvature of these trajectories, called \textbf{tracks}, allows to estimate the  particle's momentum vector and electric charge~\cite{CMS:TRK2014,1674-1137-40-10-100001}. The most common long-lived charged particles from the SM are the so called light leptons (electrons $e$ and muons $\mu$) and hadrons (pions $\pi$, kaons $K$, and protons $p$)~\cite{1674-1137-40-10-100001}. In some detectors, the ID is complemented by a Cherenkov light detector (RICH) to measure particle velocity and aid particle identification~\cite{1674-1137-40-10-100001,Leo_1994}. Combined with the momentum measurement, this velocity helps determine the particle mass, allowing for differentiation between pions, kaons, and protons~\cite{1674-1137-40-10-100001,Leo_1994}.

After the tracker, particles enter the electromagnetic calorimeter (ECAL), which is designed to fully absorb photons, electrons, and positrons~\cite{CMS_EGM_17001,CMS:2008xjf}. These particles deposit all their energy in the ECAL by initiating an electromagnetic shower via bremsstrahlung and $e^{+}e^{-}$ pair production~\cite{CMS_EGM_17001}. Electrons are identified as charged tracks that point to a compact, high-energy deposit in the ECAL~\cite{CMS_EGM_17001}.

The hadronic calorimeter (HCAL) surrounds the ECAL and is built to absorb hadrons and measure their energy through hadronic interactions~\cite{CMS:2008xjf,deFavereau:2013fsa}. High-energy quarks and gluons hadronize into collimated sprays of hadrons known as \textbf{jets}. The energy of jets is measured by combining calorimeter deposits with track momenta. The reconstruction of particles using the information of the different detector subsystems is formalized in particle‑flow reconstruction~\cite{CMS:PF2017,Cacciari:2011ma,Cacciari:2008gp}.

Muons are unique as they can penetrate the calorimeters; a dedicated muon system outside the calorimeters identifies and measures muons, and muon tracks in the ID are matched to tracks in the muon chambers~\cite{CMS_MUON_17001,CMS:2008xjf}.

Since the detector is nearly hermetic (covering almost the full solid angle), momentum conservation in the plane transverse to the beam line ($x$-$y$ plane) is a powerful tool. The vector sum of the momenta in the transverse plane ($\vec{p}_T$) of all detected particles should be zero. Any significant imbalance indicates the presence of undetected  neutral particles that did not interact with the detector, such as neutrinos or new hypothetical dark matter particles. This imbalance is referred to as missing transverse momentum ($\vec{p}_T^{\text{miss}} $) and is formally defined as:
\begin{equation}
  \vec{p}_T^{\text{miss}} \equiv -\sum_i \vec{p}_{T,i},
  \label{eq:ptmiss}
\end{equation}
where the sum runs over all reconstructed particles (e.g., leptons, photons, jets) or calorimeter deposits in the event.

The detector design, optimized for identifying and measuring SM particles, also makes it a powerful instrument to search for BSM physics.

\subsection{Collision Parameters}\label{sec:cross_sec_lumi}

One of the main objectives of particle physics experiments is to quantify how frequently different processes occur and to characterize the properties of the particles involved. The expected rate of a given process, either from the SM or from new physics, is quantified using production \textbf{cross-sections}, a theoretical estimate, and the \textbf{luminosity}, a parameter that accounts for the amount of data delivered by the accelerator.

In essence, the  cross-section ($\sigma$) quantifies the probability for a specific process to occur. Formally, it represents the effective area of a target particle presented to an incoming beam particle for an interaction to happen. It has units of area, typically barn (b), where $1\,\text{b} = 10^{-28}\,\text{m}^2$.

\marginpar{\footnotesize The scales $\mu_F$ and $\mu_R$ are unphysical parameters introduced in perturbativ calculations (typically QCD calculations). In an exact calculation to all orders, physical observables would be independent of these scales. However, at finite perturbative order (NLO, NNLO, etc.), a residual scale dependence remains, proportional to the next uncalculated order in $g_s$. This dependence is used to estimate theoretical uncertainties by varying $\mu_R, \mu_F \in [Q/2, 2Q]$ with $1/2 \leq \mu_R/\mu_F \leq 2$, where $Q$ is the characteristic hard scale of the process.}
In the context of $pp$ collisions at the LHC, the concept is generalized. Since both colliding particles are composite, the cross-section for a specific process is calculated by considering the interactions between their constituent partons (quarks and gluons). The total cross-section for a process $pp \to X$ is given by the convolution of the PDFs and the partonic cross-section $\hat{\sigma}_{ij \to X}$ \parencite[Eq. 19.45]{Rubbia:2022hry}:
\begin{equation}
\sigma(pp \to X) = \sum_{i,j} \int_0^1\int_0^1 dx_1 dx_2\, f_i(x_1, \mu_F^2) f_j(x_2, \mu_F^2)\, \hat{\sigma}_{ij \to X}(\hat{s}, \mu_F^2, \mu_R^2),
\label{eq:cross_section}
\end{equation}
where:
\begin{itemize}
    \item the sum runs over all possible parton types $i, j$ (e.g., $u, d, g$) in the two protons.
    \item $f_i(x, \mu_F^2)$ is the PDF, representing the probability density to find a parton of type $i$ carrying a fraction $x$ of the proton's momentum at a factorization scale $\mu_F$.
		\item $\hat{s} = x_1 x_2 s$ is the square of the center-of-mass energy for the colliding partons, with $s$ being the square of the $pp$ center-of-mass energy (e.g., 13.6 TeV).
    \item $\mu_F$ is the factorization scale, which separates the long-distance physics (PDFs) from the short-distance hard scattering ($\hat{\sigma}$).
    \item $\mu_R$ is the renormalization scale, at which the QCD coupling constant $g_s(\mu_R)$ is evaluated in the perturbative calculation of $\hat{\sigma}$.
    \item $\hat{\sigma}_{ij \to X}$ is the partonic cross-section for the hard scattering process $ij \to X$.
\end{itemize}

Then, on one side, the cross-section $\sigma$ is a theoretical quantity that encapsulates the fundamental physics of the interaction, independent of the accelerator's performance. On the other side, the \textbf{luminosity} ($\mathcal{L}$) is a property of the particle accelerator and beams. It measures the density of particles in the colliding beams and thus the rate at which interactions can occur. The instantaneous luminosity is defined by:
\begin{equation}
\mathcal{L} = \frac{\mathcal{F} n_1 n_2}{4\pi \sigma_x \sigma_y},
\end{equation}
where $\mathcal{F}$ is the revolution frequency of the bunches, $n_1$ and $n_2$ are the numbers of particles in each bunch, and $\sigma_x$ and $\sigma_y$ are the transverse dimensions of the beams at the interaction point. The integrated luminosity is the integral of the instantaneous luminosity over time:
\begin{equation}
L = \int \mathcal{L}\, dt.
\end{equation}
The primary unit of integrated luminosity is the inverse barn ($\text{b}^{-1}$), commonly $\text{fb}^{-1}$.

Theoretically, the expected number of events of a SM or BSM process is estimated as

\begin{equation}
N_{\text{theory}} = \sigma \cdot L.
\label{eq:N_sigma_theory}
\end{equation}


Under the context of a collider experiment one has to include the detector acceptance and efficiency of the particle identification and selection criteria used to discriminate the signal of interest among other processes. The variable $\epsilon$ is defined as the product of the acceptance ($\mathcal{A}$) and the cumulative efficiency of all the selection criteria used to estimate the signal rate above the backgrounds: 

\begin{equation}
    \epsilon = \qty( \prod_i \epsilon_i ) \times \mathcal{A}. 
\end{equation}

Therefore, the expected number of events of a process of interest, for either a SM or BSM process, is estimated using the following equation~\cite{CMS:PF2017,deFavereau:2013fsa}:

\begin{equation}
N = \sigma \cdot L \cdot \epsilon.
\label{eq:N_sigma_experiment}
\end{equation}

Equation \ref{eq:N_sigma_experiment} allows one to estimate the expected number of observed events, accounting for reconstruction, particle identification, detector resolution, and acceptance effects, among other experimental considerations~\cite{CMS:PF2017}.

Note that the integrated luminosity $L$ is a parameter that can be measured from the accelerator's performance~\cite{lumiRef}, while $\epsilon$ can be estimated using information from the detector calibration and simulation (including event generation, parton shower, and detector simulation)~\cite{Alwall:2014hca,Sjostrand:2014zea,deFavereau:2013fsa}. For a known process, if we know the expected number of events, we can use  Equation~\ref{eq:N_sigma_experiment} and solve for $\sigma$ to extract a measurement of the production rate, through a statistical interpretation based on likelihood methods~\cite{Cowan:2011}. 

In the case of searches for BSM physics, we calculate the expected background $N_\text{bkg}$ from known SM processes using Monte Carlo and data-driven techniques~\cite{Alwall:2014hca,Cacciari:2011ma}. Then, one studies the agreement in the event rates and shapes of various kinematic and topological distributions of interest, between the observed number of events in data ($N_\text{obs}$) and the expected backgrounds from SM processes. Any significant deviation in a specific region in one of the relevant observables, for example in a reconstructed mass, $N_\text{obs} - N_\text{bkg}$, can be interpreted as a potential signal. Then, the significance of such difference—both local and global—can be determined using a profile-binned likelihood test, to determine the probability that a signal process of interest explains, within the associated statistical and systematic uncertainties, the discrepancy between data and the background~\cite{Read:2002,Rolke:2005,FeldmanCousins:1998,Segura:2024srj}. 


 
\section{Detectors and Subsystems}

A typical collider experiment comprises several main detector subsystems that are used jointly to detect and measure the properties of particles produced in the collision. A \textit{schematic representation} of such a generic multipurpose detector is shown in Fig.~\ref{fig_detector}. The detector is typically composed of several concentric layers, each designed to measure different properties of the particles produced in the collisions. 

\begin{center}
	\includegraphics[width=0.8\textwidth]{Images/transversal_detector.pdf}
	\captionof{figure}{Schematic representation of transversal section of a generic multipurpose detector. The inner detector (ID) is used to measure the trajectories of charged particles, the electromagnetic calorimeter (ECAL) measures the energy of photons and electrons, the hadronic calorimeter (HCAL) measures the energy of hadrons, and the muon system (MS) measures the trajectories of muons. The missing transverse energy (MET) is measured by combining information from all subsystems..}\label{fig_detector}
\end{center}

The innermost subsystem, called the inner detector (ID), is designed to detect electrically charged particles that are long-lived enough to traverse the ID. The most common such particles from the SM are two charged leptons (the electron $e$ and the muon $\mu$ ) and three hadrons (the pion $\pi$, kaon $K$, and proton $p$ ). Regions of ionization produced by such a particle in solid-state or gaseous detector sensors are detected as spatial hits that are fit into a trajectory, referred to as a \textbf{track}. The direction and curvature of the track in a magnetic field yield the particle's momentum vector and electric charge. In some detectors, the ID is enclosed in a Cherenkov-light detector used to measure the velocity of the tracked particles. Combined with the momentum measurement in the ID, this yields the particle mass with sufficient resolution to differentiate between pions, kaons, and protons in a relevant momentum range.

After passing through the tracker, particles produced in the collisions typically enter an electromagnetic calorimeter (ECAL), designed to measure the energies of photons, electrons and positrons. The energy measurement exploits the properties of electromagnetic shower production via photon radiation and $e^{+} e^{-}$ pair production, resulting from the interaction of energetic particles with the ECAL material.

Hadrons deposit energy via hadronic interactions with the detector material. Since this process involves large fluctuations and a variety of energy-deposition mechanisms, precise hadron-energy measurement is achievable only at high-energy colliders, where fluctuations are effectively averaged out. In particular, high-energy quarks and gluons hadronize into a collimated spray of hadrons known as a \textbf{jet}. Containing the jet requires use of a deep hadronic calorimeter (HCAL) beyond the ECAL. While a jet can be identified solely in the calorimeters, its energy is nowadays measured from a combination of the momenta of tracks in the ID and the signals integrated in the ECAL and HCAL. 

The signals from the calorimeters, know as \textbf{towers}, are grouped into jets using a jet clustering algorithm. If and hadronic particle is neutral, it will not leave a track in the ID, but it will still deposit energy in the towers. So, the towers are used to measure the energy of neutral particles, such as photons and neutral hadrons, while the ID tracks are used to measure the energy of charged particles. This approach is known as particle flow (PF) reconstruction and provides a more accurate measurement of the energy of jets. 

Muons do not undergo hadronic interactions, and are heavy enough that they lose energy due to ionization at a low rate. Therefore, they lose only a few GeV while traversing a typical LHC-detector calorimeter. Using this property to identify them, a muon system (MS) is built outside the calorimeter. In high-energy collider detectors, the MS is usually immersed in a magnetic field in order to measure the momenta of muons. Tracks reconstructed in the MS are often combined with tracks in the ID to obtain a high-quality momentum measurement.

When studying final states that include long-lived, weakly interacting particles, such as neutrinos in the SM or dark matter candidates in BSM models, an important reconstructed quantity is missing momentum.  Using three-momentum conservation and the approximate hermeticity of the detector, it is possible to measure the momentum imbalance in the event and to infer the combined momentum of the invisible set of particles. Since the interacting partons in proton collisions generally carry different fractions of the momenta of the incoming hadrons and many of the particles produced fall outside of the acceptance of the sensitive detector, the summed momenta of measured final-state particles along the beam axis $z$ are not expected to cancel. Therefore, experiments at the LHC measure the missing transverse momentum, denoted $E_{\mathrm{T}}^{\text {miss }}$ known as Missing Energy Transverse (\textbf{MET}), where momentum balance is assumed only in the $x-y$ plane transverse to the beam direction.

Collider detectors are mostly designed and constructed for optimal detection of SM particles produced in the collision. However, they can also be used to search for new physics (NP) beyond the SM. In this case, the detector is used to search for signatures of NP, such as new particles or interactions that are not predicted by the SM. The detector subsystems are designed to be sensitive to a wide range of particles and interactions, allowing for the detection of a variety of NP signatures.


\subsection{Jets Reconstruction}

Quarks and gluons are never observed as free particles because of colour confinement. Nevertheless, perturbative QCD treats them as the relevant short-distance degrees of freedom: factorization theorems and asymptotic freedom justify computing hard-scattering matrix elements for incoming and outgoing partons even though QCD becomes non-perturbative at low scales. The strong coupling \(\alpha_s\) grows large and effectively ``blows up'' around the confinement scale \(\Lambda_{\mathrm{QCD}}\); consequently something must happen to quarks and gluons before they reach the detector. In practice the gluon and all quarks except the top hadronize, producing cascades of baryons and mesons that themselves undergo further decays. At the LHC these hadrons typically carry energies comparable to the electroweak scale, and relativistic boosts tend to collimate their decay products into narrow bunches. Those collimated collections of hadrons are the jets we measure at hadron colliders and the objects we use to infer the partons produced in the hard interaction.

Each high-energy parton produced in a collision, such as a quark from the process $gg \rightarrow q\bar{q}$, undergoes hadronization over a distance scale of $\sim10^{-15}\mathrm{m}$, producing a jet of hadrons. The energy composition of these jets is phenomenologically well-established: on average, approximately $60\%$ of the energy is carried by charged particles (mostly $\pi^{\pm}, K^{\pm}$), $30\%$ by photons from $\pi^0 \rightarrow \gamma\gamma$ decays, and $10\%$ by neutral hadrons (mostly neutrons, $K^0, \Lambda^0$). In high-energy jets, the particles are often too collimated to be resolved individually by the calorimeter segmentation. Nevertheless, the jet's energy and momentum can be reconstructed from the total energy deposited.

Phenomenologically one usually assumes that each high-energy parton yields a jet and that the measured jet four-momentum can, to useful accuracy, be related to the original parton four-momentum. Jets are therefore defined operationally using recombination (clustering) algorithms such as Cambridge-Aachen or the (anti-)kT family. Experimentally this means grouping a large number of energy depositions (or particle-flow candidates) observed in the calorimeters and tracker into a much smaller set of jets or subjets. Nothing in the raw detector data, however, indicates a priori how many jets there should be: the clustering procedure and the choice of a resolution scale fix the outcome. In practice one must either specify the desired number of final jets or choose a resolution/stop criterion (for example a distance parameter \(R\), a clustering distance cut, or a jet-mass/subjet-resolution threshold) that determines the smallest substructure to be considered a separate parton-like object.

Modern reconstruction at the LHC typically uses particle-flow (PF) candidates as input together with infrared- and collinear-safe clustering algorithms to define jet four-momenta. The anti-\(k_T\) algorithm~\parencite{Cacciari:2008gp}, implemented in \texttt{FastJet}~\parencite{Cacciari:2011ma}, is widely used in ATLAS and CMS; it groups candidates by proximity in the rapidity–azimuth \((y,\phi)\) plane with a typical distance parameter \(R\sim0.4\)–0.6 and is relatively insensitive to soft radiation and pileup. After clustering, jet energy corrections (JEC) derived from simulation and in-situ calibrations compensate for detector response, pileup, and underlying-event effects, while jet-substructure and tagging algorithms help infer the flavour and origin of the initiating parton.

\subsubsection{Jet algorithms}

Recombination (or sequential clustering) algorithms formalise the intuitive idea that parton showering produces collinear and soft splittings: two nearby and kinematically compatible subjets are merged if they are more likely to have originated from a single parton. A practical implementation requires a measure of ``distance'' between objects; common choices combine an angular separation in the rapidity–azimuth plane, \(\Delta R_{ij}\), with a transverse-momentum weighting. Typical distance measures are
\[
\begin{array}{lll}
k_T: & y_{ij}=\dfrac{\Delta R_{ij}}{R}\min(p_{T,i},p_{T,j}), & y_{iB}=p_{T,i},\\[6pt]
\mathrm{C/A}: & y_{ij}=\dfrac{\Delta R_{ij}}{R}, & y_{iB}=1,\\[6pt]
\text{anti-}k_T: & y_{ij}=\dfrac{\Delta R_{ij}}{R}\min(p_{T,i}^{-1},p_{T,j}^{-1}), & y_{iB}=p_{T,i}^{-1}.
\end{array}
\]
The parameter \(R\) balances jet–jet and jet–beam criteria and sets the geometric size of jets; in LHC analyses typical values are \(R\sim0.4\text{--}0.7\) depending on the physics target.

Two operational modes are useful to distinguish. In an exclusive algorithm one supplies a resolution scale \(y_{\text{cut}}\) and proceeds iteratively:
\begin{enumerate}
  \item compute \(y^{\min}=\min_{i,j}\{y_{ij},y_{iB}\}\);
  \item if \(y^{\min}=y_{ij}<y_{\text{cut}}\) merge \(i\) and \(j\) and repeat;
  \item if \(y^{\min}=y_{iB}<y_{\text{cut}}\) remove \(i\) as beam radiation and repeat;
  \item stop when \(y^{\min}>y_{\text{cut}}\) and keep remaining subjets as jets.
\end{enumerate}
An inclusive algorithm omits \(y_{\text{cut}}\) and instead declares a subjet a final-state jet when its jet–beam distance is the smallest quantity; iteration continues until no inputs remain. Inclusive algorithms therefore produce a variable number of jets, while exclusive algorithms deliver a scale-dependent fixed set.

A practical question is how to combine the kinematics of merged objects. The most common choice in modern experiments is the E-scheme: four-vectors are added, which preserves energy–momentum and yields a physical jet mass useful for substructure and boosted-object tagging. An alternative is to sum three-momenta and rescale the energy to enforce a massless jet; this can be appropriate when the analysis targets massless parton kinematics, but it discards potentially useful jet-mass information.

From a theoretical and experimental viewpoint important properties are infrared and collinear safety: a jet algorithm should give stable results under emission of soft particles or collinear splittings. The \(k_T\), C/A and anti-\(k_T\) families are constructed to satisfy these requirements. Their practical behaviour differs: \(k_T\) naturally follows the physical shower history (soft-first clustering), C/A is purely geometric (useful for declustering and substructure studies), while anti-\(k_T\) produces regular, cone-like jets that are robust and convenient experimentally.

Corrections for pileup and the underlying event are necessary at the LHC. These corrections depend on the jet area (a well-defined concept for sequential algorithms) and are typically performed by estimating an event-wide transverse-momentum density and subtracting the corresponding contribution proportional to the jet area. Finally, because inclusive algorithms can produce jets arbitrarily close to the beam, a minimum jet \(p_T\) threshold (commonly 20–100 GeV depending on the analysis) is imposed to ensure experimental observability and theoretical control.

\subsection{$\tau$ Tagging at Multipurpose Detectors}

The $\tau$ lepton decays hadronically with a probability of $\sim65\%$, producing a narrow ``$\tau$-jet'' that contains only a few charged and neutral hadrons. Hadronic decays are dominated by one- and three-prong topologies and often include neutral pions that promptly convert to photons, giving a sizable electromagnetic fraction in the calorimeters. When the $\tau$ momentum is large compared to its mass the decay products are highly collimated: for $p_T>50\ \mathrm{GeV}$ roughly $90\%$ of the visible energy is contained within a cone of radius $R=\sqrt{(\Delta\eta)^2+(\Delta\varphi)^2}=0.2$. These properties motivate the use of small signal cones and narrow isolation annuli in reconstruction.

Identification exploits three complementary classes of observables:

\begin{itemize}
  \item Calorimetric isolation and shower-shape variables: hadronic $\tau$ decays deposit localized energy in ECAL+HCAL. Experiments use isolation sums and shape ratios to quantify peripheral activity. Example variables are
  \[
    \Delta E_T^{12}=\frac{\sum_{\;0.1<\Delta R<0.2} E_{T,j}}{\sum_{\;\Delta R<0.4} E_{T,i}},\qquad
    P_{\mathrm{ISOL}}=\sum_{\Delta R<0.40}E_T - \sum_{\Delta R<0.13}E_T,
  \]
  which suppress QCD jets that populate the isolation ring.
  \item Charged-track isolation and prong topology: the few, collimated charged tracks of a $\tau$ allow powerful selections. A common procedure defines a matching cone of radius $R_{\mathrm{m}}$ around the calorimeter jet axis to select candidate tracks above a $p_T^{\min}$ threshold. The leading track (tr$_1$) is found and a narrow signal cone $R_{\mathrm{S}}$ around tr$_1$ is used to count associated tracks (1 or 3 prongs preferred). A larger isolation cone $R_{\mathrm{I}}$ is scanned for additional tracks: if no extra tracks with $\Delta z_{\text{impact}}$ consistent with tr$_1$ are found, the candidate is isolated. Typical CMS/ATLAS choices are $R_{\mathrm{S}}\sim0.07$–0.15, $R_{\mathrm{I}}\sim0.3$–0.4, and $p_T^{\min}\sim1$–2 GeV, although values depend on analysis and working point.
  \item Lifetime and vertexing observables: the finite $\tau$ lifetime ($c\tau\approx87\ \mu\mathrm{m}$) produces displaced tracks and, for multi-prong decays, a reconstructible secondary vertex. Impact-parameter significances (2D or 3D) and secondary-vertex properties (mass, flight length significance) are used to separate genuine taus from prompt jets or leptons.
\end{itemize}

Additional discriminants include the invariant mass of the visible decay products computed from tracks and calorimeter clusters (with care to avoid double counting), electromagnetic energy fractions (sensitive to $\pi^0\to\gamma\gamma$), and dedicated shower-strip grouping for nearby photons. For example, invariant-mass reconstruction commonly uses a jet cone $\Delta R_{\text{jet}}\lesssim0.4$ while excluding calorimeter clusters matched to tracks by a minimum separation $\Delta R_{\text{track}}\gtrsim0.08$ to reduce double counting.

Reconstruction algorithms combine these inputs. CMS's Hadron-Plus-Strips (HPS) and modern DeepTau methods explicitly build decay-mode hypotheses and use strip-clustering of photons plus multivariate or deep-learning discriminators to reject jets, electrons, and muons~\parencite{CMS:2022ydz,CMS_DeepTau}. ATLAS employs analogous calorimeter+track based MVAs and BDTs~\parencite{ATLAS:2022fgo}. Typical working points trade efficiency versus background: medium points often give $\tau_{\mathrm{h}}$ efficiencies of order 50–70\% with light-jet misidentification rates in the per-mille to percent range, depending on kinematics and pileup.

Practical implementations tune cone sizes, isolation thresholds, and MVA inputs to the kinematic region and analysis goals; the choice of working point is driven by the signal-to-background optimization for the search or measurement at hand.

\subsection{B Tagging at Multipurpose Detectors}

Jets originating from bottom quarks ($b$-jets) exhibit several distinctive properties that enable their identification. The relatively long lifetime of $b$ hadrons (order 1.5 ps) produces displaced charged tracks and often reconstructible secondary vertices a few millimetres from the primary interaction point. The large $b$-hadron mass yields decay products with sizable transverse momentum relative to the jet axis, and semileptonic branching fractions produce soft electrons or muons inside the jet. These features form the basis for $b$-tagging.

Practical algorithms exploit individual signatures or combine them:
\begin{itemize}
  \item \textbf{Track-counting:} counts tracks with large impact-parameter significance to identify a $b$-like topology.
  \item \textbf{Jet-probability:} evaluates the compatibility of the jet's track impact-parameter distribution with the primary vertex hypothesis.
  \item \textbf{Secondary-vertex:} explicitly reconstructs displaced vertices and uses their kinematic properties (decay length significance, vertex mass).
  \item \textbf{Soft-lepton taggers:} identify low-$p_T$ leptons inside jets from semileptonic $b$ decays.
\end{itemize}

Modern taggers combine many observables in multivariate or deep-learning classifiers to maximize discrimination power. Contemporary approaches exploit rich, low-level inputs (track-by-track and PF-candidate information, vertex features and kinematics) and advanced network architectures:

\begin{itemize}
  \item Deep feed-forward networks (e.g. DeepCSV/DeepJet) ingest a large set of high-level and per-track inputs to produce powerful binary or multi-class discriminants that separate $b$, $c$ and light-flavour jets.
  \item Sequence models and recurrent networks (RNN-based taggers) process an arbitrary ordered list of track-level variables, improving sensitivity by directly exploiting per-track correlations and order-dependent information (impact-parameter sequences, track kinematics).
  \item Graph- and set-based architectures and combined particle+vertex networks (sometimes referred to as ``DeepFlavour''-style models) aggregate heterogeneous inputs and return per-flavour probabilities, enabling natural multi-classification and calibrated operating points.
\end{itemize}

These developments yield measurable performance gains: modern deep classifiers typically improve $b$ efficiency at fixed mistag rate (or reduce mistag rates at fixed efficiency) relative to classical taggers. The continuous output of such networks permits analyses to choose operating points (loose/medium/tight) corresponding to desired efficiencies or mistag targets. Calibration remains essential: data-driven scale factors derived from control samples (e.g. $t\bar t$, multijet, dilepton) are applied to correct simulation, and systematic uncertainties from the calibration, flavour composition, and kinematic extrapolation are propagated to physics results.

Examples in use are CMS DeepCSV / DeepJet and ATLAS MV2 / DL1~\parencite{CMS:2017wtu,ATLAS:2019bwq}, which illustrate the transition from expert-designed high-level variables to large-scale machine learning leveraging low-level detector information. Typical medium working points yield $b$-tag efficiencies of order 60–80\% with light-jet misidentification rates at or below the percent level; the precise choice of working point is tuned per analysis to optimise sensitivity while accounting for calibration and systematic uncertainties.

\subsection{The CMS Detector}

CMS is a general-purpose detector at the LHC~\parencite{CMS_2008}. With a length of 21.6~m, a diameter of 14.6~m, and a weight of 14,000 tonnes, its cylindrical geometry is divided into a central barrel section and two endcaps. This design provides hermetic coverage to accurately measure momentum and energy balance, which is crucial for identifying non-interacting particles like neutrinos through missing transverse energy.

The detector is constructed from concentric layers of sub-detectors, as illustrated in Figure~\ref{fig_cms}. The innermost component is the silicon tracker, comprising a pixel detector and silicon strip tracker. It reconstructs the trajectories of charged particles and measures their transverse momenta ($p_T$) with a resolution of $\approx 0.7\%$ for 10~GeV particles within a pseudorapidity range of $|\eta| < 2.5$.

\begin{center}
	\includegraphics[width=0.98\textwidth]{Images/CMS.png}
	\captionof{figure}{Diagram of the CMS detector showing its inner components (retrieved from~\parencite{CMS_2008}).}\label{fig_cms}
\end{center}

Surrounding the tracker is the calorimetric system. The electromagnetic calorimeter (ECAL) is made of lead-tungstate crystals. It is designed to measure electrons and photons with a high resolution of $\approx 0.6\%$ for 50~GeV electrons. The hadronic calorimeter (HCAL), located outside the ECAL, is a brass-scintillator sampling calorimeter that measures hadrons (e.g., charged pions, kaons, protons) with an energy resolution of $\approx 18\%$ for 50~GeV pions. Together, the ECAL and HCAL cover $|\eta| < 3$. The coverage is extended to $|\eta| < 5$ with steel and quartz-fiber hadron calorimeters in the forward regions.

A key feature of CMS is its large superconducting solenoid, which encloses the tracker and calorimeters. The solenoid is constructed from a niobium-titanium alloy and cooled to 4.2~K with liquid helium. It generates a uniform magnetic field of 3.8~T throughout the tracking volume, enabling precise momentum measurement from the curvature of charged particle tracks.

The outermost system is dedicated to muon identification and measurement. Gas-ionization detectors are embedded in the steel flux-return yoke that surrounds the solenoid. This system provides triggering and tracking capabilities for muons up to $|\eta| < 2.4$. The combination of the inner tracker and the muon system allows for a robust identification and momentum measurement of muons across a wide kinematic range.

The geometrical segmentation of the barrel and endcaps defines the detector's acceptance in terms of pseudorapidity. The central barrel provides optimal coverage for $|\eta| \lesssim 1.5$, while the endcaps extend the acceptance to $|\eta| \lesssim 2.5$ for the tracker and calorimeters, and to $|\eta| \lesssim 2.4$ for the muon system.

This segmentation impacts the detection efficiency. The silicon trackers are highly efficient in the barrel, where particles cross the layers perpendicularly. In the endcaps, the reduced hit multiplicity from shallow-angle traversals leads to a slight decrease in tracking efficiency and resolution. The calorimeters are also optimized to maintain performance across $\eta$, though the material budget and granularity vary.

Muon reconstruction performance exhibits regional differences. In the barrel, drift tubes (DTs) provide high spatial resolution, while in the endcaps, cathode strip chambers (CSCs) and resistive plate chambers (RPCs) are used to handle higher background rates and non-uniform magnetic fields. The assumed identification efficiency for muons (electrons) is 95\% (85\%), with a mis-identification rate of 0.3\% (0.6\%)~\parencite{CMS-PAS-FTR-13-014,CMS_MUON_17001,CMS_EGM_17001}.

For the identification of heavy-flavor jets, we adopt the DeepCSV algorithm~\parencite{CMS_BTV2016}. We use its ``medium'' working point, which provides a $b$-tagging efficiency of 70\% with a light-flavor jet misidentification rate of approximately 1\% across the entire $p_T$ spectrum. The ``loose'' (85\% efficiency, 10\% mis-id) and ``tight'' (45\% efficiency, 0.1\% mis-id) working points were also explored during the analysis optimization.

For hadronically decaying $\tau$ leptons ($\tau_h$), we use the DeepTau algorithm~\parencite{CMS_DeepTau}, which employs a deep neural network combining isolation and lifetime information to identify $\tau_h$ decay modes. The ``medium'' working point is chosen for this analysis, providing a $\tau_h$ identification efficiency of 70\% and a misidentification rate of 0.5\% for jets originating from light quarks and gluons. This working point was selected through an optimization process that maximized the discovery reach of the analysis.
 
\section{Phenomenological Pipeline}

The estimation of signal and background event yields is performed through a comprehensive Monte Carlo (MC) simulation pipeline~\cite{Christensen:2008py,Alloul:2013bka,Degrande:2011ua,Alwall:2014hca}. This approach, a cornerstone of high-energy physics research, enables robust studies of Beyond the Standard Model (BSM) scenarios by emulating the entire data collection and processing chain of a collider experiment~\cite{Sjostrand:2014zea,deFavereau:2013fsa}. The key advantages of this methodology include~\cite{Alwall:2014hca,Cacciari:2011ma}:

\begin{itemize}
    \item The ability to perform automated calculations of theoretical quantities such as cross-sections and decay widths for complex processes.
    \item Conducting feasibility studies and optimizing analysis strategies prior to data acquisition.
    \item Estimating the efficiency of complex event selection criteria and the geometric acceptance of the detector.
    \item Predicting the rates and kinematical distributions of both irreducible and reducible background processes.
    \item Comparing and distinguishing between different theoretical hypotheses for a potential discovered signal.
\end{itemize}

The simulation workflow is modular, reflecting the logical progression from a theoretical Lagrangian to simulated detector-level observables~\cite{Christensen:2008py,Alloul:2013bka,Degrande:2011ua,Alwall:2014hca}. A schematic view of this pipeline is presented in Figure~\ref{fig:sim_workflow}~\cite{Alwall:2014hca,deFavereau:2013fsa}. The process begins with the implementation of the theoretical model in \texttt{FeynRules} (v2.3.43)~\parencite{Christensen:2008py,Alloul:2013bka}. The Lagrangian of the new physics scenario, including all particle definitions, parameters, and interactions, is translated into a set of Feynman rules and exported in the Universal FeynRules Output (UFO) format~\cite{Degrande:2011ua}, interoperable with modern matrix-element generators~\cite{Alwall:2014hca}.

This UFO module, accompanied by a parameter card defining numerical values for masses and couplings, serves as input to \texttt{MadGraph5\_aMC@NLO} (v3.5.7)~\parencite{Alwall:2014bza,Alwall:2014hca}. Within MadGraph the hard process and corresponding matrix elements are generated and stored in Les Houches event (LHE) files; PDF choices (here NNPDF3.0 NLO~\parencite{NNPDF:2014otw}) and matching/merging settings (MLM/CKKW-type) are configured to control radiation and multi-jet overlap~\cite{Alwall:2007fs,Buckley:2015}. Parton-level LHE events are passed to \texttt{PYTHIA 8} for showering, hadronization and decays~\parencite{Sjostrand:2014zea}, and the generator output is exchanged in HepMC format for downstream processing~\cite{Dobbs:2001}.

To accurately model processes featuring significant interference effects between the new physics signal (e.g., a $\zb'$ boson) and the Standard Model backgrounds, the full squared amplitude (often referred to as the Signal-Discriminated Events or SDE strategy) is employed for the phase-space integration. The \texttt{MadEvent} submodule then generates unweighted parton-level events, which are stored in the Les Houches Event (LHE) format, containing the four-momenta of all final-state particles. The generation is optimized through careful configuration of the \texttt{run\_card}, setting appropriate kinematic cuts on final-state partons to avoid wasting computational resources on events that would subsequently be rejected by the detector simulation.

Given the presence of additional jet radiation, the MLM matching scheme~\parencite{Alwall:2007fs} is applied to mitigate the double-counting of jet emission between the matrix element calculation and the subsequent parton shower. This ensures a smooth transition between the hard process and softer radiative effects.

The parton-level LHE events are then passed to \texttt{PYTHIA} (v8.2.44)~\parencite{Sjostrand:2014zea} for the modeling of QCD and QED radiation (parton showering), hadronization, and particle decays. This step translates the colored partons into stable, color-singlet hadrons and resonances that form the observable final state. The resulting events, which include a full list of generator-level particles, are saved in the HepMC2 format.

Detector effects are simulated using \texttt{DELPHES} (v3.4.2)~\parencite{deFavereau:2013fsa}, a fast parametric detector simulation framework. The \texttt{delphes\_card\_CMS.tcl} configuration card is used to emulate the response of the CMS detector, including the geometric acceptance, tracking efficiency, calorimeter energy resolution and segmentation, and magnetic field. Key reconstruction algorithms are applied within DELPHES:
\begin{itemize}
    \item Jets are clustered from calorimeter towers using the anti-$k_t$ algorithm~\parencite{Cacciari:2008gp} with a distance parameter of $R=0.4$, and $b$-tagging is simulated based on the efficiency and mis-tag rate of the CMS performance.
    \item Muons and electrons are identified with efficiency maps that are functions of $p_T$ and $\eta$.
    \item The missing transverse energy (MET) is calculated from the negative vector sum of all reconstructed particle momenta.
\end{itemize}
The final output, containing reconstructed physics objects (jets, leptons, MET), is stored in ROOT format~\parencite{Brun:1997pa}.

At this stage, the analysis of the simulated samples converges with the methodology applied to real collider data. The subsequent steps involve applying event selection criteria, calibrating and scaling the reconstructed objects (e.g., applying Jet Energy Corrections), and performing statistical interpretation. The reliability of the simulation is validated by comparing the modeling of well-known Standard Model processes (e.g., Drell-Yan, $t\bar{t}$ production) against published results and data-driven control regions. Dominant theoretical systematic uncertainties, such as those arising from the choice of factorization and renormalization scales, PDF variations, and parton shower modeling, are evaluated and propagated through the analysis.

 
\section{Measurement of the Power of an Analysis}
\label{sec:power_analysis}

In high-energy physics experiments, data is often discretized into bins (e.g., histograms of collision events versus energy or momentum) to test competing hypotheses~\cite{BakerCousins:1984}. The fundamental framework compares two scenarios: the \textit{null hypothesis} ($H_0$), representing background-only processes ($b_i$ in each bin $i$), and the \textit{alternative hypothesis} ($H_1$), including both signal and background ($s_i + b_i$)~\cite{NeymanPearson:1933}. Given the Poissonian nature of event counts $n_i$, the likelihood for observing the data under each hypothesis is the product of Poisson probabilities per bin and is therefore written as a binned Poisson likelihood~\cite{BakerCousins:1984,Cowan:2011}:
\begin{equation}
    \mathcal{L}(n_i \mid \lambda_i) = \frac{e^{-\lambda_i} \lambda_i^{n_i}}{n_i!}, \quad \text{where } \lambda_i = 
    \begin{cases}
        b_i & \text{for } H_0, \\
        s_i + b_i & \text{for } H_1.
    \end{cases}
\end{equation}
The Neyman-Pearson lemma~\parencite{NeymanPearson:1933,Segura:2024srj} provides a rigorous framework for hypothesis testing by establishing that the \textit{likelihood ratio} $Q = \mathcal{L}(\text{data} \mid H_1)/\mathcal{L}(\text{data} \mid H_0)$ is the most powerful test statistic for distinguishing between two simple hypotheses, $H_0$ and $H_1$~\cite{NeymanPearson:1933,Cowan:2011}. This forms the basis for quantifying the evidence for new physics signals against known backgrounds~\cite{Cowan:2011,Read:2002}. For binned analyses in particle physics, we define the likelihood ratio $Q_i$ for each bin $i$ as~\cite{BakerCousins:1984,Cowan:2011},
\begin{equation}
Q_i = \frac{\mathcal{L}(n_i \mid s_i + b_i)}{\mathcal{L}(n_i \mid b_i)} = e^{-s_i} \left( 1+\frac{s_i}{b_i} \right)^{n_i},
\end{equation}
where $n_i$ is the observed event count, $s_i$ the expected signal, and $b_i$ the expected background in bin $i$~\cite{BakerCousins:1984,Cowan:2011}. 

The test for the full analysis is constructed as the product of individual bin likelihood ratios~\cite{BakerCousins:1984,Cowan:2011}:
\begin{equation}
Q = \prod_{i=1}^{N} Q_i,
\end{equation}
where $N$ is the total number of bins~\cite{BakerCousins:1984}. Under this formulation, each bin is treated as an independent experiment, allowing us to analyze the data in a modular way; this is convenient when combining results from multiple search channels or energy ranges~\cite{Read:2002,Cowan:2011}. 

For convenience and to connect with asymptotic results, one commonly works with the log-likelihood ratio:
\begin{equation}
-2\ln Q = 2\sum_{i=1}^{N}\left[s_i - n_i \ln\left(1 + \frac{s_i}{b_i}\right)\right],
\end{equation}
and, by Wilks' theorem, its asymptotic distribution under the null hypothesis is chi-square distributed in regular cases~\cite{Wilks:1938,Cowan:2011}.

In practice, the Neyman-Pearson lemma motivates the use of a test statistic $t$ that quantifies the evidence for a signal against the background-only hypothesis, which can be written as
\begin{equation}
t=-2\ln Q = \sum_{i=1}^{N} \left[2s_i - 2n_i w_i\right],
\end{equation}
with the optimal weight of each bin given by $w_i = \ln\!\left(1 + \frac{s_i}{b_i}\right)$.




The discovery significance $\kappa$ quantifies the statistical separation of $t$ if $n$ is distributed according to the background-only hypothesis ($H_0$) versus the signal-plus-background hypothesis ($H_1$), normalized by the standard deviation of the $H_1$ distribution,
\begin{equation}
\kappa = \frac{\braket{t}_{H_0} - \braket{t}_{H_1}}{\sigma_{H_1}}.
\end{equation}
The expected behavior differs under the signal-plus-background ($H_1$) and background-only ($H_0$) hypotheses:

\begin{itemize}
	\item \textbf{Under $H_1$} we expect that the $n_i$ data distribution follows $\text{Pois}(s_i + b_i)$:
	\begin{equation}
	\langle -2\ln Q \rangle_{s+b} = \sum_i \left[2s_i - 2(s_i + b_i)w_i\right]
	\implies \sigma^2_{s+b} = 4\sum_i (s_i + b_i) w_i^2.
	\end{equation}

	\item \textbf{Under $H_0$} we expect that the $n_i$ data distribution follows $\text{Pois}(b_i)$
	\begin{equation}
	\langle -2\ln Q \rangle_{b} = \sum_i \left[2s_i - 2b_i w_i\right]
	\implies \sigma^2_{b} = 4\sum_i b_i w_i^2
	\end{equation}
\end{itemize}
Substituting in $\kappa$ gives a useful expression for the discovery significance,
\begin{align}
\kappa = \frac{\sum s_i w_i}{\sqrt{\sum (s_i + b_i) w_i^2}}
\end{align}
It quantifies the separation between the signal+background ($s+b$) and background-only hypotheses in units of standard deviations ($\sigma$), where $\kappa = 5$ corresponds to the traditional $5\sigma$ discovery threshold, $\kappa =3$ to a $3\sigma$ evidence to the traditional anomaly detection threshold, and $\kappa = 1.69$ to the $95\%$ confidence level (CL) exclusion limit.


This figure of merit automatically optimizes sensitivity through the logarithmic weights $w_i = \ln(1 + s_i/b_i)$, which naturally emphasize bins with either high signal-to-background ratios ($s_i/b_i$) or large absolute signal contributions ($s_i$). In asymptotic limits, $\kappa$ simplifies to intuitive forms: for dominant signals ($s_i \gg b_i$), it approaches $\sqrt{\sum s_i}$ (Poisson counting), while in background-dominated regimes ($s_i \ll b_i$), it reduces to an inverse-variance-weighted sum $\sum s_i / \sqrt{\sum b_i (s_i/b_i)^2}$. This dual behavior ensures optimal discrimination power across all signal regimes.

In practice, we must take into account systematic effects by incorporating nuisance parameters into the likelihood and profiling over uncertainty ranges. The power calculation can be extended to include systematic uncertainties by modifying the denominator as,
\begin{equation}
	\boxed{
	\kappa_{\text{sys}} = \frac{\sum_i s_i w_i}{\sqrt{\sum_i \left[(s_i + b_i) + \delta^2_{\text{sys,signal},i} + \delta^2_{\text{sys,bkg},i}\right] w_i^2}},
}
\label{eq:kappa_with_systematics}
\end{equation}
where $\delta_{\text{sys}}$ terms represent the systematic uncertainties on signal and background predictions.

This framework not only provides a figure of merit for an analysis but also serves as a roadmap for experimental optimization. The expected signal and background in each bin, $s_i$ and $b_i$, are not fundamental inputs but are themselves products of the experimental setup and analysis choices. They can be expressed in terms of more fundamental experimental parameters (with acceptance absorbed into the selection efficiencies):
\begin{align*}
    s_i &= \sigma_{s,i} \cdot \mathcal{L} \cdot \epsilon_{s,i}, \\
    b_i &= \sigma_{b,i} \cdot \mathcal{L} \cdot \epsilon_{b,i},
\end{align*}
where $\sigma_{s,i}$ and $\sigma_{b,i}$ are the fiducial cross-sections for signal and background processes in bin $i$, $\mathcal{L}$ is the integrated luminosity, and $\epsilon_{s,i}$ and $\epsilon_{b,i}$ are the effective efficiencies (selection efficiency combined with detector acceptance and reconstruction effects).

Substituting these expressions into the significance $\kappa$ reveals the multidimensional parameter space available for optimization:
\[
\kappa = \frac{\sum_i \sigma_{s,i} \cdot \epsilon_{s,i} \cdot w_i}
{\sqrt{\sum_i \left[ (\sigma_{s,i}\epsilon_{s,i} + \sigma_{b,i}\epsilon_{b,i}) + \delta^2_{\text{sys}} \right] \cdot w_i^2}} \cdot \sqrt{\mathcal{L}}.
\]

This decomposition shows that the discovery significance can be enhanced through several distinct strategies. The primary handles are:

\begin{itemize}
    \item \textbf{Increasing integrated luminosity} ($\mathcal{L}$): The $\sqrt{\mathcal{L}}$ scaling represents the fundamental statistical limit - doubling sensitivity requires quadrupling data collection time. This drives the construction of higher-luminosity colliders and longer data-taking campaigns.
    
    \item \textbf{Reducing systematic uncertainties}: The $\delta_{\text{sys}}$ terms encompass uncertainties from theoretical predictions, detector calibration, background estimation methods, and luminosity measurement. Their reduction requires dedicated calibration measurements, improved Monte Carlo simulations, and sophisticated data-driven background estimation techniques.

		\item \textbf{Improving detector performance}: Effective efficiencies $\epsilon_{s,i}$ and $\epsilon_{b,i}$ can be improved through better detector design, increased coverage, and enhanced reconstruction and calibration algorithms that recover and correctly identify more signal events while controlling backgrounds.

    \item \textbf{Choosing optimal observables}: The weights $w_i = \ln(1 + s_i/b_i)$ are maximized when the analysis uses variables that provide the best separation between signal and background. This motivates the development of advanced feature engineering and the use of multivariate methods that automatically learn the most discriminating variables.

    \item \textbf{Optimizing selection criteria}: Signal efficiency $\epsilon_{s,i}$ can be maximized while background efficiency $\epsilon_{b,i}$ is minimized through sophisticated trigger algorithms, multivariate analysis techniques, and machine learning classifiers that exploit subtle differences between signal and background event features.

\end{itemize}

Therefore, the power of an analysis, quantified by $\kappa$, is the result of a concerted effort across accelerator operation, detector performance, and analysis strategy.

The key limitation of the binned formulation in Eq.~\ref{eq:kappa_with_systematics} is its treatment of bins as independent experiments, which discards valuable information from inter-bin correlations. This approximation becomes particularly evident in regions of high sensitivity, where the shape information of distributions becomes crucial. In such cases, multivariate methods that exploit the full correlation structure (such as matrix element methods, deep learning classifiers, or template fits) typically outperform simple binned significance estimates.

However, the formalism presented here provides theoretical insight and a useful approximation for quick sensitivity estimates. In the asymptotic limit and for counting experiments, this approach yields results consistent with statistical packages commonly used in high-energy physics, such as \texttt{RooStats} and \texttt{RooFit}. These frameworks implement more rigorous statistical procedures that fully account for the likelihood structure, parameter correlations, and systematic uncertainties through nuisance parameters.


Despite this limitation, the $\kappa$ metric remains invaluable for establishing \textit{experimental sensitivity}, which is defined as the minimum signal strength required to achieve a certain significance level (e.g., 95\% CL exclusion or $5\sigma$ discovery potential). It provides a practical tool for guiding analysis design, optimizing selection criteria, and prioritizing experimental efforts. 

For experimental final results and interpretation, full statistical treatments using profile likelihood methods within frameworks like \texttt{RooStats} remain the gold standard, as they properly account for all correlations and systematic uncertainties. In this work, we are not interested in the final statistical interpretation of data, but rather in understanding and optimizing the experimental sensitivity to new physics signals. Therefore, the $\kappa$ metric serves as a practical and insightful tool for guiding analysis design and experimental strategy.
 
