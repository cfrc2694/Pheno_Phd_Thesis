\section{Discussion and conclusions}
\label{sec:discusion}

Experimental searches for $\lq$s with preferential couplings to third generation fermions are currently of great interest due to their potential to explain observed tensions in the $R_D$ and $R_{D^*}$ decay ratios of $\Bm$ mesons with respect to the SM predictions. Although the LHC has a broad physics program on searches for $\lq$s, it is very important to consider the impact of each search within wide range of different theoretical assumptions within a specific model. In addition, in order to improve the sensitivity to detect possible signs of physics beyond the SM, it is also important to strongly consider new computational techniques based on machine learning (ML). Therefore, we have studied the production of $U_1$ $\lq$s with preferential couplings to third generation fermions, considering different couplings, masses and chiral currents. These studies have been performed considering $\mathrm{p}\,\mathrm{p}$ collisions at $\sqrt{s} = 13\tev$ and $13.6\tev$ and different luminosity scenarios, including projections for the high luminosity LHC. A ML algorithm based on boosted decision trees is used to maximize the signal significance. The signal to background discrimination output of the algorithm is taken as input to perform a profile binned-likelihood test statistic to extract the expected signal significance. 

The expected signal significance for s$\lq$, d$\lq$ and non-res production, and their combination, is presented as contours on a two dimensional plane of $g_U$ versus $M_U$. We present results for the case of exclusive couplings to left-handed, mixed, and exclusive right-handed currents. For the first two, the region of the phase space that could explain the $\Bm$ meson anomalies is also presented. We confirm the findings of previous works that the largest production cross-section and best overall significance comes from the combination of d$\lq$ and non-res production channels. We also find that the sensitivity to probe the parameter space of the model is highly dependent on the chirality of the couplings. Nevertheless, the region solving the $\Bm$-meson anomalies also changes with each choice, such that in all evaluated cases we find ourselves just starting to probe this region at large $M_U$.

Our studies compare our exclusion regions with respect to the latest reported results from the ATLAS and CMS Collaborations. The comparison suggests that our ML approach has a better sensitivity than the standard cut-based analyses, especially at large values of $g_U$. In addition, our projections for the HL-LHC cover the whole region solving the B-anomalies, for masses up to $5.00\tev$.

Finally, we consider the effects of a companion $\zb^{\prime}$ boson on non-res production. We find that such a contribution can have a considerable impact on the LQ sensitivity regions, depending on the specific masses and couplings. In spite of this, we still consider non-res production as an essential channel for probing LQs in the future.