
\chapter{$U(1)_{T^3_R}$ Gauge Extension of the Standard Model}

Extensions of the Standard Model (SM) that introduce new $U(1)$ gauge symmetries are among the most widely studied scenarios for physics beyond the SM. In particular, the $U(1)_{T^3_R}$ symmetry, where right-handed SM fermions and possible new states such as right-handed neutrinos are charged, has been explored in the context of left-right symmetric models~\parencite{Assad:2017iib, MohapatraPati1975, SenjanovicMohapatra1975}. 

In these frameworks, $U(1)_{T^3_R}$ is identified as the diagonal, electrically neutral generator of $SU(2)_R$, and is often related to $U(1)_{B-L}$ through the breaking pattern $U(1)_{B-L} \times U(1)_{T^3_R} \rightarrow U(1)_Y$. This motivates the existence of a massive, electrically neutral $\textrm{Z}'$ gauge boson~\parencite{DiLuzio2018, Baker2019, Michaels:2020fzj, Dev:2021otb, Florez2023}. The Higgs doublet, being a singlet under $U(1)_{B-L}$, acquires its hypercharge from $U(1)_{T^3_R}$, so its vacuum expectation value (VEV) links the symmetry-breaking scales of $U(1)_Y$ and $U(1)_{T^3_R}$. Alternatively, these scales can be decoupled by introducing an additional $U(1)_G$ group, under which SM fermions are singlets but the Higgs is charged, so that~\parencite{Dutta:2022qvn}
\begin{equation}
Y=Q_{T^3_R}+\frac{1}{2}Q_{B-L} + Q_G.
\end{equation}
More generally, scenarios can be constructed where the hypercharge is not directly related to $U(1)_{T^3_R}$.

Recent theoretical and phenomenological work has focused on models where the low-energy gauge symmetry of the SM is extended by an Abelian $U(1)_{T^3_R}$ group whose spontaneous breaking is not tied to electroweak symmetry breaking~\parencite{Dutta2019, Dutta2020, Dutta2020b, Dutta2022, PhysRevD.107.095019, Dutta2023}. In these models, the $U(1)_{T^3_R}$ gauge boson is associated with a massive dark photon $A'$, whose longitudinal mode arises from a Higgs-like mechanism involving a complex scalar $\phi$. This field is a singlet under the SM, with its CP-odd component providing the $A'$ mass and its CP-even component giving rise to a dark Higgs, $\phi'$. To ensure anomaly cancellation, a right-handed neutrino $\nu_R$ is required for each SM generation that couples to $U(1)_{T^3_R}$. In addition, a set of new vector-like fermions $(\chi_\mathrm{u}, \chi_d, \chi_\ell, \chi_\nu)$ is introduced to generate fermion masses in a UV-complete theory, following the universal see-saw mechanism~\parencite{Berezhiani, Chang1987, Davidson1987, Rajpoot1987, Babu1989, Babu1990}.

This chapter presents a phenomenological study of LHC search strategies for a light (GeV-scale) scalar $\phi'$ produced in association with a heavy (TeV-scale) top-partner $\chi_\mathrm{u}$, via a previously unexplored production channel. We focus on the process $\mathrm{pp}\to \mathrm{t}\chi_\mathrm{u} \phi'$, which contrasts with the more commonly studied $\mathrm{pp}\to \mathrm{T}\mathrm{T}\to \mathrm{t}\phi'\mathrm{t}\phi'$ and di-photon $\phi'$ decay channels~\parencite{Bhardwaj_2022, Bhardwaj_2022_2, Bardhan_2023, Banerjee_2016, Alves_2024}. The non-trivial $\chi - \mathrm{t} -\phi'$ coupling allows for $\mathrm{t}\chi_\mathrm{u} \phi'$ final states through $\chi_\mathrm{u}$--$\mathrm{t}$ fusion (see Figure~\ref{fig:qqfusion}), and since $\chi_\mathrm{u}$ couples to SM quarks and gluons, it can be copiously produced. Its energetic decay products, together with a $\phi'$ mediator carrying significant transverse momentum, can be efficiently detected, especially if $\phi'$ decays to visible SM particles in the central detector region. This strategy is effective for reducing SM backgrounds and enhances the LHC discovery potential for heavy top partners and GeV-scale mediators, which are otherwise challenging to probe at hadron colliders. Moreover, $\mathrm{t}\chi_\mathrm{u} \phi'$ final states can also arise from $\chi_\mathrm{u}\bar\chi_\mathrm{u}$ production via QCD vertices, where one $\chi_\mathrm{u}$ decays to $\mathrm{t}\phi'$ (see Figure~\ref{fig:ggfusion}). The presence of energetic decay products and a mediator with substantial transverse momentum provides greater sensitivity than searches considering $\chi_\mathrm{u}$ or $\phi'$ alone.

We consider the case where $\phi'$ has family non-universal couplings to fermions, as proposed in~\parencite{Dutta2020}, which can address several open questions in the SM. The analysis focuses on $\phi'\to\mu^+\mu^-$ decays, as muons are efficiently reconstructed and identified, allowing for low $p_{\mathrm{T}}(\mu)$ triggers and clean signatures to suppress QCD multijet backgrounds. A central component of this study is the use of a machine learning (ML) analysis based on Boosted Decision Trees (BDT)~\parencite{friedman_greedy_2001}, whose output is used in a profile-binned likelihood test to determine the signal significance for each model. The effectiveness of BDTs and other ML algorithms has been demonstrated in numerous experimental and phenomenological studies~\parencite{Ai:2022qvs, ATLAS:2017fak, Biswas:2018snp, Chung:2020ysf, Feng:2021eke, ttZprime, Chigusa:2022svv, Florez2023, Arganda2024, Ajmal_2024, Dutta_2015}, and our results show that the BDT approach significantly improves sensitivity.

The remainder of this chapter is organized as follows. Section~\ref{sec:model} describes the minimal $U(1)_{T^3_R}$ model. Section~\ref{sec:exp} reviews current relevant LHC results. Section~\ref{sec:sims} details the Monte Carlo (MC) simulation samples used in this study. Section~\ref{sec:ML} discusses the motivation and implementation of the machine learning workflow, and Section~\ref{sec:results} presents the main results.

