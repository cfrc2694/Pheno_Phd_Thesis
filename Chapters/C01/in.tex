\chapter{Standard Model of Particle Physics}\label{ch:sm}

The Standard Model (SM) of particle physics is a quantum field theory (QFT) that describes matter as fermionic particles and their fundamental interactions. The forces are incorporated through the gauge principle, where force-carrying particles---vector bosons with spin one, arising from the adjoint representation of symmetry groups (\textit{gauge groups})---mediate the interactions between matter particles~\parencite{greiner2000relativistic,pokorski2000gauge}. However, this elegant formulation is not sufficient to account for particle masses. These are generated through Yukawa interactions, which are scalar-fermion couplings between the Higgs field and the fermion fields. While the Yukawa interactions themselves are not gauge interactions, their allowed structure---specifically, which fermions they can couple and their transformation properties---is strictly dictated by the gauge symmetry of the theory. This combined framework of gauge and Yukawa sectors successfully describes three of the four fundamental forces in nature.

In this chapter, we contextualize the SM by introducing the basic concepts of quantum field theory, including the notion of fields and symmetries. We then present the particle content of the SM, its gauge group, and the Lagrangian density that describes its dynamics. The Higgs mechanism and its role in providing mass to the weak gauge bosons and fermions are also discussed. Finally, we address the main deficiencies of the SM and review the experimental evidence that motivates the search for physics beyond the SM.

\section{Fields}
Relativistic quantum fields are degrees of freedom in QFT. Formally, they are \textit{operator-valued functions on spacetime that transform under a representation of the Lorentz group on an invariant subspace}~\parencite{Tong1995}. The different representations of the Lorentz group are mainly characterized by their spin, and their fields obey a different equation of motion (see table~\ref{tab-repLorentz2}). 

In classical field theory, a variational principle is established which generates the equations that govern the dynamics of the different fields in a theory, \textit{the equations of motion}. Hamilton's principle, or principle of minimal action, indicates that all possible physical configurations for a set of fields $\varphi^I$, with $I=1,2,3,\cdots,n$, are those for which the action $S$ is  minimal~\parencite{Goldstein,jose1998classical}:
\begin{equation}\label{eq-action}
	S=\int \mathcal{L}(\varphi^I,\partial_\mu\varphi^I) d^4x.
\end{equation}
Here, $d^4x=dx^0dx^1 dx^2dx^3$ and $x\equiv(ct,x^1,x^2,x^3)\equiv(x^0,x^1,x^2,x^3)\in\mathcal{M}^4$, are the space-time coordinates in the Minkowskian spacetime ($\mathcal M^4$), and the function $\lag(\varphi^I,\partial_\mu\varphi^I)$ is called \textit{the Lagrangian density} of a theory~\parencite{greiner2000relativistic,Goldstein}. The problem in classical field dynamics is to find the functions $\varphi^I(x)$ in a space-time $\mathcal{M}^4$, fixing their boundary conditions. The solution to this classical problem is given by the Euler-Lagrange equations:
\begin{equation}\label{eq_EulerLag}
	\dpr{\mathcal{L}}{\varphi^I}-\dpr{}{x^\mu}\dpr{\lag}{\fac{\partial_\mu \varphi^I}}=0,
\end{equation}
and they are used to obtain the equations of motion of the set of fields $\varphi^I$~\parencite{jose1998classical}. 

While in classical field theory the Euler–Lagrange equations directly determines the dynamics of the system, in QFT the approach changes: if we adopt the path-integral formulation~\parencite{martinez2002,Weinberg}, the idea of an equation of motion vanishes and we move on to searching for correlations between free particle states. However, the notion of action remains the cornerstone in the description of these observables.

Explicitly, the correlation functions are calculated through the Lehmann-Symanzik-Zimmermann (LSZ) reduction formula, which connects these correlators with physical scattering amplitudes. These are computed from the path integral~\parencite{greiner1996qft,peskin}:
\begin{equation}
	\begin{aligned}
		Z[J]&=\braket{\text { out, } 0| 0, \text { in }}
		\\&=\mathcal{N}\int \mathcal{D}(\varphi, \bar{\varphi})  e^{i S[\varphi]} e^{i \int J_I\varphi^I  d^{4} x}
		\\&=\mathcal{N}\int \mathcal{D}(\varphi, \bar{\varphi})  e^{i \int d^{4} x \mathcal{L}} e^{i \int J_I\varphi^I  d^{4} x},
	\end{aligned}
\end{equation}
taken over the space of fields $\varphi$ with an appropriate measure $\mathcal{D}(\varphi, \bar{\varphi})$ and normalized by $\mathcal{N}$. The quantity $Z$ is known as the partition function of the theory and gives the transition amplitude from the initial vacuum $\ket{0,\text{ in}}$ to the final vacuum $\ket{0,\text{ out}}$ in the presence of a source $J(x)$ producing particles~\parencite{birrell75900}.


\begin{center}
    \begin{tabular}{|l|c|c|l|}\hline\bigstrut
        Name							& Field				& Spin & Free-Lagrangian	\\\hline\hline\bigstrut
        Klein-Gordon				&	$\phi$					& $0$			&	$\lag=\fac{\partial^\mu\bar \phi\partial_\mu \phi-m^2 \bar \phi\phi}$						\\\hline\bigstrut
        Dirac								& $\chi$			& $1/2$	&$\lag=\bar\chi\fac{i\pmb\gamma^\mu \partial_\mu -m\pmb 1}\chi$\\\hline\bigstrut
        Proca (Massive Vector)	        & $A^\mu$ 		& $1$		&$\lag=-\frac{1}{4} F^{\mu\nu} F_{\mu\nu} + \frac{1}{2}m^2 A^\mu A_\mu $\\\hline
    \end{tabular}
	\captionof{table}{Some relevant representations of the Lorentz group in  $4$-dimensional space-time. In this notation $\eta_{\mu\nu}=\diag(1,-1,-1,-1)$, $\pmb \gamma^\mu$ are the Dirac matrices, $F_{\mu \nu}=\partial_{\mu} A_{\nu}-\partial_{\nu} A_{\mu}$ is the abelian field strength tensor. All equations are written in natural units with $c=\hbar=1$. Fields are shown in their standard representations.}\label{tab-repLorentz2}
\end{center}

Therefore, the dynamics, at both the classical and quantum levels, are entirely determined by the Lagrangian density. For free fields (i.e., non-interacting), the Lagrangian is quadratic in the fields and the path integral can be evaluated exactly. Tab.~\ref{tab-repLorentz2} records the Lagrangian density for these free fields. However, to describe physics, we must include interactions, which render the path integral impossible to compute exactly.

The framework of \textit{perturbation theory} addresses this by expanding the interaction part of the Lagrangian as a power series. This expansion is organized using \textit{Feynman diagrams}, which provides a pictorial representation of each term, and a set of \textit{Feynman rules}, which provides a precise dictionary to translate these diagrams into mathematical expressions for scattering amplitudes~\parencite{peskin,Weinberg}. The importance of these rules cannot be overstated, as they are the practical computational tools of perturbative QFT.


In this paradigm, our task is to propose a Lagrangian density for a set of fields that correctly models the propagation and interactions of fundamental particles. The free part defines the particle content and propagators, while the interaction part defines the vertices and possible scattering processes.

\subsection{Interactions and Symmetries} 
The form of the Lagrangian density is not arbitrary: it is shaped by a small set of physical and mathematical principles. These principles act as ``rules'' that guide the construction of consistent theories, ensuring both their internal consistency and their predictive power. In particular, if we want a relativistic and renormalizable theory, the Lagrangian must satisfy several conditions that strongly restrict the kind of terms that can appear.

\textcolor{red}{The need for these restrictions is evident from the path integral formulation itself. If we split the Lagrangian into a free part and an interaction part in the form
\begin{equation}
    \mathcal{L} = \mathcal{L}_0 + \mathcal{L}_{\text{int}}.
\end{equation}
......AF: Encuentro esta parte desconectda y la idea no se entiende.}
The generating functional $Z[J]$ can then be expressed as an operator acting on the free functional $Z_0[J]$:
\begin{equation}
    Z[J] = \mathcal{N} \exp\left[i \int d^4x\, \mathcal{L}_{\text{int}}\left(-i \frac{\delta}{\delta J(x)}\right)\right] Z_0[J].
\end{equation}
The exponential operator generates an infinite perturbation series. \textcolor{red}{The $n$-point correlation function is found by taking functional derivatives of $Z[J]$ with respect to the sources $J(x_i)$ and setting $J=0$. Each term in this series is represented by a \textbf{Feynman diagram}:.....AF: Hay que incluir un diagrama de Feynman que ilustre lo que se dice en los bullets abajo y resaltar cada componente}

\begin{itemize}
    \item \textbf{External Lines:} Represent incoming and outgoing physical particles.
    \item \textbf{Internal Lines:} Represent virtual particles propagating between interactions, corresponding to the free-field propagators derived from $\mathcal{L}_0$.
    \item \textbf{Vertices:} Represent interactions, derived from the terms in $\mathcal{L}_{\text{int}}$. Each vertex has an associated coupling constant and enforces momentum conservation.
\end{itemize}

For this series to be a predictive and well-defined computational tool, the individual terms must yield finite results. This requirement of \textit{renormalizability} is a powerful constraint on $\mathcal{L}_{\text{int}}$. Furthermore, the structure of both $\mathcal{L}_0$ and $\mathcal{L}_{\text{int}}$ is profoundly constrained by the requirement that the theory possesses certain \textit{symmetries}.

To begin with, relativistic invariance demands that the equations of motion remain the same in all inertial frames. This requirement is implemented by asking the action to be invariant under Poincaré transformations~\parencite{pall}. Equivalently, the Lagrangian density must transform as a Lorentz scalar and may change under translations at most by a total derivative~\parencite{jose1998classical}. 

Another basic condition is Hermiticity: the Lagrangian density must be Hermitian such that observables are real and the time evolution of the theory is unitary~\parencite{pall,peskin}. In addition, dimensional analysis places further restrictions. In natural units, $\mathcal{L}$ carries dimensions of mass to the fourth power ($[\mathcal{L}] = [mass]^4$), which corresponds with an  energy density. This means that the interaction terms that we can add must be such that the overall operator has the correct dimension, which already rules out many possibilities. 

In quantum field theory, loop corrections to scattering amplitudes typically produce divergences. A theory is called renormalizable if all these divergences can be absorbed into a redefinition of a \emph{finite set} of physical parameters (such as masses and couplings). In practice, this requirement translates into a restriction on the operators that may appear in the Lagrangian: only terms of mass dimension $\leq 4$ lead to renormalizable interactions. Higher-dimensional operators are still allowed, but they correspond to \emph{effective} interactions that are suppressed at low energies and signal the presence of new physics at higher scales~\parencite{peskin,Weinberg}. 

A classical symmetry of the Lagrangian may not always survive the process of quantization. If it fails to do so, it is said to be anomalous. Chiral anomalies, specifically, arise from the regularization of fermion loops in triangle diagrams and can break the gauge symmetry at the quantum level. Since the gauge symmetry is the very principle that dictates the form of interactions and removes unphysical states, its violation would destroy the renormalizability and unitarity of the theory. Therefore, the particle content must be carefully chosen so that these potential anomalies cancel among fermions, a non-trivial condition famously satisfied by the quarks and leptons of the SM.

Furthermore, the stability of the vacuum is a prerequisite for a physically meaningful theory. This is ensured by demanding that the scalar potential, which governs the self-interactions of the Higgs field, is bounded from below. If the potential were unbounded, it would imply that the system could lower its energy indefinitely by rolling down the potential to field values of ever-greater magnitude, meaning no stable ground state could exist. For a renormalizable potential, this stability condition typically translates into the requirement that the quartic coupling constant $\lambda > 0$. However, this condition must hold not just at tree-level but also at the quantum level, as running couplings can change sign at different energy scales, potentially leading to metastability or instability of the vacuum.

Summarizing, the main constraints that a relativistic and renormalizable Lagrangian density must satisfy are:
\begin{itemize}
	\item \textbf{Poincaré (global) invariance:} the action must be invariant under Lorentz transformations and translations; the Lagrangian density is a Lorentz scalar and may change by at most a total derivative~\parencite{pall,jose1998classical}. \marginpar{\footnotesize In QFT, Poincaré invariance is assumed to be global. Promoting it to a local symmetry leads to gravity, with spin-2 fields (the graviton) as mediators. Perturbatively, such a theory is not renormalizable, so it lacks predictivity at high energies, although it can still be understood as an effective field theory.}
	\item \textbf{Hermiticity:} $\mathcal{L}$ must be Hermitian to ensure real observables and unitary evolution~\parencite{pall,peskin}.
	\item \textbf{Renormalizability and Operator Dimension:} the theory must be perturbatively renormalizable, meaning all ultraviolet divergences can be absorbed into a finite number of parameters. This requirement, determined via power-counting arguments, restricts interaction operators to have a \textbf{mass dimension $\leq 4$}. In natural units, where $\mathcal{L}$ has dimension [mass]$^4$, this allows only Yukawa couplings (dim 4), scalar $\phi^4$ interactions (dim 4), and gauge interactions (dim 4), while forbidding non-renormalizable operators like $\phi^6$ (dim 6)~\parencite{peskin,Weinberg}.
	\item \textbf{Absence of chiral anomalies:} gauge symmetries must be free of chiral (gauge) anomalies to ensure the consistency and unitarity of the quantum theory~\parencite{peskin,Weinberg,bertlmann1996anomalies}. In the Standard Model, the particle content is such that all gauge anomalies cancel exactly.
	\item \textbf{Stability of the potential:} the scalar potential must be bounded from below to guarantee the existence of a stable vacuum state. This typically requires that the quartic couplings in the potential are positive at the relevant energy scales.
\end{itemize}

These constraints drastically reduce the number of possible terms in the Lagrangian. As a result, the renormalizable interaction structures that typically arise are limited to: Yukawa couplings between fermions and scalars, scalar self-interactions (up to quartic order), and gauge interactions between matter fields and vector bosons.

Despite these powerful constraints, a vast number of possible interaction terms between the allowed fields remain. To further restrict the form of the Lagrangian and to describe fundamental forces, the concept of \textit{symmetry}—specifically \textit{gauge symmetry}—has proven to be our most powerful guiding principle.

The procedure is systematic: first, the spin$-0$ and spin$-1/2$ fields are organized into representations of a unitary (gauge) group $G$, such that the Lagrangian density is globally invariant under $G$. This global symmetry is then ``promoted'' to a \textit{local symmetry} (where the group parameters can vary in spacetime) by replacing the ordinary derivatives $\partial_\mu$ with \textit{covariant derivatives} $\Dcov_\mu$ that incorporate new \textit{gauge fields} $B_\mu^A$~\parencite{pokorski2000gauge,freedman2012supergravity, Gallego2016,VanProeyen1999,Martin2012}.
This ``promotion'' is described in more detail below.

Given a Lagrangian density $\lag(\varphi^I, \partial_\mu \varphi^I)$, where $I$ is an index enumerating the different fields $\varphi^{I}$ in the model, it is said to be \textit{globally symmetric} under unitary transformations if the action remains invariant under field variations. At infinitesimal level, these variations are given by:
\begin{equation}
	\delta_G \varphi^I = i\theta^A (T_A)^I_J \varphi^J,
\end{equation}
where $\theta^{A}$ are constant parameters of the transformation and the $T_{A}$ are the generators of the group $G$ in the appropriate representation. The corresponding finite unitary transformation is
\begin{equation}
	\mathcal{U}_G \equiv U(\theta)=\exp(i\theta^A T_A).
\end{equation}
Note that the $T_A$  generators  satisfy the same Lie algebra of the group $G$:
\begin{equation}
	[T_A, T_B] = i f_{AB}^{\;\;C}T_C,
\end{equation}
where $f_{AB}^{\;\;C}$ are the structure constants of $G$.

To promote the global symmetry to a local one ($\theta^A \to \theta^A(x)$), the ordinary derivative $\partial_\mu$ is replaced by a \textit{covariant derivative} $\Dcov_\mu$. This new derivative is designed to transform covariantly under the gauge group, meaning $\Dcov_\mu \varphi \to U(x) (\Dcov_\mu \varphi)$, so that the kinetic terms $\lag_{\text{kin}} \sim (\Dcov_\mu \varphi)^\dagger (\Dcov^\mu \varphi)$ remain invariant. This is achieved by introducing a gauge field $B_\mu^A$ for each generator $T_A$ and defining:
\begin{equation}
	\Dcov_\mu = \partial_\mu - i g B_\mu^A T_A,
\end{equation}
where $g$ is the gauge coupling constant. The transformation law for the gauge fields that ensures the covariant transformation of $\Dcov_\mu$ is:
\begin{equation}
	\delta B_\mu^A = \partial_\mu \theta^A + g f_{BC}{}^A \theta^B B_\mu^C.\label{eq:gauge-transformation}
\end{equation}

The introduction of the gauge fields $B_\mu^A$ requires the addition of a kinetic term for them to the Lagrangian. This is constructed from the \textit{field strength tensor} $F_{\mu\nu}^A$, defined as the curvature of the covariant derivative:
\begin{equation}
	F_{\mu\nu}^A T_A = -\frac{i}{g} [\Dcov_\mu, \Dcov_\nu] = \partial_\mu B^A_\nu - \partial_\nu B^A_\mu + g f_{BC}{}^A B^B_\mu B^C_\nu.
\end{equation}
The gauge-invariant kinetic Lagrangian is then:
\begin{equation}
	\lag_{\text{gauge}} = -\frac{1}{4} \delta_{AB} F^A_{\mu\nu} F^{\mu\nu B}.
\end{equation}
Often, the rescaling $B_\mu^A \to g B_\mu^A$ is performed, which moves the coupling constant $g$ from the kinetic term to the covariant derivative, resulting in the more conventional form $\Dcov_\mu = \partial_\mu - i g B_\mu^A T_A$ and $\lag_{\text{gauge}} = -\frac{1}{4g^2} \delta_{AB} F^A_{\mu\nu} F^{\mu\nu B}$.


A general, archetypal Lagrangian, embodying these structures, can be written as:
\begin{equation}\label{eq:generic-renorm-lag}
	\mathcal{L} = -\frac{1}{4} F_{\mu \nu}^A F^{A \mu \nu} + i \bar{\psi}^i \gamma^\mu \mathcal{D}_\mu \psi^i + \left(\bar{\psi}_L^j \, \Gamma^j_k \, \Phi \, \psi_R^k + \text{h.c.}\right) + |\mathcal{D}_\mu \Phi|^2 - V(\Phi)
\end{equation}
The terms correspond to: the kinetic term for gauge fields ($F_{\mu \nu}^A$), the kinetic term for fermions $\psi^i$, the Yukawa interactions between left- and right-handed fermions and scalars ($\Gamma^j_k$ is a Yukawa coupling matrix and $\Phi$ is a scalar field), the kinetic term for scalars, and the scalar potential $V(\Phi)$. For a renormalizable and  stable theory $V(\Phi) = \mu^2 |\Phi|^2 + \lambda |\Phi|^4$ with $\lambda > 0$.

Note the absence of explicit mass terms for the gauge fields ($\sim M^2 B_\mu B^\mu$) and fermions ($\sim m \bar{\psi}\psi$). These are forbidden by gauge invariance and for chiral fermions. Mass terms can be generated via spontaneous symmetry breaking, as discussed below.

It is important to emphasize that while the Yukawa interactions $\bar{\psi}_L^j \, \Gamma^j_k \, \Phi \, \psi_R^k$ do not involve gauge bosons directly, their structure is nonetheless \textit{completely determined by the gauge symmetry}. Specifically, gauge invariance dictates which fermion fields can couple to which scalar fields, and constrains the form of the coupling matrix $\Gamma^j_k$. For a Yukawa term to be gauge-invariant, the product $\bar{\psi}_L^j \, \Phi \, \psi_R^k$ must be a singlet under the gauge group. This requirement arises because the left-handed and right-handed fermions typically transform in different representations of the gauge group, and the scalar field $\Phi$ must carry the appropriate quantum numbers to make the overall combination invariant. In the Standard Model, for instance, the left-handed fermions are $SU(2)_L$ doublets while the right-handed fermions are singlets, and the Higgs doublet provides the necessary quantum numbers to form gauge-invariant Yukawa couplings. Thus, even though Yukawa interactions are scalar-mediated rather than gauge-mediated, the gauge principle is the fundamental organizing principle that determines their allowed structure.

\subsubsection{Example}
To illustrate these concepts, let us consider a renormalizable theory with a real scalar $\phi$ and a Dirac spinor $\psi$, and suppose that this theory is globally invariant under $U(1)$ phase transformations, i.e. the fields $\varphi\in\{\phi,\psi\}$ transform as $\varphi\mapsto e^{i\theta \hat Q}\varphi $ such that $\hat Q \psi = q \psi$ and $\hat Q \phi=0$. The free Lagrangian density is
\begin{equation}
	\mathcal L_{\text{free}}=\frac{1}{2} \partial^{\mu} \phi \partial_{\mu} \phi-\frac{1}{2}\mu^2\phi^2+\bar{\psi}(i \gamma_\mu  \partial^\mu-m) \psi.
\end{equation}
\marginpar{\footnotesize Note that for this vector-like $U(1)$ theory, the explicit fermion mass term $m\bar{\psi}\psi$ is gauge-invariant. This will not be the case for chiral gauge theories like the Standard Model.}

To add globally symmetric interaction terms, we must consider operators of mass dimension $\leq 4$. The most general renormalizable Lagrangian, invariant under the global $U(1)$ symmetry, is
\begin{equation}
	\begin{aligned}
		\mathcal L_{\text{global}}&=\frac{1}{2} \partial^{\mu} \phi \partial_{\mu} \phi-V(\phi)+\bar{\psi}(i \gamma_\mu  \partial^\mu-m) \psi + k_1 \phi\bar\psi\psi,
		\\
		V(\phi)&=\frac{1}{2}\mu^2\phi^2 +\frac{\alpha}{3!}\phi^3+\frac{\lambda}{4!}\phi^4.
	\end{aligned}
\end{equation}
The cubic and quartic terms in $V(\phi)$ are allowed as $\phi$ is neutral. The Yukawa coupling $k_1 \phi\bar\psi\psi$ is also gauge-invariant since the charges of $\bar\psi$, $\phi$, and $\psi$ sum to zero ($-q + 0 + q = 0$).

Promoting the global symmetry to a local one ($\theta \to \theta(x)$) requires introducing a gauge field $A_\mu$ and replacing ordinary derivatives with covariant derivatives:
\begin{equation}
	\mathcal D_\mu\varphi=(\partial_{\mu}-i g A_\mu\hat Q )\varphi
	\quad\Longrightarrow\quad
	\begin{cases}
		\mathcal D_\mu\phi=\partial_\mu \phi, & (\text{since } \hat Q\phi=0)\\
		\mathcal D_\mu\psi=(\partial_\mu - i g q A_\mu) \psi.
	\end{cases}
\end{equation}
The field strength tensor for the abelian $U(1)$ field is defined as $F_{\mu\nu} = \partial_\mu A_\nu - \partial_\nu A_\mu$. The locally invariant Lagrangian is then:
\begin{multline}
	\mathcal L_{\text{local}}=\frac{1}{2} \mathcal D^{\mu} \phi \mathcal D_{\mu} \phi-V(\phi)
	+\bar{\psi}i \gamma_\mu  \mathcal D^{\mu} \psi - m \bar{\psi}\psi
	+ k_1 \phi\bar\psi\psi-\frac{1}{4} F_{\mu\nu}F^{\mu\nu}.
\end{multline}


With these ingredients and principles, we are now equipped to understand the structure of the SM Lagrangian, which will be discussed in the next section.

\begin{figure}[h!]
    \centering
    \begin{subfigure}[b]{0.48\textwidth}
        \centering
        \begin{fmffile}{feyngraphs/feyngraph1} 
			\vspace{0.5cm}
            \begin{fmfgraph*}(80,60)
                \fmfleft{i1}
                \fmfright{o1,o2}
                
                \fmf{dashes,tension=2.0}{i1,v1}
                \fmf{fermion}{o1,v1}
                \fmf{fermion}{v1,o2}

                \fmflabel{$\phi$}{i1}
                \fmflabel{$\bar\psi$}{o1}
                \fmflabel{$\psi$}{o2}
            \end{fmfgraph*}
			\vspace{0.5cm}
        \end{fmffile}
        \caption{Yukawa coupling with a scalar $\phi$.}
        \label{fig-yukawa-scalar}
    \end{subfigure}
    \hfill
    \begin{subfigure}[b]{0.48\textwidth}
        \centering
        \begin{fmffile}{feyngraphs/feyngraph2}
			\vspace{0.5cm}
            \begin{fmfgraph*}(80,60)
                \fmfleft{i1}
                \fmfright{o1,o2}
                
                \fmf{photon,tension=2.0}{i1,v1}
                \fmf{fermion}{o1,v1}
                \fmf{fermion}{v1,o2}

                \fmflabel{$\gamma$}{i1}
                \fmflabel{$\bar\psi$}{o1}
                \fmflabel{$\psi$}{o2}
            \end{fmfgraph*}
			\vspace{0.5cm}
        \end{fmffile}
        \caption{Interaction with a photon $\gamma$.}
        \label{fig-qed-photon}
    \end{subfigure}
	\begin{subfigure}[b]{0.48\textwidth}
        \centering
		\begin{fmffile}{feyngraphs/feyngraph3}
			\vspace{1.0cm}
			\begin{fmfgraph*}(80,60)
				\fmfleft{i1}
				\fmfright{o1,o2}

				\fmf{dashes}{i1,v1}
				\fmf{dashes}{v1,o1}
				\fmf{dashes}{v1,o2}

				\fmflabel{$\phi$}{i1}
				\fmflabel{$\phi$}{o1}
				\fmflabel{$\phi$}{o2}
			\end{fmfgraph*}
			\vspace{0.5cm}
		\end{fmffile}
		\caption{Triple scalar coupling.}
		\label{fig-triple-scalar}
	\end{subfigure}
	\begin{subfigure}[b]{0.48\textwidth}
        \centering
		\begin{fmffile}{feyngraphs/feyngraph4}
			\vspace{1.0cm}
			\begin{fmfgraph*}(80,60)
				\fmfleft{i1,i2}
				\fmfright{o1,o2}

				\fmf{dashes}{i1,v1}
				\fmf{dashes}{i2,v1}
				\fmf{dashes}{v1,o1}
				\fmf{dashes}{v1,o2}

				\fmflabel{$\phi$}{i1}
				\fmflabel{$\phi$}{i2}
				\fmflabel{$\phi$}{o1}
				\fmflabel{$\phi$}{o2}
			\end{fmfgraph*}
			\vspace{0.5cm}
		\end{fmffile}
		\caption{Quartic scalar coupling.}
		\label{fig-quartic-scalar}
	\end{subfigure}
    \caption{Feynman diagrams for Yukawa coupling, gauge boson coupling and quartic scalar coupling.}
\end{figure}
 % Fields
\section{Standard Model}

{$ $ \scriptsize \hfill Fragment extracted and adapted from~\parencite{robinson2011symmetry}}

In 1965, Tomonaga, Feynman, and Schwinger were awarded the Nobel Prize for their independent formulation of Quantum Electrodynamics (QED)~\parencite{1972physics}. Their work established renormalization as a consistent method to separate infinities from finite, physically meaningful results in quantum field theory. QED provided predictions, such as the anomalous magnetic moment of the electron, that later experiments confirmed with remarkable precision~\parencite{1674-1137-40-10-100001, PhysRev.75.486}. It became the prototypical example of a successful quantum field theory.

This success, however, did not extend to other fundamental interactions. The weak interaction was described by the chiral $V-A$ model, in which processes like beta decay were represented by four-fermion contact terms. This framework was not renormalizable: divergences could not be absorbed into a finite set of parameters, restricting its validity to low energies. A fundamental description within the quantum field theory framework was still missing.

The issue was linked to the short-range character of the weak and strong forces. In quantum field theory, the range of an interaction depends on the mass of its mediating boson. A massless boson, such as the photon, generates a long-range force with an inverse-square dependence. A massive boson, in contrast, produces a Yukawa potential of the form $\exp(-mr)/r$, which falls off rapidly with distance. A consistent theory of the weak interaction therefore required massive gauge bosons.

Here lay the apparent obstacle. A mass term for a gauge boson, such as $m_{A}^{2} A_{\mu} A^{\mu}$ in the Lagrangian, explicitly breaks gauge invariance, since it is not preserved under the transformation $A_{\mu} \mapsto A_{\mu} + \partial_{\mu}\epsilon$. This seemed to rule out gauge theories as candidates for describing short-range forces. The problem was recognized early on. For instance, in a 1954 seminar where Chen Ning Yang introduced non-Abelian gauge theories, Wolfgang Pauli objected that assigning masses to the gauge bosons would violate gauge invariance, and without such masses the theory could not describe nuclear forces. This skepticism reflected a widely shared view: gauge symmetry appeared incompatible with short-range interactions.

The resolution of this problem came from two developments that allowed gauge bosons to behave as if they had mass, without explicitly breaking gauge symmetry:
\begin{enumerate}
    \item \label{list:sol_mass_1} The Higgs mechanism. In this framework, a scalar field permeates the vacuum. While the underlying Lagrangian remains gauge invariant, the vacuum state does not respect this symmetry. Gauge bosons interacting with this vacuum acquire mass in a renormalizable way. This mechanism explains the masses of the $W$ and $Z$ bosons.
    \item \label{list:sol_mass_2} Dynamical mass generation in non-Abelian gauge theories. In Quantum Chromodynamics (QCD), gluons and nearly massless quarks are confined into hadrons with substantial masses. The appearance of a mass gap is a nonperturbative consequence of confinement. Understanding this mechanism in a rigorous way is at the core of the Yang–Mills existence and mass gap Millennium Prize problem.
\end{enumerate}

The Standard Model (SM) incorporates both solutions. Electroweak theory relies on the Higgs mechanism (\ref{list:sol_mass_1}), which provides a renormalizable description of the weak interaction. For the strong interaction, QCD employs dynamical mass generation (\ref{list:sol_mass_2}), where most of the mass of hadrons arises from confinement rather than from the small quark masses introduced by the Higgs field.

\subsection{Particle Content and Gauge Group}

First, let us talk about the chiral nature of particles: Massive half-spin particles are described at the fundamental level by a Dirac spinorial field, see table \ref{tab-repLorentz2}. However, Dirac spinors do not transform under an irreducible representation of the Lorentz group. Spinors can be decomposed into two components that do transform under irreducible representations of the Lorentz group: two \textit{Weyl spinors}. The left and right chiral projectors, $P_L$ and $P_R$, take a Dirac spinor and project it onto each of these invariant subspaces. For a massless Dirac spinor, the left and right components are dynamically decoupled, \textit{i.e.} which are independent fields obeying independent Lagrangian densities; for example, the left component of a massless spinor has the Lagrangian $\lag=-i\bar\psi\slashed{\partial}P_L\psi$ (For more details see Appendix A at~\parencite{CRodriguezUPTC}). 

The discovery of parity asymmetry in radioactive decays~\parencite{PhysRev.105.1413} indicates that the chiral description of weak interactions couples differently to the left and right chiral components of half-spin particles. Indeed, the chirality of the fermionic spectrum is possibly one of the deepest properties of the Standard Model. Describing particles in terms of Dirac spinors, it means that left- and right-chirality components actually have different EW quantum numbers. This is compatible with a gauge symmetry only if half-spin particles are considered to be massless, at least without a Dirac mass $m \overline{f_{R}} f_{L}+\text { h.c.}$ Nevertheless, half-integer spin fundamental particles, such as the electron, have a well-measured mass. Therefore, the reconciliation of chiral asymmetry and mass lies in the Higgs mechanism, where the masses of the particles result from an effective Yukawa coupling with a scalar, the Higgs boson.

With this in mind, the SM has a content of matter fields from three generations (or families) of quarks $q$ and leptons $\ell$, described as Weyl 2-component spinors, with the structure
\begin{equation}
	q_{L}=\left(
		\begin{array}{c}
			u_{L}^{i} \\
			d_{L}^{i}
		\end{array}
	\right), 
	u_{R}^{i}, d_{R}^{i}, 
	\quad \ell_L=\left(
		\begin{array}{c}
			\nu_{L}^{i} \\
			e_{L}^{i}
		\end{array}
	\right), e_{R}^{i} ; \quad i=1,2,3 .
\end{equation}
All these particles transform under a group $U$(1) with different associated (hyper)charges.
The doublets formed by the left components of the fields transform under the representation of two components of a $SU$(2) group. The right components do not transform under SU(2), therefore they are singlets.
In addition, each quark in $q_{L}$ transforms as color triplets under $SU$(3), while $u_{R}, d_{R}$ transforms as conjugate triplets. Leptons, on the other hand, turn out to be colored singlets.
Gauge quantum numbers of the Standard Model fermions are shown in table \ref{tab_qm}.

\begin{center}
	$$
	\begin{array}{|l||c|c|c||c|}
		\hline \text {\textbf{Field} } & S U(3)_C & S U(2)_{L} & U(1)_{Y} & U(1)_{EM} \bigstrut\\
		\hline q_{L}^{i}=\left(u^{i}, d^{i}\right)_{L} & \mathbf{3} & \mathbf{2} & +1 / 3 & (2/3,-1/3) \bigstrut\\
		u_{R}^{i} & \overline{\mathbf{3}} & \mathbf{1} & +4 / 3 & +2/3 \bigstrut\\
		d_{R}^{i} & \overline{\mathbf{3}} & \mathbf{1} & -2 / 3 & -1/3 \bigstrut\\
		\ell^{i}_L=\left(\nu^{i}, e^{i}\right)_{L} & \mathbf{1} & \mathbf{2} & -1  & (0,-1)\bigstrut\\
		e_{R}^{i} & \mathbf{1} & \mathbf{1} & -2 & -1 \bigstrut\\
		H=\left(H^{+}, H^{0}\right) & \mathbf{1} & \mathbf{2} & +1 & (+1,0) \bigstrut\\
		\hline \hline
	\end{array}
	$$
	\captionof{table}{Gauge quantum numbers of Standard Model quarks, leptons
		and the Higgs scalar.}\label{tab_qm}
\end{center}

Then, we consider the Standard Model as a quantum field theory based on a gauge group
\begin{equation}
	G_{\mathrm{SM}}=S U(3)_C \times S U(2)_{L} \times U(1)_{Y},
\end{equation}
with $S U(3)_C$ describing strong interactions via Quantum Chromodynamics (QCD), and $S U(2)_{L} \times U(1)_{Y}$ describing electroweak (EW) interactions. Gauge vector bosons that result from taking this group locally are eight gluons ($G^a$) from each $t^a$ color-generator of $SU(3)_C$, and a linear combination of the three ($W^\pm, Z$) weak bosons and the ($\gamma$) electromagnetic photon from the three $T^i$ isospin-generators of $SU(2)_L$ and $Y$ hyper-charge-generator of $U(1)_Y$.

Electroweak symmetry is spontaneously broken into electromagnetic symmetry $U(1)_{EM}$ via the Higgs mechanism and the Higgs boson $H$. The hypercharges $Y$ of the Standard Model fermions in table \ref{tab_qm} are related to their usual electric charges by the Gell-Mann–Nishijima relation~\parencite{10.1143/PTP.10.581} 
\begin{equation}
	Q_{\mathrm{EM}}=\frac12Y+T_{3}, \label{eq:Gell-Mann-Nishijima}
\end{equation}
where $T_{3}\dot=\operatorname{diag}\left(\frac{1}{2},-\frac{1}{2}\right)$ is an $S U(2)_{L}$ generator.  Thus, they reproduce electric charge quantization, e.g. the equality in magnitude of the proton and electron charges. Although these hypercharge assignments look rather ad hoc, their values are dictated by the quantum consistency of the theory.\marginpar{It is indeed easy to check that these are (module an irrelevant overall normalization) the only (family independent) assignments canceling all potential triangle gauge anomalies.}

\subsection{Gauge Bosons}

The Lie algebra of the gauge group $SU(3)\times SU(2)\times U(1)$ is
\begin{equation}
\begin{aligned}
	{\left[t^{a}, t^{b}\right] } &=i f^{a b c} t_{c}, \\
	{\left[T^{i}, T^{j}\right] } &=i \epsilon^{i j k} T_{k}, \\
	{\left[T^{i}, \, Y\;\right] } &=\left[t^{a}, T^{j}\right]=\left[t^{a}, Y\right]=0,
\end{aligned}
\end{equation}
where $f^{a b c}$ and $\epsilon^{i j k}$ are the structure constants of $SU(3)$ and $SU(2)$. And therefore, the gauge fields $G_\mu$, $W_\mu$, and $B_\mu$ must transform in the adjoint representation: 
\begin{equation}
	\begin{aligned}
		\delta B_{\mu} &=\partial_{\mu} \theta, \\
		\delta W_{\mu}^{i} &=\partial_{\mu} \theta^{i}-g \epsilon^{i j k} \theta^{j} W_{\mu}^{k}, \\
		\delta G_{\mu}^{a} &=\partial_{\mu} \epsilon^{a}-g_{s} f^{a b c} \epsilon^{b} G_{\mu}^{c}.
	\end{aligned}
\end{equation}
Then, the curvature strength tensors are
\begin{equation}
\begin{aligned}
	G_{\mu \nu}^{a} &=\partial_{\mu} G_{\nu}^{a}-\partial_{\nu} G_{\mu}^{a}+g_{s} f^{a b c} G_{\mu}^{b} G_{\nu}^{c} \\
	W_{\mu \nu}^{i} &=\partial_{\mu} W_{\nu}^{i}-\partial_{\nu} W_{\mu}^{i}+g \epsilon^{i j k} W_{\mu}^{j} W_{\nu}^{k} \\
	B_{\mu \nu} &=\partial_{\mu} B_{\nu}-\partial_{\nu} B_{\mu}
\end{aligned}
\end{equation}
and the ``kinetic'' term for gauge fields in the Lagrangian is  
\begin{equation}
\mathcal{L}_{\text{Gauge}}=-\frac{1}{4} G_{\mu \nu}^{a} G_{a}^{\mu \nu}-\frac{1}{4} W_{\mu \nu}^{i} W_{i}^{\mu \nu}-\frac{1}{4} B_{\mu \nu} B^{\mu \nu}.
\end{equation}
while these kinetic terms induce vertices between gauge bosons and in turn do not take into account the masses for such vector bosons, the Higgs mechanism produces the masses for them and gives us the linear combination to the physical bosons $W^\pm$, $Z$, $\gamma$:
\begin{equation}
\begin{cases}
	\begin{aligned}
		W_{\mu}^{+} &=\frac{1}{\sqrt{2}}\left(W_{\mu}^{1}-i W_{\mu}^{2}\right) \\
		W_{\mu}^{-} &=\frac{1}{\sqrt{2}}\left(W_{\mu}^{1}+i W_{\mu}^{2}\right) \\
		Z_{\mu} &=c_{w} W_{\mu}^{3}-s_{w} B_{\mu} \\
		A_{\mu} &=s_{w} W_{\mu}^{3}+c_{w} B_{\mu}
	\end{aligned}
\end{cases}
\text{where}
\;
\begin{cases}
	s_{w}=\sin \theta_{w}=\dfrac{g}{\sqrt{g^{2}+g{\prime2}}}\\
	c_{w}=\cos \theta_{w}=\dfrac{g^\prime}{\sqrt{g^{2}+g{\prime2}}}
\end{cases}
\end{equation}
where to avoid confusion with Dirac matrices, we denote as $A_\mu$ the electromagnetic potential.

\begin{figure}[h!]
    \centering
    \begin{subfigure}[b]{0.48\textwidth}
        \centering
        \begin{fmffile}{feyngraph21} 
			\vspace{0.5cm}
            \begin{fmfgraph*}(80,60)
                \fmfleft{i1}
                \fmfright{o1,o2}
                
                \fmf{gluon,tension=2.0}{i1,v1}
                \fmf{gluon}{v1,o1}
                \fmf{gluon}{v1,o2}

                \fmflabel{$g^a$}{i1}
                \fmflabel{$g^b$}{o1}
                \fmflabel{$g^c$}{o2}
            \end{fmfgraph*}
			\vspace{0.5cm}
        \end{fmffile}
        \caption{Triple gluon vertex.}
        \label{fig-triple-gluon}
    \end{subfigure}
    \hfill
    \begin{subfigure}[b]{0.48\textwidth}
        \centering
        \begin{fmffile}{feyngraph22}
			\vspace{0.5cm}
            \begin{fmfgraph*}(80,60)
                \fmfleft{i1,i2}
                \fmfright{o1,o2}
                
                \fmf{gluon}{i1,v1}
                \fmf{gluon}{i2,v1}
                \fmf{gluon}{v1,o1}
                \fmf{gluon}{v1,o2}

                \fmflabel{$g^a$}{i1}
                \fmflabel{$g^b$}{i2}
                \fmflabel{$g^c$}{o1}
                \fmflabel{$g^d$}{o2}
            \end{fmfgraph*}
			\vspace{0.5cm}
        \end{fmffile}
        \caption{Quartic gluon vertex.}
        \label{fig-quartic-gluon}
    \end{subfigure}
	\begin{subfigure}[b]{0.48\textwidth}
        \centering
		\begin{fmffile}{feyngraph23}
			\vspace{1.0cm}
			\begin{fmfgraph*}(80,60)
				\fmfleft{i1}
                \fmfright{o1,o2}

				\fmf{boson,tension=2.0}{i1,v1}
                \fmf{boson}{v1,o1}
				\fmf{boson}{v1,o2}

				\fmflabel{$Z/\gamma$}{i1}
				\fmflabel{$W^+$}{o1}
				\fmflabel{$W^-$}{o2}
			\end{fmfgraph*}
			\vspace{0.5cm}
		\end{fmffile}
		\caption{Triple $WWX$ boson vertex.}
		\label{fig-triple-w}
	\end{subfigure}
	\begin{subfigure}[b]{0.48\textwidth}
        \centering
		\begin{fmffile}{feyngraph24}
			\vspace{1.0cm}
			\begin{fmfgraph*}(80,60)
				\fmfleft{i1,i2}
                \fmfright{o1,o2}

				\fmf{boson}{i1,v1}
                \fmf{boson}{i2,v1}
				\fmf{boson}{v1,o1}
                \fmf{boson}{v1,o2}

				\fmflabel{$W^+$}{i1}
				\fmflabel{$W^-$}{i2}
				\fmflabel{$W^+$}{o1}
				\fmflabel{$W^-$}{o2}
			\end{fmfgraph*}
			\vspace{0.5cm}
		\end{fmffile}
		\caption{Quartic $W$ boson vertex.}
		\label{fig-quartic-w}
	\end{subfigure}
	\begin{subfigure}[b]{0.48\textwidth}
        \centering
		\begin{fmffile}{feyngraph25}
			\vspace{1.0cm}
			\begin{fmfgraph*}(80,60)
				\fmfleft{i1,i2}
                \fmfright{o1,o2}

				\fmf{boson}{i1,v1}
                \fmf{boson}{i2,v1}
				\fmf{boson}{v1,o1}
                \fmf{photon}{v1,o2}

				\fmflabel{$W^+$}{i1}
				\fmflabel{$W^-$}{i2}
				\fmflabel{$Z$}{o1}
				\fmflabel{$\gamma$}{o2}
			\end{fmfgraph*}
			\vspace{0.5cm}
		\end{fmffile}
		\caption{Quartic $WWZ\gamma$ vertex.}
		\label{fig-quartic-wwzgamma}
	\end{subfigure}
	\begin{subfigure}[b]{0.48\textwidth}
        \centering
		\begin{fmffile}{feyngraph26}
			\vspace{1.0cm}
			\begin{fmfgraph*}(80,60)
				\fmfleft{i1,i2}
                \fmfright{o1,o2}

				\fmf{boson}{i1,v1}
                \fmf{boson}{i2,v1}
				\fmf{boson}{v1,o1}
                \fmf{boson}{v1,o2}

				\fmflabel{$W^+$}{i1}
				\fmflabel{$W^-$}{i2}
				\fmflabel{$Z$}{o1}
				\fmflabel{$Z$}{o2}
			\end{fmfgraph*}
			\vspace{0.5cm}
		\end{fmffile}
		\caption{Quartic $WWZZ$ vertex.}
		\label{fig-quartic-wwzz}
	\end{subfigure}
    \caption{Feynman diagrams for gauge boson self-interactions. The non-Abelian structure of $SU(3)_C$ and $SU(2)_L$ gives rise to triple and quartic gauge boson vertices. While $WWZ\gamma$ and $WWZZ$ quartic vertices exist, $WW\gamma\gamma$ does not appear as a fundamental vertex in the Standard Model due to the Abelian nature of the photon field.}
    \label{fig-gauge-vertices}
\end{figure}

\subsection{Matter Fields}
We refer to the fermionic fields of the SM as the matter fields. We distinguish fermions in these two categories: leptons, fermions that do not have strong interaction, and quarks that interact both strongly and electroweakly. In table \ref{tab-generations}, we can see that there are six leptons, three charged and three neutral: each charged lepton has an associated neutrino forming between them doublets of $SU(2)_L$ and similarly for quarks. 

According to the SM, there are three generations of fermions. Each generation contains a doublet of leptons and a doublet of quarks. Among generations, particles differ by their flavour quantum number and mass, but their strong and electrical interactions are identical. Moreover, the flavour quantum number is a quantity conserved by all interactions except for the weak interaction.  Each generation is more massive than the previous one. The second and third generations are unstable and they disintegrate into the first generation. This is why ordinary matter is composed of the first generation. All three generations are produced in nuclear reactors, colliders, and cosmic rays. 

%TO DO -> Adjust to the margin

\begin{table}[h!]
\centering
	{\small
	\begin{tabular}{|c||c||l|l|l|}
		\hline \multicolumn{2}{|c||}{ \textbf{Fermion categories} } & \multicolumn{3}{c|}{\textbf{ Elementary particle generation} } \bigstrut\\
		\hline \hline Type & Subtype & First & Second & Third \bigstrut\\
		\hline\hline \multirow{2}{*}{ Quarks ($q$) }  & up-type & ($u$) up & ($c$) charm & ($t$) top  \bigstrut \\
		\cline { 2 - 5 }  & down-type & ($d$) down & ($s$) strange & ($b$) bottom  \bigstrut\\
		\hline\hline \multirow{2}{*}{ Leptons ($\ell$) } & charged & ($e$) electron & ($\mu$) muon & ($\tau$) tauon \bigstrut\\
		\cline { 2 - 5 } & neutrino & ($\nu_e$) & ($\nu_\mu$) & ($\nu_\tau$) \bigstrut\\
		\hline
	\end{tabular}
	}
	\caption{Three generations of fermions according to the Standard Model of particle physics. Each generation containing two types of leptons and two types of quarks.}\label{tab-generations}
\end{table}

Under all the constraints on local gauge invariance and renormalizability of the theory, the fermionic Lagrangian for SM is given by
\begin{equation}
	\mathcal{L}_{\mathrm{Fer}}
	=i \bar{\ell}_{L}^j \slashed{\mathcal D} \ell_{L}^j
	+i \bar{e}_{R}^j \slashed{\mathcal D} e_{R}^j
	+i{\bar{q}}_{L}^j  \slashed{\mathcal D}  q_{L}^j
	+i{\bar{u}}_{R}^j  \slashed{\mathcal D}  u_{R}^j
	+i{\bar{d}}_{R}^j  \slashed{\mathcal D}  d_{R}^j
\end{equation}
where $\slashed{\mathcal D}\equiv \gamma ^\mu \mathcal D_\mu$ with covariant derivative
\begin{equation}
	\mathcal D_\mu = \partial_\mu -ig_st_ aG^a_\mu -ig T_i W_\mu^i -ig'\frac Y2 B_\mu,
\end{equation}
and gauge fields $G^a$, $W^i$, and $B$ acting on each kind of fermion via
\begin{equation}
\begin{aligned}
	\mathcal D_{ \mu} \ell_L^i &=\fac{\partial_{\mu}-i g T_j W_{\mu}^{j}+i \frac{g^{\prime}}2 B_{\mu}} \ell_L^i \\
	\mathcal D_{ \mu} e_R^i &=\fac{\partial_{\mu} -  i g^{\prime}  B_{\mu}\vph}e_R^i \\
	\mathcal D_{ \mu} q_L^i &=\fac{\partial_{\mu}-i g_{s} t_{a} G_{\mu}^{a}-i g T_j W_{\mu}^{j}-i \frac{g^{\prime}}{6} B_{\mu}} q_L^i \\
	\mathcal D_{ \mu} u_R^i &=\fac{\partial_{\mu} -i g_{s} t_{a} G_{\mu}^{a} - i \frac{2g^{\prime}}3  B_{\mu}}u_R^i \\
	\mathcal D_{ \mu} d_R^i &=\fac{\partial_{\mu} -i g_{s} t_{a} G_{\mu}^{a} + i \frac{g^{\prime}}3  B_{\mu}}d_R^i \\
\end{aligned}
\end{equation}
which couples the fermions to the gauge bosons. As we will show below, after electroweak symmetry breaking, these interactions give rise to the familiar electromagnetic, weak, and strong forces, where the physical $\gamma$, $Z$, and $W$ bosons are a superposition of the original $B$ and $W$ fields. Feynman diagrams for these interactions are shown in Fig.~\ref{fig-gauge-interactions}.

\begin{figure}[h!]
    \centering
    \begin{subfigure}[b]{0.48\textwidth}
        \centering
        \begin{fmffile}{feyngraph5} 
			\vspace{0.5cm}
            \begin{fmfgraph*}(80,60)
                \fmfleft{i1}
                \fmfright{o1,o2}
                
                \fmf{photon}{i1,v1}
                \fmf{fermion}{o1,v1}
                \fmf{fermion}{v1,o2}

                \fmflabel{$\gamma$}{i1}
                \fmflabel{$\bar f$}{o1}
                \fmflabel{$f$}{o2}
            \end{fmfgraph*}
			\vspace{0.5cm}
        \end{fmffile}
        \caption{Electromagnetic interaction.}
        \label{fig-em-interaction}
    \end{subfigure}
    \hfill
    \begin{subfigure}[b]{0.48\textwidth}
        \centering
        \begin{fmffile}{feyngraph6}
			\vspace{0.5cm}
            \begin{fmfgraph*}(80,60)
                \fmfleft{i1}
                \fmfright{o1,o2}
                
                \fmf{boson}{i1,v1}
                \fmf{fermion}{o1,v1}
                \fmf{fermion}{v1,o2}

                \fmflabel{$W^{\pm}$}{i1}
                \fmflabel{$\bar \ell/\bar u$}{o1}
                \fmflabel{$\nu/d$}{o2}
            \end{fmfgraph*}
			\vspace{0.5cm}
        \end{fmffile}
        \caption{Charged weak interaction.}
        \label{fig-charged-weak}
    \end{subfigure}
	\begin{subfigure}[b]{0.48\textwidth}
        \centering
		\begin{fmffile}{feyngraph7}
			\vspace{1.0cm}
			\begin{fmfgraph*}(80,60)
				\fmfleft{i1}
				\fmfright{o1,o2}

				\fmf{boson}{i1,v1}
                \fmf{fermion}{o1,v1}
                \fmf{fermion}{v1,o2}

				\fmflabel{$Z$}{i1}
				\fmflabel{$\bar f$}{o1}
				\fmflabel{$f$}{o2}
			\end{fmfgraph*}
			\vspace{0.5cm}
		\end{fmffile}
		\caption{Neutral weak interaction.}
		\label{fig-neutral-weak}
	\end{subfigure}
	\begin{subfigure}[b]{0.48\textwidth}
        \centering
		\begin{fmffile}{feyngraph8}
			\vspace{1.0cm}
			\begin{fmfgraph*}(80,60)
				\fmfleft{i1}
				\fmfright{o1,o2}

				\fmf{gluon}{i1,v1}
                \fmf{fermion}{o1,v1}
                \fmf{fermion}{v1,o2}

				\fmflabel{$g$}{i1}
				\fmflabel{$\bar q$}{o1}
				\fmflabel{$q$}{o2}

				% \fmfv{lab=$ig_s\gamma^\mu t^a$, lab.dist=0.3cm, lab.angle=115}{v1}
			\end{fmfgraph*}
			\vspace{0.5cm}
		\end{fmffile}
		\caption{Strong interaction.}
		\label{fig-strong-interaction}
	\end{subfigure}
    \caption{Feynman diagrams for gauge boson interactions in the Standard Model.}\label{fig-gauge-interactions}
\end{figure}

\subsection{Electroweak Symmetry Breaking}

In the SM, the electroweak symmetry $SU(2)_{L} \times U(1)_{Y}$ is spontaneously broken down to the electromagnetic $U(1)_{\text{EM}}$ symmetry by a complex scalar Higgs field $H=\left(H^{+}, H^{0}\right)$ transforming as an $SU(2)_{L}$ doublet with hypercharge $+1$. Its dynamics are governed by the Mexican-hat potential:
\begin{equation}
    V(H)=-\mu^{2}|H|^{2}+\lambda|H|^{4} \quad \Rightarrow \quad v^{2} \equiv \langle H^{\dagger} H \rangle = \mu^{2} / 2\lambda.
\end{equation}
The vacuum expectation value (vev) aligns with the electrically neutral component, $\langle H^{0} \rangle = v/\sqrt{2} \simeq 174 \mathrm{GeV}$, generating masses for the weak gauge bosons while preserving $U(1)_{\text{EM}}$.

Fermion masses arise through Yukawa couplings, which represent the most general renormalizable interactions between the Higgs field and the fermion fields:
\begin{equation}
    \mathcal{L}_{\text{Yuk}} = y_{u}^{ij} \bar{q}_{L}^{i} u_{R}^{j} \tilde{H} + y_{d}^{ij} \bar{q}_{L}^{i} d_{R}^{j} H + y_{\ell}^{ij} \bar{\ell}_L^{i} e_{R}^{j} H + \text{h.c.},
\end{equation}
where $\tilde{H} = i\sigma_2 H^*$, and $y_{u}, y_{d}, y_{\ell}$ are arbitrary $3 \times 3$ complex matrices in flavor space. When the Higgs acquires its vev, 
\begin{equation}
		H = \begin{pmatrix} G^{+} \\ \frac{1}{\sqrt{2}}(v + h + i G^{0}) \end{pmatrix},\qquad \Longrightarrow \quad \langle H \rangle = \begin{pmatrix} 0 \\ \frac{v}{\sqrt{2}} \end{pmatrix},
\end{equation}
these couplings generate Dirac mass terms for the fermions.

\begin{figure}[h!]
    \centering
    \begin{subfigure}[b]{0.48\textwidth}
        \centering
        \begin{fmffile}{feyngraph9} 
			\vspace{0.5cm}
            \begin{fmfgraph*}(80,60)
                \fmfleft{i1,i2}
                \fmfright{o1}
                
                \fmf{fermion}{i1,v1}
                \fmf{fermion}{v1,i2}
                \fmf{dashes,tension=2.0}{v1,o1}

                \fmflabel{$\bar q_L$}{i1}
                \fmflabel{$u_R$}{i2}
                \fmflabel{$h$}{o1}
            \end{fmfgraph*}
			\vspace{0.5cm}
        \end{fmffile}
        \caption{Up-type Yukawa coupling.}
        \label{fig-yukawa-up}
    \end{subfigure}
    \hfill
    \begin{subfigure}[b]{0.48\textwidth}
        \centering
        \begin{fmffile}{feyngraph10}
				\vspace{0.5cm}
							\begin{fmfgraph*}(80,60)
									\fmfleft{i1,i2}
									\fmfright{o1}
									
									\fmf{fermion}{i1,v1}
									\fmf{fermion}{v1,i2}
									\fmf{dashes,tension=2.0}{v1,o1}

									\fmflabel{$\bar q_L$}{i1}
									\fmflabel{$d_R$}{i2}
									\fmflabel{$h$}{o1}
							\end{fmfgraph*}
				\vspace{0.5cm}
        \end{fmffile}
        \caption{Down-type Yukawa coupling.}
        \label{fig-yukawa-down}
    \end{subfigure}
	\begin{subfigure}[b]{0.48\textwidth}
        \centering
		\begin{fmffile}{feyngraph11}
			\vspace{1.0cm}
			\begin{fmfgraph*}(80,60)
				\fmfleft{i1,i2}
				\fmfright{o1}

				\fmf{fermion}{i1,v1}
				\fmf{fermion}{v1,i2}
				\fmf{dashes,tension=2.0}{v1,o1}

				\fmflabel{$\bar \ell_L$}{i1}
				\fmflabel{$e_R$}{i2}
				\fmflabel{$h$}{o1}
			\end{fmfgraph*}
			\vspace{0.5cm}
		\end{fmffile}
		\caption{Lepton Yukawa coupling.}
		\label{fig-yukawa-lepton}
	\end{subfigure}
	\begin{subfigure}[b]{0.48\textwidth}
        \centering
		\begin{fmffile}{feyngraph12}
			\vspace{1.0cm}
			\begin{fmfgraph*}(80,60)
				\fmfleft{i1}
				\fmfright{o1,o2,o3}

				\fmf{gluon}{i1,v1}
				\fmf{gluon,tension=0.8}{v1,o1}
				\fmf{gluon,tension=0.8}{v1,o2}
				\fmf{gluon,tension=0.8}{v1,o3}

				\fmflabel{$g^a$}{i1}
				\fmflabel{$g^b$}{o1}
				\fmflabel{$g^c$}{o2}
				\fmflabel{$g^d$}{o3}
			\end{fmfgraph*}
			\vspace{0.5cm}
		\end{fmffile}
		\caption{Gluon self-interaction.}
		\label{fig-gluon-self}
	\end{subfigure}
    \caption{Feynman diagrams for Yukawa couplings and gluon self-interactions in the Standard Model.}
\end{figure}


The quark mass matrices are proportional to the Yukawa matrices: $M_u = y_u v/\sqrt{2}$, $M_d = y_d v/\sqrt{2}$. Since $y_u$ and $y_d$ are general complex matrices, they cannot be simultaneously diagonalized. The physical quark masses and states are found by performing separate unitary transformations on the left- and right-handed fields:
\begin{equation}
    u_L \to V_L^u u_L, \quad u_R \to V_R^u u_R, \quad d_L \to V_L^d d_L, \quad d_R \to V_R^d d_R,
\end{equation}
such that $V_L^u M_u V_R^{u\dagger} = M_u^{\text{diag}}$ and $V_L^d M_d V_R^{d\dagger} = M_d^{\text{diag}}$ are diagonal with real, positive entries.

\begin{figure}[h!]
		\centering
		\begin{fmffile}{feyngraph13} 
		\vspace{0.5cm}
				\begin{fmfgraph*}(80,60)
						\fmfleft{i1,i2}
						\fmfright{o1}
						
						\fmf{fermion}{i1,v1,o1}
						\fmf{dashes}{v1,i2}

						\fmflabel{$\bar f_L$}{i1}
						\fmflabel{$\langle H \rangle$}{i2}
						\fmflabel{$f_R$}{o1}
				\end{fmfgraph*}
		\vspace{0.5cm}
		\end{fmffile}
		\caption{Feynman diagram for quark mass generation via Yukawa coupling and Higgs vev.}
		\label{fig-quark-mass}
\end{figure}

This diagonalization procedure has a direct consequence for the charged-current interactions mediated by the $W^{\pm}$ bosons. In the flavor basis the interaction reads
\begin{equation}
    \mathcal{L}_{W} \supset -\frac{g}{\sqrt{2}} (\bar{u}_L, \bar{c}_L, \bar{t}_L) \gamma^\mu W_\mu^+ (d_L, s_L, b_L)^T + \text{h.c.}
\end{equation}
After moving to the mass basis, the left-handed up- and down-type quarks rotate differently ($u_L \to V_L^u u_L$, $d_L \to V_L^d d_L$), and the interaction becomes
\begin{equation}
    \mathcal{L}_{W} \supset -\frac{g}{\sqrt{2}} (\bar{u}_L, \bar{c}_L, \bar{t}_L) \gamma^\mu W_\mu^+ V_{\mathrm{CKM}} (d_L, s_L, b_L)^T + \text{h.c.},
\end{equation}
where the Cabibbo–Kobayashi–Maskawa (CKM) matrix appears as the mismatch between the two rotations:
\begin{equation}
    V_{\mathrm{CKM}} \equiv V_{L}^{u} V_{L}^{d \dagger} = \begin{pmatrix}
        V_{ud} & V_{us} & V_{ub} \\
        V_{cd} & V_{cs} & V_{cb} \\
        V_{td} & V_{ts} & V_{tb}
    \end{pmatrix}.
\end{equation}
This unitary matrix encodes flavor mixing in charged-current weak interactions, and its non-diagonal structure is the origin of all quark flavor-changing processes in the Standard Model.

The situation is different for leptons in the minimal Standard Model without right-handed neutrinos. The charged-lepton mass matrix $M_\ell = y_\ell v/\sqrt{2}$ can be diagonalized by field redefinitions, but since neutrinos are massless in this framework, there is no additional rotation in the neutrino sector. As a result, the charged-current interaction
\begin{equation}
    \mathcal{L}_{W} \supset -\frac{g}{\sqrt{2}} \bar{\nu}_L \gamma^\mu W_\mu^+ \ell_L + \text{h.c.}
\end{equation}
remains diagonal in the mass basis. This implies \textit{Lepton Flavor Universality} (LFU): the electroweak gauge bosons couple to all three lepton families with identical strength. In particular, the $W$ boson couples to each $\bar{\nu}_L \gamma^\mu \ell_L$ current with coefficient $-g/\sqrt{2}$, and the $Z$ boson couplings to $\ell_L$ and $\ell_R$ are flavor-independent because the hypercharge assignments are the same for all families.

LFU means that processes differing only by the lepton flavor, such as leptonic decays or semileptonic transitions, are predicted to occur with the same rates up to well-understood effects: differences in phase space, helicity suppression, lepton-mass dependence, and small radiative corrections. The assumption of LFU is central in the extraction of CKM parameters, since experimental determinations from decays involving electrons, muons, and tau leptons can be consistently combined.

Precision tests of LFU focus on ratios of decay widths or branching fractions where theoretical and experimental uncertainties cancel to a large extent. Agreement with these tests confirms the gauge structure of the Standard Model, while deviations would point to new physics.

The Lagrangian of the scalar sector is
\begin{equation}
	\mathcal{L}_{H}= \mathcal D_{\mu} H^{\dagger} \mathcal D^{\mu} H - V\!\left(H^{\dagger}, H\right),
\end{equation}
with the covariant derivative defined as $\mathcal D_{\mu} H=\left(\partial_{\mu}+i g T_a W_{\mu}^{a}+i g^{\prime} \tfrac{Y}{2} B_{\mu}\right) H$. Substituting the Higgs vacuum expectation value, one obtains
\begin{equation}
	\begin{aligned}
		\mathcal{L}_{\langle H\rangle}
		&=-\frac{1}{8}\left(\begin{array}{ll}
			0 & v
		\end{array}\right)\left(\begin{array}{ll}
			g W_{\mu}^{3}-g' B_{\mu} & g\left(W_{\mu}^{1}-i W_{\mu}^{2}\right)\vph \\
			g\left(W_{\mu}^{1}+i W_{\mu}^{2}\right)&-g W_{\mu}^{3}-g' B_{\mu}\vph
		\end{array}\right)^{2}\left(\begin{array}{l}
			0 \\
			v
		\end{array}\right)
		\\&=
		-\frac{1}{8} v^{2} V_{\mu}^{T}\left(\begin{array}{cccc}
			g^{2} & 0 & 0 & 0 \\
			0 & g^{2} & 0 & 0 \\
			0 & 0 & g^{2} & -g' g \\
			0 & 0 & -g' g & g'^{2}
		\end{array}\right) V^{\mu},
	\end{aligned}
\end{equation} 
where $V_{\mu}^{T}=\left(W_{\mu}^{1}, W_{\mu}^{2}, W_{\mu}^{3}, B_{\mu}\right)$. Diagonalizing this mass matrix yields eigenvalues $0$, $-\tfrac{1}{8} v^{2} g^{2}$, $-\tfrac{1}{8} v^{2} g^{2}$, and $-\tfrac{1}{8} v^{2}\left(g^{2}+g'^{2}\right)$. The massless state corresponds to the photon, the heaviest to the $Z$ boson, and the two degenerate intermediate states to the charged bosons $W^\pm$, which transform under the representation of the unbroken generator $Q_{EM}$. 

\begin{figure}[h!]
    \centering
    \begin{subfigure}[b]{0.48\textwidth}
        \centering
        \begin{fmffile}{feyngraph14} 
			\vspace{0.5cm}
            \begin{fmfgraph*}(80,60)
                \fmfleft{i1,i2}
                \fmfright{o1,o2}
                
                \fmf{dashes}{i1,v1}
                \fmf{dashes}{i2,v1}
                \fmf{boson}{v1,o1}
                \fmf{boson}{v1,o2}

                \fmflabel{$h$}{i1}
                \fmflabel{$h$}{i2}
                \fmflabel{$W^+$}{o1}
                \fmflabel{$W^-$}{o2}
            \end{fmfgraph*}
			\vspace{0.5cm}
        \end{fmffile}
        \caption{Higgs-$W$ boson coupling.}
        \label{fig-higgs-w}
    \end{subfigure}
    \hfill
    \begin{subfigure}[b]{0.48\textwidth}
        \centering
        \begin{fmffile}{feyngraph15}
			\vspace{0.5cm}
            \begin{fmfgraph*}(80,60)
                \fmfleft{i1,i2}
                \fmfright{o1}
                
                \fmf{dashes}{i1,v1}
                \fmf{dashes}{i2,v1}
                \fmf{boson,tension=2.0}{v1,o1}

                \fmflabel{$H$}{i1}
                \fmflabel{$H$}{i2}
                \fmflabel{$Z$}{o1}

				\fmfv{lab=$\frac{ig^2 v}{\cos\theta_w}$, lab.dist=0.3cm, lab.angle=-65}{v1}
            \end{fmfgraph*}
			\vspace{0.5cm}
        \end{fmffile}
        \caption{Higgs-$Z$ boson coupling.}
        \label{fig-higgs-z}
    \end{subfigure}
	\begin{subfigure}[b]{0.48\textwidth}
        \centering
		\begin{fmffile}{feyngraph16}
			\vspace{1.0cm}
			\begin{fmfgraph*}(80,60)
				\fmfleft{i1}
                \fmfright{o1,o2}

				\fmf{dashes,tension=2.0}{i1,v1}
                \fmf{dashes}{v1,o1}
				\fmf{dashes}{v1,o2}

				\fmflabel{$H$}{i1}
				\fmflabel{$H$}{o1}
				\fmflabel{$H$}{o2}

				\fmfv{lab=$3i\lambda v$, lab.dist=0.25cm, lab.angle=115}{v1}
			\end{fmfgraph*}
			\vspace{0.5cm}
		\end{fmffile}
		\caption{Higgs triple self-coupling.}
		\label{fig-higgs-triple}
	\end{subfigure}
	\begin{subfigure}[b]{0.48\textwidth}
        \centering
		\begin{fmffile}{feyngraph17}
			\vspace{1.0cm}
			\begin{fmfgraph*}(80,60)
				\fmfleft{i1,i2}
                \fmfright{o1,o2}

				\fmf{dashes}{i1,v1}
                \fmf{dashes}{i2,v1}
				\fmf{dashes}{v1,o1}
                \fmf{dashes}{v1,o2}

				\fmflabel{$H$}{i1}
				\fmflabel{$H$}{i2}
				\fmflabel{$H$}{o1}
                \fmflabel{$H$}{o2}

				\fmfv{lab=$3i\lambda$, lab.dist=0.3cm, lab.angle=90}{v1}
			\end{fmfgraph*}
			\vspace{0.5cm}
		\end{fmffile}
		\caption{Higgs quartic self-coupling.}
		\label{fig-higgs-quartic}
	\end{subfigure}
    \caption{Feynman diagrams for the Higgs sector interactions in the Standard Model.}
\end{figure}

This suffices to illustrate how the Standard Model, formulated as a relativistic quantum field theory, describes the interactions of matter fields through the fundamental forces, mediated by vector bosons. The Higgs boson, also part of the Standard Model spectrum, plays the central role in generating masses for the weak bosons, the fermions, and indirectly in distinguishing the photon as the only massless gauge boson of the electroweak sector.

Since its formulation, the Standard Model has been tested extensively and has shown remarkable success, both in explaining existing data and in making accurate predictions. A well-known example is the agreement between the Standard Model prediction and the experimental measurement of the electron magnetic dipole moment, consistent to twelve significant figures~\parencite{PhysRevLett.97.030801}. The discovery of the Higgs boson in 2012 was the culmination of almost fifty years of experimental effort, confirming the mechanism incorporated into the Standard Model in the late 1960s through the unification of the electromagnetic and weak interactions by Glashow, Weinberg, and Salam~\parencite{PhysRevLett.19.1264, gl1961579}. With this discovery, the full particle spectrum predicted by the Standard Model was finally observed.
 % Standard Model

\section{Deficiencies of Standard Model and New Physics}

{\Large Pending to be updated} %TO DO -> REFRESH FOR ACTUAL STATE

While these and other successes of the Standard Model are an achievement for the field of particle physics, it is well known that this cannot be the ultimate theory of fundamental particles and interactions. Even though the Standard Model is currently the best description there is of the subatomic world, it does not explain the complete picture; there are also important questions that it does not answer and it is also surrounded by different irregularities. Some of them are completely incompatible with the current Standard Model, and strongly suggest that the Standard Model requires a consistent extension to solve experimental and theoretical problems that we will label as the cosmological problems, phenomenological problems, and theoretical problems. Below we will list very briefly the main representatives of these categories.



\subsection{Theoretical problems}

\begin{description}
	\item[Hierarchy problem] Is the problem concerning the large discrepancy between aspects of the weak force and gravity. Both of these forces involve constants of nature, the Fermi constant for the weak force and the Newtonian constant of gravitation for gravity. If the Standard Model is used to calculate the quantum corrections to Fermi's constant, it appears that Fermi's constant is surprisingly large and is expected to be closer to Newton's constant unless there is a delicate cancellation between the bare value of Fermi's constant and the quantum corrections to it. 
	
	In the Standard Model context, the Higgs boson is much lighter than the energy scale on which the standard model is considered valid (ideally the Plank mass), and the quantum corrections to the Higgs mass are on the order of this energy scale; it would inevitably make the Higgs and fermions masses huge, comparable to the scale at which new physics appears, unless there is an incredible fine-tuning cancellation between the quadratic radiative corrections and the bare mass. This level of fine-tuning is deemed unnatural.
	\item[Strong CP problem] QCD Lagrangian supports a term associated with the strength tensor dual for gluons that break CP symmetry in the strong interaction sector. Experimentally, however, no such violation has been found, implying that the coefficient of this term is fine tunned to zero. 
	\item[Quantum triviality] Suggests that it may not be possible to create a consistent quantum field theory involving elementary scalar Higgs particles because for high momentum particles the renormalization presents inconsistencies unless the renormalization of the charges becomes null, and therefore not interacting, \textit{i.e.} trivial. Nevertheless, because the Higgs boson plays a central role in the Standard Model of particle physics, the question of triviality in Higgs models is of great importance. 
	\item[Number of parameters and Unexplained relations] In total, the standard model has too many free parameters (19 in total) that are obtained experimentally, and there are indications that several of them may be correlated, however the origin of these correlations is beyond the standard model.
	
	For example, Yoshio Koide's empirical formula~\parencite{0505220}
	$$
	\frac{m_{e}+m_{\mu}+m_{\tau}}{\left(\sqrt{m_{e}}+\sqrt{m_{\mu}}+\sqrt{m_{\tau}}\right)^{2}}=0.666661(7) \approx \frac{2}{3}
	$$
	seems to indicate that there is a way to predict the masses of leptons.
	
\end{description}
\subsection{Cosmological problems}
\begin{description}
	\item[Gravity] Although the Standard Model describes the three important fundamental forces at the subatomic scale, it does not include gravity. However, at larger scales, gravity becomes present and is described by Einstein's theory of general relativity, in which gravity rather than a force is a property that measures the deformation of spacetime then, the most of the conventional machinery of perturbative QFT is profoundly incompatible with the general relativistic framework~\parencite{book:217893}, and a theory of quantum gravity with which we are enabled to perform calculations has yet to be discovered.
	
	
	
	
	\item[Dark matter] Within the framework of Einstein's general relativity, the cosmological standard model ($\Lambda$CDM) is, like the standard model of particle physics, one of the most successful theories of the 20th century. $\Lambda$CDM it is based on a very specific density of matter that can be explained with ordinary matter from the standard model of particles, baryonic matter; according to $\Lambda$CDM, in addition to baryonic matter, there is a kind of matter five times more abundant than baryonic matter, which does not interact electrically (therefore it is dark) and non-relativistic (therefore it is cold), known as cold dark matter (CDM).  Yet, the Standard Model does not supply any fundamental particles that are good dark matter candidates.
	\item[Dark energy] Moreover, according to Lambda CDM only 31\% of the energy that makes up the universe is matter, the remaining 69\% of the universe's energy should consist of the so-called dark energy, a constant energy density for the vacuum ($\Lambda$). If we try to explain dark energy in terms of vacuum energy only from the standard model lead to a mismatch of 120 orders of magnitude~\parencite{Adler1995}, sometimes called \textit{The Worst Theoretical Prediction in the History of Physics}~\parencite{book:15261}; a bit sensationalist title to indicate the fact that we do not fully understand the composition of the particle spectrum of the universe.
	
	\item[Matter-antimatter asymmetry] In the observable universe there is more matter than antimatter. In 1967, Andrei Sakharov proposed a set of three necessary conditions that a baryon-generating interaction must satisfy to produce matter and antimatter at different rates~\parencite{1967JETPL...5...24S}. While the standard model can satisfy these three conditions~\parencite{PhysRevLett.37.8,ph/0609145},  it satisfies them at three different energy scales and therefore presents difficulties in the capability to explain the  matter-antimatter asymmetry~\parencite{robinson2011symmetry}. 
	
\end{description}
\subsection{Phenomenological problems}\label{pheno_bsm}
\begin{description}
	\item[Neutrino masses] In the standard model, the right chiral component of neutrinos is not part of the composition of fermionic fields because if they were present they would not interact and consequently neutrinos have no mass. However, the precision measurement~\parencite{Abe_2008} of the mixing matrix for neutrino oscillations has shown that neutrinos change flavour in free flight and in turn that the three neutrino flavours cannot have identical mass, meaning that all three cannot have zero mass. There is no single way to extend the standard model to include masses to neutrinos and even more to explain their value so close to zero and results in the open problem confirmed at the phenomenological level present in the standard model.
	\item[Anomalous B-mesons decay] A B-meson is a bound state made up of an quark-antiquark pair where one of them comes from a $b$-quark. Various experimental results~\parencite{PhysRevLett.109.101802, PhysRevLett.115.111803,Altmannshofer_2015, Hurth_2016,arxiv.2103.11769} have suggested a surplus over Standard Model predictions in its decays to D-mesons along with a $\tau$, $\nu_\tau$ doublet. While none of them have reached the statistical threshold of 5 $\sigma$ to declare a break from the standard model, the Capdevilaa's meta-analysis of all available data reported a $5.0\sigma$ deviation from SM~\parencite{Capdevila_2018}. 
	\item[Anomalous magnetic dipole moment of muon]  Unlike the extraoirdinary agreement between theory and experiment with the magnetic dipole moment of the electron~\parencite{PhysRevLett.97.030801}; in the case of the muon, the measurement of Fermilab's Muon g-2 experiment has presented an apparent discrepancy  with an accuracy of 4.2 $\sigma$~\parencite{arxiv.1311.2198, Abi_2021} which strengthen evidence of new physics in the muon sector and apparently in the violation of lepton universality of the standard model. 
	\item[Anomalous mass of the W boson] Results from the CDF Collaboration, reported in April 2022, indicate that the mass of a W boson exceeds the mass predicted by the Standard Model with a significance of 7 $\sigma$~\parencite{abk1781}. However, this very highly accurate result, unlike the anomaly in B-meson Decay, is in tension with the results of Atlas, LHCb, LEP and D0 II~\parencite{Aaboud_2018,jhep012022036,Schael_2006,Abazov_2012,}. Certainly, a review of all the information we possess so far must be done to determine if this anomaly is a window into new physics beyond the standard model.
	\item[CCA and $q\bar q \mapsto e^+ e^-$] It has been observed that certain nuclear beta decays happen less frequently than expected~\parencite{PhysRevC.102.045501}. This tension, called the Cabibbo Angle anomaly (CAA), displays a significance around $3 \sigma$~\parencite{1674-1137-40-10-100001}, and can again be interpreted as a sign that electrons and muons behave more differently than predicted by the SM~\parencite{PhysRevLett.125.111801}. Furthermore, the CMS experiment at CERN observed more very high-energetic electrons in proton-proton collisions $\left(q \bar{q} \rightarrow e^{+} e^{-}\right)$ compared to muons than expected~\parencite{Sirunyan2021}.
\end{description}
 % Deficiencies of Standard Model and Evidence
\section{Lepton Flavour Universality: Tests and Anomalies}\label{sec:LFU}

As established previously, gauge interactions of charged leptons are flavour-universal~\cite{gl1961579,PhysRevLett.19.1264,1674-1137-40-10-100001} in the SM: the $SU(2)_L\times U(1)_Y$ couplings are generation-independent~\cite{gl1961579,PhysRevLett.19.1264}, meaning that after accounting for kinematic and mass effects, processes mediated by electroweak interactions predict identical couplings to electrons, muons, and tau leptons~\cite{1674-1137-40-10-100001}. While small deviations can arise from well-understood mass-dependent, phase-space, and radiative corrections~\cite{1674-1137-40-10-100001}, genuine Lepton Flavour Universality Violation (LFUV) would constitute clear evidence of new physics beyond the Standard Model~\cite{Hiller:2014yaa,Dorsner:2016wpm,Buttazzo:2017ixm}.

It is worth noting that total lepton flavour numbers are accidental symmetries of the renormalizable SM with massless neutrinos~\cite{1674-1137-40-10-100001}. While neutrino oscillations demonstrate flavour violation in the neutral sector~\cite{SuperK:1998osc,SNO:2002NC}, any significant charged-lepton flavour violation (cLFV) would unambiguously signal BSM physics~\parencite{1674-1137-40-10-100001}. This section focuses specifically on LFU tests and their current experimental status~\cite{1674-1137-40-10-100001,Ciuchini:2022wbq}.

An extensive experimental program probes LFU across various processes~\cite{1674-1137-40-10-100001}:
\begin{itemize}
  \item \textbf{Rare $B$ decays} ($b\to s\ell^+\ell^-$): Clean ratios $R_K$ and $R_{K^*}$ comparing muon to electron modes~\cite{LHCb:2014vgu,LHCb:2017avl,LHCb:2019hip,LHCb:2021trn,LHCb:2022qnv,LHCb:2022zom,Hiller:2014yaa}.
  \item \textbf{Charged-current $B$ decays} ($b\to c\ell\nu$): Ratios $R_D$ and $R_{D^*}$ comparing $\tau$ to light leptons~\cite{BaBar:2012obs,BaBar:2013mob,Belle:2015qfa,LHCb:2015gmp,LHCb:2017rln,LHCb:2023zxo,Amhis_2021}.
  \item \textbf{Light-meson decays}: Leptonic ($\pi, K\to \ell\nu$) y semileptonic ($K_{\ell3}$) universality tests~\cite{1674-1137-40-10-100001,Antonelli:2010}.
  \item \textbf{Electroweak boson decays}: $W\to \ell\nu$ y $Z\to \ell^+\ell^-$ universality~\cite{atlas2020test,LEPEW:2006}.
  \item \textbf{$\tau$ decays}: Tests of $e/\mu/\tau$ universality in leptonic and semileptonic channels~\cite{1674-1137-40-10-100001,Amhis_2021}.
\end{itemize}
Combinations of these measurements provide stringent constraints on flavour-dependent interactions beyond the SM~\parencite{1674-1137-40-10-100001}.

Most LFU tests involving light mesons, $W/Z$ bosons, and $\tau$ decays show agreement with SM predictions at the percent level~\parencite{1674-1137-40-10-100001}. The most significant tensions initially emerged in semileptonic $B$ decays~\cite{Hiller:2014yaa,Buttazzo:2017ixm,Capdevila_2018,Alonso:2015sja,Calibbi:2015kma}, particularly in the neutral-current ratios $R_K$ and $R_{K^*}$ and the charged-current ratios $R_D$ and $R_{D^*}$ (discussed in detail in the following subsection)~\cite{Altmannshofer_2015,Capdevila_2018}. Recent analyses have brought $R_{K^{(*)}}$ measurements closer to SM predictions~\parencite{LHCb:2022qnv,LHCb:2022zom,Greljo:2022jac,Ciuchini:2022wbq}, while the situation for $R_{D^{(*)}}$ remains actively investigated~\cite{Amhis_2021}, with forthcoming data expected to provide decisive insights~\cite{Belle-II:2018jsg}. Collectively, these measurements delineate viable patterns of lepton non-universal interactions and provide crucial guidance for theoretical model building~\cite{Dorsner:2016wpm,Buttazzo:2017ixm,Angelescu:2018tyl,Cornella:2021sby}.

In the renormalizable SM, any credible observation of LFV  would require BSM physics~\cite{Hiller:2014yaa,Dorsner:2016wpm}. Various theoretical frameworks can accommodate such violations, including extended gauge sectors and other scenarios that generate non-universal couplings~\cite{DiLuzio:2017vat,Greljo:2018tuh,Angelescu:2021lln}. These models typically predict correlated signals across multiple precision observables~\cite{Greljo:2022jac,Ciuchini:2022wbq,Allwicher:2022gkm} and, depending on their flavour structure, may also induce cLFV at potentially observable levels, subject to tight constraints from existing experimental limits~\parencite{Blankenburg:2012nx,Angelescu:2018tyl}.
\textcolor{red}{AF: Toda esta parte es una repetición de lo que se dijo arriba....:
In recent years, significant attention has focused on precision measurements of $B$-meson decay rates~\cite{Hiller:2014yaa,Buttazzo:2017ixm}, particularly through ratios that test LFU~\cite{Amhis_2021}. The most prominent examples are the $R_{K^{(*)}}$~\parencite{LHCb:2014vgu,LHCb:2017avl,LHCb:2019hip,LHCb:2021trn} and $R_{D^{(*)}}$~\parencite{BaBar:2012obs,BaBar:2013mob,Abdesselam:2019dgh,Hirose:2017dxl,Sato:2016svk,Hirose:2016wfn,Huschle:2015rga,LHCb:2015gmp,Aaij:2015yra,Aaij:2017uff,LHCb:2017rln,LHCb:2023zxo} ratios, which compare decay rates to different lepton families~\cite{Amhis_2021,1674-1137-40-10-100001}. These measurements generated substantial theoretical interest, with numerous proposals for new physics scenarios that could explain potential deviations from SM predictions~\cite{Dorsner:2016wpm,Angelescu:2018tyl,Bauer:2015knc,Crivellin:2017zlb}.}
\textcolor{red}{
Recent re-analyses of $R_{K^{(*)}}$ data have shown this ratio to be compatible with the SM prediction~\parencite{LHCb:2022qnv,LHCb:2022zom,Greljo:2022jac,Ciuchini:2022wbq}, while the situation for $R_{D^{(*)}}$ remains an open question~\cite{Amhis_2021} that continues to motivate the study of scenarios where new particles might have preferential couplings to third-generation fermions~\cite{Greljo:2018tuh,King:2021jeo,Cornella:2019hct}.
}
The anomalous magnetic moment of the muon, $a_\mu \equiv (g-2)_\mu/2$, represents another benchmark precision observable sensitive to new virtual states~\cite{Aoyama:2020ynm}. The latest measurements from the Fermilab Muon $g-2$ experiment report a value with sub-ppm precision~\parencite{Abi_2021}, broadly consistent with but more precise than the earlier BNL result~\cite{Muong-2:2006rrc}. This measurement shows sustained tension with certain Standard Model evaluations~\cite{Aoyama:2020ynm}.

The theoretical prediction for $a_\mu$ combines QED, electroweak, and hadronic contributions~\cite{Aoyama:2020ynm}, with the hadronic vacuum polarization and light-by-light scattering components driving the dominant uncertainties~\parencite{arxiv.1311.2198,1674-1137-40-10-100001}. The comparison between experimental results and theoretical predictions therefore serves as a powerful indirect probe of new physics scenarios that couple to leptons~\cite{Aoyama:2020ynm}.

The complementarity between $(g-2)_\mu$, LFU tests in $B$ decays, and direct searches provides a multifaceted approach to testing the Standard Model~\cite{Ciuchini:2022wbq,Greljo:2022jac}. Limits on charged lepton flavour violation further constrain possible chirality-flipping couplings that could also contribute to dipole moments~\cite{Blankenburg:2012nx,Angelescu:2018tyl}, highlighting the interconnected nature of these precision observables in the search for physics beyond the Standard Model~\cite{Dorsner:2016wpm}.

Taken together, these precision observables underscore the importance of direct searches for new physics at colliders~\cite{CMS:2020wzx,ATLAS:2019qpq}. While deviations from LFU in light-meson, $\tau$, and electroweak boson decays remain consistent with SM expectations~\cite{1674-1137-40-10-100001,LEPEW:2006}, the persistent anomalies in $B$ decays and the muon $(g-2)$ point to scenarios where new states may couple non-universally to leptons~\cite{Buttazzo:2017ixm,Aoyama:2020ynm}. A particularly well-motivated possibility is that new particles exhibit enhanced couplings to third-generation fermions~\cite{Greljo:2018tuh,DiLuzio:2018zxy}. Such flavour structures naturally alleviate existing constraints from first- and second-generation processes while offering testable signatures in collider environments~\cite{Angelescu:2021lln,Cornella:2021sby}. 

Therefore, a dedicated experimental program to search for new particles with preferential couplings to the third generation is a crucial component of the BSM search strategy~\cite{CMS:2020wzx,ATLAS:2019qpq}. This program requires not only powerful collider experiments~\cite{ATLAS:2008xda,CMS:2008xjf} but also detailed \textit{feasibility studies} to assess the discovery potential of these non-standard signatures~\cite{Faroughy:2016osc,Baker:2019sli}. Such studies are essential to optimize trigger strategies~\cite{ATLAS:2008xda,CMS:2008xjf}, refine analysis techniques~\cite{Cowan:2011}, and ultimately guide the exploration of the most promising regions of parameter space where these hypothetical particles might reveal themselves~\cite{Dorsner:2016wpm}.
 % Beyond Standard Model