\chapter{Standard Model of Particle Physics}

The standard model (SM) of particle physics is a quantum field theory (QFT) in which fundamental particles are excitations of interacting relativistic fields in the quantum vacuum~\parencite{greiner2000relativistic}. In this context, matter in nature is formed by particles that have a fermionic character, and their interactions are described by the gauge principle, where integer spin particles, defined as vector bosons from the adjoint representation of a symmetry group (\textit{gauge group}), are the messengers of the interaction~\parencite{pokorski2000gauge}.

Specifically, the SM characterizes interactions through the gauge principle, where force-carrying particles (integer spin vector bosons) originate from the adjoint representation of symmetry groups (\textit{gauge groups}) \cite{pokorski2000gauge}. This elegant formulation unifies three of the four fundamental forces in nature.

In this chapter, we contextualize the SM by introducing the basic concepts of quantum field theory, including the notion of fields and symmetries. We then present the particle content of the SM, its gauge group, and the Lagrangian density that describes its dynamics. Finally, we discuss the Higgs mechanism and its role in providing mass to the weak gauge bosons and fermions.

\section{Fields}
Relativistic quantum fields are degrees of freedom in QFT. Formally, they are \textit{operator-valued functions on spacetime that transform under a representation of the Lorentz group on an invariant subspace}~\parencite{Tong1995}. The different representations of the Lorentz group are mainly characterized by their spin, and their fields obey a different equation of motion (see table~\ref{tab-repLorentz2}). 

In classical field theory, a variational principle is established which generates the equations that govern the dynamics of the different fields in a theory, \textit{the equations of motion}. Hamilton's principle, or principle of minimal action, indicates that all possible physical configurations for a set of fields $\varphi^I$, with $I=1,2,3,\cdots,n$, are those for which the action $S$ is  minimal~\parencite{Goldstein,jose1998classical}:
\begin{equation}\label{eq-action}
	S=\int \mathcal{L}(\varphi^I,\partial_\mu\varphi^I) d^4x.
\end{equation}
Here, $d^4x=dx^0dx^1 dx^2dx^3$ and $x\equiv(ct,x^1,x^2,x^3)\equiv(x^0,x^1,x^2,x^3)\in\mathcal{M}^4$, are the space-time coordinates in the Minkowskian spacetime ($\mathcal M^4$), and the function $\lag(\varphi^I,\partial_\mu\varphi^I)$ is called \textit{the Lagrangian density} of a theory~\parencite{greiner2000relativistic,Goldstein}. The problem in classical field dynamics is to find the functions $\varphi^I(x)$ in a space-time $\mathcal{M}^4$, fixing their boundary conditions. The solution to this classical problem is given by the Euler-Lagrange equations:
\begin{equation}\label{eq_EulerLag}
	\dpr{\mathcal{L}}{\varphi^I}-\dpr{}{x^\mu}\dpr{\lag}{\fac{\partial_\mu \varphi^I}}=0,
\end{equation}
and they are used to obtain the equations of motion of the set of fields $\varphi^I$~\parencite{jose1998classical}. 

While in classical field theory the Euler–Lagrange equations directly determines the dynamics of the system, in QFT the approach changes: if we adopt the path-integral formulation~\parencite{martinez2002,Weinberg}, the idea of an equation of motion vanishes and we move on to searching for correlations between free particle states. However, the notion of action remains the cornerstone in the description of these observables.

Explicitly, the correlation functions are calculated through the Lehmann-Symanzik-Zimmermann (LSZ) reduction formula, which connects these correlators with physical scattering amplitudes. These are computed from the path integral~\parencite{greiner1996qft,peskin}:
\begin{equation}
	\begin{aligned}
		Z[J]&=\braket{\text { out, } 0| 0, \text { in }}
		\\&=\mathcal{N}\int \mathcal{D}(\varphi, \bar{\varphi})  e^{i S[\varphi]} e^{i \int J_I\varphi^I  d^{4} x}
		\\&=\mathcal{N}\int \mathcal{D}(\varphi, \bar{\varphi})  e^{i \int d^{4} x \mathcal{L}} e^{i \int J_I\varphi^I  d^{4} x},
	\end{aligned}
\end{equation}
taken over the space of fields $\varphi$ with an appropriate measure $\mathcal{D}(\varphi, \bar{\varphi})$ and normalized by $\mathcal{N}$. The quantity $Z$ is known as the partition function of the theory and gives the transition amplitude from the initial vacuum $\ket{0,\text{ in}}$ to the final vacuum $\ket{0,\text{ out}}$ in the presence of a source $J(x)$ producing particles~\parencite{birrell75900}.


\begin{center}
    \begin{tabular}{|l|c|c|l|}\hline\bigstrut
        Name							& Field				& Spin & Free-Lagrangian	\\\hline\hline\bigstrut
        Klein-Gordon				&	$\phi$					& $0$			&	$\lag=\frac{1}{2}\fac{\partial^\mu \phi\partial_\mu \phi-m^2 \phi\phi}$						\\\hline\bigstrut
        Dirac								& $\chi$			& $1/2$	&$\lag=\bar\chi\fac{i\pmb\gamma^\mu \partial_\mu -m\pmb 1}\chi$\\\hline\bigstrut
        Proca (Massive Vector)	        & $A^\mu$ 		& $1$		&$\lag=-\frac{1}{4} F^{\mu\nu} F_{\mu\nu} + \frac{1}{2}m^2 A^\mu A_\mu $\\\hline
    \end{tabular}
	\captionof{table}{Some relevant representations of the Lorentz group in  $4$-dimensional space-time. In this notation $\eta_{\mu\nu}=\diag(1,-1,-1,-1)$, $\pmb \gamma^\mu$ are the Dirac matrices, $F_{\mu \nu}=\partial_{\mu} A_{\nu}-\partial_{\nu} A_{\mu}$ is the abelian field strength tensor. All equations are written in natural units with $c=\hbar=1$. Fields are shown in their standard representations.}\label{tab-repLorentz2}
\end{center}

Therefore, the dynamics, at both the classical and quantum levels, are entirely determined by the Lagrangian density. For free fields (i.e., non-interacting), the Lagrangian is quadratic in the fields and the path integral can be evaluated exactly. Tab.~\ref{tab-repLorentz2} records the Lagrangian density for these free fields. However, to describe physics, we must include interactions, which render the path integral impossible to compute exactly.

The framework of \textit{perturbation theory} addresses this by expanding the interaction part of the Lagrangian as a power series. This expansion is organized using \textit{Feynman diagrams}, which provides a pictorial representation of each term, and a set of \textit{Feynman rules}, which provides a precise dictionary to translate these diagrams into mathematical expressions for scattering amplitudes~\parencite{peskin,Weinberg}. The importance of these rules cannot be overstated, as they are the practical computational tools of perturbative QFT.


In this paradigm, our task is to propose a Lagrangian density for a set of fields that correctly models the propagation and interactions of fundamental particles. The free part defines the particle content and propagators, while the interaction part defines the vertices and possible scattering processes.

\subsection{Interactions and Symmetries} 

The structure of the Lagrangian density in a quantum field theory is not arbitrary; it is constrained by fundamental principles that ensure the theory is physically consistent and mathematically well-defined. These principles act as ``rules'' that guide the construction of viable theories. In what follows, we systematically develop these constraints, starting from the practical requirements of perturbation theory and building up to the fundamental symmetry principles.

To perform calculations, we typically split the Lagrangian into a free part, which describes non-interacting fields, and an interaction part:
\begin{equation}
    \mathcal{L} = \mathcal{L}_0 + \mathcal{L}_{\text{int}}.
\end{equation}
This splitting is the starting point for perturbation theory. In the path integral formulation, the generating functional $Z[J]$ can then be expressed as an operator acting on the free functional $Z_0[J]$:
\begin{equation}
    Z[J] = \mathcal{N} \exp\left[i \int d^4x\, \mathcal{L}_{\text{int}}\left(-i \frac{\delta}{\delta J(x)}\right)\right] Z_0[J].
\end{equation}
The exponential operator generates an infinite series known as the perturbation series. The $n$-point correlation function is found by taking functional derivatives of $Z[J]$ with respect to the sources $J(x_i)$ and setting $J=0$. Each term in this series is represented by a \textbf{Feynman diagram}, whose components are:

\begin{itemize}
    \item \textbf{External Lines:} Represent incoming and outgoing physical particles.
    \item \textbf{Internal Lines:} Represent virtual particles propagating between interactions, corresponding to the free-field propagators derived from $\mathcal{L}_0$.
    \item \textbf{Vertices:} Represent interactions, derived from the terms in $\mathcal{L}_{\text{int}}$. Each vertex has an associated coupling constant and enforces momentum conservation.
\end{itemize}

\begin{figure}[h!]
    \centering
    \begin{fmffile}{feyngraphs/feyngraph0}
        \vspace{0.5cm}
        \begin{fmfgraph*}(120,80)
            \fmfleft{i1,i2}
            \fmfright{o1,o2}
            
            % External incoming lines
            \fmf{fermion}{i1,v1}
            \fmf{fermion}{i2,v2}
            
            % Internal propagator
            \fmf{photon,label=$\gamma$,label.side=left}{v1,v2}
            
            % External outgoing lines
            \fmf{fermion}{v1,o1}
            \fmf{fermion}{v2,o2}
            
            % Labels for external particles
            \fmflabel{$e^-$}{i1}
            \fmflabel{$e^+$}{i2}
            \fmflabel{$e^-$}{o1}
            \fmflabel{$e^+$}{o2}
            
            % Vertex labels
            \fmfv{label=$v_1$,label.angle=180,label.dist=0.3cm}{v1}
            \fmfv{label=$v_2$,label.angle=0,label.dist=0.3cm}{v2}
        \end{fmfgraph*}
        \vspace{0.5cm}
    \end{fmffile}
    \caption{Example of a Feynman diagram for $e^+e^- \to e^+e^-$ scattering. \textbf{External lines} (solid arrows at the edges) represent the incoming and outgoing electrons and positrons. The \textbf{internal line} (wavy line) represents the virtual photon propagator. The \textbf{vertices} ($v_1$ and $v_2$) represent the electromagnetic interaction points where the coupling constant $e$ (electric charge) enters and momentum is conserved.}
    \label{fig:feynman-components}
\end{figure}

For this perturbation series to be a predictive computational tool, it must yield finite physical results. However, individual terms in the series (i.e., individual Feynman diagrams) often lead to divergent integrals when loop corrections are included. The key is that in a \textit{renormalizable} theory, these divergences from all diagrams can be systematically absorbed into a finite number of parameters (like masses and coupling constants) through a redefinition procedure known as renormalization. It is important to note that while individual Feynman diagrams may diverge, the requirement is that the combination of all contributions at a given order yields finite, physically meaningful results after renormalization.

This requirement of renormalizability imposes a powerful constraint on the form of $\mathcal{L}_{\text{int}}$. Through power-counting arguments, one finds that only operators of mass dimension $\leq 4$ lead to renormalizable interactions. In natural units, where $\mathcal{L}$ has dimension $[\text{mass}]^4$, this means that $\mathcal{L}_{\text{int}}$ can be expressed as a truncated polynomial containing only terms up to dimension 4. Specifically, this allows Yukawa couplings (dim 4), quartic scalar interactions (dim 4), and gauge interactions (dim 4), while forbidding non-renormalizable operators like $\phi^6$ (dim 6). Higher-dimensional operators are still allowed in effective field theories, but they correspond to interactions that are suppressed at low energies and signal the presence of new physics at higher scales~\parencite{peskin,Weinberg}.

This is why we express $\mathcal{L}_{\text{int}}$ as a truncated polynomial: renormalizability demands that we include only a finite set of operators with dimension $\leq 4$, ensuring that the theory remains predictive at all accessible energy scales.

An additional crucial requirement is the \textit{stability of the vacuum}. For a theory to be physically meaningful, it must possess a stable ground state. This is ensured by demanding that the scalar potential, which governs the self-interactions of scalar fields, is bounded from below. If the potential were unbounded, the system could lower its energy indefinitely by evolving toward field configurations of ever-greater magnitude, meaning no stable vacuum would exist.

For a renormalizable theory, the scalar potential can contain at most quartic terms. A general scalar potential for a set of scalar fields $\{\phi_i\}$ takes the form:
\begin{equation}
    V(\phi_i) = \sum_i \mu_i^2 |\phi_i|^2 + \sum_{i,j} \lambda_{ij} |\phi_i|^2 |\phi_j|^2 + \cdots
\end{equation}
where the ellipsis denotes possible cubic and mixed quartic terms allowed by the symmetries of the theory. The stability condition requires that the quartic couplings $\lambda_{ij}$ satisfy certain positivity constraints to ensure that $V \to +\infty$ as $|\phi_i| \to \infty$ in any direction in field space. This is why the scalar potential is a polynomial of at most order four: renormalizability forbids higher-order terms, and stability demands that the quartic terms dominate at large field values with the correct sign.

It is important to note that this condition must hold not just at tree-level but also at the quantum level, as running couplings can change sign at different energy scales, potentially leading to metastability or instability of the vacuum.

A fundamental requirement from quantum mechanics is \textit{Hermiticity}: the Lagrangian density must be Hermitian to ensure that observables are real and the time evolution of the theory is unitary~\parencite{pall,peskin}. This is the most basic constraint that quantum theory imposes on the Lagrangian. Without Hermiticity, the theory would predict complex-valued probabilities and violate the fundamental probabilistic interpretation of quantum mechanics.

Beyond the quantum mechanical requirement of Hermiticity, special relativity imposes a fundamental constraint: \textit{Poincaré invariance}. This symmetry demands that the equations of motion remain the same in all inertial frames. Mathematically, this is implemented by requiring the action to be globally invariant under Poincaré transformations~\parencite{pall}. Equivalently, the Lagrangian density must transform as a Lorentz scalar and may change under translations at most by a total derivative~\parencite{jose1998classical}.\marginpar{\footnotesize In QFT, Poincaré invariance is assumed to be global. Promoting it to a local symmetry leads to gravity, with spin-2 fields (the graviton) as mediators. Perturbatively, such a theory is not renormalizable, so it lacks predictivity at high energies, although it can still be understood as an effective field theory.}

This constraint is extremely powerful: it eliminates all possible interaction terms that would depend on the choice of reference frame. For instance, terms that explicitly depend on spacetime coordinates or preferred directions are forbidden. Furthermore, \textit{dimensional analysis} places additional restrictions. In natural units, $\mathcal{L}$ carries dimensions of mass to the fourth power ($[\mathcal{L}] = [\text{mass}]^4$), which corresponds with an energy density. Combined with Lorentz invariance, this means that the interaction terms must be constructed from Lorentz-covariant combinations of fields and their derivatives, with the correct overall mass dimension.

The symmetries discussed so far—Poincaré invariance, Hermiticity, and dimensional analysis—are universal requirements that any relativistic quantum field theory must satisfy. However, they still leave a vast array of possible interaction terms. To further constrain the Lagrangian and to describe the fundamental forces of nature, we must consider \textit{internal symmetries}: transformations that act on the fields' internal degrees of freedom rather than on spacetime coordinates.

Internal symmetries can be either \textit{global} (where the transformation parameters are constant throughout spacetime) or \textit{local} (gauge symmetries, where the parameters can vary from point to point). The procedure for constructing gauge theories—where global symmetries are ``promoted'' to local ones by introducing gauge fields—is systematic and will be described in detail below. This gauge principle has proven to be the most powerful organizing principle in particle physics, determining not only which interactions are realized in nature but also their precise mathematical structure.

A classical symmetry of the Lagrangian may not always survive the process of quantization. If it fails to do so, it is said to be anomalous. \textit{Chiral anomalies}, specifically, arise from the regularization of fermion loops in triangle diagrams and can break gauge symmetries at the quantum level. Since gauge symmetry is the very principle that dictates the form of interactions and removes unphysical states, its violation would destroy the renormalizability and unitarity of the theory. Therefore, the particle content must be carefully chosen so that these potential anomalies cancel among fermions, a non-trivial condition famously satisfied by the quarks and leptons of the Standard Model~\parencite{peskin,Weinberg,bertlmann1996anomalies}.

In summary, the construction of a consistent relativistic quantum field theory proceeds through a hierarchy of constraints:
\begin{enumerate}
	\item \textbf{Perturbative renormalizability:} power-counting arguments restrict operators to mass dimension $\leq 4$, ensuring $\mathcal{L}_{\text{int}}$ is a truncated polynomial.
	\item \textbf{Vacuum stability:} the scalar potential must be bounded from below, requiring appropriate positivity conditions on quartic couplings.
	\item \textbf{Hermiticity:} quantum mechanics demands $\mathcal{L}$ be Hermitian for real observables and unitary evolution.
	\item \textbf{Poincaré invariance:} special relativity requires the action to be invariant under Lorentz transformations and translations, eliminating frame-dependent terms.
	\item \textbf{Internal symmetries:} global and gauge symmetries further constrain the form of interactions and determine the structure of fundamental forces.
	\item \textbf{Anomaly cancellation:} the particle content must be chosen such that chiral anomalies cancel, preserving gauge symmetry at the quantum level.
\end{enumerate}

These constraints drastically reduce the number of possible terms in the Lagrangian. The renormalizable interaction structures that survive are limited to: Yukawa couplings between fermions and scalars, quartic scalar self-interactions, and gauge interactions between matter fields and vector bosons. The precise form of these interactions is then determined by the internal (gauge) symmetries of the theory, which we now describe in detail.

The procedure is systematic: first, the spin$-0$ and spin$-1/2$ fields are organized into representations of a unitary (gauge) group $G$, such that the Lagrangian density is globally invariant under $G$. This global symmetry is then ``promoted'' to a \textit{local symmetry} (where the group parameters can vary in spacetime) by replacing the ordinary derivatives $\partial_\mu$ with \textit{covariant derivatives} $\Dcov_\mu$ that incorporate new \textit{gauge fields} $B_\mu^A$~\parencite{pokorski2000gauge,freedman2012supergravity, Gallego2016,VanProeyen1999,Martin2012}.
This ``promotion'' is described in more detail below.

Given a Lagrangian density $\lag(\varphi^I, \partial_\mu \varphi^I)$, where $I$ is an index enumerating the different fields $\varphi^{I}$ in the model, it is said to be \textit{globally symmetric} under unitary transformations if the action remains invariant under field variations. At infinitesimal level, these variations are given by:
\begin{equation}
	\delta_G \varphi^I = i\theta^A (T_A)^I_J \varphi^J,
\end{equation}
where $\theta^{A}$ are constant parameters of the transformation and the $T_{A}$ are the generators of the group $G$ in the appropriate representation. The corresponding finite unitary transformation is
\begin{equation}
	\mathcal{U}_G \equiv U(\theta)=\exp(i\theta^A T_A).
\end{equation}
Note that the $T_A$  generators  satisfy the same Lie algebra of the group $G$:
\begin{equation}
	[T_A, T_B] = i f_{AB}^{\;\;C}T_C,
\end{equation}
where $f_{AB}^{\;\;C}$ are the structure constants of $G$.

To promote the global symmetry to a local one ($\theta^A \to \theta^A(x)$), the ordinary derivative $\partial_\mu$ is replaced by a \textit{covariant derivative} $\Dcov_\mu$. This new derivative is designed to transform covariantly under the gauge group, meaning $\Dcov_\mu \varphi \to U(x) (\Dcov_\mu \varphi)$, so that the kinetic terms $\lag_{\text{kin}} \sim (\Dcov_\mu \varphi)^\dagger (\Dcov^\mu \varphi)$ remain invariant. This is achieved by introducing a gauge field $B_\mu^A$ for each generator $T_A$ and defining:
\begin{equation}
	\Dcov_\mu = \partial_\mu - i g B_\mu^A T_A,
\end{equation}
where $g$ is the gauge coupling constant. The transformation law for the gauge fields that ensures the covariant transformation of $\Dcov_\mu$ is:
\begin{equation}
	\delta B_\mu^A = \partial_\mu \theta^A + g f_{BC}{}^A \theta^B B_\mu^C.\label{eq:gauge-transformation}
\end{equation}

The introduction of the gauge fields $B_\mu^A$ requires the addition of a kinetic term for them to the Lagrangian. This is constructed from the \textit{field strength tensor} $F_{\mu\nu}^A$, defined as the curvature of the covariant derivative:
\begin{equation}
	F_{\mu\nu}^A T_A = -\frac{i}{g} [\Dcov_\mu, \Dcov_\nu] = \partial_\mu B^A_\nu - \partial_\nu B^A_\mu + g f_{BC}{}^A B^B_\mu B^C_\nu.
\end{equation}
The gauge-invariant kinetic Lagrangian is then:
\begin{equation}
	\lag_{\text{gauge}} = -\frac{1}{4} \delta_{AB} F^A_{\mu\nu} F^{\mu\nu B}.
\end{equation}
Often, the rescaling $B_\mu^A \to g B_\mu^A$ is performed, which moves the coupling constant $g$ from the kinetic term to the covariant derivative, resulting in the more conventional form $\Dcov_\mu = \partial_\mu - i g B_\mu^A T_A$ and $\lag_{\text{gauge}} = -\frac{1}{4g^2} \delta_{AB} F^A_{\mu\nu} F^{\mu\nu B}$.


A general, archetypal Lagrangian, embodying these structures, can be written as:
\begin{equation}\label{eq:generic-renorm-lag}
	\mathcal{L} = -\frac{1}{4} F_{\mu \nu}^A F^{A \mu \nu} + i \bar{\psi}^i \gamma^\mu \mathcal{D}_\mu \psi^i + \left(\bar{\psi}_L^j \, \Gamma^j_k \, \Phi \, \psi_R^k + \text{h.c.}\right) + |\mathcal{D}_\mu \Phi|^2 - V(\Phi)
\end{equation}
The terms correspond to: the kinetic term for gauge fields ($F_{\mu \nu}^A$), the kinetic term for fermions $\psi^i$, the Yukawa interactions between left- and right-handed fermions and scalars ($\Gamma^j_k$ is a Yukawa coupling matrix and $\Phi$ is a scalar field), the kinetic term for scalars, and the scalar potential $V(\Phi)$. For a renormalizable and  stable theory $V(\Phi) = \mu^2 |\Phi|^2 + \lambda |\Phi|^4$ with $\lambda > 0$.

Note the absence of explicit mass terms for the gauge fields ($\sim M^2 B_\mu B^\mu$) and fermions ($\sim m \bar{\psi}\psi$). These are forbidden by gauge invariance and for chiral fermions. Mass terms can be generated via spontaneous symmetry breaking, as discussed below.

It is important to emphasize that while the Yukawa interactions $\bar{\psi}_L^j \, \Gamma^j_k \, \Phi \, \psi_R^k$ do not involve gauge bosons directly, their structure is nonetheless \textit{completely determined by the gauge symmetry}. Specifically, gauge invariance dictates which fermion fields can couple to which scalar fields, and constrains the form of the coupling matrix $\Gamma^j_k$. For a Yukawa term to be gauge-invariant, the product $\bar{\psi}_L^j \, \Phi \, \psi_R^k$ must be a singlet under the gauge group. This requirement arises because the left-handed and right-handed fermions typically transform in different representations of the gauge group, and the scalar field $\Phi$ must carry the appropriate quantum numbers to make the overall combination invariant. In the Standard Model, for instance, the left-handed fermions are $SU(2)_L$ doublets while the right-handed fermions are singlets, and the Higgs doublet provides the necessary quantum numbers to form gauge-invariant Yukawa couplings. Thus, even though Yukawa interactions are scalar-mediated rather than gauge-mediated, the gauge principle is the fundamental organizing principle that determines their allowed structure.

\subsubsection{Example}
To illustrate these concepts, let us consider a renormalizable theory with a real scalar $\phi$ and a Dirac spinor $\psi$, and suppose that this theory is globally invariant under $U(1)$ phase transformations, i.e. the fields $\varphi\in\{\phi,\psi\}$ transform as $\varphi\mapsto e^{i\theta \hat Q}\varphi $ such that $\hat Q \psi = q \psi$ and $\hat Q \phi=0$. The free Lagrangian density is
\begin{equation}
	\mathcal L_{\text{free}}=\frac{1}{2} \partial^{\mu} \phi \partial_{\mu} \phi-\frac{1}{2}\mu^2\phi^2+\bar{\psi}(i \gamma_\mu  \partial^\mu-m) \psi.
\end{equation}
\marginpar{\footnotesize Note that for this vector-like $U(1)$ theory, the explicit fermion mass term $m\bar{\psi}\psi$ is gauge-invariant. This will not be the case for chiral gauge theories like the Standard Model.}

To add globally symmetric interaction terms, we must consider operators of mass dimension $\leq 4$. The most general renormalizable Lagrangian, invariant under the global $U(1)$ symmetry, is
\begin{equation}
	\begin{aligned}
		\mathcal L_{\text{global}}&=\frac{1}{2} \partial^{\mu} \phi \partial_{\mu} \phi-V(\phi)+\bar{\psi}(i \gamma_\mu  \partial^\mu-m) \psi + k_1 \phi\bar\psi\psi,
		\\
		V(\phi)&=\frac{1}{2}\mu^2\phi^2 +\frac{\alpha}{3!}\phi^3+\frac{\lambda}{4!}\phi^4.
	\end{aligned}
\end{equation}
The cubic and quartic terms in $V(\phi)$ are allowed as $\phi$ is neutral. The Yukawa coupling $k_1 \phi\bar\psi\psi$ is also gauge-invariant since the charges of $\bar\psi$, $\phi$, and $\psi$ sum to zero ($-q + 0 + q = 0$).

Promoting the global symmetry to a local one ($\theta \to \theta(x)$) requires introducing a gauge field $A_\mu$ and replacing ordinary derivatives with covariant derivatives:
\begin{equation}
	\mathcal D_\mu\varphi=(\partial_{\mu}-i g A_\mu\hat Q )\varphi
	\quad\Longrightarrow\quad
	\begin{cases}
		\mathcal D_\mu\phi=\partial_\mu \phi, & (\text{since } \hat Q\phi=0)\\
		\mathcal D_\mu\psi=(\partial_\mu - i g q A_\mu) \psi.
	\end{cases}
\end{equation}
The field strength tensor for the abelian $U(1)$ field is defined as $F_{\mu\nu} = \partial_\mu A_\nu - \partial_\nu A_\mu$. The locally invariant Lagrangian is then:
\begin{multline}
	\mathcal L_{\text{local}}=\frac{1}{2} \mathcal D^{\mu} \phi \mathcal D_{\mu} \phi-V(\phi)
	+\bar{\psi}i \gamma_\mu  \mathcal D^{\mu} \psi - m \bar{\psi}\psi
	+ k_1 \phi\bar\psi\psi-\frac{1}{4} F_{\mu\nu}F^{\mu\nu}.
\end{multline}


With these ingredients and principles, we are now equipped to understand the structure of the SM Lagrangian, which will be discussed in the next section.

\begin{figure}[h!]
    \centering
    \begin{subfigure}[b]{0.48\textwidth}
        \centering
        \begin{fmffile}{feyngraphs/feyngraph1} 
			\vspace{0.5cm}
            \begin{fmfgraph*}(80,60)
                \fmfleft{i1}
                \fmfright{o1,o2}
                
                \fmf{dashes,tension=2.0}{i1,v1}
                \fmf{fermion}{o1,v1}
                \fmf{fermion}{v1,o2}

                \fmflabel{$\phi$}{i1}
                \fmflabel{$\bar\psi$}{o1}
                \fmflabel{$\psi$}{o2}
            \end{fmfgraph*}
			\vspace{0.5cm}
        \end{fmffile}
        \caption{Yukawa coupling with a scalar $\phi$.}
        \label{fig-yukawa-scalar}
    \end{subfigure}
    \hfill
    \begin{subfigure}[b]{0.48\textwidth}
        \centering
        \begin{fmffile}{feyngraphs/feyngraph2}
			\vspace{0.5cm}
            \begin{fmfgraph*}(80,60)
                \fmfleft{i1}
                \fmfright{o1,o2}
                
                \fmf{photon,tension=2.0}{i1,v1}
                \fmf{fermion}{o1,v1}
                \fmf{fermion}{v1,o2}

                \fmflabel{$\gamma$}{i1}
                \fmflabel{$\bar\psi$}{o1}
                \fmflabel{$\psi$}{o2}
            \end{fmfgraph*}
			\vspace{0.5cm}
        \end{fmffile}
        \caption{Interaction with a photon $\gamma$.}
        \label{fig-qed-photon}
    \end{subfigure}
	\begin{subfigure}[b]{0.48\textwidth}
        \centering
		\begin{fmffile}{feyngraphs/feyngraph3}
			\vspace{1.0cm}
			\begin{fmfgraph*}(80,60)
				\fmfleft{i1}
				\fmfright{o1,o2}

				\fmf{dashes}{i1,v1}
				\fmf{dashes}{v1,o1}
				\fmf{dashes}{v1,o2}

				\fmflabel{$\phi$}{i1}
				\fmflabel{$\phi$}{o1}
				\fmflabel{$\phi$}{o2}
			\end{fmfgraph*}
			\vspace{0.5cm}
		\end{fmffile}
		\caption{Triple scalar coupling.}
		\label{fig-triple-scalar}
	\end{subfigure}
	\begin{subfigure}[b]{0.48\textwidth}
        \centering
		\begin{fmffile}{feyngraphs/feyngraph4}
			\vspace{1.0cm}
			\begin{fmfgraph*}(80,60)
				\fmfleft{i1,i2}
				\fmfright{o1,o2}

				\fmf{dashes}{i1,v1}
				\fmf{dashes}{i2,v1}
				\fmf{dashes}{v1,o1}
				\fmf{dashes}{v1,o2}

				\fmflabel{$\phi$}{i1}
				\fmflabel{$\phi$}{i2}
				\fmflabel{$\phi$}{o1}
				\fmflabel{$\phi$}{o2}
			\end{fmfgraph*}
			\vspace{0.5cm}
		\end{fmffile}
		\caption{Quartic scalar coupling.}
		\label{fig-quartic-scalar}
	\end{subfigure}
    \caption{Feynman diagrams for Yukawa coupling, gauge boson coupling and quartic scalar coupling.}
\end{figure}
 % Fields
\section{Standard Model}

{$ $ \scriptsize \hfill Fragment extracted and adapted from~\parencite{robinson2011symmetry}}

In 1965, Tomonaga, Feynman, and Schwinger were awarded the Nobel Prize for their independent formulation of Quantum Electrodynamics (QED)~\parencite{1972physics}. Their work established renormalization as a consistent method to separate infinities from finite, physically meaningful results in quantum field theory. QED provided predictions, such as the anomalous magnetic moment of the electron, that later experiments confirmed with remarkable precision~\parencite{1674-1137-40-10-100001, PhysRev.75.486}. It became the prototypical example of a successful quantum field theory.

This success, however, did not extend to other fundamental interactions. The weak interaction was described by the chiral $V-A$ model, in which processes like beta decay were represented by four-fermion contact terms. This framework was not renormalizable: divergences could not be absorbed into a finite set of parameters, restricting its validity to low energies. A fundamental description within the quantum field theory framework was still missing.

The issue was linked to the short-range character of the weak and strong forces. In quantum field theory, the range of an interaction depends on the mass of its mediating boson. A massless boson, such as the photon, generates a long-range force with an inverse-square dependence. A massive boson, in contrast, produces a Yukawa potential of the form $\exp(-mr)/r$, which falls off rapidly with distance. A consistent theory of the weak interaction therefore required massive gauge bosons.

Here lay the apparent obstacle. A mass term for a gauge boson, such as $m_{A}^{2} A_{\mu} A^{\mu}$ in the Lagrangian, explicitly breaks gauge invariance, since it is not preserved under the transformation $A_{\mu} \mapsto A_{\mu} + \partial_{\mu}\epsilon$. This seemed to rule out gauge theories as candidates for describing short-range forces. The problem was recognized early on. For instance, in a 1954 seminar where Chen Ning Yang introduced non-Abelian gauge theories, Wolfgang Pauli objected that assigning masses to the gauge bosons would violate gauge invariance, and without such masses the theory could not describe nuclear forces. This skepticism reflected a widely shared view: gauge symmetry appeared incompatible with short-range interactions.

The resolution of this problem came from two developments that allowed gauge bosons to behave as if they had mass, without explicitly breaking gauge symmetry:
\begin{enumerate}
    \item \label{list:sol_mass_1} The Higgs mechanism. In this framework, a scalar field permeates the vacuum. While the underlying Lagrangian remains gauge invariant, the vacuum state does not respect this symmetry. Gauge bosons interacting with this vacuum acquire mass in a renormalizable way. This mechanism explains the masses of the $W$ and $Z$ bosons.
    \item \label{list:sol_mass_2} Dynamical mass generation in non-Abelian gauge theories. In Quantum Chromodynamics (QCD), gluons and nearly massless quarks are confined into hadrons with substantial masses. The appearance of a mass gap is a nonperturbative consequence of confinement. Understanding this mechanism in a rigorous way is at the core of the Yang–Mills existence and mass gap Millennium Prize problem.
\end{enumerate}

The Standard Model (SM) incorporates both solutions. Electroweak theory relies on the Higgs mechanism (\ref{list:sol_mass_1}), which provides a renormalizable description of the weak interaction. For the strong interaction, QCD employs dynamical mass generation (\ref{list:sol_mass_2}), where most of the mass of hadrons arises from confinement rather than from the small quark masses introduced by the Higgs field.

\subsection{Particle Content and Gauge Group}

First, let us talk about the chiral nature of particles: Massive half-spin particles are described at the fundamental level by a Dirac spinorial field, see table \ref{tab-repLorentz2}. However, Dirac spinors do not transform under an irreducible representation of the Lorentz group. Spinors can be decomposed into two components that do transform under irreducible representations of the Lorentz group: two \textit{Weyl spinors}. The left and right chiral projectors, $P_L$ and $P_R$, take a Dirac spinor and project it onto each of these invariant subspaces. For a massless Dirac spinor, the left and right components are dynamically decoupled, \textit{i.e.} which are independent fields obeying independent Lagrangian densities; for example, the left component of a massless spinor has the Lagrangian $\lag=-i\bar\psi\slashed{\partial}P_L\psi$ (For more details see Appendix A at~\parencite{CRodriguezUPTC}). 

The discovery of parity asymmetry in radioactive decays~\parencite{PhysRev.105.1413} indicates that the chiral description of weak interactions couples differently to the left and right chiral components of half-spin particles. Indeed, the chirality of the fermionic spectrum is possibly one of the deepest properties of the Standard Model. Describing particles in terms of Dirac spinors, it means that left- and right-chirality components actually have different EW quantum numbers. This is compatible with a gauge symmetry only if half-spin particles are considered to be massless, at least without a Dirac mass $m \overline{f_{R}} f_{L}+\text { h.c.}$ Nevertheless, half-integer spin fundamental particles, such as the electron, have a well-measured mass. Therefore, the reconciliation of chiral asymmetry and mass lies in the Higgs mechanism, where the masses of the particles result from an effective Yukawa coupling with a scalar, the Higgs boson.

With this in mind, the SM has a content of matter fields from three generations (or families) of quarks $q$ and leptons $\ell$, described as Weyl 2-component spinors, with the structure
\begin{equation}
	q_{L}=\left(
		\begin{array}{c}
			u_{L}^{i} \\
			d_{L}^{i}
		\end{array}
	\right), 
	u_{R}^{i}, d_{R}^{i}, 
	\quad \ell_L=\left(
		\begin{array}{c}
			\nu_{L}^{i} \\
			e_{L}^{i}
		\end{array}
	\right), e_{R}^{i} ; \quad i=1,2,3 .
\end{equation}
All these particles transform under a group $U$(1) with different associated (hyper)charges.
The doublets formed by the left components of the fields transform under the representation of two components of a $SU$(2) group. The right components do not transform under SU(2), therefore they are singlets.
In addition, each quark in $q_{L}$ transforms as color triplets under $SU$(3), while $u_{R}, d_{R}$ transforms as conjugate triplets. Leptons, on the other hand, turn out to be colored singlets.
Gauge quantum numbers of the Standard Model fermions are shown in table \ref{tab_qm}.

\begin{center}
	$$
	\begin{array}{|l||c|c|c||c|}
		\hline \text {\textbf{Field} } & S U(3)_C & S U(2)_{L} & U(1)_{Y} & U(1)_{EM} \bigstrut\\
		\hline q_{L}^{i}=\left(u^{i}, d^{i}\right)_{L} & \mathbf{3} & \mathbf{2} & +1 / 3 & (2/3,-1/3) \bigstrut\\
		u_{R}^{i} & \overline{\mathbf{3}} & \mathbf{1} & +4 / 3 & +2/3 \bigstrut\\
		d_{R}^{i} & \overline{\mathbf{3}} & \mathbf{1} & -2 / 3 & -1/3 \bigstrut\\
		\ell^{i}_L=\left(\nu^{i}, e^{i}\right)_{L} & \mathbf{1} & \mathbf{2} & -1  & (0,-1)\bigstrut\\
		e_{R}^{i} & \mathbf{1} & \mathbf{1} & -2 & -1 \bigstrut\\
		H=\left(H^{+}, H^{0}\right) & \mathbf{1} & \mathbf{2} & +1 & (+1,0) \bigstrut\\
		\hline \hline
	\end{array}
	$$
	\captionof{table}{Gauge quantum numbers of Standard Model quarks, leptons
		and the Higgs scalar.}\label{tab_qm}
\end{center}

Then, we consider the Standard Model as a quantum field theory based on a gauge group
\begin{equation}
	G_{\mathrm{SM}}=S U(3)_C \times S U(2)_{L} \times U(1)_{Y},
\end{equation}
with $S U(3)_C$ describing strong interactions via Quantum Chromodynamics (QCD), and $S U(2)_{L} \times U(1)_{Y}$ describing electroweak (EW) interactions. Gauge vector bosons that result from taking this group locally are eight gluons ($G^a$) from each $t^a$ color-generator of $SU(3)_C$, and a linear combination of the three ($W^\pm, Z$) weak bosons and the ($\gamma$) electromagnetic photon from the three $T^i$ isospin-generators of $SU(2)_L$ and $Y$ hyper-charge-generator of $U(1)_Y$.

Electroweak symmetry is spontaneously broken into electromagnetic symmetry $U(1)_{EM}$ via the Higgs mechanism and the Higgs boson $H$. The hypercharges $Y$ of the Standard Model fermions in table \ref{tab_qm} are related to their usual electric charges by the Gell-Mann–Nishijima relation~\parencite{10.1143/PTP.10.581} 
\begin{equation}
	Q_{\mathrm{EM}}=\frac12Y+T_{3}, \label{eq:Gell-Mann-Nishijima}
\end{equation}
where $T_{3}\dot=\operatorname{diag}\left(\frac{1}{2},-\frac{1}{2}\right)$ is an $S U(2)_{L}$ generator.  Thus, they reproduce electric charge quantization, e.g. the equality in magnitude of the proton and electron charges. Although these hypercharge assignments look rather ad hoc, their values are dictated by the quantum consistency of the theory.\marginpar{It is indeed easy to check that these are (module an irrelevant overall normalization) the only (family independent) assignments canceling all potential triangle gauge anomalies.}

\subsection{Gauge Bosons}

The Lie algebra of the gauge group $SU(3)\times SU(2)\times U(1)$ is
\begin{equation}
\begin{aligned}
	{\left[t^{a}, t^{b}\right] } &=i f^{a b c} t_{c}, \\
	{\left[T^{i}, T^{j}\right] } &=i \epsilon^{i j k} T_{k}, \\
	{\left[T^{i}, \, Y\;\right] } &=\left[t^{a}, T^{j}\right]=\left[t^{a}, Y\right]=0,
\end{aligned}
\end{equation}
where $f^{a b c}$ and $\epsilon^{i j k}$ are the structure constants of $SU(3)$ and $SU(2)$. And therefore, the gauge fields $G_\mu$, $W_\mu$, and $B_\mu$ must transform in the adjoint representation: 
\begin{equation}
	\begin{aligned}
		\delta B_{\mu} &=\partial_{\mu} \theta, \\
		\delta W_{\mu}^{i} &=\partial_{\mu} \theta^{i}-g \epsilon^{i j k} \theta^{j} W_{\mu}^{k}, \\
		\delta G_{\mu}^{a} &=\partial_{\mu} \epsilon^{a}-g_{s} f^{a b c} \epsilon^{b} G_{\mu}^{c}.
	\end{aligned}
\end{equation}
Then, the curvature strength tensors are
\begin{equation}
\begin{aligned}
	G_{\mu \nu}^{a} &=\partial_{\mu} G_{\nu}^{a}-\partial_{\nu} G_{\mu}^{a}+g_{s} f^{a b c} G_{\mu}^{b} G_{\nu}^{c} \\
	W_{\mu \nu}^{i} &=\partial_{\mu} W_{\nu}^{i}-\partial_{\nu} W_{\mu}^{i}+g \epsilon^{i j k} W_{\mu}^{j} W_{\nu}^{k} \\
	B_{\mu \nu} &=\partial_{\mu} B_{\nu}-\partial_{\nu} B_{\mu}
\end{aligned}
\end{equation}
and the ``kinetic'' term for gauge fields in the Lagrangian is  
\begin{equation}
\mathcal{L}_{\text{Gauge}}=-\frac{1}{4} G_{\mu \nu}^{a} G_{a}^{\mu \nu}-\frac{1}{4} W_{\mu \nu}^{i} W_{i}^{\mu \nu}-\frac{1}{4} B_{\mu \nu} B^{\mu \nu}.
\end{equation}
while these kinetic terms induce vertices between gauge bosons and in turn do not take into account the masses for such vector bosons, the Higgs mechanism produces the masses for them and gives us the linear combination to the physical bosons $W^\pm$, $Z$, $\gamma$:
\begin{equation}
\begin{cases}
	\begin{aligned}
		W_{\mu}^{+} &=\frac{1}{\sqrt{2}}\left(W_{\mu}^{1}-i W_{\mu}^{2}\right) \\
		W_{\mu}^{-} &=\frac{1}{\sqrt{2}}\left(W_{\mu}^{1}+i W_{\mu}^{2}\right) \\
		Z_{\mu} &=c_{w} W_{\mu}^{3}-s_{w} B_{\mu} \\
		A_{\mu} &=s_{w} W_{\mu}^{3}+c_{w} B_{\mu}
	\end{aligned}
\end{cases}
\text{where}
\;
\begin{cases}
	s_{w}=\sin \theta_{w}=\dfrac{g}{\sqrt{g^{2}+g{\prime2}}}\\
	c_{w}=\cos \theta_{w}=\dfrac{g^\prime}{\sqrt{g^{2}+g{\prime2}}}
\end{cases}
\end{equation}
where to avoid confusion with Dirac matrices, we denote as $A_\mu$ the electromagnetic potential.
%TO DO -> Feynmann diagrams 
\subsection{Matter Fields}
We refer to the fermionic fields of the SM as the matter fields. We distinguish fermions in these two categories: leptons, fermions that do not have strong interaction, and quarks that interact both strongly and electroweakly. In table \ref{tab-generations}, we can see that there are six leptons, three charged and three neutral: each charged lepton has an associated neutrino forming between them doublets of $SU(2)_L$ and similarly for quarks. 

According to the SM, there are three generations of fermions. Each generation contains a doublet of leptons and a doublet of quarks. Among generations, particles differ by their flavour quantum number and mass, but their strong and electrical interactions are identical. Moreover, the flavour quantum number is a quantity conserved by all interactions except for the weak interaction.  Each generation is more massive than the previous one. The second and third generations are unstable and they disintegrate into the first generation. This is why ordinary matter is composed of the first generation. All three generations are produced in nuclear reactors, colliders, and cosmic rays. 

%TO DO -> Adjust to the margin
\begin{center}
	{\small
	\begin{tabular}{|c||c||l|l|l|}
		\hline \multicolumn{2}{|c||}{ \textbf{Fermion categories} } & \multicolumn{3}{c|}{\textbf{ Elementary particle generation} } \bigstrut\\
		\hline \hline Type & Subtype & First & Second & Third \bigstrut\\
		\hline\hline \multirow{2}{*}{ Quarks ($q$) }  & up-type & ($u$) up & ($c$) charm & ($t$) top  \bigstrut \\
		\cline { 2 - 5 }  & down-type & ($d$) down & ($s$) strange & ($b$) bottom  \bigstrut\\
		\hline\hline \multirow{2}{*}{ Leptons ($\ell$) } & charged & ($e$) electron & ($\mu$) muon & ($\tau$) tauon \bigstrut\\
		\cline { 2 - 5 } & neutrino & ($\nu_e$) & ($\nu_\mu$) & ($\nu_\tau$) \bigstrut\\
		\hline
	\end{tabular}
	}
	\captionof{table}{Three generations of fermions according to the Standard Model of particle physics. Each generation containing two types of leptons and two types of quarks.}\label{tab-generations}
\end{center}

Under all the constraints on local gauge invariance and renormalizability of the theory, the fermionic Lagrangian for SM is given by
\begin{equation}
	\mathcal{L}_{\mathrm{Fer}}
	=i \bar{\ell}_{L}^j \slashed{\mathcal D} \ell_{L}^j
	+i \bar{e}_{R}^j \slashed{\mathcal D} e_{R}^j
	+i{\bar{q}}_{L}^j  \slashed{\mathcal D}  q_{L}^j
	+i{\bar{u}}_{R}^j  \slashed{\mathcal D}  u_{R}^j
	+i{\bar{d}}_{R}^j  \slashed{\mathcal D}  d_{R}^j
\end{equation}
where $\slashed{\mathcal D}\equiv \gamma ^\mu \mathcal D_\mu$ with covariant derivative
\begin{equation}
	\mathcal D_\mu = \partial_\mu -ig_st_ aG^a_\mu -ig T_i W_\mu^i -ig'\frac Y2 B_\mu,
\end{equation}
and gauge fields $G^a$, $W^i$, and $B$ acting on each kind of fermion via
\begin{equation}
\begin{aligned}
	\mathcal D_{ \mu} \ell_L^i &=\fac{\partial_{\mu}-i g T_j W_{\mu}^{j}+i \frac{g^{\prime}}2 B_{\mu}} \ell_L^i \\
	\mathcal D_{ \mu} e_R^i &=\fac{\partial_{\mu} -  i g^{\prime}  B_{\mu}\vph}e_R^i \\
	\mathcal D_{ \mu} q_L^i &=\fac{\partial_{\mu}-i g_{s} t_{a} G_{\mu}^{a}-i g T_j W_{\mu}^{j}-i \frac{g^{\prime}}{6} B_{\mu}} q_L^i \\
	\mathcal D_{ \mu} u_R^i &=\fac{\partial_{\mu} -i g_{s} t_{a} G_{\mu}^{a} - i \frac{2g^{\prime}}3  B_{\mu}}u_R^i \\
	\mathcal D_{ \mu} d_R^i &=\fac{\partial_{\mu} -i g_{s} t_{a} G_{\mu}^{a} + i \frac{g^{\prime}}3  B_{\mu}}d_R^i \\
\end{aligned}
\end{equation}
which couples the fermions to the gauge bosons.
% TO DO -> Feynman Diagrams
\subsection{Electroweak Symmetry Breaking}

In the SM, the electroweak symmetry $SU(2)_{L} \times U(1)_{Y}$ is spontaneously broken down to the electromagnetic $U(1)_{\text{EM}}$ symmetry by a complex scalar Higgs field $H=\left(H^{+}, H^{0}\right)$ transforming as an $SU(2)_{L}$ doublet with hypercharge $+1$. Its dynamics are governed by the Mexican-hat potential:
\begin{equation}
    V(H)=-\mu^{2}|H|^{2}+\lambda|H|^{4} \quad \Rightarrow \quad v^{2} \equiv \langle H^{\dagger} H \rangle = \mu^{2} / 2\lambda.
\end{equation}
The vacuum expectation value (vev) aligns with the electrically neutral component, $\langle H^{0} \rangle = v/\sqrt{2} \simeq 174 \mathrm{GeV}$, generating masses for the weak gauge bosons while preserving $U(1)_{\text{EM}}$.

Fermion masses arise through Yukawa couplings, which represent the most general renormalizable interactions between the Higgs field and the fermion fields:
\begin{equation}
    \mathcal{L}_{\text{Yuk}} = y_{u}^{ij} \bar{q}_{L}^{i} u_{R}^{j} \tilde{H} + y_{d}^{ij} \bar{q}_{L}^{i} d_{R}^{j} H + y_{\ell}^{ij} \bar{\ell}_L^{i} e_{R}^{j} H + \text{h.c.},
\end{equation}
where $\tilde{H} = i\sigma_2 H^*$, and $y_{u}, y_{d}, y_{\ell}$ are arbitrary $3 \times 3$ complex matrices in flavor space. When the Higgs acquires its vev, $\langle H \rangle = (0, v/\sqrt{2})$, these couplings generate Dirac mass terms for the fermions.

The quark mass matrices are proportional to the Yukawa matrices: $M_u = y_u v/\sqrt{2}$, $M_d = y_d v/\sqrt{2}$. Since $y_u$ and $y_d$ are general complex matrices, they cannot be simultaneously diagonalized. The physical quark masses and states are found by performing separate unitary transformations on the left- and right-handed fields:
\begin{equation}
    u_L \to V_L^u u_L, \quad u_R \to V_R^u u_R, \quad d_L \to V_L^d d_L, \quad d_R \to V_R^d d_R,
\end{equation}
such that $V_L^u M_u V_R^{u\dagger} = M_u^{\text{diag}}$ and $V_L^d M_d V_R^{d\dagger} = M_d^{\text{diag}}$ are diagonal with real, positive entries.

This diagonalization procedure has a direct consequence for the charged-current interactions mediated by the $W^{\pm}$ bosons. In the flavor basis the interaction reads
\begin{equation}
    \mathcal{L}_{W} \supset -\frac{g}{\sqrt{2}} (\bar{u}_L, \bar{c}_L, \bar{t}_L) \gamma^\mu W_\mu^+ (d_L, s_L, b_L)^T + \text{h.c.}
\end{equation}
After moving to the mass basis, the left-handed up- and down-type quarks rotate differently ($u_L \to V_L^u u_L$, $d_L \to V_L^d d_L$), and the interaction becomes
\begin{equation}
    \mathcal{L}_{W} \supset -\frac{g}{\sqrt{2}} (\bar{u}_L, \bar{c}_L, \bar{t}_L) \gamma^\mu W_\mu^+ V_{\mathrm{CKM}} (d_L, s_L, b_L)^T + \text{h.c.},
\end{equation}
where the Cabibbo–Kobayashi–Maskawa (CKM) matrix appears as the mismatch between the two rotations:
\begin{equation}
    V_{\mathrm{CKM}} \equiv V_{L}^{u} V_{L}^{d \dagger} = \begin{pmatrix}
        V_{ud} & V_{us} & V_{ub} \\
        V_{cd} & V_{cs} & V_{cb} \\
        V_{td} & V_{ts} & V_{tb}
    \end{pmatrix}.
\end{equation}
This unitary matrix encodes flavor mixing in charged-current weak interactions, and its non-diagonal structure is the origin of all quark flavor-changing processes in the Standard Model.

The situation is different for leptons in the minimal Standard Model without right-handed neutrinos. The charged-lepton mass matrix $M_\ell = y_\ell v/\sqrt{2}$ can be diagonalized by field redefinitions, but since neutrinos are massless in this framework, there is no additional rotation in the neutrino sector. As a result, the charged-current interaction
\begin{equation}
    \mathcal{L}_{W} \supset -\frac{g}{\sqrt{2}} \bar{\nu}_L \gamma^\mu W_\mu^+ \ell_L + \text{h.c.}
\end{equation}
remains diagonal in the mass basis. This implies \textit{Lepton Flavor Universality} (LFU): the electroweak gauge bosons couple to all three lepton families with identical strength. In particular, the $W$ boson couples to each $\bar{\nu}_L \gamma^\mu \ell_L$ current with coefficient $-g/\sqrt{2}$, and the $Z$ boson couplings to $\ell_L$ and $\ell_R$ are flavor-independent because the hypercharge assignments are the same for all families.

LFU means that processes differing only by the lepton flavor, such as leptonic decays or semileptonic transitions, are predicted to occur with the same rates up to well-understood effects: differences in phase space, helicity suppression, lepton-mass dependence, and small radiative corrections. The assumption of LFU is central in the extraction of CKM parameters, since experimental determinations from decays involving electrons, muons, and tau leptons can be consistently combined.

Precision tests of LFU focus on ratios of decay widths or branching fractions where theoretical and experimental uncertainties cancel to a large extent. Agreement with these tests confirms the gauge structure of the Standard Model, while deviations would point to new physics.

The Lagrangian of the scalar sector is
\begin{equation}
	\mathcal{L}_{H}= \mathcal D_{\mu} H^{\dagger} \mathcal D^{\mu} H - V\!\left(H^{\dagger}, H\right),
\end{equation}
with the covariant derivative defined as $\mathcal D_{\mu} H=\left(\partial_{\mu}+i g T_a W_{\mu}^{a}+i g^{\prime} \tfrac{Y}{2} B_{\mu}\right) H$. Substituting the Higgs vacuum expectation value, one obtains
\begin{equation}
	\begin{aligned}
		\mathcal{L}_{\langle H\rangle}
		&=-\frac{1}{8}\left(\begin{array}{ll}
			0 & v
		\end{array}\right)\left(\begin{array}{ll}
			g W_{\mu}^{3}-g' B_{\mu} & g\left(W_{\mu}^{1}-i W_{\mu}^{2}\right)\vph \\
			g\left(W_{\mu}^{1}+i W_{\mu}^{2}\right)&-g W_{\mu}^{3}-g' B_{\mu}\vph
		\end{array}\right)^{2}\left(\begin{array}{l}
			0 \\
			v
		\end{array}\right)
		\\&=
		-\frac{1}{8} v^{2} V_{\mu}^{T}\left(\begin{array}{cccc}
			g^{2} & 0 & 0 & 0 \\
			0 & g^{2} & 0 & 0 \\
			0 & 0 & g^{2} & -g' g \\
			0 & 0 & -g' g & g'^{2}
		\end{array}\right) V^{\mu},
	\end{aligned}
\end{equation} 
where $V_{\mu}^{T}=\left(W_{\mu}^{1}, W_{\mu}^{2}, W_{\mu}^{3}, B_{\mu}\right)$. Diagonalizing this mass matrix yields eigenvalues $0$, $-\tfrac{1}{8} v^{2} g^{2}$, $-\tfrac{1}{8} v^{2} g^{2}$, and $-\tfrac{1}{8} v^{2}\left(g^{2}+g'^{2}\right)$. The massless state corresponds to the photon, the heaviest to the $Z$ boson, and the two degenerate intermediate states to the charged bosons $W^\pm$, which transform under the representation of the unbroken generator $Q_{EM}$. 

This suffices to illustrate how the Standard Model, formulated as a relativistic quantum field theory, describes the interactions of matter fields through the fundamental forces, mediated by vector bosons. The Higgs boson, also part of the Standard Model spectrum, plays the central role in generating masses for the weak bosons, the fermions, and indirectly in distinguishing the photon as the only massless gauge boson of the electroweak sector.

Since its formulation, the Standard Model has been tested extensively and has shown remarkable success, both in explaining existing data and in making accurate predictions. A well-known example is the agreement between the Standard Model prediction and the experimental measurement of the electron magnetic dipole moment, consistent to twelve significant figures~\parencite{PhysRevLett.97.030801}. The discovery of the Higgs boson in 2012 was the culmination of almost fifty years of experimental effort, confirming the mechanism incorporated into the Standard Model in the late 1960s through the unification of the electromagnetic and weak interactions by Glashow, Weinberg, and Salam~\parencite{PhysRevLett.19.1264, gl1961579}. With this discovery, the full particle spectrum predicted by the Standard Model was finally observed.
 % Standard Model
\section{Deficiencies of Standard Model and New Physics}
While these and other successes of the Standard Model are an achievement for the field of particle physics, it is well known that this cannot be the ultimate theory of fundamental particles and interactions. Even though the Standard Model is currently the best description there is of the subatomic world, it does not explain the complete picture; there are also important questions that it does not answer and it is also surrounded by different irregularities. Some of them are completely incompatible with the current Standard Model, and strongly suggest that the Standard Model requires a consistent extension to solve experimental and theoretical problems that we will label as the cosmological problems, phenomenological problems, and theoretical problems. Below we will list very briefly the main representatives of these categories.

\subsection{Cosmological problems}
\begin{description}
		\item[Gravity and the cosmological-constant problem] A UV-complete quantum theory of gravity remains unknown; at low energies general relativity can be treated as an effective field theory, but it is non-renormalizable~\parencite{Donoghue:1994EFT,Burgess:2004QG}. Moreover, the observed vacuum energy (cosmological constant) driving cosmic acceleration is many orders of magnitude smaller than naive quantum-field-theory estimates, posing a severe naturalness problem~\parencite{Weinberg:1989CC}.

  \item[Dark matter] Cosmological and astrophysical data require a cold, non-baryonic,  component with $\Omega_c h^2 \simeq 0.12$~\parencite{Planck2018}. The Standard Model has no suitable particle: active neutrinos are too light and hot, and baryons are limited by BBN/CMB (Baryon Acoustic Oscillations / Cosmic Microwave Background). This points to new degrees of freedom BSM. Direct-detection experiments continue to improve sensitivity with null results; recent LZ and XENONnT runs set the strongest spin-independent limits over a wide mass range~\parencite{LZ:2022first,LZ:2023full,XENONnT:2023}.

  \item[Matter--antimatter asymmetry (baryon asymmetry)] The Universe exhibits a nonzero baryon asymmetry, $\eta_B \simeq 6\times10^{-10}$ from CMB/BBN~\parencite{Planck2018}. The SM fails quantitatively: for $m_H=125$ GeV the electroweak transition is a crossover (no sufficient departure from equilibrium), and CKM CP violation is many orders of magnitude too small to generate the observed asymmetry. Therefore additional CP violation and/or new dynamics are required, e.g. leptogenesis or electroweak baryogenesis~\parencite{Sakharov:1967,Davidson:2008Leptogenesis,Morrissey:2012EWB}.

  \item[Dark energy] Late-time acceleration is consistent with a cosmological constant with $w\approx -1$~\parencite{DESIY1:2024}. In SM+GR there is no mechanism to obtain such a tiny but nonzero vacuum energy; naive quantum-field-theory estimates are vastly larger, implying extreme fine-tuning (the cosmological-constant problem)~\parencite{Weinberg:1989CC}. The $H_0$ tension persists and could reflect systematics or new physics.
\end{description}

\subsection{Theoretical problems}

\begin{description}
    \item[Hierarchy problem] Is the problem concerning the large discrepancy between aspects of the weak force and gravity. Both of these forces involve constants of nature, the Fermi constant for the weak force and the Newtonian constant of gravitation for gravity. If the Standard Model is used to calculate the quantum corrections to Fermi's constant, it appears that Fermi's constant is surprisingly large and is expected to be closer to Newton's constant unless there is a delicate cancellation between the bare value of Fermi's constant and the quantum corrections to it. 
    
    In the Standard Model context, the Higgs boson is much lighter than the energy scale on which the standard model is considered valid (ideally the Plank mass), and the quantum corrections to the Higgs mass are on the order of this energy scale; it would inevitably make the Higgs and fermions masses huge, comparable to the scale at which new physics appears, unless there is an incredible fine-tuning cancellation between the quadratic radiative corrections and the bare mass. This level of fine-tuning is deemed unnatural.

    \item[Strong CP problem] QCD Lagrangian supports a term associated with the strength tensor dual for gluons that break CP symmetry in the strong interaction sector. Experimentally, however, no such violation has been found, implying that the coefficient of this term is fine tunned to zero. 
    \item[Quantum triviality] Suggests that it may not be possible to create a consistent quantum field theory involving elementary scalar Higgs particles because for high momentum particles the renormalization presents inconsistencies unless the renormalization of the charges becomes null, and therefore not interacting, \textit{i.e.} trivial. Nevertheless, because the Higgs boson plays a central role in the Standard Model of particle physics, the question of triviality in Higgs models is of great importance. 
    \item[Number of parameters and Unexplained relations] In total, the standard model has too many free parameters (19 in total) that are obtained experimentally, and there are indications that several of them may be correlated, however the origin of these correlations is beyond the standard model.
    
    For example, Yoshio Koide's empirical formula~\parencite{0505220}
    $$
    \frac{m_{e}+m_{\mu}+m_{\tau}}{\left(\sqrt{m_{e}}+\sqrt{m_{\mu}}+\sqrt{m_{\tau}}\right)^{2}}=0.666661(7) \approx \frac{2}{3}
    $$
    seems to indicate that there is a way to predict the masses of leptons.
    
\end{description}
\subsection{Phenomenological problems}\label{pheno_bsm}
\begin{description}
  \item[Neutrino masses] Precision oscillation data continue to require non-zero neutrino masses and mixing. Global fits still prefer normal ordering, but the mass ordering and the Dirac CP phase remain unestablished; see the latest NuFIT summary~\parencite{NuFIT2024}. Direct kinematic limits from KATRIN have pushed the effective electron-neutrino mass into the sub-eV regime (about $0.5$ eV at 90\% CL)~\parencite{KATRIN2022,KATRIN2024}.

  \item[Anomalies in $b$-hadron decays] The earlier hints of lepton-flavour universality violation in $b\to s\ell\ell$ (e.g. $R_{K^{(\ast)}}$) have largely subsided. The 2022 LHCb analyses with the full Run 1+2 dataset report $R_K$ and related ratios consistent with the SM within uncertainties~\parencite{LHCb:2022RK}. Angular-observable tensions (e.g. $P'_5$) persist at lower significance and are sensitive to hadronic uncertainties. For $b\to c\tau\nu$, updated averages (Belle/Belle~II, HFLAV) have moved closer to the SM; any deviation is now at the $\sim$2--3$\sigma$ level depending on inputs~\parencite{HFLAV2023,BelleII:2023RDstar}.

  \item[Anomalous magnetic dipole moment of the muon] Fermilab's 2023 update improved the experimental precision~\parencite{FNALg2_2023}. The significance of any discrepancy with the SM now depends on the hadronic vacuum polarization input: using the 2020 theory white paper gives a $\sim$4$\sigma$ deviation~\parencite{Aoyama:2020WhitePaper}, whereas lattice-QCD evaluations (e.g. BMW) and new $e^+e^-\!\to\pi^+\pi^-$ input from CMD-3 tend to reduce the tension~\parencite{BMW:2021Nature,CMD3:2023pipii}.

  \item[W-boson mass] The precise CDF II determination remains in strong tension with the SM and with other experiments~\parencite{CDFII:2022Wmass}. Subsequent ATLAS and earlier LEP/LHCb results are compatible with the SM; see the PDG 2024 summary for a balanced overview~\parencite{PDG2024}.

  \item[CCA and $q\bar q \mapsto e^+ e^-$] First-row CKM unitarity tests still show a mild ($\sim$2--3$\sigma$) tension depending on treatment of radiative/nuclear corrections and kaon inputs~\parencite{PDG2024,Seng:2018PRL,HardyTowner:2020,Cirigliano:2022}. High-mass Drell--Yan lepton universality measurements at the LHC with Run~2/3 data are generally consistent with the SM within uncertainties.
\end{description}

 % Deficiencies of Standard Model and Evidence