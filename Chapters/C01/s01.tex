\section{Fields}
Relativistic quantum fields are degrees of freedom in QFT. Formally, they are \textit{operator-valued functions on spacetime that transform under a representation of the Lorentz group on an invariant subspace}~\parencite{Tong1995}. The different representations of the Lorentz group are mainly characterized by their spin, and their fields obey a different equation of motion (see table~\ref{tab-repLorentz2}). 

In classical field theory, a variational principle is established which generates the equations that govern the dynamics of the different fields in a theory, \textit{the equations of motion}. Hamilton's principle, or principle of minimal action, indicates that all possible physical configurations for a set of fields $\varphi^I$, with $I=1,2,3,\cdots,n$, are those for which the action $S$ is  minimal~\parencite{Goldstein,jose1998classical}:
\begin{equation}\label{eq-action}
	S=\int \mathcal{L}(\varphi^I,\partial_\mu\varphi^I) d^4x.
\end{equation}
Here, $d^4x=dx^0dx^1 dx^2dx^3$ and $x\equiv(ct,x^1,x^2,x^3)\equiv(x^0,x^1,x^2,x^3)\in\mathcal{M}^4$, are the space-time coordinates in the Minkowskian spacetime ($\mathcal M^4$), and the function $\lag(\varphi^I,\partial_\mu\varphi^I)$ is called \textit{the Lagrangian density} of a theory~\parencite{greiner2000relativistic,Goldstein}. The problem in classical field dynamics is to find the functions $\varphi^I(x)$ in a space-time $\mathcal{M}^4$, fixing their boundary conditions. The solution to this classical problem is given by the Euler-Lagrange equations:
\begin{equation}\label{eq_EulerLag}
	\dpr{\mathcal{L}}{\varphi^I}-\dpr{}{x^\mu}\dpr{\lag}{\fac{\partial_\mu \varphi^I}}=0,
\end{equation}
and they are used to obtain the equations of motion of the set of fields $\varphi^I$~\parencite{jose1998classical}. 

While in classical field theory the Euler–Lagrange equations directly determines the dynamics of the system, in QFT the approach changes: if we adopt the path-integral formulation~\parencite{martinez2002,Weinberg}, the idea of an equation of motion vanishes and we move on to searching for correlations between free particle states. However, the notion of action remains the cornerstone in the description of these observables.

Explicitly, the correlation functions are calculated through the Lehmann-Symanzik-Zimmermann (LSZ) reduction formula, which connects these correlators with physical scattering amplitudes. These are computed from the path integral~\parencite{greiner1996qft,peskin}:
\begin{equation}
	\begin{aligned}
		Z[J]&=\braket{\text { out, } 0| 0, \text { in }}
		\\&=\mathcal{N}\int \mathcal{D}(\varphi, \bar{\varphi})  e^{i S[\varphi]} e^{i \int J_I\varphi^I  d^{4} x}
		\\&=\mathcal{N}\int \mathcal{D}(\varphi, \bar{\varphi})  e^{i \int d^{4} x \mathcal{L}} e^{i \int J_I\varphi^I  d^{4} x},
	\end{aligned}
\end{equation}
taken over the space of fields $\varphi$ with an appropriate measure $\mathcal{D}(\varphi, \bar{\varphi})$ and normalized by $\mathcal{N}$. The quantity $Z$ is known as the partition function of the theory and gives the transition amplitude from the initial vacuum $\ket{0,\text{ in}}$ to the final vacuum $\ket{0,\text{ out}}$ in the presence of a source $J(x)$ producing particles~\parencite{birrell75900}.


\begin{center}
    \begin{tabular}{|l|c|c|l|}\hline\bigstrut
        Name							& Field				& Spin & Free-Lagrangian	\\\hline\hline\bigstrut
        Klein-Gordon				&	$\phi$					& $0$			&	$\lag=\frac{1}{2}\fac{\partial^\mu \phi\partial_\mu \phi-m^2 \phi\phi}$						\\\hline\bigstrut
        Dirac								& $\chi$			& $1/2$	&$\lag=\bar\chi\fac{i\pmb\gamma^\mu \partial_\mu -m\pmb 1}\chi$\\\hline\bigstrut
        Proca (Massive Vector)	        & $A^\mu$ 		& $1$		&$\lag=-\frac{1}{4} F^{\mu\nu} F_{\mu\nu} + \frac{1}{2}m^2 A^\mu A_\mu $\\\hline
    \end{tabular}
	\captionof{table}{Some relevant representations of the Lorentz group in  $4$-dimensional space-time. In this notation $\eta_{\mu\nu}=\diag(1,-1,-1,-1)$, $\pmb \gamma^\mu$ are the Dirac matrices, $F_{\mu \nu}=\partial_{\mu} A_{\nu}-\partial_{\nu} A_{\mu}$ is the abelian field strength tensor. All equations are written in natural units with $c=\hbar=1$. Fields are shown in their standard representations.}\label{tab-repLorentz2}
\end{center}

Therefore, the dynamics, at both the classical and quantum levels, are entirely determined by the Lagrangian density. For free fields (i.e., non-interacting), the Lagrangian is quadratic in the fields and the path integral can be evaluated exactly. Tab.~\ref{tab-repLorentz2} records the Lagrangian density for these free fields. However, to describe physics, we must include interactions, which render the path integral impossible to compute exactly.

The framework of \textit{perturbation theory} addresses this by expanding the interaction part of the Lagrangian as a power series. This expansion is organized using \textit{Feynman diagrams}, which provides a pictorial representation of each term, and a set of \textit{Feynman rules}, which provides a precise dictionary to translate these diagrams into mathematical expressions for scattering amplitudes~\parencite{peskin,Weinberg}. The importance of these rules cannot be overstated, as they are the practical computational tools of perturbative QFT.


In this paradigm, our task is to propose a Lagrangian density for a set of fields that correctly models the propagation and interactions of fundamental particles. The free part defines the particle content and propagators, while the interaction part defines the vertices and possible scattering processes.

\subsection{Interactions and Symmetries} 

The structure of the Lagrangian density in a quantum field theory is not arbitrary; it is constrained by fundamental principles that ensure the theory is physically consistent and mathematically well-defined. These principles act as ``rules'' that guide the construction of viable theories. In what follows, we systematically develop these constraints, starting from the practical requirements of perturbation theory and building up to the fundamental symmetry principles.

To perform calculations, we typically split the Lagrangian into a free part, which describes non-interacting fields, and an interaction part:
\begin{equation}
    \mathcal{L} = \mathcal{L}_0 + \mathcal{L}_{\text{int}}.
\end{equation}
This splitting is the starting point for perturbation theory. In the path integral formulation, the generating functional $Z[J]$ can then be expressed as an operator acting on the free functional $Z_0[J]$:
\begin{equation}
    Z[J] = \mathcal{N} \exp\left[i \int d^4x\, \mathcal{L}_{\text{int}}\left(-i \frac{\delta}{\delta J(x)}\right)\right] Z_0[J].
\end{equation}
The exponential operator generates an infinite series known as the perturbation series. The $n$-point correlation function is found by taking functional derivatives of $Z[J]$ with respect to the sources $J(x_i)$ and setting $J=0$. Each term in this series is represented by a \textbf{Feynman diagram}, whose components are:

\begin{itemize}
    \item \textbf{External Lines:} Represent incoming and outgoing physical particles.
    \item \textbf{Internal Lines:} Represent virtual particles propagating between interactions, corresponding to the free-field propagators derived from $\mathcal{L}_0$.
    \item \textbf{Vertices:} Represent interactions, derived from the terms in $\mathcal{L}_{\text{int}}$. Each vertex has an associated coupling constant and enforces momentum conservation.
\end{itemize}

\begin{figure}[h!]
    \centering
    \begin{fmffile}{feyngraphs/feyngraph0}
        \vspace{0.5cm}
        \begin{fmfgraph*}(120,80)
            \fmfleft{i1,i2}
            \fmfright{o1,o2}
            
            % External incoming lines
            \fmf{fermion}{i1,v1}
            \fmf{fermion}{i2,v2}
            
            % Internal propagator
            \fmf{photon,label=$\gamma$,label.side=left}{v1,v2}
            
            % External outgoing lines
            \fmf{fermion}{v1,o1}
            \fmf{fermion}{v2,o2}
            
            % Labels for external particles
            \fmflabel{$e^-$}{i1}
            \fmflabel{$e^+$}{i2}
            \fmflabel{$e^-$}{o1}
            \fmflabel{$e^+$}{o2}
            
            % Vertex labels
            \fmfv{label=$v_1$,label.angle=180,label.dist=0.3cm}{v1}
            \fmfv{label=$v_2$,label.angle=0,label.dist=0.3cm}{v2}
        \end{fmfgraph*}
        \vspace{0.5cm}
    \end{fmffile}
    \caption{Example of a Feynman diagram for $e^+e^- \to e^+e^-$ scattering. \textbf{External lines} (solid arrows at the edges) represent the incoming and outgoing electrons and positrons. The \textbf{internal line} (wavy line) represents the virtual photon propagator. The \textbf{vertices} ($v_1$ and $v_2$) represent the electromagnetic interaction points where the coupling constant $e$ (electric charge) enters and momentum is conserved.}
    \label{fig:feynman-components}
\end{figure}

For this perturbation series to be a predictive computational tool, it must yield finite physical results. However, individual terms in the series (i.e., individual Feynman diagrams) often lead to divergent integrals when loop corrections are included. The key is that in a \textit{renormalizable} theory, these divergences from all diagrams can be systematically absorbed into a finite number of parameters (like masses and coupling constants) through a redefinition procedure known as renormalization. It is important to note that while individual Feynman diagrams may diverge, the requirement is that the combination of all contributions at a given order yields finite, physically meaningful results after renormalization.

This requirement of renormalizability imposes a powerful constraint on the form of $\mathcal{L}_{\text{int}}$. Through power-counting arguments, one finds that only operators of mass dimension $\leq 4$ lead to renormalizable interactions. In natural units, where $\mathcal{L}$ has dimension $[\text{mass}]^4$, this means that $\mathcal{L}_{\text{int}}$ can be expressed as a truncated polynomial containing only terms up to dimension 4. Specifically, this allows Yukawa couplings (dim 4), quartic scalar interactions (dim 4), and gauge interactions (dim 4), while forbidding non-renormalizable operators like $\phi^6$ (dim 6). Higher-dimensional operators are still allowed in effective field theories, but they correspond to interactions that are suppressed at low energies and signal the presence of new physics at higher scales~\parencite{peskin,Weinberg}.

This is why we express $\mathcal{L}_{\text{int}}$ as a truncated polynomial: renormalizability demands that we include only a finite set of operators with dimension $\leq 4$, ensuring that the theory remains predictive at all accessible energy scales.

An additional crucial requirement is the \textit{stability of the vacuum}. For a theory to be physically meaningful, it must possess a stable ground state. This is ensured by demanding that the scalar potential, which governs the self-interactions of scalar fields, is bounded from below. If the potential were unbounded, the system could lower its energy indefinitely by evolving toward field configurations of ever-greater magnitude, meaning no stable vacuum would exist.

For a renormalizable theory, the scalar potential can contain at most quartic terms. A general scalar potential for a set of scalar fields $\{\phi_i\}$ takes the form:
\begin{equation}
    V(\phi_i) = \sum_i \mu_i^2 |\phi_i|^2 + \sum_{i,j} \lambda_{ij} |\phi_i|^2 |\phi_j|^2 + \cdots
\end{equation}
where the ellipsis denotes possible cubic and mixed quartic terms allowed by the symmetries of the theory. The stability condition requires that the quartic couplings $\lambda_{ij}$ satisfy certain positivity constraints to ensure that $V \to +\infty$ as $|\phi_i| \to \infty$ in any direction in field space. This is why the scalar potential is a polynomial of at most order four: renormalizability forbids higher-order terms, and stability demands that the quartic terms dominate at large field values with the correct sign.

It is important to note that this condition must hold not just at tree-level but also at the quantum level, as running couplings can change sign at different energy scales, potentially leading to metastability or instability of the vacuum.

A fundamental requirement from quantum mechanics is \textit{Hermiticity}: the Lagrangian density must be Hermitian to ensure that observables are real and the time evolution of the theory is unitary~\parencite{pall,peskin}. This is the most basic constraint that quantum theory imposes on the Lagrangian. Without Hermiticity, the theory would predict complex-valued probabilities and violate the fundamental probabilistic interpretation of quantum mechanics.

Beyond the quantum mechanical requirement of Hermiticity, special relativity imposes a fundamental constraint: \textit{Poincaré invariance}. This symmetry demands that the equations of motion remain the same in all inertial frames. Mathematically, this is implemented by requiring the action to be globally invariant under Poincaré transformations~\parencite{pall}. Equivalently, the Lagrangian density must transform as a Lorentz scalar and may change under translations at most by a total derivative~\parencite{jose1998classical}.\marginpar{\footnotesize In QFT, Poincaré invariance is assumed to be global. Promoting it to a local symmetry leads to gravity, with spin-2 fields (the graviton) as mediators. Perturbatively, such a theory is not renormalizable, so it lacks predictivity at high energies, although it can still be understood as an effective field theory.}

This constraint is extremely powerful: it eliminates all possible interaction terms that would depend on the choice of reference frame. For instance, terms that explicitly depend on spacetime coordinates or preferred directions are forbidden. Furthermore, \textit{dimensional analysis} places additional restrictions. In natural units, $\mathcal{L}$ carries dimensions of mass to the fourth power ($[\mathcal{L}] = [\text{mass}]^4$), which corresponds with an energy density. Combined with Lorentz invariance, this means that the interaction terms must be constructed from Lorentz-covariant combinations of fields and their derivatives, with the correct overall mass dimension.

The symmetries discussed so far—Poincaré invariance, Hermiticity, and dimensional analysis—are universal requirements that any relativistic quantum field theory must satisfy. However, they still leave a vast array of possible interaction terms. To further constrain the Lagrangian and to describe the fundamental forces of nature, we must consider \textit{internal symmetries}: transformations that act on the fields' internal degrees of freedom rather than on spacetime coordinates.

Internal symmetries can be either \textit{global} (where the transformation parameters are constant throughout spacetime) or \textit{local} (gauge symmetries, where the parameters can vary from point to point). The procedure for constructing gauge theories—where global symmetries are ``promoted'' to local ones by introducing gauge fields—is systematic and will be described in detail below. This gauge principle has proven to be the most powerful organizing principle in particle physics, determining not only which interactions are realized in nature but also their precise mathematical structure.

A classical symmetry of the Lagrangian may not always survive the process of quantization. If it fails to do so, it is said to be anomalous. \textit{Chiral anomalies}, specifically, arise from the regularization of fermion loops in triangle diagrams and can break gauge symmetries at the quantum level. Since gauge symmetry is the very principle that dictates the form of interactions and removes unphysical states, its violation would destroy the renormalizability and unitarity of the theory. Therefore, the particle content must be carefully chosen so that these potential anomalies cancel among fermions, a non-trivial condition famously satisfied by the quarks and leptons of the Standard Model~\parencite{peskin,Weinberg,bertlmann1996anomalies}.

In summary, the construction of a consistent relativistic quantum field theory proceeds through a hierarchy of constraints:
\begin{enumerate}
	\item \textbf{Perturbative renormalizability:} power-counting arguments restrict operators to mass dimension $\leq 4$, ensuring $\mathcal{L}_{\text{int}}$ is a truncated polynomial.
	\item \textbf{Vacuum stability:} the scalar potential must be bounded from below, requiring appropriate positivity conditions on quartic couplings.
	\item \textbf{Hermiticity:} quantum mechanics demands $\mathcal{L}$ be Hermitian for real observables and unitary evolution.
	\item \textbf{Poincaré invariance:} special relativity requires the action to be invariant under Lorentz transformations and translations, eliminating frame-dependent terms.
	\item \textbf{Internal symmetries:} global and gauge symmetries further constrain the form of interactions and determine the structure of fundamental forces.
	\item \textbf{Anomaly cancellation:} the particle content must be chosen such that chiral anomalies cancel, preserving gauge symmetry at the quantum level.
\end{enumerate}

These constraints drastically reduce the number of possible terms in the Lagrangian. The renormalizable interaction structures that survive are limited to: Yukawa couplings between fermions and scalars, quartic scalar self-interactions, and gauge interactions between matter fields and vector bosons. The precise form of these interactions is then determined by the internal (gauge) symmetries of the theory, which we now describe in detail.

The procedure is systematic: first, the spin$-0$ and spin$-1/2$ fields are organized into representations of a unitary (gauge) group $G$, such that the Lagrangian density is globally invariant under $G$. This global symmetry is then ``promoted'' to a \textit{local symmetry} (where the group parameters can vary in spacetime) by replacing the ordinary derivatives $\partial_\mu$ with \textit{covariant derivatives} $\Dcov_\mu$ that incorporate new \textit{gauge fields} $B_\mu^A$~\parencite{pokorski2000gauge,freedman2012supergravity, Gallego2016,VanProeyen1999,Martin2012}.
This ``promotion'' is described in more detail below.

Given a Lagrangian density $\lag(\varphi^I, \partial_\mu \varphi^I)$, where $I$ is an index enumerating the different fields $\varphi^{I}$ in the model, it is said to be \textit{globally symmetric} under unitary transformations if the action remains invariant under field variations. At infinitesimal level, these variations are given by:
\begin{equation}
	\delta_G \varphi^I = i\theta^A (T_A)^I_J \varphi^J,
\end{equation}
where $\theta^{A}$ are constant parameters of the transformation and the $T_{A}$ are the generators of the group $G$ in the appropriate representation. The corresponding finite unitary transformation is
\begin{equation}
	\mathcal{U}_G \equiv U(\theta)=\exp(i\theta^A T_A).
\end{equation}
Note that the $T_A$  generators  satisfy the same Lie algebra of the group $G$:
\begin{equation}
	[T_A, T_B] = i f_{AB}^{\;\;C}T_C,
\end{equation}
where $f_{AB}^{\;\;C}$ are the structure constants of $G$.

To promote the global symmetry to a local one ($\theta^A \to \theta^A(x)$), the ordinary derivative $\partial_\mu$ is replaced by a \textit{covariant derivative} $\Dcov_\mu$. This new derivative is designed to transform covariantly under the gauge group, meaning $\Dcov_\mu \varphi \to U(x) (\Dcov_\mu \varphi)$, so that the kinetic terms $\lag_{\text{kin}} \sim (\Dcov_\mu \varphi)^\dagger (\Dcov^\mu \varphi)$ remain invariant. This is achieved by introducing a gauge field $B_\mu^A$ for each generator $T_A$ and defining:
\begin{equation}
	\Dcov_\mu = \partial_\mu - i g B_\mu^A T_A,
\end{equation}
where $g$ is the gauge coupling constant. The transformation law for the gauge fields that ensures the covariant transformation of $\Dcov_\mu$ is:
\begin{equation}
	\delta B_\mu^A = \partial_\mu \theta^A + g f_{BC}{}^A \theta^B B_\mu^C.\label{eq:gauge-transformation}
\end{equation}

The introduction of the gauge fields $B_\mu^A$ requires the addition of a kinetic term for them to the Lagrangian. This is constructed from the \textit{field strength tensor} $F_{\mu\nu}^A$, defined as the curvature of the covariant derivative:
\begin{equation}
	F_{\mu\nu}^A T_A = -\frac{i}{g} [\Dcov_\mu, \Dcov_\nu] = \partial_\mu B^A_\nu - \partial_\nu B^A_\mu + g f_{BC}{}^A B^B_\mu B^C_\nu.
\end{equation}
The gauge-invariant kinetic Lagrangian is then:
\begin{equation}
	\lag_{\text{gauge}} = -\frac{1}{4} \delta_{AB} F^A_{\mu\nu} F^{\mu\nu B}.
\end{equation}
Often, the rescaling $B_\mu^A \to g B_\mu^A$ is performed, which moves the coupling constant $g$ from the kinetic term to the covariant derivative, resulting in the more conventional form $\Dcov_\mu = \partial_\mu - i g B_\mu^A T_A$ and $\lag_{\text{gauge}} = -\frac{1}{4g^2} \delta_{AB} F^A_{\mu\nu} F^{\mu\nu B}$.


A general, archetypal Lagrangian, embodying these structures, can be written as:
\begin{equation}\label{eq:generic-renorm-lag}
	\mathcal{L} = -\frac{1}{4} F_{\mu \nu}^A F^{A \mu \nu} + i \bar{\psi}^i \gamma^\mu \mathcal{D}_\mu \psi^i + \left(\bar{\psi}_L^j \, \Gamma^j_k \, \Phi \, \psi_R^k + \text{h.c.}\right) + |\mathcal{D}_\mu \Phi|^2 - V(\Phi)
\end{equation}
The terms correspond to: the kinetic term for gauge fields ($F_{\mu \nu}^A$), the kinetic term for fermions $\psi^i$, the Yukawa interactions between left- and right-handed fermions and scalars ($\Gamma^j_k$ is a Yukawa coupling matrix and $\Phi$ is a scalar field), the kinetic term for scalars, and the scalar potential $V(\Phi)$. For a renormalizable and  stable theory $V(\Phi) = \mu^2 |\Phi|^2 + \lambda |\Phi|^4$ with $\lambda > 0$.

Note the absence of explicit mass terms for the gauge fields ($\sim M^2 B_\mu B^\mu$) and fermions ($\sim m \bar{\psi}\psi$). These are forbidden by gauge invariance and for chiral fermions. Mass terms can be generated via spontaneous symmetry breaking, as discussed below.

It is important to emphasize that while the Yukawa interactions $\bar{\psi}_L^j \, \Gamma^j_k \, \Phi \, \psi_R^k$ do not involve gauge bosons directly, their structure is nonetheless \textit{completely determined by the gauge symmetry}. Specifically, gauge invariance dictates which fermion fields can couple to which scalar fields, and constrains the form of the coupling matrix $\Gamma^j_k$. For a Yukawa term to be gauge-invariant, the product $\bar{\psi}_L^j \, \Phi \, \psi_R^k$ must be a singlet under the gauge group. This requirement arises because the left-handed and right-handed fermions typically transform in different representations of the gauge group, and the scalar field $\Phi$ must carry the appropriate quantum numbers to make the overall combination invariant. In the Standard Model, for instance, the left-handed fermions are $SU(2)_L$ doublets while the right-handed fermions are singlets, and the Higgs doublet provides the necessary quantum numbers to form gauge-invariant Yukawa couplings. Thus, even though Yukawa interactions are scalar-mediated rather than gauge-mediated, the gauge principle is the fundamental organizing principle that determines their allowed structure.

\subsubsection{Example}
To illustrate these concepts, let us consider a renormalizable theory with a real scalar $\phi$ and a Dirac spinor $\psi$, and suppose that this theory is globally invariant under $U(1)$ phase transformations, i.e. the fields $\varphi\in\{\phi,\psi\}$ transform as $\varphi\mapsto e^{i\theta \hat Q}\varphi $ such that $\hat Q \psi = q \psi$ and $\hat Q \phi=0$. The free Lagrangian density is
\begin{equation}
	\mathcal L_{\text{free}}=\frac{1}{2} \partial^{\mu} \phi \partial_{\mu} \phi-\frac{1}{2}\mu^2\phi^2+\bar{\psi}(i \gamma_\mu  \partial^\mu-m) \psi.
\end{equation}
\marginpar{\footnotesize Note that for this vector-like $U(1)$ theory, the explicit fermion mass term $m\bar{\psi}\psi$ is gauge-invariant. This will not be the case for chiral gauge theories like the Standard Model.}

To add globally symmetric interaction terms, we must consider operators of mass dimension $\leq 4$. The most general renormalizable Lagrangian, invariant under the global $U(1)$ symmetry, is
\begin{equation}
	\begin{aligned}
		\mathcal L_{\text{global}}&=\frac{1}{2} \partial^{\mu} \phi \partial_{\mu} \phi-V(\phi)+\bar{\psi}(i \gamma_\mu  \partial^\mu-m) \psi + k_1 \phi\bar\psi\psi,
		\\
		V(\phi)&=\frac{1}{2}\mu^2\phi^2 +\frac{\alpha}{3!}\phi^3+\frac{\lambda}{4!}\phi^4.
	\end{aligned}
\end{equation}
The cubic and quartic terms in $V(\phi)$ are allowed as $\phi$ is neutral. The Yukawa coupling $k_1 \phi\bar\psi\psi$ is also gauge-invariant since the charges of $\bar\psi$, $\phi$, and $\psi$ sum to zero ($-q + 0 + q = 0$).

Promoting the global symmetry to a local one ($\theta \to \theta(x)$) requires introducing a gauge field $A_\mu$ and replacing ordinary derivatives with covariant derivatives:
\begin{equation}
	\mathcal D_\mu\varphi=(\partial_{\mu}-i g A_\mu\hat Q )\varphi
	\quad\Longrightarrow\quad
	\begin{cases}
		\mathcal D_\mu\phi=\partial_\mu \phi, & (\text{since } \hat Q\phi=0)\\
		\mathcal D_\mu\psi=(\partial_\mu - i g q A_\mu) \psi.
	\end{cases}
\end{equation}
The field strength tensor for the abelian $U(1)$ field is defined as $F_{\mu\nu} = \partial_\mu A_\nu - \partial_\nu A_\mu$. The locally invariant Lagrangian is then:
\begin{multline}
	\mathcal L_{\text{local}}=\frac{1}{2} \mathcal D^{\mu} \phi \mathcal D_{\mu} \phi-V(\phi)
	+\bar{\psi}i \gamma_\mu  \mathcal D^{\mu} \psi - m \bar{\psi}\psi
	+ k_1 \phi\bar\psi\psi-\frac{1}{4} F_{\mu\nu}F^{\mu\nu}.
\end{multline}


With these ingredients and principles, we are now equipped to understand the structure of the SM Lagrangian, which will be discussed in the next section.

\begin{figure}[h!]
    \centering
    \begin{subfigure}[b]{0.48\textwidth}
        \centering
        \begin{fmffile}{feyngraphs/feyngraph1} 
			\vspace{0.5cm}
            \begin{fmfgraph*}(80,60)
                \fmfleft{i1}
                \fmfright{o1,o2}
                
                \fmf{dashes,tension=2.0}{i1,v1}
                \fmf{fermion}{o1,v1}
                \fmf{fermion}{v1,o2}

                \fmflabel{$\phi$}{i1}
                \fmflabel{$\bar\psi$}{o1}
                \fmflabel{$\psi$}{o2}
            \end{fmfgraph*}
			\vspace{0.5cm}
        \end{fmffile}
        \caption{Yukawa coupling with a scalar $\phi$.}
        \label{fig-yukawa-scalar}
    \end{subfigure}
    \hfill
    \begin{subfigure}[b]{0.48\textwidth}
        \centering
        \begin{fmffile}{feyngraphs/feyngraph2}
			\vspace{0.5cm}
            \begin{fmfgraph*}(80,60)
                \fmfleft{i1}
                \fmfright{o1,o2}
                
                \fmf{photon,tension=2.0}{i1,v1}
                \fmf{fermion}{o1,v1}
                \fmf{fermion}{v1,o2}

                \fmflabel{$\gamma$}{i1}
                \fmflabel{$\bar\psi$}{o1}
                \fmflabel{$\psi$}{o2}
            \end{fmfgraph*}
			\vspace{0.5cm}
        \end{fmffile}
        \caption{Interaction with a photon $\gamma$.}
        \label{fig-qed-photon}
    \end{subfigure}
	\begin{subfigure}[b]{0.48\textwidth}
        \centering
		\begin{fmffile}{feyngraphs/feyngraph3}
			\vspace{1.0cm}
			\begin{fmfgraph*}(80,60)
				\fmfleft{i1}
				\fmfright{o1,o2}

				\fmf{dashes}{i1,v1}
				\fmf{dashes}{v1,o1}
				\fmf{dashes}{v1,o2}

				\fmflabel{$\phi$}{i1}
				\fmflabel{$\phi$}{o1}
				\fmflabel{$\phi$}{o2}
			\end{fmfgraph*}
			\vspace{0.5cm}
		\end{fmffile}
		\caption{Triple scalar coupling.}
		\label{fig-triple-scalar}
	\end{subfigure}
	\begin{subfigure}[b]{0.48\textwidth}
        \centering
		\begin{fmffile}{feyngraphs/feyngraph4}
			\vspace{1.0cm}
			\begin{fmfgraph*}(80,60)
				\fmfleft{i1,i2}
				\fmfright{o1,o2}

				\fmf{dashes}{i1,v1}
				\fmf{dashes}{i2,v1}
				\fmf{dashes}{v1,o1}
				\fmf{dashes}{v1,o2}

				\fmflabel{$\phi$}{i1}
				\fmflabel{$\phi$}{i2}
				\fmflabel{$\phi$}{o1}
				\fmflabel{$\phi$}{o2}
			\end{fmfgraph*}
			\vspace{0.5cm}
		\end{fmffile}
		\caption{Quartic scalar coupling.}
		\label{fig-quartic-scalar}
	\end{subfigure}
    \caption{Feynman diagrams for Yukawa coupling, gauge boson coupling and quartic scalar coupling.}
\end{figure}
