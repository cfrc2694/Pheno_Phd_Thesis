\section{Fields}
Relativistic quantum fields are degrees of freedom in QFT. Formally, they are \textit{operator-valued functions of the spacetime that transform under a representation of the Lorentz group within an invariant subspace}~\parencite{Tong1995,CRodriguezUPTC}. The different representations of the Lorentz group are mainly characterized by their spin, and their fields obey a different equation of motion (see table~\ref{tab-repLorentz2}). 

In classical field theory, a variational principle is established which generates the equations that govern the dynamics of the different fields in a theory, \textit{the equations of motion}. Hamilton's principle, or principle of minimal action, indicates that all possible physical configurations for a set of fields $\varphi^I$, with $I=1,2,3,\cdots,n$, are those where the integral of the action $S$ is a minimal~\parencite{Goldstein,jose1998classical}:
\begin{equation}\label{eq-action}
	S=\int \mathcal{L}(\varphi^I,\partial_\mu\varphi^I) d^4x,
\end{equation}
here, $d^4x=dx^0dx^1 dx^2dx^3$ and $x\equiv(ct,x^1,x^2,x^3)\equiv(x^0,x^1,x^2,x^3)\in\mathcal{M}^4$ are the space-time coordinates in the Minkowskian spacetime $\mathcal M^4$, and the function $\lag(\varphi^I,\partial_\mu\varphi^I)$ is called \textit{the Lagrangian density} of a theory~\parencite{greiner2000relativistic,Goldstein}. The problem in classical field dynamics is to find the functions $\varphi^I(x)$ in a space-time $\mathcal{M}^4$, fixing their boundary conditions. The solution to this classical problem is given by the Euler-Lagrange equations:
\begin{equation}\label{eq_EulerLag}
	\dpr{\mathcal{L}}{\varphi^I}-\dpr{}{x^\mu}\dpr{\lag}{\fac{\partial_\mu \varphi^I}}=0,
\end{equation}
and they are used to obtain the equations of motion of the set of fields $\varphi^I$~\parencite{jose1998classical}. 

While in classical field theory the Euler–Lagrange equations directly determine the dynamics, in quantum field theory the picture changes: if we adopt the path-integral formulation~\parencite{martinez2002,Weinberg}, the idea of an equation of motion vanishes and we move on to searching for correlations between free particle states. However, the notion of action remains the cornerstone in the description of these observables.

Explicitly, the correlation functions are calculated through the Lehmann-Symanzik-Zimmermann (LSZ) reduction formula, which connects these correlators with physical scattering amplitudes. These are computed from the path integral~\parencite{greiner1996qft,peskin}
\begin{equation}
	\begin{aligned}
		Z[J]&=\braket{\text { out, } 0| 0, \text { in }}
		\\&=\mathcal{N}\int \mathcal{D}(\varphi, \bar{\varphi})  e^{i S[\varphi]} e^{i \int J_I\varphi^I  d^{4} x}
		\\&=\mathcal{N}\int \mathcal{D}(\varphi, \bar{\varphi})  e^{i \int d^{4} x \mathcal{L}} e^{i \int J_I\varphi^I  d^{4} x},
	\end{aligned}
\end{equation}
taken over the space of fields $\varphi$ with an appropriate measure $\mathcal{D}(\varphi, \bar{\varphi})$ and normalized by $\mathcal{N}$. The quantity $Z$ is known as the partition function of the theory and gives the transition amplitude from the initial vacuum $\ket{0,\text{ in}}$ to the final vacuum $\ket{0,\text{ out}}$ in the presence of a source $J(x)$ producing particles~\parencite{birrell75900}.


\begin{center}
    \begin{tabular}{|l|c|c|l|}\hline\bigstrut
        Name							& Field				& Spin & Free-Lagrangian	\\\hline\hline\bigstrut
        Klein-Gordón				&	$\phi$					& $0$			&	$\lag=\fac{\partial^\mu\bar \phi\partial_\mu \phi-m^2 \bar \phi\phi}$						\\\hline\bigstrut
        Dirac								& $\chi$			& $1/2$	&$\lag=\bar\chi\fac{i\pmb\gamma^\mu \partial_\mu -m\pmb 1}\chi$\\\hline\bigstrut
        Proca (Massive Vector)	        & $A^\mu$ 		& $1$		&$\lag=-\frac{1}{4} F^{\mu\nu} F_{\mu\nu} + \frac{1}{2}m^2 A^\mu A_\mu $\\\hline
    \end{tabular}
	\captionof{table}{Some relevant representations of the Lorentz group in  $4$-dimensional space-time. In this notation $\eta_{\mu\nu}=\diag(1,-1,-1,-1)$, $\pmb \gamma^\mu$ are the Dirac matrices, $F_{\mu \nu}=\partial_{\mu} A_{\nu}-\partial_{\nu} A_{\mu}$ is the abelian field strength tensor. All equations are written in natural units with $c=\hbar=1$. Fields are shown in their standard representations.}\label{tab-repLorentz2}
\end{center}

Therefore, the dynamics, at both the classical and quantum levels, are entirely determined by the Lagrangian density. For free fields (i.e., non-interacting), the Lagrangian is quadratic in the fields and the path integral can be evaluated exactly. Table~\ref{tab-repLorentz2} records the Lagrangian density for these free fields. However, to describe physics, we must include interactions, which render the path integral impossible to compute exactly.

The framework of \textit{perturbation theory} addresses this by expanding the interaction part of the Lagrangian as a power series. This expansion is organized using \textit{Feynman diagrams}, which provides a pictorial representation of each term, and a set of \textit{Feynman rules}, which provides a precise dictionary to translate these diagrams into mathematical expressions for scattering amplitudes~\parencite{peskin,Weinberg}. The importance of these rules cannot be overstated, as they are the practical computational tools of perturbative QFT.


In this paradigm, our task is to propose a Lagrangian density for a set of fields that correctly models the propagation and interactions of fundamental particles. The free part defines the particle content and propagators, while the interaction part defines the vertices and possible scattering processes.

\subsection{Interactions and Symmetries} 
The form of the Lagrangian density is not arbitrary: it is shaped by a small set of physical and mathematical principles. These principles act as ``rules'' that guide the construction of consistent theories, ensuring both their internal consistency and their predictive power. In particular, if we want a relativistic and renormalizable theory, the Lagrangian must satisfy several conditions that strongly restrict the kind of terms that can appear.

The need for these restrictions is evident from the path integral formulation itself. If we split the Lagrangian into a free part and an interaction part in the form
\begin{equation}
    \mathcal{L} = \mathcal{L}_0 + \mathcal{L}_{\text{int}}.
\end{equation}
The generating functional $Z[J]$ can then be expressed as an operator acting on the free functional $Z_0[J]$:
\begin{equation}
    Z[J] = \mathcal{N} \exp\left[i \int d^4x\, \mathcal{L}_{\text{int}}\left(-i \frac{\delta}{\delta J(x)}\right)\right] Z_0[J].
\end{equation}
The exponential operator generates an infinite perturbation series. The $n$-point correlation function is found by taking functional derivatives of $Z[J]$ with respect to the sources $J(x_i)$ and setting $J=0$.

Each term in this series is represented by a \textbf{Feynman diagram}:
\begin{itemize}
    \item \textbf{External Lines:} Represent incoming and outgoing physical particles.
    \item \textbf{Internal Lines:} Represent virtual particles propagating between interactions, corresponding to the free-field propagators derived from $\mathcal{L}_0$.
    \item \textbf{Vertices:} Represent interactions, derived from the terms in $\mathcal{L}_{\text{int}}$. Each vertex has an associated coupling constant and enforces momentum conservation.
\end{itemize}

For this series to be a predictive and well-defined computational tool, the individual terms must yield finite results. This requirement of \textit{renormalizability} is a powerful constraint on $\mathcal{L}_{\text{int}}$. Furthermore, the structure of both $\mathcal{L}_0$ and $\mathcal{L}_{\text{int}}$ is profoundly constrained by the requirement that the theory possesses certain \textit{symmetries}.

To begin with, relativistic invariance demands that the equations of motion look the same in all inertial frames. This requirement is implemented by asking the action to be invariant under Poincaré transformations~\parencite{pall}. Equivalently, the Lagrangian density must transform as a Lorentz scalar and may change under translations at most by a total derivative~\parencite{jose1998classical}. 

Another basic condition is Hermiticity: the Lagrangian density must be Hermitian so that observables are real and the time evolution of the theory is unitary~\parencite{pall,peskin}. In addition, dimensional analysis places further restrictions. In natural units, $\mathcal{L}$ carries dimensions of [mass]$^4$ (an energy density). This means that the interaction terms that we can add must be such that the overall operator has the correct dimension, which already rules out many possibilities. 

In quantum field theory, loop corrections to scattering amplitudes typically produce divergences. A theory is called renormalizable if all these divergences can be absorbed into a redefinition of a \emph{finite set} of physical parameters (such as masses and couplings). In practice, this requirement translates into a restriction on the operators that may appear in the Lagrangian: only terms of mass dimension $\leq 4$ lead to renormalizable interactions. Higher-dimensional operators are still allowed, but they correspond to \emph{effective} interactions that are suppressed at low energies and signal the presence of new physics at higher scales~\parencite{peskin,Weinberg}. 

A classical symmetry of the Lagrangian may not always survive the process of quantization. If it fails to do so, it is said to be anomalous. Chiral anomalies, specifically, arise from the regularization of fermion loops in triangle diagrams and can break the gauge symmetry at the quantum level. Since the gauge symmetry is the very principle that dictates the form of interactions and removes unphysical states, its violation would destroy the renormalizability and unitarity of the theory. Therefore, the particle content must be carefully chosen so that these potential anomalies cancel amongst the fermions, a non-trivial condition famously satisfied by the quarks and leptons of the Standard Model.

Furthermore, the stability of the vacuum is a prerequisite for a physically meaningful theory. This is ensured by demanding that the scalar potential, which governs the self-interactions of the Higgs field, is bounded from below. If the potential were unbounded, it would imply that the system could lower its energy indefinitely by rolling down the potential to field values of ever-greater magnitude, meaning no stable ground state could exist. For a renormalizable potential, this stability condition typically translates into the requirement that the quartic coupling constant $\lambda > 0$. However, this condition must hold not just at tree-level but also at the quantum level, as running couplings can change sign at different energy scales, potentially leading to metastability or instability of the vacuum.

Summarizing, the main constraints that a relativistic and renormalizable Lagrangian density must satisfy are:
\begin{itemize}
	\item \textbf{Poincaré (global) invariance:} the action must be invariant under Lorentz transformations and translations; the Lagrangian density is a Lorentz scalar and may change by at most a total derivative~\parencite{pall,jose1998classical}. \marginpar{\footnotesize In QFT, Poincaré invariance is assumed to be global. Promoting it to a local symmetry leads to gravity, with spin-2 fields (the graviton) as mediators. Perturbatively, such a theory is not renormalizable, so it lacks predictivity at high energies, although it can still be understood as an effective field theory.}
	\item \textbf{Hermiticity:} $\mathcal{L}$ must be Hermitian to ensure real observables and unitary evolution~\parencite{pall,peskin}.
	\item \textbf{Renormalizability and Operator Dimension:} The theory must be perturbatively renormalizable, meaning all ultraviolet divergences can be absorbed into a finite number of parameters. This requirement, determined via power-counting arguments, restricts interaction operators to have a \textbf{mass dimension $\leq 4$}. In natural units, where $\mathcal{L}$ has dimension [mass]$^4$, this allows only Yukawa couplings (dim 4), scalar $\phi^4$ interactions (dim 4), and gauge interactions (dim 4), while forbidding non-renormalizable operators like $\phi^6$ (dim 6)~\parencite{peskin,Weinberg}.
	\item \textbf{Absence of chiral anomalies:} gauge symmetries must be free of chiral (gauge) anomalies to ensure the consistency and unitarity of the quantum theory~\parencite{peskin,Weinberg,bertlmann1996anomalies}. In the Standard Model, the particle content is such that all gauge anomalies cancel exactly.
	\item \textbf{Stability of the potential:} the scalar potential must be bounded from below to guarantee the existence of a stable vacuum state. This typically requires that the quartic couplings in the potential are positive at the relevant energy scales.
\end{itemize}

These constraints drastically reduce the number of possible terms in the Lagrangian. As a result, the renormalizable interaction structures that typically arise are limited to: Yukawa couplings between fermions and scalars, scalar self-interactions (up to quartic order), and gauge interactions between matter fields and vector bosons.

Despite these powerful constraints, a vast number of possible interaction terms between the allowed fields remain. To further restrict the form of the Lagrangian and to describe fundamental forces, the concept of \textit{symmetry}—specifically \textit{gauge symmetry}—has proven to be our most powerful guiding principle.

The procedure is systematic: first, the spin 0 and spin 1/2 fields are organized into representations of a unitary (gauge) group $G$ such that the Lagrangian density is a global scalar (invariant) under $G$. This global symmetry is then ``promoted'' to a \textit{local symmetry} (where the group parameters can vary in spacetime) by replacing the ordinary derivatives $\partial_\mu$ with \textit{covariant derivatives} $\Dcov_\mu$ that incorporate new \textit{gauge fields} $B_\mu^A$~\parencite{pokorski2000gauge,freedman2012supergravity, Gallego2016,VanProeyen1999,Martin2012}.
This ``promotion'' is described in more detail below.

Given a Lagrangian density $\lag(\varphi_i,\partial_\mu \varphi_i)$, a given field $\varphi$ is said to be \textit{globally symmetric} under unitary transformations if the action is invariant under the variations of the fields $\varphi^{I}$ which are given, at the infinitesimal level, by:
\begin{equation}
	\delta_G \varphi^I = i\theta^A (T_A)^I_J \varphi^J,
\end{equation}
where $\theta^{A}$ are constant parameters of the transformation and the $T_{A}$ are the generators of the group $G$ in the appropriate representation. The corresponding finite unitary transformation is
\begin{equation}
	\mathcal{U}_G \equiv U(\theta)=\exp(i\theta^A T_A).
\end{equation}
The generators $T_A$ satisfy the Lie algebra of $G$:
\begin{equation}
	[T_A, T_B] = i f_{AB}^{\;\;C}T_C,
\end{equation}
where $f_{AB}^{\;\;C}$ are the structure constants of $G$.

To promote the global symmetry to a local one ($\theta^A \to \theta^A(x)$), the ordinary derivative $\partial_\mu$ is replaced by a \textit{covariant derivative} $\Dcov_\mu$. This new derivative is designed to transform covariantly under the gauge group, meaning $\Dcov_\mu \varphi \to U(x) (\Dcov_\mu \varphi)$, so that the kinetic terms $\lag_{\text{kin}} \sim (\Dcov_\mu \varphi)^\dagger (\Dcov^\mu \varphi)$ remain invariant. This is achieved by introducing a gauge field $B_\mu^A$ for each generator $T_A$ and defining:
\begin{equation}
	\Dcov_\mu = \partial_\mu - i g B_\mu^A T_A,
\end{equation}
where $g$ is the gauge coupling constant. The transformation law for the gauge fields that ensures the covariant transformation of $\Dcov_\mu$ is:
\begin{equation}
	\delta B_\mu^A = \partial_\mu \theta^A + g f_{BC}{}^A \theta^B B_\mu^C.\label{eq:gauge-transformation}
\end{equation}

The introduction of the gauge fields $B_\mu^A$ necessitates the addition of a kinetic term for them to the Lagrangian. This is constructed from the \textit{field strength tensor} $F_{\mu\nu}^A$, defined as the curvature of the covariant derivative:
\begin{equation}
	F_{\mu\nu}^A T_A = -\frac{i}{g} [\Dcov_\mu, \Dcov_\nu] = \partial_\mu B^A_\nu - \partial_\nu B^A_\mu + g f_{BC}{}^A B^B_\mu B^C_\nu.
\end{equation}
The gauge-invariant kinetic Lagrangian is then:
\begin{equation}
	\lag_{\text{gauge}} = -\frac{1}{4} \delta_{AB} F^A_{\mu\nu} F^{\mu\nu B}.
\end{equation}
Often, the rescaling $B_\mu^A \to g B_\mu^A$ is performed, which moves the coupling constant $g$ from the kinetic term to the covariant derivative, resulting in the more conventional form $\Dcov_\mu = \partial_\mu - i g B_\mu^A T_A$ and $\lag_{\text{gauge}} = -\frac{1}{4g^2} \delta_{AB} F^A_{\mu\nu} F^{\mu\nu B}$.

\marginpar{Note the absence of explicit mass terms for the gauge fields ($\sim M^2 B_\mu B^\mu$) and fermions ($\sim m \bar{\psi}\psi$). These are forbidden by gauge invariance for non-abelian fields and for chiral fermions. Mass terms can be generated via spontaneous symmetry breaking, as discussed below.}

A general, archetypal Lagrangian embodying these structures can be written as:
\begin{equation}\label{eq:generic-renorm-lag}
	\mathcal{L} = -\frac{1}{4} F_{\mu \nu}^A F^{A \mu \nu} + i \bar{\psi}^i \gamma^\mu D_\mu \psi^i + \left(\bar{\psi}_L^j \, \Gamma^j_k \, \Phi \, \psi_R^k + \text{h.c.}\right) + |D_\mu \Phi|^2 - V(\Phi)
\end{equation}
The terms correspond to: the kinetic term for gauge fields ($F_{\mu \nu}^A$); the kinetic term for fermions $\psi^i$ ($D_\mu$ is the gauge covariant derivative); Yukawa interactions between left- and right-handed fermions and scalars ($\Gamma^j_k$ is a Yukawa coupling matrix and $\Phi$ is a scalar field); the kinetic term for scalars; and the scalar potential $V(\Phi)$, which for a renormalizable, stable theory is $V(\Phi) = \mu^2 |\Phi|^2 + \lambda |\Phi|^4$ with $\lambda > 0$.

\subsubsection{Example}
As a way of illustration let us consider a renormalizable theory with a real scalar $\phi$ and a Dirac spinor $\psi$, and suppose that this theory is globally invariant under $U(1)$ phase transformations, i.e. the fields $\varphi\in\{\phi,\psi\}$ transform as $\varphi\mapsto e^{i\theta \hat Q}\varphi $ such that $\hat Q \psi = q \psi$ and $\hat Q \phi=0$. The free Lagrangian density is
\begin{equation}
	\mathcal L_{\text{free}}=\frac{1}{2} \partial^{\mu} \phi \partial_{\mu} \phi-\frac{1}{2}\mu^2\phi^2+\bar{\psi}(i \gamma_\mu  \partial^\mu-m) \psi.
\end{equation}
\marginpar{\footnotesize Note that for this vector-like $U(1)$ theory, the explicit fermion mass term $m\bar{\psi}\psi$ is gauge-invariant. This will not be the case for chiral gauge theories like the Standard Model.}

To add globally symmetric interaction terms, we must consider operators of mass dimension $\leq 4$. The most general renormalizable Lagrangian, invariant under the global $U(1)$ symmetry, is
\begin{equation}
	\begin{aligned}
		\mathcal L_{\text{global}}&=\frac{1}{2} \partial^{\mu} \phi \partial_{\mu} \phi-V(\phi)+\bar{\psi}(i \gamma_\mu  \partial^\mu-m) \psi + k_1 \phi\bar\psi\psi,
		\\
		V(\phi)&=\frac{1}{2}\mu^2\phi^2 +\frac{\alpha}{3!}\phi^3+\frac{\lambda}{4!}\phi^4.
	\end{aligned}
\end{equation}
The cubic and quartic terms in $V(\phi)$ are allowed as $\phi$ is neutral. The Yukawa coupling $k_1 \phi\bar\psi\psi$ is also gauge-invariant since the charges of $\bar\psi$, $\phi$, and $\psi$ sum to zero ($-q + 0 + q = 0$).

Promoting the global symmetry to a local one ($\theta \to \theta(x)$) requires introducing a gauge field $A_\mu$ and replacing ordinary derivatives with covariant derivatives:
\begin{equation}
	\mathcal D_\mu\varphi=(\partial_{\mu}-i g A_\mu\hat Q )\varphi
	\quad\Longrightarrow\quad
	\begin{cases}
		\mathcal D_\mu\phi=\partial_\mu \phi, & (\text{since } \hat Q\phi=0)\\
		\mathcal D_\mu\psi=(\partial_\mu - i g q A_\mu) \psi.
	\end{cases}
\end{equation}
The field strength tensor for the abelian $U(1)$ field is defined as $F_{\mu\nu} = \partial_\mu A_\nu - \partial_\nu A_\mu$. The locally invariant Lagrangian is then:
\begin{multline}
	\mathcal L_{\text{local}}=\frac{1}{2} \mathcal D^{\mu} \phi \mathcal D_{\mu} \phi-V(\phi)
	+\bar{\psi}i \gamma_\mu  \mathcal D^{\mu} \psi - m \bar{\psi}\psi
	+ k_1 \phi\bar\psi\psi-\frac{1}{4} F_{\mu\nu}F^{\mu\nu}.
\end{multline}


With these ingredients and principles, we are now equipped to understand the structure of the Standard Model Lagrangian, which will be discussed in the next section.

\begin{figure}[h!]
    \centering
    \begin{subfigure}[b]{0.48\textwidth}
        \centering
        \begin{fmffile}{feyngraph1} 
			\vspace{0.5cm}
            \begin{fmfgraph*}(80,60)
                \fmfleft{i1}
                \fmfright{o1,o2}
                
                \fmf{dashes,tension=2.0}{i1,v1}
                \fmf{fermion}{o1,v1}
                \fmf{fermion}{v1,o2}

                \fmflabel{$\phi$}{i1}
                \fmflabel{$\bar\psi$}{o1}
                \fmflabel{$\psi$}{o2}

				\fmfv{lab=$ik_1$, lab.dist=0.25cm, lab.angle=115}{v1}
            \end{fmfgraph*}
			\vspace{0.5cm}
        \end{fmffile}
        \caption{Yukawa coupling with a scalar $\phi$.}
        \label{fig-yukawa-scalar}
    \end{subfigure}
    \hfill
    \begin{subfigure}[b]{0.48\textwidth}
        \centering
        \begin{fmffile}{feyngraph2}
			\vspace{0.5cm}
            \begin{fmfgraph*}(80,60)
                \fmfleft{i1}
                \fmfright{o1,o2}
                
                \fmf{photon,tension=2.0}{i1,v1}
                \fmf{fermion}{o1,v1}
                \fmf{fermion}{v1,o2}

                \fmflabel{$\gamma$}{i1}
                \fmflabel{$\bar\psi$}{o1}
                \fmflabel{$\psi$}{o2}

				\fmfv{lab=$ig$, lab.dist=0.25cm, lab.angle=115}{v1}
            \end{fmfgraph*}
			\vspace{0.5cm}
        \end{fmffile}
        \caption{Interaction with a photon $\gamma$.}
        \label{fig-qed-photon}
    \end{subfigure}
	\begin{subfigure}[b]{0.48\textwidth}
        \centering
		\begin{fmffile}{feyngraph3}
			\vspace{1.0cm}
			\begin{fmfgraph*}(80,60)
				\fmfleft{i1}
				\fmfright{o1,o2}

				\fmf{dashes}{i1,v1}
				\fmf{dashes}{v1,o1}
				\fmf{dashes}{v1,o2}

				\fmflabel{$\phi$}{i1}
				\fmflabel{$\phi$}{o1}
				\fmflabel{$\phi$}{o2}

				\fmfv{lab=$i\alpha$, lab.dist=0.25cm, lab.angle=115}{v1}
			\end{fmfgraph*}
			\vspace{0.5cm}
		\end{fmffile}
		\caption{Triple scalar coupling.}
		\label{fig-triple-scalar}
	\end{subfigure}
	\begin{subfigure}[b]{0.48\textwidth}
        \centering
		\begin{fmffile}{feyngraph4}
			\vspace{1.0cm}
			\begin{fmfgraph*}(80,60)
				\fmfleft{i1,i2}
				\fmfright{o1,o2}

				\fmf{dashes}{i1,v1}
				\fmf{dashes}{i2,v1}
				\fmf{dashes}{v1,o1}
				\fmf{dashes}{v1,o2}

				\fmflabel{$\phi$}{i1}
				\fmflabel{$\phi$}{i2}
				\fmflabel{$\phi$}{o1}
				\fmflabel{$\phi$}{o2}

				\fmfv{lab=$i\lambda$, lab.dist=0.3cm, lab.angle=90}{v1}
			\end{fmfgraph*}
			\vspace{0.5cm}
		\end{fmffile}
		\caption{Quartic scalar coupling.}
		\label{fig-quartic-scalar}
	\end{subfigure}
    \caption{Feynman diagrams for Yukawa coupling, gauge boson coupling and quartic scalar coupling.}
\end{figure}

%TO DO -> Feynman Diagramans
%TO DO -> Extend about the higgs mechanism
