
\section{Deficiencies of Standard Model and New Physics}

{\Large Pending to be updated} %TO DO -> REFRESH FOR ACTUAL STATE

While these and other successes of the Standard Model are an achievement for the field of particle physics, it is well known that this cannot be the ultimate theory of fundamental particles and interactions. Even though the Standard Model is currently the best description there is of the subatomic world, it does not explain the complete picture; there are also important questions that it does not answer and it is also surrounded by different irregularities. Some of them are completely incompatible with the current Standard Model, and strongly suggest that the Standard Model requires a consistent extension to solve experimental and theoretical problems that we will label as the cosmological problems, phenomenological problems, and theoretical problems. Below we will list very briefly the main representatives of these categories.



\subsection{Theoretical problems}

\begin{description}
	\item[Hierarchy problem] Is the problem concerning the large discrepancy between aspects of the weak force and gravity. Both of these forces involve constants of nature, the Fermi constant for the weak force and the Newtonian constant of gravitation for gravity. If the Standard Model is used to calculate the quantum corrections to Fermi's constant, it appears that Fermi's constant is surprisingly large and is expected to be closer to Newton's constant unless there is a delicate cancellation between the bare value of Fermi's constant and the quantum corrections to it. 
	
	In the Standard Model context, the Higgs boson is much lighter than the energy scale on which the standard model is considered valid (ideally the Plank mass), and the quantum corrections to the Higgs mass are on the order of this energy scale; it would inevitably make the Higgs and fermions masses huge, comparable to the scale at which new physics appears, unless there is an incredible fine-tuning cancellation between the quadratic radiative corrections and the bare mass. This level of fine-tuning is deemed unnatural.
	\item[Strong CP problem] QCD Lagrangian supports a term associated with the strength tensor dual for gluons that break CP symmetry in the strong interaction sector. Experimentally, however, no such violation has been found, implying that the coefficient of this term is fine tunned to zero. 
	\item[Quantum triviality] Suggests that it may not be possible to create a consistent quantum field theory involving elementary scalar Higgs particles because for high momentum particles the renormalization presents inconsistencies unless the renormalization of the charges becomes null, and therefore not interacting, \textit{i.e.} trivial. Nevertheless, because the Higgs boson plays a central role in the Standard Model of particle physics, the question of triviality in Higgs models is of great importance. 
	\item[Number of parameters and Unexplained relations] In total, the standard model has too many free parameters (19 in total) that are obtained experimentally, and there are indications that several of them may be correlated, however the origin of these correlations is beyond the standard model.
	
	For example, Yoshio Koide's empirical formula~\parencite{0505220}
	$$
	\frac{m_{e}+m_{\mu}+m_{\tau}}{\left(\sqrt{m_{e}}+\sqrt{m_{\mu}}+\sqrt{m_{\tau}}\right)^{2}}=0.666661(7) \approx \frac{2}{3}
	$$
	seems to indicate that there is a way to predict the masses of leptons.
	
\end{description}
\subsection{Cosmological problems}
\begin{description}
	\item[Gravity] Although the Standard Model describes the three important fundamental forces at the subatomic scale, it does not include gravity. However, at larger scales, gravity becomes present and is described by Einstein's theory of general relativity, in which gravity rather than a force is a property that measures the deformation of spacetime then, the most of the conventional machinery of perturbative QFT is profoundly incompatible with the general relativistic framework~\parencite{book:217893}, and a theory of quantum gravity with which we are enabled to perform calculations has yet to be discovered.
	
	
	
	
	\item[Dark matter] Within the framework of Einstein's general relativity, the cosmological standard model ($\Lambda$CDM) is, like the standard model of particle physics, one of the most successful theories of the 20th century. $\Lambda$CDM it is based on a very specific density of matter that can be explained with ordinary matter from the standard model of particles, baryonic matter; according to $\Lambda$CDM, in addition to baryonic matter, there is a kind of matter five times more abundant than baryonic matter, which does not interact electrically (therefore it is dark) and non-relativistic (therefore it is cold), known as cold dark matter (CDM).  Yet, the Standard Model does not supply any fundamental particles that are good dark matter candidates.
	\item[Dark energy] Moreover, according to Lambda CDM only 31\% of the energy that makes up the universe is matter, the remaining 69\% of the universe's energy should consist of the so-called dark energy, a constant energy density for the vacuum ($\Lambda$). If we try to explain dark energy in terms of vacuum energy only from the standard model lead to a mismatch of 120 orders of magnitude~\parencite{Adler1995}, sometimes called \textit{The Worst Theoretical Prediction in the History of Physics}~\parencite{book:15261}; a bit sensationalist title to indicate the fact that we do not fully understand the composition of the particle spectrum of the universe.
	
	\item[Matter-antimatter asymmetry] In the observable universe there is more matter than antimatter. In 1967, Andrei Sakharov proposed a set of three necessary conditions that a baryon-generating interaction must satisfy to produce matter and antimatter at different rates~\parencite{1967JETPL...5...24S}. While the standard model can satisfy these three conditions~\parencite{PhysRevLett.37.8,ph/0609145},  it satisfies them at three different energy scales and therefore presents difficulties in the capability to explain the  matter-antimatter asymmetry~\parencite{robinson2011symmetry}. 
	
\end{description}
\subsection{Phenomenological problems}\label{pheno_bsm}
\begin{description}
	\item[Neutrino masses] In the standard model, the right chiral component of neutrinos is not part of the composition of fermionic fields because if they were present they would not interact and consequently neutrinos have no mass. However, the precision measurement~\parencite{Abe_2008} of the mixing matrix for neutrino oscillations has shown that neutrinos change flavour in free flight and in turn that the three neutrino flavours cannot have identical mass, meaning that all three cannot have zero mass. There is no single way to extend the standard model to include masses to neutrinos and even more to explain their value so close to zero and results in the open problem confirmed at the phenomenological level present in the standard model.
	\item[Anomalous B-mesons decay] A B-meson is a bound state made up of an quark-antiquark pair where one of them comes from a $b$-quark. Various experimental results~\parencite{PhysRevLett.109.101802, PhysRevLett.115.111803,Altmannshofer_2015, Hurth_2016,arxiv.2103.11769} have suggested a surplus over Standard Model predictions in its decays to D-mesons along with a $\tau$, $\nu_\tau$ doublet. While none of them have reached the statistical threshold of 5 $\sigma$ to declare a break from the standard model, the Capdevilaa's meta-analysis of all available data reported a $5.0\sigma$ deviation from SM~\parencite{Capdevila_2018}. 
	\item[Anomalous magnetic dipole moment of muon]  Unlike the extraoirdinary agreement between theory and experiment with the magnetic dipole moment of the electron~\parencite{PhysRevLett.97.030801}; in the case of the muon, the measurement of Fermilab's Muon g-2 experiment has presented an apparent discrepancy  with an accuracy of 4.2 $\sigma$~\parencite{arxiv.1311.2198, Abi_2021} which strengthen evidence of new physics in the muon sector and apparently in the violation of lepton universality of the standard model. 
	\item[Anomalous mass of the W boson] Results from the CDF Collaboration, reported in April 2022, indicate that the mass of a W boson exceeds the mass predicted by the Standard Model with a significance of 7 $\sigma$~\parencite{abk1781}. However, this very highly accurate result, unlike the anomaly in B-meson Decay, is in tension with the results of Atlas, LHCb, LEP and D0 II~\parencite{Aaboud_2018,jhep012022036,Schael_2006,Abazov_2012,}. Certainly, a review of all the information we possess so far must be done to determine if this anomaly is a window into new physics beyond the standard model.
	\item[CCA and $q\bar q \mapsto e^+ e^-$] It has been observed that certain nuclear beta decays happen less frequently than expected~\parencite{PhysRevC.102.045501}. This tension, called the Cabibbo Angle anomaly (CAA), displays a significance around $3 \sigma$~\parencite{1674-1137-40-10-100001}, and can again be interpreted as a sign that electrons and muons behave more differently than predicted by the SM~\parencite{PhysRevLett.125.111801}. Furthermore, the CMS experiment at CERN observed more very high-energetic electrons in proton-proton collisions $\left(q \bar{q} \rightarrow e^{+} e^{-}\right)$ compared to muons than expected~\parencite{Sirunyan2021}.
\end{description}
