\section{Deficiencies of SM and New Physics}
Despite its remarkable successes, the SM cannot be the ultimate theory of fundamental particles and interactions. Although it provides the best current description of the subatomic world, it faces various shortcomings and leaves important questions unanswered. Some of these limitations are fundamentally incompatible with the SM framework and strongly motivate consistent extensions to address cosmological, phenomenological, and theoretical challenges.

In this thesis, we are particularly interested in \emph{phenomenological} signatures and experimental anomalies that point toward physics beyond the Standard Model. Therefore, we begin by highlighting the phenomenological problems most relevant to our work---direct experimental observations that the SM cannot accommodate or explain. We then briefly discuss cosmological and theoretical motivations for new physics to provide a broader context.

\subsection{Phenomenological problems}\label{pheno_bsm}
\begin{description}
  \item[Anomalies in $b$-hadron decays] Lepton-flavour universality (LFU) is a fundamental prediction of the SM, stating that the electroweak couplings of leptons are independent of their flavour. Measurements of ratios such as $R_K = \mathcal{B}(B\to K\mu^+\mu^-)/\mathcal{B}(B\to Ke^+e^-)$ and $R_{K^*}$ in $b\to s\ell\ell$ transitions showed persistent deviations from unity at the $\sim$2--3$\sigma$ level in earlier LHCb data, hinting at possible new physics coupling preferentially to muons. However, the 2022 LHCb analyses with the full Run~1+2 dataset report $R_K$ and related ratios consistent with the SM within uncertainties~\parencite{LHCb:2022RK}, and the tension has largely subsided. Angular-observable tensions (e.g. $P'_5$) persist at lower significance and are sensitive to hadronic uncertainties. For $b\to c\tau\nu$, updated averages (Belle/Belle~II, HFLAV) have moved closer to the SM; any deviation is now at the $\sim$2--3$\sigma$ level depending on inputs~\parencite{HFLAV2023,BelleII:2023RDstar}.

  \item[Anomalous magnetic dipole moment of the muon] The muon's magnetic moment can be expressed as $\vec{\mu} = g_\mu (e/2m_\mu)\vec{S}$, where the gyromagnetic ratio $g_\mu$ is predicted to be exactly 2 by the Dirac equation. Quantum corrections shift it slightly, and the anomalous magnetic moment $a_\mu = (g_\mu - 2)/2$ provides a sensitive probe of virtual contributions from all SM particles and potential new physics. A long-standing discrepancy has existed between the SM prediction and the experimental value measured at Brookhaven (BNL). Fermilab's 2023 update improved the experimental precision and confirmed the BNL result~\parencite{FNALg2_2023}. The significance of the discrepancy with the SM now depends critically on the hadronic vacuum polarization input: using the 2020 theory white paper gives a $\sim$4--5$\sigma$ deviation~\parencite{Aoyama:2020WhitePaper}, whereas recent lattice-QCD evaluations (e.g. BMW) and new $e^+e^-\!\to\pi^+\pi^-$ data from CMD-3 tend to reduce the tension~\parencite{BMW:2021Nature,CMD3:2023pipii}.
  
  \item[Neutrino masses] Precision oscillation data require non-zero neutrino masses and mixing. Global fits prefer normal ordering, but the mass ordering and the Dirac CP phase remain unestablished; see the latest NuFIT summary~\cite{NuFIT2024}. Direct kinematic limits from KATRIN have pushed the effective electron-neutrino mass into the sub-eV regime (about $0.5$ eV at 90\% CL)~\parencite{KATRIN2022,KATRIN2024}.

  \item[W-boson mass] The precise CDF II determination remains in strong tension with the SM and with other experiments~\parencite{CDFII:2022Wmass}. Subsequent ATLAS and earlier LEP/LHCb results are compatible with the SM; see the PDG 2024 summary for a balanced overview~\parencite{PDG2024}.

  \item[CCA and $q\bar q \mapsto e^+ e^-$] First-row CKM unitarity tests show a mild ($\sim$2--3$\sigma$) tension depending on treatment of radiative/nuclear corrections and kaon inputs~\parencite{PDG2024,Seng:2018PRL,HardyTowner:2020}. High-mass Drell--Yan lepton universality measurements at the LHC with Run~2/3 data are generally consistent with the SM within uncertainties.
\end{description}

\subsection{Cosmological problems}
\begin{description}
		\item[Gravity and the cosmological-constant problem] A UV-complete quantum theory of gravity remains unknown. At low energies, general relativity can be treated as an effective field theory, but it is non-renormalizable~\parencite{Donoghue:1994EFT,Burgess:2004QG}. Moreover, the observed vacuum energy (cosmological constant) driving cosmic acceleration is many orders of magnitude smaller than naive quantum-field-theory estimates, posing a severe naturalness problem~\parencite{Weinberg:1989CC}.

  \item[Dark matter] Cosmological and astrophysical data require a cold, non-baryonic,  component with $\Omega_c h^2 \simeq 0.12$~\parencite{Planck2018}, consistent with the thermal weakly interacting massive particle (WIMP) production hypothesis. The SM does not have a  suitable dark matter particle candidate: active neutrinos are too light and hot, and baryons are limited by BBN/CMB (Baryon Acoustic Oscillations / Cosmic Microwave Background) \textcolor{red}{AF: Por favor incluir algunas referencias}. This points to new degrees of freedom BSM. Dark matter direct-detection experiments continue to improve sensitivity with null results. Recent LZ and XENONnT runs set the strongest spin-independent limits over a wide mass range~\parencite{LZ:2022first,LZ:2023full,XENONnT:2023}.

  \item[Matter--antimatter asymmetry (baryon asymmetry)] The Universe exhibits a nonzero baryon asymmetry, $\eta_B \simeq 6\times10^{-10}$ from CMB/BBN~\parencite{Planck2018}. The SM fails quantitatively: for $m_H=125$ GeV the electroweak transition is a crossover (no sufficient departure from equilibrium), and CKM CP violation is many orders of magnitude too small to generate the observed asymmetry. Therefore, additional CP violation and/or new dynamics are required, e.g. leptogenesis or electroweak baryogenesis~\parencite{Sakharov:1967,Davidson:2008Leptogenesis,Morrissey:2012EWB}.

  \item[Dark energy] Late-time acceleration is consistent with a cosmological constant with $w\approx -1$~\parencite{DESIY1:2024}. In SM+GR there is no mechanism to obtain such a tiny but nonzero vacuum energy; naive quantum-field-theory estimates are vastly larger, implying extreme fine-tuning (the cosmological-constant problem)~\parencite{Weinberg:1989CC}. The $H_0$ tension persists and could reflect systematics or new physics.
\end{description}

\subsection{Theoretical problems}

\begin{description}
    \item[Hierarchy problem] Is the problem concerning the large discrepancy between aspects of the weak force and gravity. Both of these forces involve constants of nature: the Fermi constant for the weak force and the Newtonian constant of gravitation for gravity. If the SM is used to calculate the quantum corrections to Fermi's constant, it appears that Fermi's constant is surprisingly large and is expected to be closer to Newton's constant, unless there is a delicate cancellation between the bare value of Fermi's constant and the quantum corrections to it. 
    
    In the SM context, the Higgs boson mass is much lighter than the energy scale on which the SM is considered valid (ideally, around the Plank mass at $1.221\times 10^{19}\,GeV$), and the quantum corrections to the Higgs mass are on the order of this energy scale. This  would inevitably make the Higgs and fermions masses huge, comparable to the scale at which new physics appears, unless there is an incredible fine-tuning cancellation between the quadratic radiative corrections and the bare mass. This level of fine-tuning is deemed unnatural.

    \item[Strong CP problem] The QCD Lagrangian supports a term associated with the strength tensor dual for gluons that break CP symmetry in the strong interaction sector. Experimentally, however, no such violation has been found, implying that the coefficient of this term is fine tunned to zero. 
    \item[Quantum triviality] Suggests that it may not be possible to create a consistent quantum field theory involving elementary scalar Higgs particles, because, for high momentum particles, the renormalization presents inconsistencies unless the renormalization of the charges becomes null, and therefore not interacting, \textit{i.e.} trivial. Nevertheless, because the Higgs boson plays a central role in the SM of particle physics, the question of triviality in Higgs models is of great importance. 
    \item[Number of parameters and Unexplained relations] In total, the SM has too many free parameters (19 in total) that are obtained experimentally. There are indications that several of these free parameters may be correlated. However, the origin of these correlations is beyond the SM. For example, Yoshio Koide's empirical formula~\parencite{0505220}
    $$
    \frac{m_{e}+m_{\mu}+m_{\tau}}{\left(\sqrt{m_{e}}+\sqrt{m_{\mu}}+\sqrt{m_{\tau}}\right)^{2}}=0.666661(7) \approx \frac{2}{3},
    $$
    \textcolor{red}{seems to indicate that there is a way to predict the masses of leptons.....AF: Esto se siente inconcluso: ¿cómo concluye esto?}
    
\end{description}

