\section{Deficiencies of Standard Model and New Physics}
While these and other successes of the Standard Model are an achievement for the field of particle physics, it is well known that this cannot be the ultimate theory of fundamental particles and interactions. Even though the Standard Model is currently the best description there is of the subatomic world, it does not explain the complete picture; there are also important questions that it does not answer and it is also surrounded by different irregularities. Some of them are completely incompatible with the current Standard Model, and strongly suggest that the Standard Model requires a consistent extension to solve experimental and theoretical problems that we will label as the cosmological problems, phenomenological problems, and theoretical problems. Below we will list very briefly the main representatives of these categories.

\subsection{Cosmological problems}
\begin{description}
		\item[Gravity and the cosmological-constant problem] A UV-complete quantum theory of gravity remains unknown; at low energies general relativity can be treated as an effective field theory, but it is non-renormalizable~\parencite{Donoghue:1994EFT,Burgess:2004QG}. Moreover, the observed vacuum energy (cosmological constant) driving cosmic acceleration is many orders of magnitude smaller than naive quantum-field-theory estimates, posing a severe naturalness problem~\parencite{Weinberg:1989CC}.

  \item[Dark matter] Cosmological and astrophysical data require a cold, non-baryonic,  component with $\Omega_c h^2 \simeq 0.12$~\parencite{Planck2018}. The Standard Model has no suitable particle: active neutrinos are too light and hot, and baryons are limited by BBN/CMB (Baryon Acoustic Oscillations / Cosmic Microwave Background). This points to new degrees of freedom BSM. Direct-detection experiments continue to improve sensitivity with null results; recent LZ and XENONnT runs set the strongest spin-independent limits over a wide mass range~\parencite{LZ:2022first,LZ:2023full,XENONnT:2023}.

  \item[Matter--antimatter asymmetry (baryon asymmetry)] The Universe exhibits a nonzero baryon asymmetry, $\eta_B \simeq 6\times10^{-10}$ from CMB/BBN~\parencite{Planck2018}. The SM fails quantitatively: for $m_H=125$ GeV the electroweak transition is a crossover (no sufficient departure from equilibrium), and CKM CP violation is many orders of magnitude too small to generate the observed asymmetry. Therefore additional CP violation and/or new dynamics are required, e.g. leptogenesis or electroweak baryogenesis~\parencite{Sakharov:1967,Davidson:2008Leptogenesis,Morrissey:2012EWB}.

  \item[Dark energy] Late-time acceleration is consistent with a cosmological constant with $w\approx -1$~\parencite{DESIY1:2024}. In SM+GR there is no mechanism to obtain such a tiny but nonzero vacuum energy; naive quantum-field-theory estimates are vastly larger, implying extreme fine-tuning (the cosmological-constant problem)~\parencite{Weinberg:1989CC}. The $H_0$ tension persists and could reflect systematics or new physics.
\end{description}

\subsection{Theoretical problems}

\begin{description}
    \item[Hierarchy problem] Is the problem concerning the large discrepancy between aspects of the weak force and gravity. Both of these forces involve constants of nature, the Fermi constant for the weak force and the Newtonian constant of gravitation for gravity. If the Standard Model is used to calculate the quantum corrections to Fermi's constant, it appears that Fermi's constant is surprisingly large and is expected to be closer to Newton's constant unless there is a delicate cancellation between the bare value of Fermi's constant and the quantum corrections to it. 
    
    In the Standard Model context, the Higgs boson is much lighter than the energy scale on which the standard model is considered valid (ideally the Plank mass), and the quantum corrections to the Higgs mass are on the order of this energy scale; it would inevitably make the Higgs and fermions masses huge, comparable to the scale at which new physics appears, unless there is an incredible fine-tuning cancellation between the quadratic radiative corrections and the bare mass. This level of fine-tuning is deemed unnatural.

    \item[Strong CP problem] QCD Lagrangian supports a term associated with the strength tensor dual for gluons that break CP symmetry in the strong interaction sector. Experimentally, however, no such violation has been found, implying that the coefficient of this term is fine tunned to zero. 
    \item[Quantum triviality] Suggests that it may not be possible to create a consistent quantum field theory involving elementary scalar Higgs particles because for high momentum particles the renormalization presents inconsistencies unless the renormalization of the charges becomes null, and therefore not interacting, \textit{i.e.} trivial. Nevertheless, because the Higgs boson plays a central role in the Standard Model of particle physics, the question of triviality in Higgs models is of great importance. 
    \item[Number of parameters and Unexplained relations] In total, the standard model has too many free parameters (19 in total) that are obtained experimentally, and there are indications that several of them may be correlated, however the origin of these correlations is beyond the standard model.
    
    For example, Yoshio Koide's empirical formula~\parencite{0505220}
    $$
    \frac{m_{e}+m_{\mu}+m_{\tau}}{\left(\sqrt{m_{e}}+\sqrt{m_{\mu}}+\sqrt{m_{\tau}}\right)^{2}}=0.666661(7) \approx \frac{2}{3}
    $$
    seems to indicate that there is a way to predict the masses of leptons.
    
\end{description}
\subsection{Phenomenological problems}\label{pheno_bsm}
\begin{description}
  \item[Neutrino masses] Precision oscillation data continue to require non-zero neutrino masses and mixing. Global fits still prefer normal ordering, but the mass ordering and the Dirac CP phase remain unestablished; see the latest NuFIT summary~\parencite{NuFIT2024}. Direct kinematic limits from KATRIN have pushed the effective electron-neutrino mass into the sub-eV regime (about $0.5$ eV at 90\% CL)~\parencite{KATRIN2022,KATRIN2024}.

  \item[Anomalies in $b$-hadron decays] The earlier hints of lepton-flavour universality violation in $b\to s\ell\ell$ (e.g. $R_{K^{(\ast)}}$) have largely subsided. The 2022 LHCb analyses with the full Run 1+2 dataset report $R_K$ and related ratios consistent with the SM within uncertainties~\parencite{LHCb:2022RK}. Angular-observable tensions (e.g. $P'_5$) persist at lower significance and are sensitive to hadronic uncertainties. For $b\to c\tau\nu$, updated averages (Belle/Belle~II, HFLAV) have moved closer to the SM; any deviation is now at the $\sim$2--3$\sigma$ level depending on inputs~\parencite{HFLAV2023,BelleII:2023RDstar}.

  \item[Anomalous magnetic dipole moment of the muon] Fermilab's 2023 update improved the experimental precision~\parencite{FNALg2_2023}. The significance of any discrepancy with the SM now depends on the hadronic vacuum polarization input: using the 2020 theory white paper gives a $\sim$4$\sigma$ deviation~\parencite{Aoyama:2020WhitePaper}, whereas lattice-QCD evaluations (e.g. BMW) and new $e^+e^-\!\to\pi^+\pi^-$ input from CMD-3 tend to reduce the tension~\parencite{BMW:2021Nature,CMD3:2023pipii}.

  \item[W-boson mass] The precise CDF II determination remains in strong tension with the SM and with other experiments~\parencite{CDFII:2022Wmass}. Subsequent ATLAS and earlier LEP/LHCb results are compatible with the SM; see the PDG 2024 summary for a balanced overview~\parencite{PDG2024}.

  \item[CCA and $q\bar q \mapsto e^+ e^-$] First-row CKM unitarity tests still show a mild ($\sim$2--3$\sigma$) tension depending on treatment of radiative/nuclear corrections and kaon inputs~\parencite{PDG2024,Seng:2018PRL,HardyTowner:2020,Cirigliano:2022}. High-mass Drell--Yan lepton universality measurements at the LHC with Run~2/3 data are generally consistent with the SM within uncertainties.
\end{description}

