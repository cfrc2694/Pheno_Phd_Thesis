\section{Deficiencies of SM and New Physics}
Despite its remarkable successes, the SM cannot be the ultimate theory of fundamental particles and interactions. Although it provides the best current description of the subatomic world, it faces various shortcomings and leaves important questions unanswered. Some of these limitations are fundamentally incompatible with the SM framework and strongly motivate consistent extensions to address cosmological, phenomenological, and theoretical challenges.

In this thesis, we are particularly interested in \emph{phenomenological} signatures and experimental anomalies that point toward physics beyond the Standard Model. Therefore, we begin by highlighting the phenomenological problems most relevant to our work---direct experimental observations that the SM cannot accommodate or explain. We then briefly discuss cosmological and theoretical motivations for new physics to provide a broader context.

\subsection{Phenomenological problems}\label{pheno_bsm}
\begin{description}
  \item[Anomalies in $b$-hadron decays] Lepton-flavour universality (LFU) is a fundamental prediction of the SM, stating that the electroweak couplings of leptons are independent of their flavour. Measurements of ratios such as $R_K = \mathcal{B}(B\to K\mu^+\mu^-)/\mathcal{B}(B\to Ke^+e^-)$ and $R_{K^*}$ in $b\to s\ell\ell$ transitions showed persistent deviations from unity at the $\sim$2--3$\sigma$ level in earlier LHCb data, hinting at possible new physics coupling preferentially to muons. However, the 2022 LHCb analyses with the full Run~1+2 dataset report $R_K$ and related ratios consistent with the SM within uncertainties~\parencite{LHCb:2022RK}, and the tension has largely subsided. Angular-observable tensions (e.g. $P'_5$) persist at lower significance and are sensitive to hadronic uncertainties. For $b\to c\tau\nu$, updated averages (Belle/Belle~II, HFLAV) have moved closer to the SM; any deviation is now at the $\sim$2--3$\sigma$ level depending on inputs~\parencite{HFLAV2023,BelleII:2023RDstar}. While current data are largely consistent with the SM, the pattern of earlier deviations motivates searches for flavour-violating interactions beyond the SM gauge structure.

  \item[Anomalous magnetic dipole moment of the muon] The gyromagnetic ratio $g_\mu$ of the muon is predicted to be exactly 2 by the Dirac equation, but quantum corrections shift it slightly. The anomalous magnetic moment $a_\mu = (g_\mu - 2)/2$ provides a sensitive probe of virtual contributions from all SM particles and potential new physics. A long-standing discrepancy has existed between the SM prediction and the experimental value measured at Brookhaven (BNL). Fermilab's 2023 update improved the experimental precision and confirmed the BNL result~\parencite{FNALg2_2023}. The significance of the discrepancy with the SM now depends critically on the hadronic vacuum polarization input: using the 2020 theory white paper gives a $\sim$4--5$\sigma$ deviation~\parencite{Aoyama:2020WhitePaper}, whereas recent lattice-QCD evaluations (e.g. BMW) and new $e^+e^-\!\to\pi^+\pi^-$ data from CMD-3 tend to reduce the tension~\parencite{BMW:2021Nature,CMD3:2023pipii}. The persistent tension between experiment and theory-white-paper predictions suggests either missing SM contributions or new weakly-coupled particles contributing to the muon's quantum corrections.
  
  \item[Neutrino masses] Precision oscillation data require non-zero neutrino masses and mixing. Global fits prefer normal ordering, but the mass ordering and the Dirac CP phase remain unestablished; see the latest NuFIT summary~\cite{NuFIT2024}. Direct kinematic limits from KATRIN have pushed the effective electron-neutrino mass into the sub-eV regime (about $0.5$ eV at 90\% CL)~\parencite{KATRIN2022,KATRIN2024}. This evidence for neutrino oscillations suggests that the SM must be extended to accommodate neutrino masses.

  \item[W-boson mass] The precise CDF II determination, $m_W = 80433.5 \pm 9.4$ MeV, remains in strong tension with the SM prediction and with other experimental measurements~\parencite{CDFII:2022Wmass}. Recent measurements from ATLAS, CMS (2024), and earlier LEP/LHCb results are all compatible with the SM; see the PDG 2024 summary for a balanced overview~\parencite{PDG2024}. The CDF~II result stands as an outlier, and if confirmed by future measurements, would require modifications to electroweak radiative corrections or new particles contributing to the $W$ mass; otherwise, it likely reflects unaccounted systematic effects in the CDF~II analysis.

  \item[CCA and $q\bar q \mapsto e^+ e^-$] First-row CKM unitarity tests show a mild ($\sim$2--3$\sigma$) tension depending on treatment of radiative/nuclear corrections and kaon inputs~\parencite{PDG2024,Seng:2018PRL,HardyTowner:2020}. High-mass Drell--Yan lepton universality measurements at the LHC with Run~2/3 data are generally consistent with the SM within uncertainties. Persistent CKM unitarity violations would signal either incomplete treatment of SM radiative corrections or new physics affecting weak decays.
\end{description}

\subsection{Theoretical problems}

\begin{description}
   \item[Hierarchy problem] The hierarchy problem concerns the enormous disparity between the electroweak scale ($v \sim 246$ GeV) and the Planck scale ($M_{\mathrm{Pl}} \sim 10^{19}$ GeV), reflected in the ratio $G_F/G_N \sim 10^{33}$ between Fermi's constant (weak force) and Newton's constant (gravity)~\parencite{Veltman:1981Hierarchy}. In the SM, the Higgs mass receives quadratically divergent quantum corrections from loops of heavy particles, which would naturally drive it to the highest energy scale at which the theory is valid. Keeping the Higgs mass at its observed value requires an unnatural fine-tuning at the level of one part in $10^{34}$~\parencite{Susskind:1979Technicolor,Barbieri:1988Naturalness}, suggesting a fundamental problem on the understanding of the electroweak scale. New physics at the TeV scale is required to stabilize the Higgs mass and explain the hierarchy without extreme fine-tuning.

    \item[Strong CP problem] The QCD Lagrangian allows a CP-violating term proportional to a dimensionless parameter $\theta$. This term would induce a non-zero electric dipole moment (EDM) of the neutron proportional to $\theta$. However, experimental measurements constrain $\theta < 10^{-10}$~\parencite{Abel:2020nEDM}, implying an unnatural fine-tuning of this parameter to nearly zero without any known symmetry principle to explain it. Naively, one would expect $\theta \sim \mathcal{O}(1)$, yet nature requires it to be smaller than one part in $10^{10}$, suggesting the presence of unexplained physics. A dynamical mechanism or new symmetry principle beyond the SM is required to explain why $\theta$ is so small.
    
    \item[Quantum triviality] The SM with an elementary scalar Higgs may not be a consistent quantum field theory at arbitrarily high energies. The Higgs self-coupling $\lambda$ runs with energy scale according to its renormalization group equation, and for the observed Higgs mass ($m_H \sim 125$ GeV), $\lambda$ grows at high energies and eventually diverges at a finite energy scale known as the Landau pole, estimated to be around $10^{17}$--$10^{18}$ GeV~\parencite{Callaway:1988Triviality,Hambye:1996Landau}. This implies that the SM cannot be extrapolated to arbitrarily high scales as a fundamental theory and must be replaced by new physics or viewed as an effective theory valid only below the Landau pole. Alternatively, if $\lambda$ is required to remain finite at all scales, it must asymptotically vanish, rendering the Higgs non-interacting (\textit{i.e.}, trivial) at high energies~\parencite{Luscher:1988Triviality}. The SM must either be replaced by a more fundamental theory at high scales or supplemented with new degrees of freedom to regulate the Higgs self-coupling. 
    
    \item[Number of parameters and Unexplained relations] The SM contains 19 free parameters that must be determined experimentally: 9 fermion masses, 3 gauge couplings, 3 CKM mixing angles and 1 CP-violating phase, 2 Higgs sector parameters, and the QCD $\theta$ parameter. The theory provides no explanation for the observed values or patterns among these parameters. Intriguingly, several empirical relations suggest hidden structure. For example, the charged-lepton masses satisfy Yoshio Koide's formula~\parencite{0505220}
    $$
    \frac{m_{e}+m_{\mu}+m_{\tau}}{\left(\sqrt{m_{e}}+\sqrt{m_{\mu}}+\sqrt{m_{\tau}}\right)^{2}}=0.666661(7) \approx \frac{2}{3},
    $$
    accurate to better than $0.01\%$, sugesting that this particular choice flavour structure in the leptons is not accidental. These patterns hint at an underlying flavor symmetry or dynamical mechanism beyond the SM that could reduce the number of free parameters and explain the observed mass spectrum and mixing structure. A more fundamental theory with fewer parameters and predictive power for fermion masses and mixings is needed to replace the SM's arbitrary Yukawa couplings.
    
\end{description}

\subsection{Cosmological problems}
\begin{description}
	\item[Gravity and the cosmological-constant problem] A UV-complete quantum theory of gravity remains unknown. At low energies, general relativity can be treated as an effective field theory, but it is non-renormalizable~\parencite{Donoghue:1994EFT,Burgess:2004QG}. Moreover, the observed vacuum energy (cosmological constant) driving cosmic acceleration is many orders of magnitude smaller than naive quantum-field-theory estimates, posing a severe naturalness problem~\parencite{Weinberg:1989CC}. A quantum theory of gravity and a resolution of the cosmological-constant problem require physics beyond the SM and classical general relativity.

  \item[Dark matter] Cosmological and astrophysical data require a cold, non-baryonic,  component with $\Omega_c h^2 \simeq 0.12$~\parencite{Planck2018}, consistent with the thermal weakly interacting massive particle (WIMP) production hypothesis. The SM does not have a  suitable dark matter particle candidate: active neutrinos are too light and hot, and baryons are limited by Big Bang Nucleosynthesis (BBN) and Cosmic Microwave Background (CMB) observations~\parencite{Planck2018,PDG2024,Cyburt:2015mya,Pitrou:2018cgg,Fields:2019pfx}. This points to new degrees of freedom BSM. Dark matter direct-detection experiments continue to improve sensitivity with null results. Recent LZ and XENONnT runs set the strongest spin-independent limits over a wide mass range~\parencite{LZ:2022first,LZ:2023full,XENONnT:2023}. The SM must be extended with at least one stable, electrically neutral, non-baryonic particle to account for the observed dark matter relic density.

  \item[Matter--antimatter asymmetry (baryon asymmetry)] The Universe exhibits a nonzero baryon asymmetry, $\eta_B \simeq 6\times10^{-10}$ from CMB/BBN~\parencite{Planck2018}. The SM fails quantitatively: for $m_H=125$ GeV the electroweak transition is a crossover (no sufficient departure from equilibrium), and CKM CP violation is many orders of magnitude too small to generate the observed asymmetry. Therefore, additional CP violation and/or new dynamics are required, e.g. leptogenesis or electroweak baryogenesis~\parencite{Sakharov:1967,Davidson:2008Leptogenesis,Morrissey:2012EWB}. New sources of CP violation and out-of-equilibrium dynamics beyond the SM are necessary to explain the matter--antimatter asymmetry of the Universe.

  \item[Dark energy] Observations of late-time cosmic acceleration require a dark energy component with equation-of-state parameter $w = p/\rho \approx -1$~\parencite{DESIY1:2024}, consistent with a cosmological constant ($w = -1$ exactly). This implies negative pressure driving accelerated expansion. The observed vacuum energy density is $\rho_{\Lambda} \sim (10^{-3}\,\text{eV})^4$, yet naive quantum-field-theory estimates based on summing zero-point energies up to the Planck scale yield $\rho_{\text{QFT}} \sim M_{\mathrm{Pl}}^4$, a discrepancy of approximately $10^{120}$ orders of magnitude---the worst fine-tuning problem in physics~\parencite{Weinberg:1989CC}. Even using the electroweak scale as a cutoff still leaves a gap of $\sim 10^{56}$. The $H_0$ tension persists and could reflect systematics or new physics. A fundamental understanding of vacuum energy and its contribution to cosmic acceleration demands new physics that can naturally suppress or cancel the cosmological constant.
\end{description}

Among the challenges outlined above, the anomalies observed in the lepton sector stand out as particularly compelling hints of BSM physics. These observations share a common thread: they suggest that the SM's assumption of lepton universality may be an approximate symmetry rather than a fundamental one, potentially violated. Understanding whether these tensions represent genuine new physics or subtle theoretical/experimental effects requires a detailed examination of lepton universality tests and the theoretical frameworks that could accommodate violations of this fundamental SM principle.

In the following section, we examine in detail the concept of lepton flavour universality, its experimental tests across various processes, and the theoretical implications of potential violations. This analysis will establish the foundation for understanding how new particles with non-universal lepton couplings could simultaneously address multiple anomalies while remaining consistent with the wealth of precision measurements that confirm the SM.
