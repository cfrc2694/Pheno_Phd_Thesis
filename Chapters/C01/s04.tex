\section{Lepton Universality: Tests and Anomalies}\label{sec:LU}

As established previously, gauge interactions of charged leptons are flavour-universal~\cite{gl1961579,PhysRevLett.19.1264,1674-1137-40-10-100001} in the SM: the $SU(2)_L\times U(1)_Y$ couplings are generation-independent~\cite{gl1961579,PhysRevLett.19.1264}, meaning that after accounting for kinematic and mass effects, processes mediated by electroweak interactions predict identical couplings to electrons, muons, and tau leptons~\cite{1674-1137-40-10-100001}. While small deviations can arise from well-understood mass-dependent, phase-space, and radiative corrections~\cite{1674-1137-40-10-100001}, genuine Lepton Universality Violation (LUV) would constitute evidence of new physics beyond the Standard Model~\cite{Hiller:2014yaa,Dorsner:2016wpm,Buttazzo:2017ixm}.

Lepton Universality Violation (LUV) differs from Lepton Flavour Violation (LFV). LFV refers to processes where the initial-state lepton flavour quantum number does not match the final state---analogous to charge conservation but for lepton flavour. Neutrino oscillations already demonstrate LFV in the neutral sector~\cite{SuperK:1998osc,SNO:2002NC}, constituting evidence of BSM physics. Charged-lepton flavour violation (cLFV), such as $\mu\to e\gamma$ or $\tau\to\mu\gamma$, remains unobserved and is highly suppressed in the SM even with massive neutrinos~\cite{1674-1137-40-10-100001}. In contrast, LUV concerns whether the gauge couplings themselves differ across lepton generations---a violation of the SM's generation-independent coupling structure. One can have cLFV without LUV (e.g., through neutrino mixing alone), and conversely, LUV does not necessarily imply large cLFV rates. This section focuses specifically on LU tests and their current experimental status~\cite{1674-1137-40-10-100001,Ciuchini:2022wbq}.

Experimental programs probe LU across various processes~\cite{1674-1137-40-10-100001}:
\begin{itemize}
  \item \textbf{Rare $B$ decays} ($b\to s\ell^+\ell^-$): Ratios $R_K$ and $R_{K^*}$ comparing muon to electron modes~\cite{LHCb:2014vgu,LHCb:2017avl,LHCb:2019hip,LHCb:2021trn,LHCb:2022qnv,LHCb:2022zom,Hiller:2014yaa}.
  \item \textbf{Charged-current $B$ decays} ($b\to c\ell\nu$): Ratios $R_D$ and $R_{D^*}$ comparing $\tau$ to light leptons~\cite{BaBar:2012obs,BaBar:2013mob,Belle:2015qfa,LHCb:2015gmp,LHCb:2017rln,LHCb:2023zxo,Amhis_2021}.
  \item \textbf{Light-meson decays}: Leptonic ($\pi, K\to \ell\nu$) and semileptonic ($K_{\ell3}$) universality tests~\cite{1674-1137-40-10-100001,Antonelli:2010}.
  \item \textbf{Electroweak boson decays}: $W\to \ell\nu$ and $Z\to \ell^+\ell^-$ universality~\cite{atlas2020test,LEPEW:2006}.
  \item \textbf{$\tau$ decays}: Tests of $e/\mu/\tau$ universality in leptonic and semileptonic channels~\cite{1674-1137-40-10-100001,Amhis_2021}.
\end{itemize}
Combinations of these measurements provide constraints on flavour dependent interactions beyond the SM~\parencite{1674-1137-40-10-100001}.

Most LU tests involving light mesons, $W/Z$ bosons, and $\tau$ decays show agreement with SM predictions at the percent level~\parencite{1674-1137-40-10-100001}. Initially significant tensions emerged in semileptonic $B$ decays~\cite{Hiller:2014yaa,Buttazzo:2017ixm,Capdevila_2018,Alonso:2015sja,Calibbi:2015kma}, particularly in the neutral-current ratios $R_K$ and $R_{K^*}$~\cite{LHCb:2014vgu,LHCb:2017avl,LHCb:2019hip,LHCb:2021trn,Altmannshofer_2015} and the charged-current ratios $R_D$ and $R_{D^*}$~\cite{BaBar:2012obs,BaBar:2013mob,Belle:2015qfa,Abdesselam:2019dgh,Hirose:2017dxl,Sato:2016svk,Hirose:2016wfn,Huschle:2015rga,LHCb:2015gmp,Aaij:2015yra,Aaij:2017uff,LHCb:2017rln,LHCb:2023zxo,Amhis_2021} (discussed in detail in the following subsection). These measurements generated theoretical interest, with proposals for new physics scenarios that could explain potential deviations from SM predictions~\cite{Dorsner:2016wpm,Angelescu:2018tyl,Bauer:2015knc,Crivellin:2017zlb}. Recent re-analyses have brought $R_{K^{(*)}}$ measurements closer to SM predictions~\parencite{LHCb:2022qnv,LHCb:2022zom,Greljo:2022jac,Ciuchini:2022wbq}, while the situation for $R_{D^{(*)}}$ remains under investigation~\cite{Amhis_2021}, with forthcoming data expected to provide insights~\cite{Belle-II:2018jsg}. These results motivate the study of scenarios where new particles might have preferential couplings to third-generation fermions~\cite{Greljo:2018tuh,King:2021jeo,Cornella:2019hct,Cornella:2021sby}.

Various theoretical frameworks can accommodate LUV, including extended gauge sectors with non-universal couplings and models where new particles couple to SM fermions through Yukawa-like interactions with generation-dependent hierarchies~\cite{DiLuzio:2017vat,Greljo:2018tuh,Angelescu:2021lln}. These models typically predict correlated signals across multiple precision observables~\cite{Greljo:2022jac,Ciuchini:2022wbq,Allwicher:2022gkm}. Depending on their flavour structure---whether arising from non-universal gauge couplings or from fermion mixing patterns after spontaneous symmetry breaking---such models may also induce cLFV processes at potentially observable levels, subject to constraints from existing experimental limits~\parencite{Blankenburg:2012nx,Angelescu:2018tyl}. LUV and cLFV are distinct phenomena: models with universal gauge couplings before symmetry breaking can still exhibit effective non-universality and flavour violation in the mass basis through fermion mixing, analogous to how CKM mixing generates quark flavour violation despite universal $SU(2)_L$ couplings in the SM.

These precision observables motivate direct searches for new physics at colliders~\cite{CMS:2020wzx,ATLAS:2019qpq}. While deviations from LU in light-meson, $\tau$, and electroweak boson decays remain consistent with SM expectations~\cite{1674-1137-40-10-100001,LEPEW:2006}, the anomalies observed in $B$ decays point to scenarios where new states may couple non-universally to leptons~\cite{Buttazzo:2017ixm}. Other tensions in the leptonic sector, such as the muon $(g-2)$ anomaly~\cite{Abi_2021,Aoyama:2020ynm,ALIBERTI20251}, provide motivation for exploring such scenarios, though their experimental status remains under investigation. 

New particles may exhibit enhanced couplings to third-generation fermions~\cite{Greljo:2018tuh,DiLuzio:2018zxy}. This scenario has several motivations. First, the pattern of anomalies---particularly in $b\to c\tau\nu$ transitions---directly involves the heaviest fermion generation. Second, such flavour structures are less constrained by precision measurements involving first- and second-generation fermions: processes like $K$ and $D$ meson decays, electroweak precision tests at LEP, and rare decays involving electrons and muons impose bounds on new physics that couples democratically to all generations~\cite{1674-1137-40-10-100001,LEPEW:2006}. By concentrating new-physics effects in the third generation, these models can remain consistent with existing limits while providing signatures in $B$ physics and processes involving top quarks, bottom quarks, and tau leptons~\cite{Angelescu:2021lln,Cornella:2021sby}. Third, generation-dependent couplings can arise in models with flavour symmetries or in scenarios where the Yukawa hierarchy of the SM is reflected in the couplings of new particles~\cite{Greljo:2018tuh,DiLuzio:2018zxy}.

Therefore, searches for new particles with preferential couplings to the third generation are a component of the BSM search strategy~\cite{CMS:2020wzx,ATLAS:2019qpq}. However, the experimental signatures of such models are often complex and challenging to separate from SM backgrounds. Given the range of theoretical possibilities and the finite resources of experimental collaborations, it is useful to perform phenomenological feasibility studies before undertaking full experimental searches~\cite{Faroughy:2016osc,Baker:2019sli}. These studies bridge the gap between theoretical model-building and experimental implementation: by simulating signal and background processes, evaluating detector effects, and optimizing analysis strategies, phenomenological studies can identify promising signatures, estimate discovery potential or exclusion reach with available data, and guide the experimental program~\cite{Cowan:2011}. Such studies also allow for exploration of model parameter space, identifying regions where experimental constraints are less stringent and where new searches could have sensitivity~\cite{Dorsner:2016wpm}.
