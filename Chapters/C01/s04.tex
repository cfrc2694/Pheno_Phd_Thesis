\section{Lepton Flavour Universality: Tests and Anomalies}\label{sec:LFU}

As established previously, Standard Model gauge interactions of charged leptons are flavour-universal~\cite{gl1961579,PhysRevLett.19.1264,1674-1137-40-10-100001}: the $SU(2)_L\times U(1)_Y$ couplings are generation-independent~\cite{gl1961579,PhysRevLett.19.1264}, meaning that after accounting for kinematic and mass effects, processes mediated by electroweak interactions predict identical couplings to electrons, muons, and tau leptons~\cite{1674-1137-40-10-100001}. While small deviations can arise from well-understood mass-dependent, phase-space, and radiative corrections~\cite{1674-1137-40-10-100001}, genuine Lepton Flavour Universality Violation (LFUV) would constitute clear evidence of new physics beyond the Standard Model~\cite{Hiller:2014yaa,Dorsner:2016wpm,Buttazzo:2017ixm}.

It is worth noting that total lepton flavour numbers are accidental symmetries of the renormalizable SM with massless neutrinos~\cite{1674-1137-40-10-100001}. While neutrino oscillations demonstrate flavour violation in the neutral sector~\cite{SuperK:1998osc,SNO:2002NC}, any significant charged-lepton flavour violation (cLFV) would unambiguously signal physics beyond the Standard Model~\parencite{1674-1137-40-10-100001}. This section focuses specifically on LFU tests and their current experimental status~\cite{1674-1137-40-10-100001,Ciuchini:2022wbq}.

An extensive experimental program probes LFU across various processes~\cite{1674-1137-40-10-100001}:
\begin{itemize}
  \item \textbf{Rare $B$ decays} ($b\to s\ell^+\ell^-$): Clean ratios $R_K$ and $R_{K^*}$ comparing muon to electron modes~\cite{LHCb:2014vgu,LHCb:2017avl,LHCb:2019hip,LHCb:2021trn,LHCb:2022qnv,LHCb:2022zom,Hiller:2014yaa}.
  \item \textbf{Charged-current $B$ decays} ($b\to c\ell\nu$): Ratios $R(D)$ and $R(D^*)$ comparing $\tau$ to light leptons~\cite{BaBar:2012obs,BaBar:2013mob,Belle:2015qfa,LHCb:2015gmp,LHCb:2017rln,LHCb:2023zxo,Amhis_2021}.
  \item \textbf{Light-meson decays}: Leptonic ($\pi, K\to \ell\nu$) y semileptonic ($K_{\ell3}$) universality tests~\cite{1674-1137-40-10-100001,Antonelli:2010}.
  \item \textbf{Electroweak boson decays}: $W\to \ell\nu$ y $Z\to \ell^+\ell^-$ universality~\cite{atlas2020test,LEPEW:2006}.
  \item \textbf{$\tau$ decays}: Tests of $e/\mu/\tau$ universality in leptonic and semileptonic channels~\cite{1674-1137-40-10-100001,Amhis_2021}.
\end{itemize}
Combinations of these measurements provide stringent constraints on flavour-dependent interactions beyond the SM~\parencite{1674-1137-40-10-100001}.

Most LFU tests involving light mesons, $W/Z$ bosons, and $\tau$ decays show agreement with SM predictions at the percent level~\parencite{1674-1137-40-10-100001}. The most significant tensions have emerged in semileptonic $B$ decays~\cite{Hiller:2014yaa,Buttazzo:2017ixm,Capdevila_2018,Alonso:2015sja,Calibbi:2015kma}, particularly in the neutral-current ratios $R_K$ and $R_{K^*}$ and the charged-current ratios $R(D)$ and $R(D^*)$ (discussed in detail in the following subsection)~\cite{Altmannshofer_2015,Capdevila_2018}. Recent analyses have brought $R_{K^{(*)}}$ measurements closer to SM predictions~\parencite{LHCb:2022qnv,LHCb:2022zom,Greljo:2022jac,Ciuchini:2022wbq}, while the situation for $R(D^{(*)})$ remains actively investigated~\cite{Amhis_2021}, with forthcoming data expected to provide decisive insights~\cite{Belle-II:2018jsg}. Collectively, these measurements delineate viable patterns of lepton non-universal interactions and provide crucial guidance for theoretical model building~\cite{Dorsner:2016wpm,Buttazzo:2017ixm,Angelescu:2018tyl,Cornella:2021sby}.

In the renormalizable Standard Model, gauge interactions are flavour-universal~\cite{gl1961579,PhysRevLett.19.1264,1674-1137-40-10-100001}; therefore, any credible observation of Lepton Flavour Universality (LFU) violation would require new physics beyond the SM~\cite{Hiller:2014yaa,Dorsner:2016wpm}. Various theoretical frameworks can accommodate such violations, including extended gauge sectors and other scenarios that generate non-universal couplings~\cite{DiLuzio:2017vat,Greljo:2018tuh,Angelescu:2021lln}. These models typically predict correlated signals across multiple precision observables~\cite{Greljo:2022jac,Ciuchini:2022wbq,Allwicher:2022gkm} and, depending on their flavour structure, may also induce charged lepton flavour violation (cLFV) at potentially observable levels, subject to tight constraints from existing experimental limits~\parencite{Blankenburg:2012nx,Angelescu:2018tyl}.

In recent years, significant attention has focused on precision measurements of $B$-meson decay rates~\cite{Hiller:2014yaa,Buttazzo:2017ixm}, particularly through ratios that test lepton flavour universality~\cite{Amhis_2021}. The most prominent examples are the $R_{K^{(*)}}$~\parencite{LHCb:2014vgu,LHCb:2017avl,LHCb:2019hip,LHCb:2021trn} and $R_{D^{(*)}}$~\parencite{BaBar:2012obs,BaBar:2013mob,Abdesselam:2019dgh,Hirose:2017dxl,Sato:2016svk,Hirose:2016wfn,Huschle:2015rga,LHCb:2015gmp,Aaij:2015yra,Aaij:2017uff,LHCb:2017rln,LHCb:2023zxo} ratios, which compare decay rates to different lepton families~\cite{Amhis_2021,1674-1137-40-10-100001}. These measurements generated substantial theoretical interest, with numerous proposals for new physics scenarios that could explain potential deviations from SM predictions~\cite{Dorsner:2016wpm,Angelescu:2018tyl,Bauer:2015knc,Crivellin:2017zlb}.

Recent re-analyses of $R_{K^{(*)}}$ data have shown this ratio to be compatible with the SM prediction~\parencite{LHCb:2022qnv,LHCb:2022zom,Greljo:2022jac,Ciuchini:2022wbq}, while the situation for $R_{D^{(*)}}$ remains an open question~\cite{Amhis_2021} that continues to motivate the study of scenarios where new particles might have preferential couplings to third-generation fermions~\cite{Greljo:2018tuh,King:2021jeo,Cornella:2019hct}.

The anomalous magnetic moment of the muon, $a_\mu \equiv (g-2)_\mu/2$, represents another benchmark precision observable sensitive to new virtual states~\cite{Aoyama:2020ynm}. The latest measurements from the Fermilab Muon $g-2$ experiment report a value with sub-ppm precision~\parencite{Abi_2021}, broadly consistent with but more precise than the earlier BNL result~\cite{Muong-2:2006rrc}. This measurement shows sustained tension with certain Standard Model evaluations~\cite{Aoyama:2020ynm}.

The theoretical prediction for $a_\mu$ combines QED, electroweak, and hadronic contributions~\cite{Aoyama:2020ynm}, with the hadronic vacuum polarization and light-by-light scattering components driving the dominant uncertainties~\parencite{arxiv.1311.2198,1674-1137-40-10-100001}. The comparison between experimental results and theoretical predictions therefore serves as a powerful indirect probe of new physics scenarios that couple to leptons~\cite{Aoyama:2020ynm}.

The complementarity between $(g-2)_\mu$, LFU tests in $B$ decays, and direct searches provides a multifaceted approach to testing the Standard Model~\cite{Ciuchini:2022wbq,Greljo:2022jac}. Limits on charged lepton flavour violation further constrain possible chirality-flipping couplings that could also contribute to dipole moments~\cite{Blankenburg:2012nx,Angelescu:2018tyl}, highlighting the interconnected nature of these precision observables in the search for physics beyond the Standard Model~\cite{Dorsner:2016wpm}.

Taken together, these precision observables underscore the importance of direct searches for new physics at colliders~\cite{CMS:2020wzx,ATLAS:2019qpq}. While deviations from LFU in light-meson, $\tau$, and electroweak boson decays remain consistent with SM expectations~\cite{1674-1137-40-10-100001,LEPEW:2006}, the persistent anomalies in $B$ decays and the muon $(g-2)$ point to scenarios where new states may couple non-universally to leptons~\cite{Buttazzo:2017ixm,Aoyama:2020ynm}. A particularly well-motivated possibility is that new particles exhibit enhanced couplings to third-generation fermions~\cite{Greljo:2018tuh,DiLuzio:2018zxy}. Such flavour structures naturally alleviate existing constraints from first- and second-generation processes while offering testable signatures in collider environments~\cite{Angelescu:2021lln,Cornella:2021sby}. 

Therefore, a dedicated experimental programme to search for new particles with preferential couplings to the third generation is a crucial component of the BSM search strategy~\cite{CMS:2020wzx,ATLAS:2019qpq}. This programme requires not only powerful collider experiments~\cite{ATLAS:2008xda,CMS:2008xjf} but also detailed \textit{feasibility studies} to assess the discovery potential of these non-standard signatures~\cite{Faroughy:2016osc,Baker:2019sli}. Such studies are essential to optimize trigger strategies~\cite{ATLAS:2008xda,CMS:2008xjf}, refine analysis techniques~\cite{Cowan:2011}, and ultimately guide the exploration of the most promising regions of parameter space where these hypothetical particles might reveal themselves~\cite{Dorsner:2016wpm}.
