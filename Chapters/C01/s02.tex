\section{Standard Model}

{$ $ \scriptsize \hfill Fragment extracted and adapted from~\parencite{robinson2011symmetry}}

In 1965, Tomonaga, Feynman, and Schwinger were awarded the Nobel Prize for their independent formulation of Quantum Electrodynamics (QED)~\parencite{1972physics}. Their work established renormalization as a consistent method to separate infinities from finite, physically meaningful results in quantum field theory. QED provided predictions, such as the anomalous magnetic moment of the electron, that later experiments confirmed with remarkable precision~\parencite{1674-1137-40-10-100001, PhysRev.75.486}. It became the prototypical example of a successful quantum field theory.

This success, however, did not extend to other fundamental interactions. The weak interaction was described by the chiral $V-A$ model, in which processes such as the beta decay were represented by four-fermion contact terms. This framework was not renormalizable: divergences could not be absorbed into a finite set of parameters, restricting its validity to low energies. A fundamental description within the quantum field theory framework was still missing.


The issue was linked to the short-range character of the weak and strong forces. In quantum field theory, the range of an interaction depends on the mass of its mediating boson. A massless boson, such as the photon, generates a long-range force with an inverse-square dependence. A boson with mass $m_A$, in contrast, produces a Yukawa potential of the form $\exp(-m_A r)/r$, which falls off rapidly with distance. Therefore, a consistent theory for weak interactions required massive gauge bosons.

Here lay the apparent obstacle. A mass term for a gauge boson, such as $m_{A}^{2} A_{\mu} A^{\mu}$ in the Lagrangian, explicitly breaks gauge invariance, since it is not preserved under the transformation $A_{\mu} \mapsto A_{\mu} + \partial_{\mu}\epsilon$. This seemed to rule out gauge theories as candidates for describing short-range forces. The problem was recognized early on. For example, during a 1954 seminar by Chen Ning Yang on non-Abelian gauge theories, Wolfgang Pauli raised a fundamental objection: if the gauge bosons had mass, gauge invariance would be violated, yet without mass, the theory could not account for the short range of nuclear forces. This skepticism reflected a widely shared view: gauge symmetry appeared incompatible with short-range interactions.

The resolution of this problem came from two developments that allowed gauge bosons to behave as if they had mass, without explicitly breaking gauge symmetry:
\begin{enumerate}
    \item \label{list:sol_mass_1} The Higgs mechanism. In this framework, a scalar field permeates the vacuum. While the underlying Lagrangian remains gauge invariant, the vacuum state does not respect this symmetry. Gauge bosons interacting with this vacuum acquire mass in a renormalizable way. This mechanism explains the masses of the $W$ and $Z$ bosons.
    \item \label{list:sol_mass_2} Dynamical mass generation in non-Abelian gauge theories. In Quantum Chromodynamics (QCD), gluons and quarks (which are nearly massless at a fundamental level) are confined into hadrons that possess substantial masses. The appearance of this "mass gap" is a non-perturbative consequence of confinement. It is crucial to understand that most of the proton's mass, for example, does not come from the rest masses of its constituent quarks, but rather from the interaction energy of the confined gluon field via $E=mc^2$. This mass generation mechanism is conceptually distinct from the Higgs mechanism. Understanding it rigorously in the framework of gauge theories with non-Abelian symmetries (which allow for self-interactions among gauge bosons, like gluons) is the core of the Yang–Mills existence and mass gap Millennium Prize problem \cite{ClayMath2000}.
\end{enumerate}

The SM incorporates both solutions. Electroweak theory relies on the Higgs mechanism (\ref{list:sol_mass_1}), which provides a renormalizable description of the weak interaction. For the strong interaction, QCD employs dynamical mass generation (\ref{list:sol_mass_2}), where most of the mass of hadrons arises from confinement rather than from the small quark masses introduced by the Higgs field.

\subsection{Particle Content and Gauge Group}
\label{sec:particlecontent}

We begin by considering the chiral nature of elementary particles. While massive half-spin particles are fundamentally described by Dirac spinor fields (see Table \ref{tab-repLorentz2}), these do not transform under irreducible representations of the Lorentz group. Instead, Dirac spinors decompose into two Weyl spinors that do form irreducible representations. The left and right chiral projectors, $P_L$ and $P_R$, extract these components from a Dirac spinor. For massless fermions, these chiral components decouple dynamically, becoming independent fields described by separate Lagrangian densities—for instance, the left-handed component obeys $\lag=-i\bar\psi\slashed{\partial}P_L\psi$ (see Appendix A in~\parencite{CRodriguezUPTC}).

The observed parity violation in weak interactions, first discovered in radioactive decays~\parencite{PhysRev.105.1413}, reveals that these forces couple asymmetrically to the left and right chiral components of fermions. This fundamental asymmetry motivated the electroweak theory proposed by Glashow, Weinberg, and Salam~\parencite{gl1961579,PhysRevLett.19.1264,Salam:1968rm}, based on the gauge group $SU(2)_{L} \times U(1)_{Y}$. Within this framework, the $SU(2)_L$ factor acts exclusively on left-chiral fermion components, while $U(1)_Y$ (hypercharge) acts on both chiralities with component-specific charges.

The Standard Model further incorporates the strong interaction through the concept of color charge, fundamentally dividing fermions into two categories: color-charged particles (quarks), which transform as triplets under $SU(3)_C$, and color-neutral particles (leptons), which are singlets under $SU(3)_C$.

The complete gauge symmetry of the Standard Model is therefore:
\begin{equation}
    G_{\mathrm{SM}} = SU(3)_C \times SU(2)_{L} \times U(1)_{Y},
\end{equation}
where:
\begin{itemize}
    \item $SU(3)_C$ governs the strong interaction (Quantum Chromodynamics),
    \item $SU(2)_L \times U(1)_Y$ describes the electroweak interaction.
\end{itemize}

This gauge structure organizes the fermionic content into doublets and singlets under $SU(2)_L$, while distinguishing between colored and colorless particles. We refer to the fermionic fields of the SM as the matter fields, distinguishing them into two categories: leptons and quarks.

\begin{itemize}
    \item \textbf{Leptons}: As $SU(3)_C$ singlets, leptons do not participate in strong interactions. Each generation consists of a left-handed $SU(2)_L$ doublet containing a neutrino and a charged lepton, along with a right-handed singlet:
        \begin{equation}
            \ell_L = \begin{pmatrix} \nu_{eL} \\ e_L \end{pmatrix}, \quad
            e_R.
        \end{equation}
    In the minimal model, the right-handed neutrino $\nu_R$ is a complete gauge singlet, making it sterile.

    \item \textbf{Quarks}: As $SU(3)_C$ triplets, quarks carry color charge and participate in all interactions. Each generation forms a left-handed $SU(2)_L$ doublet together with right-handed singlets for both up-type and down-type quarks:
        \begin{equation}
            q_L = \begin{pmatrix} u_L \\ d_L \end{pmatrix}, \quad
            u_R, \quad d_R.
        \end{equation}
\end{itemize}

Remarkably, this entire pattern of gauge representations and hypercharge assignments is not unique to a single set of fermions. Instead, it repeats almost identically across \textbf{three generations}, each sharing the same gauge quantum numbers but differing in mass and flavor. This replication gives rise to the rich structure of flavor physics and underlies phenomena such as quark mixing and neutrino oscillations. The existence of three fermion families is one of the most intriguing empirical facts of the Standard Model, hinting at possible structures beyond it.

%The full fermionic content of the Standard Model, organized by generation, can therefore be written as:
%\begin{equation}
%    q_{L}^{i}=\left(
%        \begin{array}{c}
%            u_{L}^{i} \\
%            d_{L}^{i}
%        \end{array}
%    \right), \,
%    u_{R}^{i}, \, d_{R}^{i}, 
%    \quad \ell_L^{i}=\left(
%        \begin{array}{c}
%            \nu_{L}^{i} \\
%            e_{L}^{i}
%        \end{array}
%    \right),\, e_{R}^{i}; \quad i=1,2,3.
%\end{equation}

All fermions transform under $U(1)_Y$ with generation-independent hypercharges. Left-handed components form $SU(2)_L$ doublets, while right-handed components are singlets. Quarks transform as color triplets under $SU(3)_C$, whereas leptons are color singlets. 

The chirality of the fermionic spectrum stands as one of the deepest properties of the Standard Model. The fact that left- and right-chirality components transform differently under the electroweak gauge group poses a fundamental challenge: such chiral asymmetry is compatible with gauge symmetry only if fermions are massless, since a direct Dirac mass term $m \overline{f_{R}} f_{L} + \text{h.c.}$ would explicitly break the gauge invariance. Yet, fermions like the electron are unequivocally massive. This apparent contradiction finds its resolution in the Higgs mechanism, where masses emerge effectively from Yukawa couplings with the Higgs field.

This chiral gauge structure not only explains the parity-violating nature of weak interactions but also enables electroweak symmetry breaking, which generates masses for fermions and weak gauge bosons. The electroweak symmetry $SU(2)_{L} \times U(1)_{Y}$ spontaneously breaks to the electromagnetic symmetry $U(1)_{\mathrm{EM}}$ via the Higgs mechanism, whose quantum excitation is the Higgs boson $h$. The hypercharge assignments in Table \ref{tab_qm} relate to electric charges through the Gell-Mann–Nishijima relation~\parencite{10.1143/PTP.10.581}:
\begin{equation}
    Q_{\mathrm{EM}} = \frac{1}{2}Y + T_{3}, \label{eq:Gell-Mann-Nishijima}
\end{equation}
where $T_{3} \equiv \operatorname{diag}\left(\frac{1}{2},-\frac{1}{2}\right)$ is an $SU(2)_{L}$ generator. This relation ensures charge quantization—for instance, the exact equality in magnitude of the proton and electron charges. Although these hypercharge assignments might appear arbitrary, they are uniquely determined by the requirement of quantum consistency.

\begin{center}
    \begin{tabular}{|l||c|c|c||c|}
        \hline \textbf{Field} & $SU(3)_C$ & $SU(2)_{L}$ & $U(1)_{Y}$ & $U(1)_{EM}$ \bigstrut
        \\ \hline \hline 
        $q_{L}^{i}=\left(u^{i}, d^{i}\right)_{L}$ & $\mathbf{3}$ & $\mathbf{2}$ & $+1/3$ & $(2/3,-1/3)$ \bigstrut\\
        $u_{R}^{i}$ & $\overline{\mathbf{3}}$ & $\mathbf{1}$ & $+4/3$ & $+2/3$\bigstrut \\
        $d_{R}^{i}$ & $\overline{\mathbf{3}}$ & $\mathbf{1}$ & $-2/3$ & $-1/3$\bigstrut \\
        $\ell^{i}_L=\left(\nu^{i}, e^{i}\right)_{L}$ & $\mathbf{1}$ & $\mathbf{2}$ & $-1$  & $(0,-1)$\bigstrut \\
        $e_{R}^{i}$ & $\mathbf{1}$ & $\mathbf{1}$ & $-2$ & $-1$\bigstrut \\
        $H=\left(H^{+}, H^{0}\right)$ & $\mathbf{1}$ & $\mathbf{2}$ & $+1$ & $(+1,0)$\bigstrut \\
        \hline \hline
    \end{tabular}
    \captionof{table}{Gauge quantum numbers of Standard Model quarks, leptons, and the Higgs scalar.}\label{tab_qm}
\end{center}

\subsection{Gauge Bosons}

In the previous section, we have seen that the fundamental interactions of SM raise the gauge group
\begin{equation}
    G_{\mathrm{SM}} = SU(3)_C \times SU(2)_L \times U(1)_Y.
\end{equation}
Each factor in this group corresponds to a fundamental interaction and requires the introduction of gauge fields that ensure the local symmetry of the theory. These vector fields mediate the forces among matter particles and transform in the adjoint representation of their respective gauge groups.

The gauge bosons associated with each group are:
\begin{itemize}
    \item From $SU(3)_C$: Eight massless gluons ($G_\mu^a$) that mediate the strong force between color-charged particles.
    \item From $SU(2)_L$: Three weak isospin bosons ($W_\mu^i$) responsible for weak interactions.
    \item From $U(1)_Y$: A single hypercharge boson ($B_\mu$) associated with the abelian symmetry of hypercharge.
\end{itemize}

Before symmetry breaking, these are the gauge eigenstates of the theory. After electroweak symmetry breaking, the $SU(2)_L$ and $U(1)_Y$ fields mix to produce the physical gauge bosons: the massive $W^\pm$ and $Z$ bosons that mediate weak interactions, and the massless photon ($\gamma$) responsible for electromagnetism.

The Lie algebra of the gauge group $S U(3)_C \times S U(2)_{L} \times U(1)_{Y}$ follows
\begin{equation}
\begin{aligned}
	{\left[t^{a}, t^{b}\right] } &=i f^{a b c} t_{c}, \\
	{\left[T^{i}, T^{j}\right] } &=i \epsilon^{i j k} T_{k}, \\
	{\left[T^{i}, \, Y\;\right] } &=\left[t^{a}, T^{j}\right]=\left[t^{a}, Y\right]=0,
\end{aligned}
\end{equation}
where $f^{a b c}$ and $\epsilon^{i j k}$ are the structure constants of $SU(3)$ and $SU(2)$. Therefore, the gauge fields $G_\mu$, $W_\mu$, and $B_\mu$ must transform in the adjoint representation: 
\begin{equation}
	\begin{aligned}
		\delta B_{\mu} &=\partial_{\mu} \theta, \\
		\delta W_{\mu}^{i} &=\partial_{\mu} \theta^{i}-g \epsilon^{i j k} \theta^{j} W_{\mu}^{k}, \\
		\delta G_{\mu}^{a} &=\partial_{\mu} \epsilon^{a}-g_{s} f^{a b c} \epsilon^{b} G_{\mu}^{c}.
	\end{aligned}
\end{equation}
Then, the curvature strength tensors are
\begin{equation}\label{eq:field-strength-tensors}
\begin{aligned}
	G_{\mu \nu}^{a} &=\partial_{\mu} G_{\nu}^{a}-\partial_{\nu} G_{\mu}^{a}+g_{s} f^{a b c} G_{\mu}^{b} G_{\nu}^{c} \\
	W_{\mu \nu}^{i} &=\partial_{\mu} W_{\nu}^{i}-\partial_{\nu} W_{\mu}^{i}+g \epsilon^{i j k} W_{\mu}^{j} W_{\nu}^{k} \\
	B_{\mu \nu} &=\partial_{\mu} B_{\nu}-\partial_{\nu} B_{\mu}
\end{aligned}
\end{equation}
and the ``kinetic'' term for gauge fields in the Lagrangian is  
\begin{equation}\label{eq:gauge-lagrangian}
\mathcal{L}_{\text{Gauge}}=-\frac{1}{4} G_{\mu \nu}^{a} G_{a}^{\mu \nu}-\frac{1}{4} W_{\mu \nu}^{i} W_{i}^{\mu \nu}-\frac{1}{4} B_{\mu \nu} B^{\mu \nu}.
\end{equation}
These kinetic terms induce vertices between gauge bosons and in turn do not take into account the masses for such vector bosons. Their masses are generated via the Higgs mechanism, thorough linear combinations, giving rise to the physical $W^\pm$, $Z$, and $\gamma$  bosons. The field redefinitions leading to mass eigenstates are
\begin{equation}
\begin{cases}
	\begin{aligned}
		W_{\mu}^{+} &=\frac{1}{\sqrt{2}}\left(W_{\mu}^{1}-i W_{\mu}^{2}\right) \\
		W_{\mu}^{-} &=\frac{1}{\sqrt{2}}\left(W_{\mu}^{1}+i W_{\mu}^{2}\right) \\
		Z_{\mu} &=c_{w} W_{\mu}^{3}-s_{w} B_{\mu} \\
		A_{\mu} &=s_{w} W_{\mu}^{3}+c_{w} B_{\mu}
	\end{aligned}
\end{cases}
\text{where}
\;
\begin{cases}
	s_{w}=\sin \theta_{w}=\dfrac{g}{\sqrt{g^{2}+g{\prime2}}}\\
	c_{w}=\cos \theta_{w}=\dfrac{g^\prime}{\sqrt{g^{2}+g{\prime2}}}
\end{cases}
\end{equation}
To avoid confusions with the Dirac matrices, we denote the electromagnetic potential as $A_\mu$.


\marginpar{Before symmetry breaking we work in a general renormalizable $R_\xi$ gauge; for the discussion of observable tree-level vertices below we will adopt the \textit{unitary gauge}, in which the unphysical Goldstone bosons are removed from the spectrum and only the physical transverse and longitudinal polarizations of $W^\pm$ and $Z$ remain. (Ghost and Goldstone interactions present in a covariant gauge are therefore omitted from the figures.)}%
Due to the non-abelian nature of the $SU(3)_C$ and $SU(2)_L$ gauge groups, the corresponding gauge bosons also interact among themselves. From the strength tensors in Eq.~\eqref{eq:field-strength-tensors} in the kinetic term of the gauge Lagrangian in Eq.~\eqref{eq:gauge-lagrangian}, we obtain three- and four-point self-interaction vertices for vector bosons from the $SU(3)_C$ and $SU(2)_L$ sectors (see Fig.~\ref{fig-gauge-vertices}), whose structure follows directly from the commutation relations of the Lie algebra.
\begin{figure}[h!]
    \centering
    \begin{subfigure}[b]{0.48\textwidth}
        \centering
        \begin{fmffile}{feyngraphs/feyngraph21} 
			\vspace{0.5cm}
            \begin{fmfgraph*}(80,60)
                \fmfleft{i1}
                \fmfright{o1,o2}
                
                \fmf{gluon,tension=2.0}{i1,v1}
                \fmf{gluon}{v1,o1}
                \fmf{gluon}{v1,o2}

                \fmflabel{$g^a$}{i1}
                \fmflabel{$g^b$}{o1}
                \fmflabel{$g^c$}{o2}
            \end{fmfgraph*}
			\vspace{0.5cm}
        \end{fmffile}
        \caption{Triple gluon vertex.}
        \label{fig-triple-gluon}
    \end{subfigure}
    \hfill
    \begin{subfigure}[b]{0.48\textwidth}
        \centering
        \begin{fmffile}{feyngraphs/feyngraph22}
			\vspace{0.5cm}
            \begin{fmfgraph*}(80,60)
                \fmfleft{i1,i2}
                \fmfright{o1,o2}
                
                \fmf{gluon}{i1,v1}
                \fmf{gluon}{i2,v1}
                \fmf{gluon}{v1,o1}
                \fmf{gluon}{v1,o2}

                \fmflabel{$g^a$}{i1}
                \fmflabel{$g^b$}{i2}
                \fmflabel{$g^c$}{o1}
                \fmflabel{$g^d$}{o2}
            \end{fmfgraph*}
			\vspace{0.5cm}
        \end{fmffile}
        \caption{Quartic gluon vertex.}
        \label{fig-quartic-gluon}
    \end{subfigure}
	\begin{subfigure}[b]{0.48\textwidth}
        \centering
		\begin{fmffile}{feyngraphs/feyngraph23}
			\vspace{1.0cm}
			\begin{fmfgraph*}(80,60)
				\fmfleft{i1}
                \fmfright{o1,o2}

				\fmf{boson,tension=2.0}{i1,v1}
                \fmf{boson}{v1,o1}
				\fmf{boson}{v1,o2}

				\fmflabel{$Z/\gamma$}{i1}
				\fmflabel{$W^+$}{o1}
				\fmflabel{$W^-$}{o2}
			\end{fmfgraph*}
			\vspace{0.5cm}
		\end{fmffile}
		\caption{Triple $WWX$ boson vertex.}
		\label{fig-triple-w}
	\end{subfigure}
	\begin{subfigure}[b]{0.48\textwidth}
        \centering
		\begin{fmffile}{feyngraphs/feyngraph24}
			\vspace{1.0cm}
			\begin{fmfgraph*}(80,60)
				\fmfleft{i1,i2}
                \fmfright{o1,o2}

				\fmf{boson}{i1,v1}
                \fmf{boson}{i2,v1}
				\fmf{boson}{v1,o1}
                \fmf{boson}{v1,o2}

				\fmflabel{$W^+$}{i1}
				\fmflabel{$W^-$}{i2}
				\fmflabel{$W^+$}{o1}
				\fmflabel{$W^-$}{o2}
			\end{fmfgraph*}
			\vspace{0.5cm}
		\end{fmffile}
		\caption{Quartic $W$ boson vertex.}
		\label{fig-quartic-w}
	\end{subfigure}
	\begin{subfigure}[b]{0.48\textwidth}
        \centering
		\begin{fmffile}{feyngraphs/feyngraph25}
			\vspace{1.0cm}
			\begin{fmfgraph*}(80,60)
				\fmfleft{i1,i2}
                \fmfright{o1,o2}

				\fmf{boson}{i1,v1}
                \fmf{boson}{i2,v1}
				\fmf{boson}{v1,o1}
                \fmf{photon}{v1,o2}

				\fmflabel{$W^+$}{i1}
				\fmflabel{$W^-$}{i2}
				\fmflabel{$Z$}{o1}
				\fmflabel{$\gamma$}{o2}
			\end{fmfgraph*}
			\vspace{0.5cm}
		\end{fmffile}
		\caption{Quartic $WWZ\gamma$ vertex.}
		\label{fig-quartic-wwzgamma}
	\end{subfigure}
	\begin{subfigure}[b]{0.48\textwidth}
        \centering
		\begin{fmffile}{feyngraphs/feyngraph26}
			\vspace{1.0cm}
			\begin{fmfgraph*}(80,60)
				\fmfleft{i1,i2}
                \fmfright{o1,o2}

				\fmf{boson}{i1,v1}
                \fmf{boson}{i2,v1}
				\fmf{boson}{v1,o1}
                \fmf{boson}{v1,o2}

				\fmflabel{$W^+$}{i1}
				\fmflabel{$W^-$}{i2}
				\fmflabel{$Z/\gamma$}{o1}
			\fmflabel{$Z/\gamma$}{o2}
			\end{fmfgraph*}
			\vspace{0.5cm}
		\end{fmffile}
				\caption{Quartic $WWXX$ vertex ($X=Z,\gamma$).}
				\label{fig-quartic-wwxx}
	\end{subfigure}
			\caption{Feynman diagrams for gauge boson self-interactions (unitary gauge). We denote by $X=Z,\gamma$ a neutral electroweak gauge boson. }
    \label{fig-gauge-vertices}
\end{figure}

\subsection{Matter Fields}

In table \ref{tab-generations}, we can see that there are six leptons, three charged and three neutral: each charged lepton has an associated neutrino. Therefore, the electrically charged and the associated neutral leptons can be arranged as  $SU(2)_L$ doublets. In the case of quarks, we can divide these particles as up-and-down-type quarks, also arranged as $SU(2)_L$ doublets, one per generation. 

\begin{table}[h!]
\centering
	{\small
	\begin{tabular}{|c||c||l|l|l|}
		\hline \multicolumn{2}{|c||}{ \textbf{Fermion categories} } & \multicolumn{3}{c|}{\textbf{ Elementary particle generation} } \bigstrut\\
		\hline \hline Type & Subtype & First & Second & Third \bigstrut\\
		\hline\hline \multirow{2}{*}{ Quarks ($q$) }  & up-type & ($u$) up & ($c$) charm & ($t$) top  \bigstrut \\
		\cline { 2 - 5 }  & down-type & ($d$) down & ($s$) strange & ($b$) bottom  \bigstrut\\
		\hline\hline \multirow{2}{*}{ Leptons ($\ell$) } & charged & ($e$) electron & ($\mu$) muon & ($\tau$) tauon \bigstrut\\
		\cline { 2 - 5 } & neutrino & ($\nu_e$) & ($\nu_\mu$) & ($\nu_\tau$) \bigstrut\\
		\hline
	\end{tabular}
	}
	\caption{Three generations of fermions according to the SM of particle physics. Each generation containing two types of leptons and two types of quarks.}\label{tab-generations}
\end{table}

According to the SM, there are three generations (families) of fermions. Each generation contains a doublet of leptons and a doublet of quarks. Among generations, particles differ by their flavour quantum number and mass, but their strong and electrical interactions are identical. Moreover, the flavour quantum number is a quantity conserved by all interactions except for the weak interaction.  Each generation is more massive than the previous one. The second and third generations are unstable and they disintegrate into the first generation. This is why ordinary matter is composed of  first generation fermions. All three generations are produced in nuclear reactors, colliders, and cosmic rays. 

Under all the constraints on local gauge invariance and renormalizability of the theory, the fermionic Lagrangian for SM is given by
\begin{equation}
	\mathcal{L}_{\mathrm{Fer}}
	=i \bar{\ell}_{L}^j \slashed{\mathcal D} \ell_{L}^j
	+i \bar{e}_{R}^j \slashed{\mathcal D} e_{R}^j
	+i{\bar{q}}_{L}^j  \slashed{\mathcal D}  q_{L}^j
	+i{\bar{u}}_{R}^j  \slashed{\mathcal D}  u_{R}^j
	+i{\bar{d}}_{R}^j  \slashed{\mathcal D}  d_{R}^j
\end{equation}
where $\slashed{\mathcal D}\equiv \gamma ^\mu \mathcal D_\mu$, for the covariant derivative
\begin{equation}
	\mathcal D_\mu = \partial_\mu -ig_st_ aG^a_\mu -ig T_i W_\mu^i -ig'\frac Y2 B_\mu,
\end{equation}
with the gauge fields $G^a$, $W^i$, and $B$ acting on each kind of fermion via
\begin{equation}
\begin{aligned}
	\mathcal D_{ \mu} \ell_L^i &=\fac{\partial_{\mu}-i g T_j W_{\mu}^{j}+i \frac{g^{\prime}}2 B_{\mu}} \ell_L^i \\
	\mathcal D_{ \mu} e_R^i &=\fac{\partial_{\mu} -  i g^{\prime}  B_{\mu}\vph}e_R^i \\
	\mathcal D_{ \mu} q_L^i &=\fac{\partial_{\mu}-i g_{s} t_{a} G_{\mu}^{a}-i g T_j W_{\mu}^{j}-i \frac{g^{\prime}}{6} B_{\mu}} q_L^i \\
	\mathcal D_{ \mu} u_R^i &=\fac{\partial_{\mu} -i g_{s} t_{a} G_{\mu}^{a} - i \frac{2g^{\prime}}3  B_{\mu}}u_R^i \\
	\mathcal D_{ \mu} d_R^i &=\fac{\partial_{\mu} -i g_{s} t_{a} G_{\mu}^{a} + i \frac{g^{\prime}}3  B_{\mu}}d_R^i ,\\
\end{aligned}
\end{equation}
coupling the fermions to the gauge bosons. As we will show below, after electroweak symmetry breaking, these interactions give rise to the familiar electromagnetic, weak, and strong forces, where the physical $\gamma$, $Z$, and $W$ bosons are a superposition of the original $B$ and $W$ fields, as it was mentioned before. Representative tree-level gauge–fermion interaction vertices are displayed in Fig.~\ref{fig-gauge-interactions}.

\begin{figure}[h!]
    \centering
    \begin{subfigure}[b]{0.48\textwidth}
        \centering
        \begin{fmffile}{feyngraphs/feyngraph5} 
			\vspace{0.5cm}
            \begin{fmfgraph*}(80,60)
                \fmfleft{i1}
                \fmfright{o1,o2}
                
                \fmf{photon}{i1,v1}
                \fmf{fermion}{o1,v1}
                \fmf{fermion}{v1,o2}

                \fmflabel{$\gamma$}{i1}
                \fmflabel{$\bar f$}{o1}
                \fmflabel{$f$}{o2}
            \end{fmfgraph*}
			\vspace{0.5cm}
        \end{fmffile}
        \caption{Electromagnetic vertex.}
        \label{fig-em-interaction}
    \end{subfigure}
    \hfill
    \begin{subfigure}[b]{0.48\textwidth}
        \centering
        \begin{fmffile}{feyngraphs/feyngraph6}
			\vspace{0.5cm}
            \begin{fmfgraph*}(80,60)
                \fmfleft{i1}
                \fmfright{o1,o2}
                
                \fmf{boson}{i1,v1}
                \fmf{fermion}{o1,v1}
                \fmf{fermion}{v1,o2}

                \fmflabel{$W^{\pm}$}{i1}
                \fmflabel{$\bar \ell/\bar u$}{o1}
                \fmflabel{$\nu/d$}{o2}
            \end{fmfgraph*}
			\vspace{0.5cm}
        \end{fmffile}
        \caption{Charged weak vertex.}
        \label{fig-charged-weak}
    \end{subfigure}
	\begin{subfigure}[b]{0.48\textwidth}
        \centering
		\begin{fmffile}{feyngraphs/feyngraph7}
			\vspace{1.0cm}
			\begin{fmfgraph*}(80,60)
				\fmfleft{i1}
				\fmfright{o1,o2}

				\fmf{boson}{i1,v1}
                \fmf{fermion}{o1,v1}
                \fmf{fermion}{v1,o2}

				\fmflabel{$Z$}{i1}
				\fmflabel{$\bar f$}{o1}
				\fmflabel{$f$}{o2}
			\end{fmfgraph*}
			\vspace{0.5cm}
		\end{fmffile}
		\caption{Neutral weak vertex.}
		\label{fig-neutral-weak}
	\end{subfigure}
	\begin{subfigure}[b]{0.48\textwidth}
        \centering
		\begin{fmffile}{feyngraphs/feyngraph8}
			\vspace{1.0cm}
			\begin{fmfgraph*}(80,60)
				\fmfleft{i1}
				\fmfright{o1,o2}

				\fmf{gluon}{i1,v1}
                \fmf{fermion}{o1,v1}
                \fmf{fermion}{v1,o2}

				\fmflabel{$g$}{i1}
				\fmflabel{$\bar q$}{o1}
				\fmflabel{$q$}{o2}

				% \fmfv{lab=$ig_s\gamma^\mu t^a$, lab.dist=0.3cm, lab.angle=115}{v1}
			\end{fmfgraph*}
			\vspace{0.5cm}
		\end{fmffile}
		\caption{Strong interaction.}
		\label{fig-strong-interaction}
	\end{subfigure}
    \caption{Feynman diagrams for gauge boson interactions in the Standard Model.}\label{fig-gauge-interactions}
\end{figure}

\subsection{Electroweak Symmetry Breaking}

As it was mentioned in Section \ref{sec:particlecontent}, in the SM the EW symmetry, $SU(2)_{L} \times U(1)_{Y}$, is spontaneously broken down to the electromagnetic $U(1)_{\text{EM}}$ group by a complex scalar Higgs field. This field transforms as an $SU(2)_{L}$ doublet, $H=\left(H^{+}, H^{0}\right)$, with hypercharge $+1$. \textcolor{red}{Its dynamics is governed by the Mexican-hat potential ......AF: Si se va a nombrar el sombrero mexicano, sería deacuado incluir una figura del potencial} (the resulting scalar self-interactions are shown in Fig.~\ref{fig-higgs-self-interactions}):
\begin{equation}
    V(H)=-\mu^{2}|H|^{2}+\lambda|H|^{4} \quad \Rightarrow \quad v^{2} \equiv \langle H^{\dagger} H \rangle = \mu^{2} / 2\lambda.
\end{equation}
The vacuum expectation value (vev) aligns with the electrically neutral component, $\langle H^{0} \rangle = v/\sqrt{2} \simeq 174 \,\mathrm{GeV}$, generating masses for the weak gauge bosons while preserving $U(1)_{\text{EM}}$.

We adopt the Kibble (polar) parametrization of the Higgs doublet in terms of one physical scalar $h$ and three would-be Goldstone bosons $G^{\pm},G^0$:
\begin{equation}
\label{eq:KibbleParam}
H = \begin{pmatrix} G^{+} \\ \frac{1}{\sqrt{2}}(v + h + i G^{0}) \end{pmatrix},
\end{equation}
which in a unitary gauge reduces to $H = \tfrac{1}{\sqrt{2}}(0, v+h)^T$ after gauging away $G^{\pm},G^0$.

\begin{figure}[h!]
    \centering
    \begin{subfigure}[b]{0.48\textwidth}
        \centering
        \begin{fmffile}{feyngraphs/feyngraph16}
            \vspace{1.0cm}
            \begin{fmfgraph*}(80,60)
                \fmfleft{i1}
                \fmfright{o1,o2}

                \fmf{dashes,tension=2.0}{i1,v1}
                \fmf{dashes}{v1,o1}
                \fmf{dashes}{v1,o2}

                \fmflabel{$h$}{i1}
                \fmflabel{$h$}{o1}
                \fmflabel{$h$}{o2}
            \end{fmfgraph*}
            \vspace{0.5cm}
        \end{fmffile}
        \caption{Higgs triple self-coupling.}
        \label{fig-higgs-triple}
    \end{subfigure}
    \hfill
    \begin{subfigure}[b]{0.48\textwidth}
        \centering
        \begin{fmffile}{feyngraphs/feyngraph17}
            \vspace{1.0cm}
            \begin{fmfgraph*}(80,60)
                \fmfleft{i1,i2}
                \fmfright{o1,o2}

                \fmf{dashes}{i1,v1}
                \fmf{dashes}{i2,v1}
                \fmf{dashes}{v1,o1}
                \fmf{dashes}{v1,o2}

                \fmflabel{$h$}{i1}
                \fmflabel{$h$}{i2}
                \fmflabel{$h$}{o1}
                \fmflabel{$h$}{o2}
            \end{fmfgraph*}
            \vspace{0.5cm}
        \end{fmffile}
        \caption{Higgs quartic self-coupling.}
        \label{fig-higgs-quartic}
    \end{subfigure}
    \caption{Feynman diagrams for Higgs self-interactions arising from the potential $V(H^\dagger, H) = -\mu^2|H|^2 + \lambda|H|^4$ in the Higgs Lagrangian.}
    \label{fig-higgs-self-interactions}
\end{figure}

Fermion masses arise through Yukawa couplings, which represent the most general renormalizable interactions between the Higgs field and the fermion fields (see the diagrammatic decomposition in Fig.~\ref{fig-yukawa-up}–\ref{fig-yukawa-mass}):
\begin{equation}
    \mathcal{L}_{\text{Yuk}} = y_{u}^{ij} \bar{q}_{L}^{i} u_{R}^{j} \tilde{H} + y_{d}^{ij} \bar{q}_{L}^{i} d_{R}^{j} H + y_{\ell}^{ij} \bar{\ell}_L^{i} e_{R}^{j} H + \text{h.c.},
\end{equation}
where $\tilde{H} = i\sigma_2 H^*$, and $y_{u}, y_{d}, y_{\ell}$ are arbitrary $3 \times 3$ complex matrices in flavor space. When the Higgs acquires its vev, 
\begin{equation}
		\langle H \rangle = \begin{pmatrix} 0 \\ \frac{v}{\sqrt{2}} \end{pmatrix},
\end{equation}
these couplings generate Dirac mass terms for the fermions.

\begin{figure}[h!]
    \centering
    \begin{subfigure}[b]{0.48\textwidth}
        \centering
        \begin{fmffile}{feyngraphs/feyngraph9} 
			\vspace{0.5cm}
            \begin{fmfgraph*}(80,60)
                \fmfleft{i1,i2}
                \fmfright{o1}
                
                \fmf{fermion}{i1,v1}
                \fmf{fermion}{v1,i2}
                \fmf{dashes,tension=2.0}{v1,o1}

                \fmflabel{$\bar q_L$}{i1}
                \fmflabel{$u_R$}{i2}
                \fmflabel{$h$}{o1}
            \end{fmfgraph*}
			\vspace{0.5cm}
        \end{fmffile}
        \caption{Up-type Yukawa coupling.}
        \label{fig-yukawa-up}
    \end{subfigure}
    \hfill
    \begin{subfigure}[b]{0.48\textwidth}
        \centering
        \begin{fmffile}{feyngraphs/feyngraph10}
				\vspace{0.5cm}
							\begin{fmfgraph*}(80,60)
									\fmfleft{i1,i2}
									\fmfright{o1}
									
									\fmf{fermion}{i1,v1}
									\fmf{fermion}{v1,i2}
									\fmf{dashes,tension=2.0}{v1,o1}

									\fmflabel{$\bar q_L$}{i1}
									\fmflabel{$d_R$}{i2}
									\fmflabel{$h$}{o1}
							\end{fmfgraph*}
				\vspace{0.5cm}
        \end{fmffile}
        \caption{Down-type Yukawa coupling.}
        \label{fig-yukawa-down}
    \end{subfigure}
	\begin{subfigure}[b]{0.48\textwidth}
        \centering
		\begin{fmffile}{feyngraphs/feyngraph11}
			\vspace{1.0cm}
			\begin{fmfgraph*}(80,60)
				\fmfleft{i1,i2}
				\fmfright{o1}

				\fmf{fermion}{i1,v1}
				\fmf{fermion}{v1,i2}
				\fmf{dashes,tension=2.0}{v1,o1}

				\fmflabel{$\bar \ell_L$}{i1}
				\fmflabel{$e_R$}{i2}
				\fmflabel{$h$}{o1}
			\end{fmfgraph*}
			\vspace{0.5cm}
		\end{fmffile}
		\caption{Lepton Yukawa coupling.}
		\label{fig-yukawa-lepton}
	\end{subfigure}
	\begin{subfigure}[b]{0.48\textwidth}
        \centering
		\begin{fmffile}{feyngraphs/feyngraph13} 
		\vspace{0.5cm}
				\begin{fmfgraph*}(80,60)
						\fmfleft{i1,i2}
						\fmfright{o1}
						
						\fmf{fermion}{i1,v1,o1}
						\fmf{dashes}{v1,i2}

						\fmflabel{$\bar f_L$}{i1}
						\fmflabel{$\langle H \rangle$}{i2}
						\fmflabel{$f_R$}{o1}
				\end{fmfgraph*}
		\vspace{0.5cm}
		\caption{Fermion mass term from vev.}
		\label{fig-yukawa-mass}
		\end{fmffile}
	\end{subfigure}
    \caption{Feynman diagrams for Yukawa couplings.}
\end{figure}


The quark mass matrices are proportional to the Yukawa matrices: $M_u = y_u v/\sqrt{2}$, $M_d = y_d v/\sqrt{2}$. Since $y_u$ and $y_d$ are general complex matrices, they cannot be simultaneously diagonalized. The physical quark masses and states are found by performing separate unitary transformations on the left- and right-handed fields:
\begin{equation}
    u_L \to V_L^u u_L, \quad u_R \to V_R^u u_R, \quad d_L \to V_L^d d_L, \quad d_R \to V_R^d d_R,
\end{equation}
such that $V_L^u M_u V_R^{u\dagger} = M_u^{\text{diag}}$ and $V_L^d M_d V_R^{d\dagger} = M_d^{\text{diag}}$ are diagonal with real, positive entries.

This diagonalization procedure has a direct consequence for the charged-current interactions mediated by the $W^{\pm}$ bosons. In the flavor basis the interaction reads
\begin{equation}
    \mathcal{L}_{W} \supset -\frac{g}{\sqrt{2}} (\bar{u}_L, \bar{c}_L, \bar{t}_L) \gamma^\mu W_\mu^+ (d_L, s_L, b_L)^T + \text{h.c.}
\end{equation}
After moving to the mass basis, the left-handed up- and down-type quarks rotate differently ($u_L \to V_L^u u_L$, $d_L \to V_L^d d_L$), and the interaction becomes
\begin{equation}
    \mathcal{L}_{W} \supset -\frac{g}{\sqrt{2}} (\bar{u}_L, \bar{c}_L, \bar{t}_L) \gamma^\mu W_\mu^+ V_{\mathrm{CKM}} (d_L, s_L, b_L)^T + \text{h.c.},
\end{equation}
where the Cabibbo–Kobayashi–Maskawa (CKM) matrix appears as the mismatch between the two rotations:
\begin{equation}
    V_{\mathrm{CKM}} \equiv V_{L}^{u} V_{L}^{d \dagger} = \begin{pmatrix}
        V_{ud} & V_{us} & V_{ub} \\
        V_{cd} & V_{cs} & V_{cb} \\
        V_{td} & V_{ts} & V_{tb}
    \end{pmatrix}.
\end{equation}
This unitary matrix encodes flavor mixing in charged-current weak interactions, and its non-diagonal structure is the origin of all quark flavor-changing processes in the SM.

The situation is different for leptons in the minimal SM without right-handed neutrinos. The charged-lepton mass matrix $M_\ell = y_\ell v/\sqrt{2}$ can be diagonalized by field redefinitions, but since neutrinos are massless in this framework, there is no additional rotation in the neutrino sector. As a result, the charged-current interaction
\begin{equation}
    \mathcal{L}_{W} \supset -\frac{g}{\sqrt{2}} \bar{\nu}_L \gamma^\mu W_\mu^+ \ell_L + \text{h.c.}
\end{equation}
remains diagonal in the mass basis. This implies \textit{Lepton Flavor Universality} (LFU): the electroweak gauge bosons couple to all three lepton families with identical strength. In particular, the $W$ boson couples to each $\bar{\nu}_L \gamma^\mu \ell_L$ current with coefficient $-g/\sqrt{2}$, and the $Z$ boson couplings to $\ell_L$ and $\ell_R$ are flavor-independent because the hypercharge assignments are the same for all families.

LFU means that processes differing only by the lepton flavor, such as leptonic decays or semileptonic transitions, are predicted to occur with the same rates up to well-understood effects: differences in phase space, helicity suppression, lepton-mass dependence, and small radiative corrections. The assumption of LFU is central in the extraction of CKM parameters, since experimental determinations from decays involving electrons, muons, and tau leptons can be consistently combined.

Precision tests of LFU focus on ratios of decay widths or branching fractions where theoretical and experimental uncertainties cancel to a large extent. Agreement with these tests confirms the gauge structure of the SM, while deviations would point to new physics.

The Lagrangian of the scalar sector is
\begin{equation}
	\mathcal{L}_{H}= \mathcal D_{\mu} H^{\dagger} \mathcal D^{\mu} H - V\!\left(H^{\dagger}, H\right),
\end{equation}
with the covariant derivative defined as $\mathcal D_{\mu} H=\left(\partial_{\mu}+i g T_a W_{\mu}^{a}+i g^{\prime} \tfrac{Y}{2} B_{\mu}\right) H$. Substituting the Higgs vacuum expectation value, one obtains
\begin{equation}
	\begin{aligned}
		\mathcal{L}_{\langle H\rangle}
		&=-\frac{1}{8}\left(\begin{array}{ll}
			0 & v
		\end{array}\right)\left(\begin{array}{ll}
			g W_{\mu}^{3}-g' B_{\mu} & g\left(W_{\mu}^{1}-i W_{\mu}^{2}\right)\vph \\
			g\left(W_{\mu}^{1}+i W_{\mu}^{2}\right)&-g W_{\mu}^{3}-g' B_{\mu}\vph
		\end{array}\right)^{2}\left(\begin{array}{l}
			0 \\
			v
		\end{array}\right)
		\\&=
		-\frac{1}{8} v^{2} V_{\mu}^{T}\left(\begin{array}{cccc}
			g^{2} & 0 & 0 & 0 \\
			0 & g^{2} & 0 & 0 \\
			0 & 0 & g^{2} & -g' g \\
			0 & 0 & -g' g & g'^{2}
		\end{array}\right) V^{\mu},
	\end{aligned}
\end{equation} 
where $V_{\mu}^{T}=\left(W_{\mu}^{1}, W_{\mu}^{2}, W_{\mu}^{3}, B_{\mu}\right)$. Diagonalizing this mass matrix yields the following  eigenvalues 

\begin{equation*}
    0,\, -\tfrac{1}{8} v^{2} g^{2},\, -\tfrac{1}{8} v^{2} g^{2},\, \text{and} \,-\tfrac{1}{8} v^{2}\left(g^{2}+g'^{2}\right).
\end{equation*}
 The massless state corresponds to the photon, the heaviest to the $Z$ boson, and the two degenerate intermediate states to the charged bosons $W^\pm$, which transform under the representation of the unbroken generator $Q_{EM}$. The resulting Higgs--vector boson interaction structures are summarized in Fig.~\ref{fig-higgs-gauge-interactions}.

\begin{figure}[h!]
    \centering
    \begin{subfigure}[b]{0.48\textwidth}
        \centering
        \begin{fmffile}{feyngraphs/feyngraph14} 
            \vspace{0.5cm}
            \begin{fmfgraph*}(80,60)
                \fmfleft{i1,i2}
                \fmfright{o1,o2}
                
                \fmf{dashes}{i1,v1}
                \fmf{dashes}{i2,v1}
                \fmf{boson}{v1,o1}
                \fmf{boson}{v1,o2}

                \fmflabel{$h$}{i1}
                \fmflabel{$h$}{i2}
                \fmflabel{$Z/W^+$}{o1}
                \fmflabel{$Z/W^-$}{o2}
            \end{fmfgraph*}
            \vspace{0.5cm}
        \end{fmffile}
        \caption{Quartinc $hhVV$ coupling.}
        \label{fig-higgs-w}
    \end{subfigure}
    \hfill
    \begin{subfigure}[b]{0.48\textwidth}
        \centering
        \begin{fmffile}{feyngraphs/feyngraph15} 
            \vspace{0.5cm}
            \begin{fmfgraph*}(80,60)
                \fmfleft{i1,i2}
                \fmfright{o1,o2}
                
                \fmf{dashes}{i1,v1}
                \fmf{boson}{i2,v1}
                \fmf{boson}{v1,o1}
                \fmf{boson}{v1,o2}

                \fmflabel{$h$}{i1}
                \fmflabel{$Z/\gamma$}{i2}
                \fmflabel{$W^+$}{o1}
                \fmflabel{$W^-$}{o2}
            \end{fmfgraph*}
            \vspace{0.5cm}
        \end{fmffile}
	\caption{Quartic $hXWW$ coupling ($X=Z,\gamma$).}
	\label{fig-hxww}
    \end{subfigure}
    \begin{subfigure}[b]{0.48\textwidth}
        \centering
        \begin{fmffile}{feyngraphs/feyngraph18}
            \vspace{0.5cm}
            \begin{fmfgraph*}(80,60)
                \fmfleft{i1}
                \fmfright{o1,o2}
                
                \fmf{dashes,tension=2.0}{i1,v1}
                \fmf{boson}{v1,o1}
                \fmf{boson}{v1,o2}

                \fmflabel{$h$}{i1}
                \fmflabel{$Z/W^+$}{o1}
                \fmflabel{$Z/W^-$}{o2}
            \end{fmfgraph*}
            \vspace{0.5cm}
        \end{fmffile}
	\caption{Triple $hVV$ coupling ($V=W^\pm,Z$).}
	\label{fig-hvv}
    \end{subfigure}
	\begin{subfigure}[b]{0.48\textwidth}
        \centering
		\begin{fmffile}{feyngraphs/feyngraph19} 
		\vspace{0.5cm}
				\begin{fmfgraph*}(80,60)
						\fmfleft{i1,i2}
						\fmfright{o1}
						
						\fmf{boson}{i1,v1,o1}
						\fmf{dashes}{v1,i2}

						\fmflabel{$V$}{i1}
						\fmflabel{$\langle H \rangle$}{i2}
						\fmflabel{$V$}{o1}
				\end{fmfgraph*}
		\vspace{0.5cm}
		\caption{Vector boson mass insertion from the vev ($V=W^\pm,Z$).}
		\label{fig-vector-mass}
		\end{fmffile}
	\end{subfigure}
	\caption{Feynman diagrams for Higgs--gauge boson interactions (unitary gauge) arising from $\mathcal{D}_\mu H^\dagger \mathcal{D}^\mu H$ and a vev insertion. Here $V=W^\pm,Z$ and $X=Z,\gamma$.}
    \label{fig-higgs-gauge-interactions}
\end{figure}



This suffices to illustrate how the SM, formulated as a relativistic quantum field theory, describes the interactions of matter fields through the fundamental forces, mediated by vector bosons. The Higgs boson, also part of the SM spectrum, plays the central role in generating masses for the weak bosons and the fermions, while  indirectly distinguishing the photon as the only massless gauge boson of the EW sector.

Since its formulation, the SM has been tested extensively and has shown remarkable success, both in explaining existing data and in making accurate predictions. A well-known example is the agreement between its prediction and the experimental measurement of the electron magnetic dipole moment, consistent to twelve significant figures~\parencite{PhysRevLett.97.030801}. The discovery of the Higgs boson in 2012 was the culmination of almost fifty years of experimental efforts, confirming the mechanism incorporated into the SM in the late 1960s through the unification of the electromagnetic and weak interactions by Glashow, Weinberg, and Salam~\parencite{PhysRevLett.19.1264, gl1961579}. With this discovery, the full particle spectrum predicted by the SM was finally observed.
