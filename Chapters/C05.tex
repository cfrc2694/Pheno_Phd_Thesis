\chapter{Conclusions and Outlook}\label{ch:discussion}

The Standard Model of particle physics has achieved remarkable success in describing the fundamental particles and their interactions with extraordinary precision. However, both theoretical considerations and experimental observations point to the necessity of physics beyond the SM. Among the most compelling experimental hints are the persistent anomalies in lepton universality measurements, particularly in $B$-meson decays and the muon anomalous magnetic moment, which suggest the existence of new particles with preferential couplings to second and third-generation fermions.

This thesis has presented two comprehensive phenomenological studies exploring novel search strategies at the LHC to probe BSM physics models that can address these anomalies. Both studies share a common methodological framework: the use of advanced Monte Carlo simulations to accurately model signal and background processes, the application of machine learning algorithms to maximize sensitivity in complex multi-dimensional phase spaces, and the development of statistical tools to quantify discovery potential. These studies demonstrate that carefully designed analysis strategies, leveraging modern computational techniques, can significantly extend the reach of the LHC experiments to probe unexplored regions of BSM parameter space.


\section{Summary of Main Results}

\subsection{Search for Light Scalars in the $U(1)_{T^3_R}$ Model}

In Chapter~\ref{ch:U1T3R}, we investigated the phenomenology of a minimal $U(1)_{T^3_R}$ gauge extension of the SM. This model, motivated by left-right symmetric theories and the embedding of $U(1)_{B-L}$, provides a framework to address the hierarchy problem, dark matter, and lepton universality violation. The model predicts a light scalar boson $\phi'$ (the ``dark Higgs''), with masses potentially below the electroweak scale, and TeV-scale vector-like quarks $\chi_\mathrm{u}$ with QCD couplings.

We proposed a novel search strategy focusing on the previously unexplored production channel $\mathrm{pp}\to \mathrm{t}\chi_\mathrm{u}\phi'$, where the light scalar is produced in association with a heavy top-partner and a SM top quark. The key insight is that the simultaneous production of heavy QCD-coupled particles naturally provides the large momentum required for the decay products of the light $\phi'$ to be efficiently detected in the central regions of the CMS and ATLAS detectors. This ``boosted light particle'' strategy represents a paradigm shift: rather than searching for light scalars through direct low-energy production, we exploit the UV completion of the model by accessing the heavy degrees of freedom that produce the light scalar with substantial transverse momentum.

Our analysis considered the scenario where $\phi'$ has family non-universal couplings and decays primarily to muon pairs ($\phi'\to\mu^+\mu^-$) for $m(\phi') \ge 1$~GeV. The final state signature comprises three muons, a boosted top-quark system, at least one $b$-tagged jet, and large missing transverse momentum. To discriminate this complex signature from SM backgrounds (primarily $\mathrm{t}\bar{\mathrm{t}}\mu^+\mu^-$ and $\mathrm{b}\bar{\mathrm{b}}\mu\mu\mu\nu$), we developed a BDT-based machine learning classifier trained on a comprehensive set of kinematic variables. The classifier output was used in a profile binned-likelihood framework to extract signal significance.

The main results demonstrate that the LHC can probe $\phi'$ masses from $5$ to $325$~GeV and $\chi_\mathrm{u}$ masses up to almost $2$~TeV at the HL-LHC with $3000~\mathrm{fb}^{-1}$. This represents a significant extension of the accessible parameter space compared to previous searches. The analysis reveals that detection prospects for low-mass particles are substantially enhanced when it is kinematically possible to simultaneously access the heavy degrees of freedom in the UV completion. This strategy is particularly powerful for scenarios with generationally dependent couplings that would be difficult to probe at low-energy beam experiments.


\subsection{Sensitivity to Vector Leptoquarks at the LHC}

Chapter~\ref{ch:vector_lq} focused on the phenomenology of the $U_1$ vector leptoquark, transforming as $({\bf 3},\,{\bf 1},\,2/3)$ under the SM gauge group. This particle is particularly interesting as it represents one of the few viable single-mediator solutions capable of addressing both the $R_{D^{(*)}}$ anomalies in charged-current $B$-meson decays and providing signatures accessible at the LHC. The key feature of this model is that the leptoquark couples preferentially to third-generation fermions through $\mathrm{b}\tau$ and $\mathrm{t}\nu_\tau$ vertices.

We developed a comprehensive analysis strategy combining three complementary production channels: single leptoquark production (sLQ), pair production (dLQ), and non-resonant production (non-res), where the leptoquark mediates $\mathrm{pp}\to\tau^+\tau^-$ processes at the amplitude level. Each channel has distinct sensitivity to the leptoquark mass $M_U$ and its coupling to fermions $g_U$. The single-LQ production cross section scales as $g_U^2$, making it particularly sensitive to the coupling strength, while pair production depends primarily on the QCD couplings and provides model-independent constraints on the mass.

A critical aspect of this work was the systematic study of how the chiral structure of the leptoquark couplings affects the phenomenology. We considered three scenarios: exclusive couplings to left-handed currents, mixed chirality, and exclusive right-handed currents. Each scenario predicts different kinematic distributions and different regions of parameter space capable of explaining the $B$-meson anomalies. We found that while the sensitivity is highly dependent on chirality, in all cases the combination of production channels allows for comprehensive coverage of the parameter space.

The analysis employed a BDT-based machine learning approach to maximize signal-background discrimination, considering final states with varying $\tau$-lepton and $b$-jet multiplicities. Our results show that the ML approach significantly improves sensitivity compared to traditional cut-based analyses, particularly at large values of $g_U$. The HL-LHC projections indicate complete coverage of the parameter space solving the $B$-anomalies for leptoquark masses up to $5.0$~TeV.

An important finding concerns the impact of additional particles predicted by complete gauge theories. We evaluated the effects of a companion $Z'$ boson on non-resonant production and found that interference effects can have considerable impact on the sensitivity regions, depending on the specific masses and couplings. This demonstrates the importance of considering the full particle spectrum of UV-complete models when interpreting LHC searches.


\section{Common Themes and Methodological Contributions}

Both studies in this thesis exemplify several key principles in modern collider phenomenology:

\paragraph{Machine Learning as an Essential Tool.} Traditional cut-based analyses, while transparent and robust, often fail to fully exploit the information contained in the high-dimensional phase space of LHC events. Both our studies demonstrate that machine learning algorithms, particularly BDTs, can learn complex, non-linear correlations between kinematic variables to construct powerful discriminators that significantly enhance sensitivity. The improvements are most pronounced in scenarios with low signal-to-background ratios and overlapping kinematic distributions---precisely the challenging cases where new physics is most likely to remain hidden.

\paragraph{Importance of UV Completions.} The $U(1)_{T^3_R}$ study illustrates a crucial point: low-energy phenomena are often best probed by accessing the high-energy degrees of freedom that arise in the UV completion of the theory. The conventional approach of searching for light particles through direct low-energy production may miss important regions of parameter space. By considering associated production with heavy partners, we achieve sensitivity to light scalars that would otherwise be inaccessible at hadron colliders.

\paragraph{Model-Dependent Phenomenology.} The leptoquark study emphasizes that the detailed phenomenology---production rates, decay modes, kinematic distributions, and optimal search strategies---depends critically on the specific theoretical framework. The chiral structure of couplings, the presence of companion particles, and interference effects all play essential roles. Comprehensive phenomenological studies must explore the full parameter space and account for correlations between observables.

\paragraph{Complementarity of Search Channels.} Both studies demonstrate that different production mechanisms provide complementary sensitivity to different regions of parameter space. In the $U(1)_{T^3_R}$ case, $\chi_\mathrm{u}$-t fusion and $\chi_\mathrm{u}\bar{\chi}_\mathrm{u}$ production probe different couplings. In the leptoquark case, single, pair, and non-resonant production have different dependencies on $M_U$ and $g_U$. Optimal search strategies must consider all relevant channels and their combination.


\section{Outlook and Future Directions}

The work presented in this thesis opens several avenues for future research, both in extending the current analyses and in exploring related phenomenological questions.


\subsection{Advanced Analysis Techniques}

\paragraph{Tau Polarization Measurements.} In both studies, $\tau$ leptons play important roles in the final states. The polarization of $\tau$ leptons carries information about the chiral structure of the new physics couplings. Developing machine learning algorithms specifically designed to extract $\tau$ polarization information from the kinematics of hadronic $\tau$ decays could provide an additional discriminating variable to enhance signal sensitivity and to distinguish between different theoretical scenarios. This would require sophisticated reconstruction of the $\tau$ decay products and potentially the application of deep learning techniques to model the complex correlations between the visible decay products and the initial $\tau$ polarization state.

\paragraph{Attention Mechanisms and Graph Neural Networks.} While BDTs have proven highly effective, recent developments in deep learning offer promising alternatives. Graph Neural Networks (GNNs) can naturally handle the variable-multiplicity jet and lepton content of LHC events by representing each event as a graph where particles are nodes and their relationships are encoded in edges. Attention mechanisms allow the network to dynamically focus on the most relevant features for each event. These architectures have shown impressive performance in jet tagging and event classification tasks and may offer further improvements in signal-background discrimination for the complex final states considered in this thesis.

\paragraph{Unbinned Likelihood Methods.} Our statistical analyses employed binned likelihood approaches, which are standard in HEP but inevitably lose some information through binning. Recent work on unbinned likelihood methods using neural density estimation could provide more powerful hypothesis tests. These methods use neural networks to learn the full probability density of events in the multidimensional feature space, allowing for optimal statistical inference without binning losses.


\subsection{Extended Phenomenology of the $U(1)_{T^3_R}$ Model}

\paragraph{Dark Matter in the Neutrino Sector.} The $U(1)_{T^3_R}$ model predicts right-handed neutrinos $\nu_R$ that are required for anomaly cancellation. In certain parameter regimes, the lightest right-handed neutrino or a combination of neutrino states could serve as a viable dark matter candidate. A detailed study of the dark matter phenomenology in this sector would involve:
\begin{itemize}
    \item Computing the relic abundance through freeze-out mechanisms, considering annihilation channels mediated by the $Z'$ gauge boson and the dark Higgs $\phi'$.
    \item Evaluating direct detection constraints from neutrino-nucleon scattering mediated by $Z'$ exchange.
    \item Studying indirect detection signatures from dark matter annihilation in galactic halos, particularly into charged leptons through $\phi'$ mediation.
    \item Investigating collider signatures of dark matter production in association with visible particles, such as mono-jet, mono-$Z$, or mono-Higgs channels.
\end{itemize}
This would require a comprehensive analysis of the neutrino mass matrix, mixing patterns, and the interplay between the $U(1)_{T^3_R}$ breaking scale and the neutrino mass scale.

\paragraph{Electroweak Precision Tests and $Z'$ Phenomenology.} The $Z'$ gauge boson arising from the spontaneously broken $U(1)_{T^3_R}$ symmetry has not been the primary focus of this thesis but warrants detailed study. The $Z'$ couples to right-handed SM fermions and could contribute to precision electroweak observables through loop corrections. A comprehensive study would include:
\begin{itemize}
    \item Computing one-loop corrections to $Z$-pole observables and comparing with LEP precision measurements.
    \item Evaluating contributions to rare processes such as $B_s$-$\bar{B}_s$ mixing and $\mu\to e$ conversion in nuclei.
    \item Studying direct production of the $Z'$ at the LHC through Drell-Yan processes and its subsequent decays to di-leptons or di-jets.
    \item Investigating the interplay between $Z'$ mass, coupling strength, and the constraints from both high-energy LHC searches and low-energy precision measurements.
\end{itemize}

\paragraph{$\phi'$ Production in Rare Decays.} While we focused on $\phi'$ production in association with vector-like quarks, the dark Higgs could also be produced in rare decays of SM particles, such as $B\to K\phi'$ or $t\to c\phi'$. These processes would provide complementary probes of the scalar couplings and could be accessible at Belle~II and future colliders. The branching ratios depend sensitively on the flavor structure of the Yukawa couplings after mixing with vector-like fermions.


\subsection{Leptoquark Phenomenology Beyond the LHC}

\paragraph{Complementarity with Low-Energy Experiments.} While the LHC provides direct sensitivity to leptoquark masses up to several TeV, low-energy precision experiments offer complementary constraints on the coupling strength. Future $B$-factory experiments such as Belle~II will measure $R_{D^{(*)}}$ with improved precision, while proposed charged lepton flavor violation experiments will probe $\tau\to\mu\gamma$ and $\mu\to e$ conversion with unprecedented sensitivity. A global fit combining LHC searches with low-energy constraints would provide the most comprehensive picture of the allowed parameter space.

\paragraph{Future Colliders.} The phenomenology of leptoquarks at future colliders---such as a potential High-Energy LHC (HE-LHC) at $27$~TeV, a Future Circular Collider (FCC-hh) at $100$~TeV, or a muon collider---would differ significantly from the studies presented here. The higher center-of-mass energies would allow direct production of much heavier leptoquarks, while the cleaner initial state of lepton colliders would enable precision measurements of leptoquark properties if they are discovered. Phenomenological studies for these future facilities would be valuable in optimizing their design and physics programs.

\paragraph{Simplified Model Framework.} The leptoquark analysis in this thesis considered a specific gauge model with defined couplings. A complementary approach would be to develop a simplified model framework parameterizing leptoquark couplings in a model-independent way, similar to the effective field theory approach. This would allow for interpretation of experimental searches in terms of general coupling structures and facilitate comparisons between different theoretical models.


\subsection{Broader Phenomenological Questions}

\paragraph{Multi-Component New Physics.} Real BSM theories typically predict multiple new particles that can appear simultaneously in LHC events. Future work could explore scenarios where both leptoquarks and $Z'$ bosons, or vector-like quarks and dark Higgs bosons, are produced in the same event. The interference and correlation effects in such scenarios could provide powerful discrimination between different theoretical frameworks.

\paragraph{Systematic Uncertainties and Detector Effects.} The phenomenological studies in this thesis included realistic detector effects through DELPHES fast simulation but did not explore the full range of systematic uncertainties that would affect experimental analyses. A more detailed treatment would include variations of parton distribution functions, renormalization and factorization scales, jet energy scales, $b$-tagging efficiencies, and pileup modeling. Understanding the impact of these uncertainties on the final sensitivity is crucial for translating phenomenological projections into experimental search programs.

\paragraph{Connection to Cosmology.} Many BSM models motivated by collider anomalies also have cosmological implications. The $U(1)_{T^3_R}$ model's dark matter candidate could contribute to structure formation and leave signatures in cosmic microwave background anisotropies. Leptoquark models with heavy right-handed neutrinos could play a role in leptogenesis and the matter-antimatter asymmetry. Exploring these connections would strengthen the theoretical motivation for the collider searches and potentially provide additional constraints on the parameter space.


\section{Closing Remarks}

The Large Hadron Collider has entered a new era with Run~3 and will continue to accumulate data for at least another decade through the High-Luminosity program. This unprecedented dataset offers remarkable opportunities to search for physics beyond the Standard Model, but also presents significant challenges in extracting weak signals from enormous backgrounds.

This thesis has demonstrated that carefully designed phenomenological studies, leveraging modern computational tools including machine learning, are essential for maximizing the discovery potential of the LHC. By identifying optimal search strategies, quantifying sensitivity to specific BSM scenarios, and understanding the interplay between different observables, phenomenological work provides crucial guidance for the experimental program.

The two studies presented here---searching for light scalars in the $U(1)_{T^3_R}$ model and probing vector leptoquarks---represent concrete examples of this approach. Both analyses explore BSM scenarios motivated by persistent experimental anomalies and develop novel search strategies that could significantly extend the reach of the ATLAS and CMS experiments. The methodologies developed, particularly the application of machine learning for signal-background discrimination and the framework for statistical interpretation, are broadly applicable to other BSM searches.

As the LHC continues to probe the energy frontier and Belle~II explores the intensity frontier, the coming years promise to be an exciting time for particle physics. Whether these experiments discover new particles or further constrain the parameter space of BSM models, the result will advance our understanding of nature's fundamental laws. The phenomenological tools and strategies developed in this thesis contribute to this ongoing quest, bridging the gap between theoretical ideas and experimental reality, and helping to ensure that if new physics exists within the reach of current experiments, we will find it.

The search for physics beyond the Standard Model is far from over. The path forward requires continued synergy between theory, phenomenology, and experiment, with each discipline informing and challenging the others. It is in this collaborative spirit that the work presented in this thesis is offered as a contribution to the collective effort to unravel the mysteries that the Standard Model leaves unanswered.
