\chapter{Discussion and results}
%\label{ch:discussion}
\section{U1}

The LHC will continue to run with pp collisions at $\sqrt{s} = 13.6$~\textrm{TeV} for the next decade. Given the increase in the integrated luminosity expected from the high-luminosity program, it is important to consider unexplored new physics phase space that diverges from the conventional assumptions made in many BSM theories, and which could have remained hidden in processes that have not yet been thoroughly examined. It is additionally crucial to explore advanced analysis techniques, in particular the use of artificial intelligence algorithms, to enhance the probability of detecting these rare corners where production cross sections are lower and discrimination from SM backgrounds is difficult. 

In this work, we examine a model based on a $U(1)_{T^3_R}$ extension of the SM, which can address various conceptual and experimental issues with the SM, including the mass hierarchy between generations of fermions, the thermal dark matter abundance, and the muon $g - 2$, $R_{(D)}$, and $R_{(D^*)}$ anomalies. This model contains a light scalar boson $\phi'$, with potential masses below the electroweak scale, and~\textrm{TeV}-scale vector-like quarks $\chi_\mathrm{u}$. We consider the scenario where the scalar $\phi'$ has family non-universal fermion couplings and $m(\phi') \ge 1$~\textrm{GeV}, as was suggested in Ref.~\parencite{Dutta2020}, and thus the $\phi^{\prime}$ can primarily decay to a pair of muons. Previous works in Refs.~\parencite{Dutta2023, Banerjee_2016} considered scenarios motivating a search methodology with a merged diphoton system from $\phi' \to \gamma\gamma$ decays. The authors of Ref~\parencite{Dutta2023}, in which $m(\phi') < 1$~\textrm{GeV},  indeed pointed out that if the $\phi'$ is heavier than about 1~\textrm{GeV}, then decays to $\mu^+ \mu^-$ can become the preferable mode for discovery, which is the basis for the work presented in this paper. We further note that the final state topology studied in this paper would represent the most important mode for discovery at $m(\phi') < 2 m_{\mathrm{t}}$ where the $\phi' \to \mathrm{t\bar{t}}$ decay is kinematically forbidden. 

The main result of this paper is that we have shown that the LHC can probe the visible decays of new bosons with masses below the electroweak scale, down to the~\textrm{GeV}-scale, by considering the simultaneous production of heavy QCD-coupled particles, which then decay to the SM particles that contain large momentum values and can be observed in the central regions of the CMS and ATLAS detectors. The boosted system combined with innovative machine learning algorithms allows for the signal extraction above the lower-energy SM background. The LHC search strategy described here can be used to discover the prompt decay of new light particles.  An important conclusion from this paper is that the detection prospects for low-mass particles are enhanced when it is kinematically possible to simultaneously access the heavy degrees of freedom which arise in the UV completion of the low-energy model.  This specific scenario in which the couplings of the light scalars are generationally dependent, with important coupling values to the top quark, is an ideal example which would be difficult to directly probe at low energy beam experiments.

The proposed data analysis represents a competitive alternative 
to complement searches already being conducted at the LHC, allowing us to probe $\phi'$ masses from 1 to 325 \textrm{GeV}, for $m(\chi_{\mathrm{u}})$ values up to almost 2~\textrm{TeV}, at the HL-LHC. Therefore, we strongly encourage the ATLAS and CMS Collaborations to consider the proposed analysis strategy in future new physics searches. 


\section{Leptoquark}


Experimental searches for $\lq$s with preferential couplings to third generation fermions are currently of great interest due to their potential to explain observed tensions in the $R(D)$ and $R(D^{*})$ decay ratios of $\Bm$ mesons with respect to the SM predictions. Although the LHC has a broad physics program on searches for $\lq$s, it is very important to consider the impact of each search within wide range of different theoretical assumptions within a specific model. In addition, in order to improve the sensitivity to detect possible signs of physics beyond the SM, it is also important to strongly consider new computational techniques based on machine learning (ML). Therefore, we have studied the production of $U_1$ $\lq$s with preferential couplings to third generation fermions, considering different couplings, masses and chiral currents. These studies have been performed considering $\mathrm{p}\,\mathrm{p}$ collisions at $\sqrt{s} = 13\tev$ and $13.6\tev$ and different luminosity scenarios, including projections for the high luminosity LHC. A ML algorithm based on boosted decision trees is used to maximize the signal significance. The signal to background discrimination output of the algorithm is taken as input to perform a profile binned-likelihood test statistic to extract the expected signal significance. 

The expected signal significance for s$\lq$, d$\lq$ and non-res production, and their combination, is presented as contours on a two dimensional plane of $g_U$ versus $M_U$. We present results for the case of exclusive couplings to left-handed, mixed, and exclusive right-handed currents. For the first two, the region of the phase space that could explain the $\Bm$ meson anomalies is also presented. We confirm the findings of previous works that the largest production cross-section and best overall significance comes from the combination of d$\lq$ and non-res production channels. We also find that the sensitivity to probe the parameter space of the model is highly dependent on the chirality of the couplings. Nevertheless, the region solving the $\Bm$-meson anomalies also changes with each choice, such that in all evaluated cases we find ourselves just starting to probe this region at large $M_U$.

Our studies compare our exclusion regions with respect to the latest reported results from the ATLAS and CMS Collaborations. The comparison suggests that our ML approach has a better sensitivity than the standard cut-based analyses, especially at large values of $g_U$. In addition, our projections for the HL-LHC cover the whole region solving the B-anomalies, for masses up to $5.00\tev$.

Finally, we consider the effects of a companion $\zb^{\prime}$ boson on non-res production. We find that such a contribution can have a considerable impact on the LQ sensitivity regions, depending on the specific masses and couplings. In spite of this, we still consider non-res production as an essential channel for probing LQs in the future.
