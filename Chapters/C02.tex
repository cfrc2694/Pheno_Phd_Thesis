\chapter{LHC Conditions, and Phenomenological Framework}

Since its formulation, the Standard Model (SM) has proven remarkably successful in describing the fundamental particles and interactions, and its parameters have been measured with increasing precision over several decades; however, as we have commented in the last chapter, various theoretical and experimental observations suggest that the SM is incomplete and certain details of the standard model seem to demand a better explanation, motivating the exploration of new physics (NP) beyond the standard model (BSM) and in turn methods in the search for this new physics. This pursuit requires both the development of theoretical models and the design of experimental strategies to test them. Particle physics phenomenology plays a crucial role in this endeavor by bridging theoretical predictions with experimental observations, feasibility and searches, particularly in high-energy experiments such as those conducted at the Large Hadron Collider (LHC), as well as in high-precision low-energy measurements.

The LHC is a proton-proton ($pp$) collider that has been operating since 2009, achieving center-of-mass collision energies ranging from $7~\mathrm{TeV}$ to $13.6~\mathrm{TeV}$. During its Run~I~(2010-2013), the LHC reached $7~\mathrm{TeV}$ in 2010-2011 and $8~\mathrm{TeV}$ in 2012, leading to landmark discoveries such as the Higgs boson in 2012. Run~II~(2015-2018) operated at $13~\mathrm{TeV}$ and achieved an instantaneous luminosity of $1.5 \times 10^{34}~\mathrm{cm}^{-2}~\mathrm{s}^{-1}$, yielding approximately 1000 top-quark pairs and 50 Higgs bosons per minute. Run~III~(2022-2025) is currently underway with collisions at a record energy of $13.6~\mathrm{TeV}$ and even higher luminosities. Following this, the High-Luminosity LHC (HL-LHC) is expected to begin operations around 2029. This major upgrade aims to increase the integrated luminosity by more than an order of magnitude, targeting up to $3\,\mathrm{ab}^{-1}$ of data per experiment. The HL-LHC will significantly enhance the sensitivity to rare processes, improve the precision of Standard Model measurements, and boost the discovery potential for BSM phenomena.

These collisions take place at four main interaction points, each equipped with a sophisticated particle detector designed to record and analyze the outcomes. Two of the largest and most comprehensive experiments at the LHC are the Compact Muon Solenoid (CMS) and ATLAS detectors. Both are multipurpose detectors with broad physics programs, designed to explore a wide range of phenomena. They perform precision measurements within the electroweak sector of the SM, investigate the dynamics of quarks and gluons (including through heavy-ion collisions), and carry out extensive searches for BSM signatures using $pp$ collision data. While CMS and ATLAS differ in their detector designs and reconstruction strategies, their physics goals are largely overlapping, and their results are complementary. Throughout this work, phenomenological studies and comparisons are primarily developed in the context of CMS, although several results from ATLAS are also referenced, given the close alignment in sensitivity and scope.

\section{Coordinate System and Collision Parameters}
To fully describe the CMS experiment, some of its parameters should be outlined. Measurements performed at CMS adopt the coordinate system whose origin lies at the collision point, with the $y$-axis pointing vertically upward, the $x$-axis pointing radially inward towards the centre of the LHC and the $z$-axis along the beam direction. The azimuthal angle $\phi$ is measured in the $x y$-plane from the $x$-axis and the polar angle, $\theta$, is measured from the $z$-axis, as shown in the Fig.~\ref{fig_coordinates}. 
\begin{center}
	\includegraphics[width=0.8\textwidth]{Images/coordinatechart.png}
	\captionof{figure}{Coordinate system employed by the CMS experiment (retrieved from~\parencite{cmsplots}).}\label{fig_coordinates}
\end{center}

The event rate $R$ for a physical process (e.g., $pp \to X$) is governed by the accelerator's luminosity $\mathcal{L}$ and the process cross section $\sigma$. Luminosity quantifies the performance of a collider to produce interactions, establishing the proportionality,
\begin{equation}
	\frac{d R}{d t} = \mathcal{L} \sigma,
\end{equation}
where $\sigma$ (typically measured in barns, $1\,\text{b} = 10^{-24}\,\text{cm}^2$) encodes the interaction probability. For LHC proton bunches colliding head-on with Gaussian transverse profiles, the instantaneous luminosity is~\parencite{Herr:941318,book:1123430}:
\begin{equation}
	\mathcal{L} = \frac{f N_{b}}{4\pi} \frac{N_{1} N_{2}}{\sigma_{x} \sigma_{y}}\label{eq_lumi}
\end{equation}
Here, $N_{1,2}$ are proton counts per bunch, $f$ is the bunch collision frequency, $N_{b}$ is the number of bunches, and $\sigma_{x,y}$ are transverse beam widths. 

Integrating $\mathcal{L}$ over time yields the total integrated luminosity $L$, linking directly to the observed event count $N$:
\begin{equation}
	L = \int \mathcal{L}\, dt \quad \Rightarrow \quad N = L \sigma.
\end{equation}
%At Run~II luminosities ($\mathcal{L} \approx 1.5 \times 10^{34}\,\text{cm}^{-2}\text{s}^{-1}$), a $1\,\text{pb}$ cross section produces $\sim\!15$ events per second. This framework underpins all LHC physics analyses, from Higgs measurements ($\sigma_{pp\to H} \sim 50\,\text{pb}$ at $13\,\text{TeV}$) to BSM searches for rare processes ($\sigma_{\text{BSM}} \ll 1\,\text{fb}$).
  
The Gaussian beam approximation in~\eqref{eq_lumi} ignores hourglass effects (beam divergence near interaction points) and dynamic $\sigma_{x,y}$ variations during fills. CMS mitigates these via real-time luminosity monitoring using pixel clusters~\parencite{Sirunyan2021}, with systematic uncertainties below $2\%$. High $\mathcal{L}$ also introduces pileup—multiple $pp$ interactions per bunch crossing—which complicates $\eta$/$\phi$ measurements but is corrected using vertex isolation algorithms.


The most important parameters of an accelerator are the following:
\begin{description}

	\item[Centre-of-mass energy]  $(\sqrt{s})$ The energy of the colliding particles in the centre-of-mass (CM) frame. Since the total momentum of the particles in the CM frame is zero, the CM energy is simply the square-root of the total 4-momentum:
	$$
	\begin{gathered}
		\sqrt{s}=\sqrt{P^{\mu} P_{\mu}}, \\
		P^{\mu}=\sum_{i} p_{i}^{\mu}.
	\end{gathered}
	$$
	The energy in the CM frame must be greater than the total mass of the particles being produced. Thus, the CM energy determines which particles can be studied in the accelerator.
\end{description}
\begin{center}
  		% CMS detector - left perspective
		\tdplotsetmaincoords{75}{50} % to reset previous setting
		\begin{tikzpicture}[scale=2.6,tdplot_main_coords,rotate around x=90]
			
			% VARIABLES
			\def\rvec{\L/2/cos(\thetavec)}
			\def\thetavec{18}
			\def\phivec{60}
			\def\L{3.3}    % detector length
			\def\R{0.75}   % detector cylinder radius
			\def\l{4.3}    % beam pipe length
			\def\r{0.04}   % beam pipe radius
			\def\rt{0.042} % beam pipe radius + line thickness
			\def\xmax{1}   % maximum x axis
			\def\ymax{1}   % maximum y axis
			\def\zmin{-\l/2-0.2} % minimum z axis
			\def\zmax{\l/2+0.3}  % maximum z axis
			\def\w{0.3}
			\coordinate (O) at (0,0,0);
			\coordinate (Z) at (0,0,\L/2);
			\tdplotsetcoord{O'}{0.022}{\thetavec}{\phivec} % slightly shifted origin
			\tdplotsetcoord{O''}{0.018}{90}{\phivec} % slightly shifted origin
			\tdplotsetcoord{P}{\rvec}{\thetavec}{\phivec}
			
			% CYLINDER behind
			\def\ang{19} % rotate lines to simulate cylinder
			\fill[top color=red!50!black!4,bottom color=red!60!black!2,rotate around z=\ang]
			(0,\R,\L/2) --++ (0,0,-\L) arc(90:270:\R) --++ (0,0,\L) arc(270:90:\R) -- cycle;
			\fill[detector surface] % transverse plane at z=L/2
			(0,0,\L/2) --++ (0,\R,0) arc(90:270:\R) -- cycle;
			\fill[detector surface] % transverse plane at z=-L/2
			(0,0,-\L/2) --++ (0,\R,0) arc(90:270:\R) -- cycle;
			\tdplotdrawarc[detector]{(0,0,\L/2)}{\R}{0}{360}{}{}
			\tdplotdrawarc[detector,thin]{(0,0,-\L/2)}{\R}{0}{360}{}{}
			%\draw[detector,canvas is yx plane at z=-\L/2] (0,0,0) circle(\R);
			\draw[detector,thin, dashed] % transverse plane at z=0
			(90-\ang:\R) arc (90-\ang:270:\R);
			\draw[detector] (0,0,-\L/2)++(90:\R) --++ (0,0,\L); % top horizontal
			\draw[detector] (0,0,-\L/2)++(-90:\R) --++ (0,0,\L); % bottom horizontal
			
			% BEAM PIPE
			\tdplotdrawarc[beam pipe]{(0,0,\l/2)}{\r}{0}{360}{}{}
			%\tdplotdrawarc[beam pipe]{(0,0,-\l/2)}{\r}{\ang-90}{90}{}{}
			%\draw[beam pipe] % cylindric beam pipe
			%  (0,\r,-\l/2) --++ (0,0,\l) arc(90:-90:\r)
			%  --++ (0,0,-\l) arc(-90:90:\r);
			\draw[beam pipe] % beam pipe, thinner in middle
			(0,\r,-\l/2) -- (0,\r,-0.2*\l) -- (90:0.5*\r)
			-- (0,\r,0.2*\l) -- (0,\r,0.5*\l) arc(90:-90:\r)
			-- (0,-\r,0.2*\l) -- (-90:0.5*\r) --
			(0,-\r,-0.2*\l) -- (0,-\r,-\l/2) arc(-90:90:\r);
			\draw[beam pipe] (0,0,\l/2) circle(\r);
			
			% AXES
			%\draw[thick,->] (0,0,0) -- (0,0,1) node[below right]{$z$}; % short
			\draw[axis,-] (0,0,\zmin) -- (0,0,0); % long
			\fill[CMScol] (O) circle(0.5pt) node[right=1,below=1] {IP};
			\draw[axis] (0,0,0.020) -- (0,0,\zmax) node[right=3,above=0.1]{$z$}; % long
			\draw[axis] (0,0.019,0) -- (0,\ymax,0) node[below left]{$y$};
			\draw[axis] (0.022,0,0) -- (\xmax,0,0) node[below=1,right=-2]{$x$};
			
			% LABELS
			\node[mydarkred,above] at (0,\ymax,0) {$\eta=0$};
			\node[mydarkred,above=0.6, left] at (0,\R,0.3*\L) {$\eta>0$};
			\node[mydarkred,above=0.7, right] at (0,\R,-0.2*\L) {$\eta<0$};
			\node[mydarkred,below=1,left] at (0,0,\zmax) {$\eta=\infty$};
			\node[mydarkred,above=1,right] at (0,0,\zmin) {$\eta=-\infty$};
			
			% VECTORS
			%\fill[radius=0.4,red] (P) circle;
			\draw[dashed,myred] (P)  -- (Pxy);
			\draw[dashed,myred] (Py) -- (Pxy);
			\draw[dashed,myred] (P) -- (Pz);
			
			
			\draw[->,miverde,line cap=round,draw opacity=0.9] (O') -- (P) node[anchor=-30] {\contour{white}{$\va*{p}$}};
			\draw[->,miverde,line cap=round] (O') -- (P) node[anchor=-30] {$\va*{p}$};
			
			\draw[->,azulF,line cap=round,draw opacity=0.9] (O') -- (Pxy) node[right, anchor=-100] {\contour{white}{$\va*{p}_T$}};
			% \draw[->,azulF,line cap=round] (O') -- (Pxy) node[right , anchor=-100] {$\va*{p}_T$};
			
			
			% CYLINDER front
			\draw[beam pipe,fill=none] (0,\r,-\l/2) arc(90:-90:\r);
			\fill[detector surface] % transverse plane at z=L/2
			(0,\rt,\L/2) --++ (0,\R-\rt,0) arc(90:-90:\R) --++ (0,\R-\rt,0) arc(-90:90:\rt);
			\fill[detector surface] % transverse plane at z=-L/2
			(0,\rt,-\L/2) --++ (0,\R-\rt,0) arc(90:-90:\R) --++ (0,\R-\rt,0) arc(-90:90:\rt);
			\tdplotdrawarc[detector]{(0,0,\L/2)}{\R}{-90}{90}{}{} % transverse plane at z=L/2
			\tdplotdrawarc[detector]{(0,0,-\L/2)}{\R}{-90}{90}{}{} % transverse plane at z=-L/2
			\draw[beam pipe,fill=none] (0,\r,\l/2) arc(90:-90:\r);
			\draw[detector,very thin, dashed] % transverse plane at z=0
			(90-\ang:\R) arc (90-\ang:-90:\R);
			
			% ANGLES
			\tdplotdrawarc[thick,red!57!black!3] % contour
			{(O)}{0.2}{4}{0.7*\phivec}{}{}

			% white to contour
			\tdplotdrawarc[draw=azulF, line width=0.6pt, draw opacity=0.9]{(O)}{0.2}{0}{\phivec}{above=2,right=0.75,anchor=-30,text=black}{\contour{white}{$\phi$}}
			\tdplotdrawarc[->, azulF]{(O)}{0.2}{0}{\phivec}{above=2,right=0.75,anchor=-30}{$\phi$}


			\tdplotdrawarc[->,rotate around z=\phivec-90,rotate around y=-90]
			{(O)}{0.88}{0}{\thetavec}{anchor=mid east}{$\theta$}
			\tdplotdrawarc[thick,red!58!black!4,rotate around z=\phivec-90,rotate around y=-90] % contour
			{(O)}{0.3}{88}{0.5*(90+\thetavec)}{}{}
			\tdplotdrawarc[-{>[flex'=1]},rotate around z=\phivec-90,rotate around y=-90,line cap=round]
			{(O)}{0.3}{90}{\thetavec}{above=4.5,right=0.5,anchor=mid east}{$\eta$}
			\draw[mydarkred] (0,0,\L/2) --++ (\R,0,0);
			\tdplotdrawarc[thick,red!60!black!6] % contour
			{(Z)}{0.2}{4}{0.7*\phivec}{}{}
			\tdplotdrawarc[draw=none,opacity=0.8]{(Z)}{0.2}{0}{\phivec}{above=2,right=0.7,anchor=-30}{\contour{red!60!black!6}{$\phi$}}
			\tdplotdrawarc[->]{(Z)}{0.2}{0}{\phivec}{above=2,right=0.7,anchor=-30}{$\phi$}
			
			% COMPASS - CMS-ATLAS axis has a ~12° declination (http://googlecompass.com)
			\begin{scope}[shift={(1.1*\R,-\R,0.2*\L)},rotate around y=12]
				\draw[<->,black!50] (-\w,0,0) -- (\w,0,0);
				\draw[<->,black!50] (0,0,-\w) -- (0,0,\w);
				\node[left,black!50,scale=0.6] at (-\w,0,0) {N};
				\node[below=3,left=-2,green!20!black!50,scale=0.6] at (0,0,\w) {Jura};
				%\node[below=1,right,black!50,scale=0.6,align=center] at (\w,0,0) {center of\\the LHC};
				%\node[below=1,right,blue!30!black!50,scale=0.6] at (\w,0,0) {ATLAS};
			\end{scope}
			\draw[->,thick,orange!30!black] (1.4*\w,-\R,-0.1*\L) --++ (2*\w,0,0)
			node[right,scale=0.8,align=center] {center of\\[-1pt]the LHC};
			
		\end{tikzpicture}
  \captionof{figure}{Detailed reparametrization of the coordinate system employed by the CMS experiment (retrieved from~\parencite{cmsplots})}\label{fig_cms_coor}
\end{center}
The following variables are related to the particles being produced rather than the accelerator.
\begin{description}
	\item[Decay width] ( $\Gamma)$ The decay rate is the probability that a given particle will decay per unit time. Since a particle can have multiple decay modes, the total decay rate is the sum of the decay rates for each mode~\parencite{book:1123430}. The relative frequency of a decay mode is the branching ratio, given by
	$$
	\mathrm{BR}(j)=\frac{\Gamma(j)}{\Gamma} .
	$$
	\item[Cross-section] $(\sigma)$ The cross-section is a measure of the probability that an interaction will occur from a collision. It is a quantum-mechanical analogue of the "effective size" of the particles involved in an interaction.

	\item[Pseudo-rapidity] $(\eta)$ Instead of using the polar angle, CMS measurements involve the pseudo-rapidity, defined by
	$$
	\eta=-\ln \left(\tan \frac{\theta}{2}\right)
	$$
	The main advantage of using the pseudo-rapidity is that distributions over it tend to be closer to a uniform distribution than those over the polar angle, see Fig.~\ref{fig_cms_coor}. Furthermore, the difference in pseudo-rapidity is invariant under Lorentz boosts along the beam direction~\parencite{book:1123430}.
	\item[Transverse Momentum] ($p_T$) Refers to the component of momentum which is perpendicular to the beam line. It is usually preferred over full momentum because momentum along the beamline may just be left over from the beam particles, while the transverse momentum is always associated with whatever physics happened at the vertex, see Fig.~\ref{fig_cms_coor}.
	\item[Missing transverse energy and momentum] $\left(E_{T}^{\text {miss }} \& p_{T}^{\text {miss }}\right)$ Missing energy and momentum refers to the energy and momentum that is not detected but is expected to be there as a consequence of energy conservation and momentum conservation. This momentum is often carried by particles that do not interact electromagnetically or strongly and are therefore difficult to detect~\parencite{book:1123430}. Missing energy and momentum provides an indirect measurement of undetectable particles in hadron colliders such as neutrinos. Missing momentum reconstructions focus on the transverse direction, where total momentum is expected to be zero.
\end{description}
