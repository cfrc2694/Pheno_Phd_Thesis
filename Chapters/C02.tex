\chapter{Phenomenological Framework for LHC Observables, Searches and Analysis}

Since its formulation, the Standard Model (SM) has proven remarkably successful in describing the fundamental particles and interactions, and its parameters have been measured with increasing precision over several decades. However, as discussed in the previous chapter, various theoretical and experimental observations suggest that the SM is incomplete. As outlined previously, this is motivated by theoretical shortcomings—such as the hierarchy problem, the absence of a dark matter candidate, and non-zero neutrino masses—as well as by experimental anomalies. These limitations motivate the search for new physics (NP) beyond the SM (BSM).

The search for BSM physics proceeds along two main axes: the construction of theoretical extensions to the SM, and the development of experimental methods to probe them. A necessary condition for any viable BSM model is consistency with existing experimental data, which places strong constraints on its parameter space. These constraints include lower limits on the masses of new particles from direct searches at high-energy colliders, and upper bounds on couplings and mixing angles from precision measurements at both high and low energies, which are sensitive to virtual corrections.

Phenomenology connects theoretical models to experimental observables by calculating cross sections, decay rates, and other signatures for given model parameters. A critical function of this field is to assess the experimental feasibility of BSM scenarios—evaluating whether predicted signals would be observable above background processes given the capabilities of current and future detectors. This involves estimating production rates, modeling detector acceptance and efficiency, and developing discrimination variables to maximize signal-to-background ratios. This feasibility assessment is essential for designing analysis strategies, particularly at the Large Hadron Collider (LHC), where signals of new physics must be discriminated from large Standard Model backgrounds.

The LHC is a proton-proton ($pp$) collider that has been operating since 2009, achieving center-of-mass collision energies ranging from $7~\mathrm{TeV}$ to $13.6~\mathrm{TeV}$. During its Run~I~(2010-2013), the LHC reached $7~\mathrm{TeV}$ in 2010-2011 and $8~\mathrm{TeV}$ in 2012, leading to landmark discoveries such as the Higgs boson in 2012. Run~II~(2015-2018) operated at $13~\mathrm{TeV}$ and achieved an instantaneous luminosity of $1.5 \times 10^{34}~\mathrm{cm}^{-2}~\mathrm{s}^{-1}$. Run~III~(2022-2025) is currently underway with collisions at a record energy of $13.6~\mathrm{TeV}$ and even higher luminosities. Following this, the High-Luminosity LHC (HL-LHC) is expected to begin operations around 2029. 

\section{Experiments at Colliders}

When two particle bunches from colliding beams cross each other, they generate individual interactions known as events. 
At the LHC, the beam intensity is so high that multiple interactions can take place in a single event; this phenomenon is referred to as pile-up. 
These collisions occur at four main interaction points, each hosting a large particle detector designed to record and analyze the outcomes. 
Among them, the Compact Muon Solenoid (CMS) and ATLAS are the largest and most comprehensive experiments. Both are multipurpose detectors with broad physics programs, capable of exploring a wide range of phenomena. They perform precision measurements within the electroweak sector of the Standard Model, probe the dynamics of quarks and gluons (including through heavy-ion collisions), and conduct extensive searches for physics beyond the Standard Model using $pp$ collision data. 
While CMS and ATLAS differ in their detector designs and reconstruction strategies, their physics goals are largely overlapping, and their results are complementary. 
Throughout this work, phenomenological studies and comparisons are primarily developed in the context of CMS, although several results from ATLAS are also referenced, given the close alignment in sensitivity and scope. 

To fully describe the CMS experiment, some of its parameters should be outlined. Measurements performed at CMS adopt the coordinate system whose origin lies at the collision point, with the $y$-axis pointing vertically upward, the $x$-axis pointing radially inward towards the centre of the LHC and the $z$-axis along the beam direction. The azimuthal angle $\phi$ is measured in the $x y$-plane from the $x$-axis and the polar angle, $\theta$, is measured from the $z$-axis, as shown in the Fig.~\ref{fig_coordinates}. 
\begin{center}
	\includegraphics[width=0.8\textwidth]{Images/coordinatechart.png}
	\captionof{figure}{Coordinate system employed by the CMS experiment (retrieved from~\parencite{cmsplots}).}\label{fig_coordinates}
\end{center}

The event rate $R$ for a physical process (e.g., $pp \to X$) is governed by the accelerator's luminosity $\mathcal{L}$ and the process cross section $\sigma$. Luminosity quantifies the performance of a collider to produce interactions, establishing the proportionality,
\begin{equation}
	\frac{d R}{d t} = \mathcal{L} \sigma,
\end{equation}
where $\sigma$ (typically measured in barns, $1\,\text{b} = 10^{-24}\,\text{cm}^2$) encodes the interaction probability. For LHC proton bunches colliding head-on with Gaussian transverse profiles, the instantaneous luminosity is~\parencite{Herr:941318,book:1123430}:
\begin{equation}
	\mathcal{L} = \frac{f N_{b}}{4\pi} \frac{N_{1} N_{2}}{\sigma_{x} \sigma_{y}}\label{eq_lumi}
\end{equation}
Here, $N_{1,2}$ are proton counts per bunch, $f$ is the bunch collision frequency, $N_{b}$ is the number of bunches, and $\sigma_{x,y}$ are transverse beam widths. 

Integrating $\mathcal{L}$ over time yields the total integrated luminosity $L$, linking directly to the observed event count $N$:
\begin{equation}
	L = \int \mathcal{L}\, dt \quad \Rightarrow \quad N = L \sigma.
\end{equation}

%This major upgrade aims to increase the integrated luminosity by more than an order of magnitude, targeting up to $3\,\mathrm{ab}^{-1}$ of data per experiment. The HL-LHC will significantly enhance the sensitivity to rare processes, improve the precision of Standard Model measurements, and boost the discovery potential for BSM phenomena.

  
The Gaussian beam approximation in~\eqref{eq_lumi} ignores hourglass effects (beam divergence near interaction points) and dynamic $\sigma_{x,y}$ variations during fills. CMS mitigates these via real-time luminosity monitoring using pixel clusters~\parencite{Sirunyan2021}, with systematic uncertainties below $2\%$. High $\mathcal{L}$ also introduces pileup—multiple $pp$ interactions per bunch crossing—which complicates $\eta$/$\phi$ measurements but is corrected using vertex isolation algorithms.

\begin{center}
  		% CMS detector - left perspective
		\tdplotsetmaincoords{75}{50} % to reset previous setting
		\begin{tikzpicture}[scale=2.6,tdplot_main_coords,rotate around x=90]
			
			% VARIABLES
			\def\rvec{\L/2/cos(\thetavec)}
			\def\thetavec{18}
			\def\phivec{60}
			\def\L{3.3}    % detector length
			\def\R{0.75}   % detector cylinder radius
			\def\l{4.3}    % beam pipe length
			\def\r{0.04}   % beam pipe radius
			\def\rt{0.042} % beam pipe radius + line thickness
			\def\xmax{1}   % maximum x axis
			\def\ymax{1}   % maximum y axis
			\def\zmin{-\l/2-0.2} % minimum z axis
			\def\zmax{\l/2+0.3}  % maximum z axis
			\def\w{0.3}
			\coordinate (O) at (0,0,0);
			\coordinate (Z) at (0,0,\L/2);
			\tdplotsetcoord{O'}{0.022}{\thetavec}{\phivec} % slightly shifted origin
			\tdplotsetcoord{O''}{0.018}{90}{\phivec} % slightly shifted origin
			\tdplotsetcoord{P}{\rvec}{\thetavec}{\phivec}
			
			% CYLINDER behind
			\def\ang{19} % rotate lines to simulate cylinder
			\fill[top color=red!50!black!4,bottom color=red!60!black!2,rotate around z=\ang]
			(0,\R,\L/2) --++ (0,0,-\L) arc(90:270:\R) --++ (0,0,\L) arc(270:90:\R) -- cycle;
			\fill[detector surface] % transverse plane at z=L/2
			(0,0,\L/2) --++ (0,\R,0) arc(90:270:\R) -- cycle;
			\fill[detector surface] % transverse plane at z=-L/2
			(0,0,-\L/2) --++ (0,\R,0) arc(90:270:\R) -- cycle;
			\tdplotdrawarc[detector]{(0,0,\L/2)}{\R}{0}{360}{}{}
			\tdplotdrawarc[detector,thin]{(0,0,-\L/2)}{\R}{0}{360}{}{}
			%\draw[detector,canvas is yx plane at z=-\L/2] (0,0,0) circle(\R);
			\draw[detector,thin, dashed] % transverse plane at z=0
			(90-\ang:\R) arc (90-\ang:270:\R);
			\draw[detector] (0,0,-\L/2)++(90:\R) --++ (0,0,\L); % top horizontal
			\draw[detector] (0,0,-\L/2)++(-90:\R) --++ (0,0,\L); % bottom horizontal
			
			% BEAM PIPE
			\tdplotdrawarc[beam pipe]{(0,0,\l/2)}{\r}{0}{360}{}{}
			%\tdplotdrawarc[beam pipe]{(0,0,-\l/2)}{\r}{\ang-90}{90}{}{}
			%\draw[beam pipe] % cylindric beam pipe
			%  (0,\r,-\l/2) --++ (0,0,\l) arc(90:-90:\r)
			%  --++ (0,0,-\l) arc(-90:90:\r);
			\draw[beam pipe] % beam pipe, thinner in middle
			(0,\r,-\l/2) -- (0,\r,-0.2*\l) -- (90:0.5*\r)
			-- (0,\r,0.2*\l) -- (0,\r,0.5*\l) arc(90:-90:\r)
			-- (0,-\r,0.2*\l) -- (-90:0.5*\r) --
			(0,-\r,-0.2*\l) -- (0,-\r,-\l/2) arc(-90:90:\r);
			\draw[beam pipe] (0,0,\l/2) circle(\r);
			
			% AXES
			%\draw[thick,->] (0,0,0) -- (0,0,1) node[below right]{$z$}; % short
			\draw[axis,-] (0,0,\zmin) -- (0,0,0); % long
			\fill[CMScol] (O) circle(0.5pt) node[right=1,below=1] {IP};
			\draw[axis] (0,0,0.020) -- (0,0,\zmax) node[right=3,above=0.1]{$z$}; % long
			\draw[axis] (0,0.019,0) -- (0,\ymax,0) node[below left]{$y$};
			\draw[axis] (0.022,0,0) -- (\xmax,0,0) node[below=1,right=-2]{$x$};
			
			% LABELS
			\node[mydarkred,above] at (0,\ymax,0) {$\eta=0$};
			\node[mydarkred,above=0.6, left] at (0,\R,0.3*\L) {$\eta>0$};
			\node[mydarkred,above=0.7, right] at (0,\R,-0.2*\L) {$\eta<0$};
			\node[mydarkred,below=1,left] at (0,0,\zmax) {$\eta=\infty$};
			\node[mydarkred,above=1,right] at (0,0,\zmin) {$\eta=-\infty$};
			
			% VECTORS
			%\fill[radius=0.4,red] (P) circle;
			\draw[dashed,myred] (P)  -- (Pxy);
			\draw[dashed,myred] (Py) -- (Pxy);
			\draw[dashed,myred] (P) -- (Pz);
			
			
			\draw[->,miverde,line cap=round,draw opacity=0.9] (O') -- (P) node[anchor=-30] {\contour{white}{$\va*{p}$}};
			\draw[->,miverde,line cap=round] (O') -- (P) node[anchor=-30] {$\va*{p}$};
			
			\draw[->,azulF,line cap=round,draw opacity=0.9] (O') -- (Pxy) node[right, anchor=-100] {\contour{white}{$\va*{p}_T$}};
			% \draw[->,azulF,line cap=round] (O') -- (Pxy) node[right , anchor=-100] {$\va*{p}_T$};
			
			
			% CYLINDER front
			\draw[beam pipe,fill=none] (0,\r,-\l/2) arc(90:-90:\r);
			\fill[detector surface] % transverse plane at z=L/2
			(0,\rt,\L/2) --++ (0,\R-\rt,0) arc(90:-90:\R) --++ (0,\R-\rt,0) arc(-90:90:\rt);
			\fill[detector surface] % transverse plane at z=-L/2
			(0,\rt,-\L/2) --++ (0,\R-\rt,0) arc(90:-90:\R) --++ (0,\R-\rt,0) arc(-90:90:\rt);
			\tdplotdrawarc[detector]{(0,0,\L/2)}{\R}{-90}{90}{}{} % transverse plane at z=L/2
			\tdplotdrawarc[detector]{(0,0,-\L/2)}{\R}{-90}{90}{}{} % transverse plane at z=-L/2
			\draw[beam pipe,fill=none] (0,\r,\l/2) arc(90:-90:\r);
			\draw[detector,very thin, dashed] % transverse plane at z=0
			(90-\ang:\R) arc (90-\ang:-90:\R);
			
			% ANGLES
			\tdplotdrawarc[thick,red!57!black!3] % contour
			{(O)}{0.2}{4}{0.7*\phivec}{}{}

			% white to contour
			\tdplotdrawarc[draw=azulF, line width=0.6pt, draw opacity=0.9]{(O)}{0.2}{0}{\phivec}{above=2,right=0.75,anchor=-30,text=black}{\contour{white}{$\phi$}}
			\tdplotdrawarc[->, azulF]{(O)}{0.2}{0}{\phivec}{above=2,right=0.75,anchor=-30}{$\phi$}


			\tdplotdrawarc[->,rotate around z=\phivec-90,rotate around y=-90]
			{(O)}{0.88}{0}{\thetavec}{anchor=mid east}{$\theta$}
			\tdplotdrawarc[thick,red!58!black!4,rotate around z=\phivec-90,rotate around y=-90] % contour
			{(O)}{0.3}{88}{0.5*(90+\thetavec)}{}{}
			\tdplotdrawarc[-{>[flex'=1]},rotate around z=\phivec-90,rotate around y=-90,line cap=round]
			{(O)}{0.3}{90}{\thetavec}{above=4.5,right=0.5,anchor=mid east}{$\eta$}
			\draw[mydarkred] (0,0,\L/2) --++ (\R,0,0);
			\tdplotdrawarc[thick,red!60!black!6] % contour
			{(Z)}{0.2}{4}{0.7*\phivec}{}{}
			\tdplotdrawarc[draw=none,opacity=0.8]{(Z)}{0.2}{0}{\phivec}{above=2,right=0.7,anchor=-30}{\contour{red!60!black!6}{$\phi$}}
			\tdplotdrawarc[->]{(Z)}{0.2}{0}{\phivec}{above=2,right=0.7,anchor=-30}{$\phi$}
			
			% COMPASS - CMS-ATLAS axis has a ~12° declination (http://googlecompass.com)
			\begin{scope}[shift={(1.1*\R,-\R,0.2*\L)},rotate around y=12]
				\draw[<->,black!50] (-\w,0,0) -- (\w,0,0);
				\draw[<->,black!50] (0,0,-\w) -- (0,0,\w);
				\node[left,black!50,scale=0.6] at (-\w,0,0) {N};
				\node[below=3,left=-2,green!20!black!50,scale=0.6] at (0,0,\w) {Jura};
				%\node[below=1,right,black!50,scale=0.6,align=center] at (\w,0,0) {center of\\the LHC};
				%\node[below=1,right,blue!30!black!50,scale=0.6] at (\w,0,0) {ATLAS};
			\end{scope}
			\draw[->,thick,orange!30!black] (1.4*\w,-\R,-0.1*\L) --++ (2*\w,0,0)
			node[right,scale=0.8,align=center] {center of\\[-1pt]the LHC};
			
		\end{tikzpicture}
  \captionof{figure}{Detailed reparametrization of the coordinate system employed by the CMS experiment (retrieved from~\parencite{cmsplots})}\label{fig_cms_coor}
\end{center}
The following variables are related to the particles being produced rather than the accelerator.
\begin{description}
	\item[Decay width] ( $\Gamma)$ The decay rate is the probability that a given particle will decay per unit time. Since a particle can have multiple decay modes, the total decay rate is the sum of the decay rates for each mode~\parencite{book:1123430}. The relative frequency of a decay mode is the branching ratio, given by
	$$
	\mathrm{BR}(j)=\frac{\Gamma(j)}{\Gamma} .
	$$
	\item[Cross-section] $(\sigma)$ The cross-section is a measure of the probability that an interaction will occur from a collision. It is a quantum-mechanical analogue of the "effective size" of the particles involved in an interaction.

	\item[Pseudo-rapidity] $(\eta)$ Instead of using the polar angle, CMS measurements involve the pseudo-rapidity, defined by
	$$
	\eta=-\ln \left(\tan \frac{\theta}{2}\right)
	$$
	The main advantage of using the pseudo-rapidity is that distributions over it tend to be closer to a uniform distribution than those over the polar angle, see Fig.~\ref{fig_cms_coor}. Furthermore, the difference in pseudo-rapidity is invariant under Lorentz boosts along the beam direction~\parencite{book:1123430}.
	\item[Transverse Momentum] ($p_T$) Refers to the component of momentum which is perpendicular to the beam line. It is usually preferred over full momentum because momentum along the beamline may just be left over from the beam particles, while the transverse momentum is always associated with whatever physics happened at the vertex, see Fig.~\ref{fig_cms_coor}.
	\item[Missing transverse energy and momentum] $\left(E_{T}^{\text {miss }} \& p_{T}^{\text {miss }}\right)$ Missing energy and momentum refers to the energy and momentum that is not detected but is expected to be there as a consequence of energy conservation and momentum conservation. This momentum is often carried by particles that do not interact electromagnetically or strongly and are therefore difficult to detect~\parencite{book:1123430}. Missing energy and momentum provides an indirect measurement of undetectable particles in hadron colliders such as neutrinos. Missing momentum reconstructions focus on the transverse direction, where total momentum is expected to be zero.
\end{description}


\section{Detectors and Subsystems}

A typical collider experiment comprises several main detector subsystems that are used jointly to detect and measure the properties of particles produced in the collision. A \textit{schematic representation} of such a generic multipurpose detector is shown in Fig.~\ref{fig_detector}. The detector is typically composed of several concentric layers, each designed to measure different properties of the particles produced in the collisions. 

\begin{center}
	\includegraphics[width=0.8\textwidth]{Images/transversal_detector.pdf}
	\captionof{figure}{Schematic representation of transversal section of a generic multipurpose detector. The inner detector (ID) is used to measure the trajectories of charged particles, the electromagnetic calorimeter (ECAL) measures the energy of photons and electrons, the hadronic calorimeter (HCAL) measures the energy of hadrons, and the muon system (MS) measures the trajectories of muons. The missing transverse energy (MET) is measured by combining information from all subsystems..}\label{fig_detector}
\end{center}

The innermost subsystem, called the inner detector (ID), is designed to detect electrically charged particles that are long-lived enough to traverse the ID. The most common such particles from the SM are two charged leptons (the electron $e$ and the muon $\mu$ ) and three hadrons (the pion $\pi$, kaon $K$, and proton $p$ ). Regions of ionization produced by such a particle in solid-state or gaseous detector sensors are detected as spatial hits that are fit into a trajectory, referred to as a \textbf{track}. The direction and curvature of the track in a magnetic field yield the particle's momentum vector and electric charge. In some detectors, the ID is enclosed in a Cherenkov-light detector used to measure the velocity of the tracked particles. Combined with the momentum measurement in the ID, this yields the particle mass with sufficient resolution to differentiate between pions, kaons, and protons in a relevant momentum range.

After passing through the tracker, particles produced in the collisions typically enter an electromagnetic calorimeter (ECAL), designed to measure the energies of photons, electrons and positrons. The energy measurement exploits the properties of electromagnetic shower production via photon radiation and $e^{+} e^{-}$ pair production, resulting from the interaction of energetic particles with the ECAL material.

Hadrons deposit energy via hadronic interactions with the detector material. Since this process involves large fluctuations and a variety of energy-deposition mechanisms, precise hadron-energy measurement is achievable only at high-energy colliders, where fluctuations are effectively averaged out. In particular, high-energy quarks and gluons hadronize into a collimated spray of hadrons known as a \textbf{jet}. Containing the jet requires use of a deep hadronic calorimeter (HCAL) beyond the ECAL. While a jet can be identified solely in the calorimeters, its energy is nowadays measured from a combination of the momenta of tracks in the ID and the signals integrated in the ECAL and HCAL. 

The signals from the calorimeters, know as \textbf{towers}, are grouped into jets using a jet clustering algorithm. If and hadronic particle is neutral, it will not leave a track in the ID, but it will still deposit energy in the towers. So, the towers are used to measure the energy of neutral particles, such as photons and neutral hadrons, while the ID tracks are used to measure the energy of charged particles. This approach is known as particle flow (PF) reconstruction and provides a more accurate measurement of the energy of jets. 

Muons do not undergo hadronic interactions, and are heavy enough that they lose energy due to ionization at a low rate. Therefore, they lose only a few GeV while traversing a typical LHC-detector calorimeter. Using this property to identify them, a muon system (MS) is built outside the calorimeter. In high-energy collider detectors, the MS is usually immersed in a magnetic field in order to measure the momenta of muons. Tracks reconstructed in the MS are often combined with tracks in the ID to obtain a high-quality momentum measurement.

When studying final states that include long-lived, weakly interacting particles, such as neutrinos in the SM or dark matter candidates in BSM models, an important reconstructed quantity is missing momentum.  Using three-momentum conservation and the approximate hermeticity of the detector, it is possible to measure the momentum imbalance in the event and to infer the combined momentum of the invisible set of particles. Since the interacting partons in proton collisions generally carry different fractions of the momenta of the incoming hadrons and many of the particles produced fall outside of the acceptance of the sensitive detector, the summed momenta of measured final-state particles along the beam axis $z$ are not expected to cancel. Therefore, experiments at the LHC measure the missing transverse momentum, denoted $E_{\mathrm{T}}^{\text {miss }}$ known as Missing Energy Transverse (\textbf{MET}), where momentum balance is assumed only in the $x-y$ plane transverse to the beam direction.

Collider detectors are mostly designed and constructed for optimal detection of SM particles produced in the collision. However, they can also be used to search for new physics (NP) beyond the SM. In this case, the detector is used to search for signatures of NP, such as new particles or interactions that are not predicted by the SM. The detector subsystems are designed to be sensitive to a wide range of particles and interactions, allowing for the detection of a variety of NP signatures.


\subsection{Jets Reconstruction}

Quarks and gluons are never observed as free particles because of colour confinement. Nevertheless, perturbative QCD treats them as the relevant short-distance degrees of freedom: factorization theorems and asymptotic freedom justify computing hard-scattering matrix elements for incoming and outgoing partons even though QCD becomes non-perturbative at low scales. The strong coupling \(\alpha_s\) grows large and effectively ``blows up'' around the confinement scale \(\Lambda_{\mathrm{QCD}}\); consequently something must happen to quarks and gluons before they reach the detector. In practice the gluon and all quarks except the top hadronize, producing cascades of baryons and mesons that themselves undergo further decays. At the LHC these hadrons typically carry energies comparable to the electroweak scale, and relativistic boosts tend to collimate their decay products into narrow bunches. Those collimated collections of hadrons are the jets we measure at hadron colliders and the objects we use to infer the partons produced in the hard interaction.

Each high-energy parton produced in a collision, such as a quark from the process $gg \rightarrow q\bar{q}$, undergoes hadronization over a distance scale of $\sim10^{-15}\mathrm{m}$, producing a jet of hadrons. The energy composition of these jets is phenomenologically well-established: on average, approximately $60\%$ of the energy is carried by charged particles (mostly $\pi^{\pm}, K^{\pm}$), $30\%$ by photons from $\pi^0 \rightarrow \gamma\gamma$ decays, and $10\%$ by neutral hadrons (mostly neutrons, $K^0, \Lambda^0$). In high-energy jets, the particles are often too collimated to be resolved individually by the calorimeter segmentation. Nevertheless, the jet's energy and momentum can be reconstructed from the total energy deposited.

Phenomenologically one usually assumes that each high-energy parton yields a jet and that the measured jet four-momentum can, to useful accuracy, be related to the original parton four-momentum. Jets are therefore defined operationally using recombination (clustering) algorithms such as Cambridge-Aachen or the (anti-)kT family. Experimentally this means grouping a large number of energy depositions (or particle-flow candidates) observed in the calorimeters and tracker into a much smaller set of jets or subjets. Nothing in the raw detector data, however, indicates a priori how many jets there should be: the clustering procedure and the choice of a resolution scale fix the outcome. In practice one must either specify the desired number of final jets or choose a resolution/stop criterion (for example a distance parameter \(R\), a clustering distance cut, or a jet-mass/subjet-resolution threshold) that determines the smallest substructure to be considered a separate parton-like object.

Modern reconstruction at the LHC typically uses particle-flow (PF) candidates as input together with infrared- and collinear-safe clustering algorithms to define jet four-momenta. The anti-\(k_T\) algorithm~\parencite{Cacciari:2008gp}, implemented in \texttt{FastJet}~\parencite{Cacciari:2011ma}, is widely used in ATLAS and CMS; it groups candidates by proximity in the rapidity–azimuth \((y,\phi)\) plane with a typical distance parameter \(R\sim0.4\)–0.6 and is relatively insensitive to soft radiation and pileup. After clustering, jet energy corrections (JEC) derived from simulation and in-situ calibrations compensate for detector response, pileup, and underlying-event effects, while jet-substructure and tagging algorithms help infer the flavour and origin of the initiating parton.

\subsubsection{Jet algorithms}

Recombination (or sequential clustering) algorithms formalise the intuitive idea that parton showering produces collinear and soft splittings: two nearby and kinematically compatible subjets are merged if they are more likely to have originated from a single parton. A practical implementation requires a measure of ``distance'' between objects; common choices combine an angular separation in the rapidity–azimuth plane, \(\Delta R_{ij}\), with a transverse-momentum weighting. Typical distance measures are
\[
\begin{array}{lll}
k_T: & y_{ij}=\dfrac{\Delta R_{ij}}{R}\min(p_{T,i},p_{T,j}), & y_{iB}=p_{T,i},\\[6pt]
\mathrm{C/A}: & y_{ij}=\dfrac{\Delta R_{ij}}{R}, & y_{iB}=1,\\[6pt]
\text{anti-}k_T: & y_{ij}=\dfrac{\Delta R_{ij}}{R}\min(p_{T,i}^{-1},p_{T,j}^{-1}), & y_{iB}=p_{T,i}^{-1}.
\end{array}
\]
The parameter \(R\) balances jet–jet and jet–beam criteria and sets the geometric size of jets; in LHC analyses typical values are \(R\sim0.4\text{--}0.7\) depending on the physics target.

Two operational modes are useful to distinguish. In an exclusive algorithm one supplies a resolution scale \(y_{\text{cut}}\) and proceeds iteratively:
\begin{enumerate}
  \item compute \(y^{\min}=\min_{i,j}\{y_{ij},y_{iB}\}\);
  \item if \(y^{\min}=y_{ij}<y_{\text{cut}}\) merge \(i\) and \(j\) and repeat;
  \item if \(y^{\min}=y_{iB}<y_{\text{cut}}\) remove \(i\) as beam radiation and repeat;
  \item stop when \(y^{\min}>y_{\text{cut}}\) and keep remaining subjets as jets.
\end{enumerate}
An inclusive algorithm omits \(y_{\text{cut}}\) and instead declares a subjet a final-state jet when its jet–beam distance is the smallest quantity; iteration continues until no inputs remain. Inclusive algorithms therefore produce a variable number of jets, while exclusive algorithms deliver a scale-dependent fixed set.

A practical question is how to combine the kinematics of merged objects. The most common choice in modern experiments is the E-scheme: four-vectors are added, which preserves energy–momentum and yields a physical jet mass useful for substructure and boosted-object tagging. An alternative is to sum three-momenta and rescale the energy to enforce a massless jet; this can be appropriate when the analysis targets massless parton kinematics, but it discards potentially useful jet-mass information.

From a theoretical and experimental viewpoint important properties are infrared and collinear safety: a jet algorithm should give stable results under emission of soft particles or collinear splittings. The \(k_T\), C/A and anti-\(k_T\) families are constructed to satisfy these requirements. Their practical behaviour differs: \(k_T\) naturally follows the physical shower history (soft-first clustering), C/A is purely geometric (useful for declustering and substructure studies), while anti-\(k_T\) produces regular, cone-like jets that are robust and convenient experimentally.

Corrections for pileup and the underlying event are necessary at the LHC. These corrections depend on the jet area (a well-defined concept for sequential algorithms) and are typically performed by estimating an event-wide transverse-momentum density and subtracting the corresponding contribution proportional to the jet area. Finally, because inclusive algorithms can produce jets arbitrarily close to the beam, a minimum jet \(p_T\) threshold (commonly 20–100 GeV depending on the analysis) is imposed to ensure experimental observability and theoretical control.

\subsection{$\tau$ Tagging at Multipurpose Detectors}

The $\tau$ lepton decays hadronically with a probability of $\sim65\%$, producing a narrow ``$\tau$-jet'' that contains only a few charged and neutral hadrons. Hadronic decays are dominated by one- and three-prong topologies and often include neutral pions that promptly convert to photons, giving a sizable electromagnetic fraction in the calorimeters. When the $\tau$ momentum is large compared to its mass the decay products are highly collimated: for $p_T>50\ \mathrm{GeV}$ roughly $90\%$ of the visible energy is contained within a cone of radius $R=\sqrt{(\Delta\eta)^2+(\Delta\varphi)^2}=0.2$. These properties motivate the use of small signal cones and narrow isolation annuli in reconstruction.

Identification exploits three complementary classes of observables:

\begin{itemize}
  \item Calorimetric isolation and shower-shape variables: hadronic $\tau$ decays deposit localized energy in ECAL+HCAL. Experiments use isolation sums and shape ratios to quantify peripheral activity. Example variables are
  \[
    \Delta E_T^{12}=\frac{\sum_{\;0.1<\Delta R<0.2} E_{T,j}}{\sum_{\;\Delta R<0.4} E_{T,i}},\qquad
    P_{\mathrm{ISOL}}=\sum_{\Delta R<0.40}E_T - \sum_{\Delta R<0.13}E_T,
  \]
  which suppress QCD jets that populate the isolation ring.
  \item Charged-track isolation and prong topology: the few, collimated charged tracks of a $\tau$ allow powerful selections. A common procedure defines a matching cone of radius $R_{\mathrm{m}}$ around the calorimeter jet axis to select candidate tracks above a $p_T^{\min}$ threshold. The leading track (tr$_1$) is found and a narrow signal cone $R_{\mathrm{S}}$ around tr$_1$ is used to count associated tracks (1 or 3 prongs preferred). A larger isolation cone $R_{\mathrm{I}}$ is scanned for additional tracks: if no extra tracks with $\Delta z_{\text{impact}}$ consistent with tr$_1$ are found, the candidate is isolated. Typical CMS/ATLAS choices are $R_{\mathrm{S}}\sim0.07$–0.15, $R_{\mathrm{I}}\sim0.3$–0.4, and $p_T^{\min}\sim1$–2 GeV, although values depend on analysis and working point.
  \item Lifetime and vertexing observables: the finite $\tau$ lifetime ($c\tau\approx87\ \mu\mathrm{m}$) produces displaced tracks and, for multi-prong decays, a reconstructible secondary vertex. Impact-parameter significances (2D or 3D) and secondary-vertex properties (mass, flight length significance) are used to separate genuine taus from prompt jets or leptons.
\end{itemize}

Additional discriminants include the invariant mass of the visible decay products computed from tracks and calorimeter clusters (with care to avoid double counting), electromagnetic energy fractions (sensitive to $\pi^0\to\gamma\gamma$), and dedicated shower-strip grouping for nearby photons. For example, invariant-mass reconstruction commonly uses a jet cone $\Delta R_{\text{jet}}\lesssim0.4$ while excluding calorimeter clusters matched to tracks by a minimum separation $\Delta R_{\text{track}}\gtrsim0.08$ to reduce double counting.

Reconstruction algorithms combine these inputs. CMS's Hadron-Plus-Strips (HPS) and modern DeepTau methods explicitly build decay-mode hypotheses and use strip-clustering of photons plus multivariate or deep-learning discriminators to reject jets, electrons, and muons~\parencite{CMS:2022ydz,CMS_DeepTau}. ATLAS employs analogous calorimeter+track based MVAs and BDTs~\parencite{ATLAS:2022fgo}. Typical working points trade efficiency versus background: medium points often give $\tau_{\mathrm{h}}$ efficiencies of order 50–70\% with light-jet misidentification rates in the per-mille to percent range, depending on kinematics and pileup.

Practical implementations tune cone sizes, isolation thresholds, and MVA inputs to the kinematic region and analysis goals; the choice of working point is driven by the signal-to-background optimization for the search or measurement at hand.

\subsection{B Tagging at Multipurpose Detectors}

Jets originating from bottom quarks ($b$-jets) exhibit several distinctive properties that enable their identification. The relatively long lifetime of $b$ hadrons (order 1.5 ps) produces displaced charged tracks and often reconstructible secondary vertices a few millimetres from the primary interaction point. The large $b$-hadron mass yields decay products with sizable transverse momentum relative to the jet axis, and semileptonic branching fractions produce soft electrons or muons inside the jet. These features form the basis for $b$-tagging.

Practical algorithms exploit individual signatures or combine them:
\begin{itemize}
  \item \textbf{Track-counting:} counts tracks with large impact-parameter significance to identify a $b$-like topology.
  \item \textbf{Jet-probability:} evaluates the compatibility of the jet's track impact-parameter distribution with the primary vertex hypothesis.
  \item \textbf{Secondary-vertex:} explicitly reconstructs displaced vertices and uses their kinematic properties (decay length significance, vertex mass).
  \item \textbf{Soft-lepton taggers:} identify low-$p_T$ leptons inside jets from semileptonic $b$ decays.
\end{itemize}

Modern taggers combine many observables in multivariate or deep-learning classifiers to maximize discrimination power. Contemporary approaches exploit rich, low-level inputs (track-by-track and PF-candidate information, vertex features and kinematics) and advanced network architectures:

\begin{itemize}
  \item Deep feed-forward networks (e.g. DeepCSV/DeepJet) ingest a large set of high-level and per-track inputs to produce powerful binary or multi-class discriminants that separate $b$, $c$ and light-flavour jets.
  \item Sequence models and recurrent networks (RNN-based taggers) process an arbitrary ordered list of track-level variables, improving sensitivity by directly exploiting per-track correlations and order-dependent information (impact-parameter sequences, track kinematics).
  \item Graph- and set-based architectures and combined particle+vertex networks (sometimes referred to as ``DeepFlavour''-style models) aggregate heterogeneous inputs and return per-flavour probabilities, enabling natural multi-classification and calibrated operating points.
\end{itemize}

These developments yield measurable performance gains: modern deep classifiers typically improve $b$ efficiency at fixed mistag rate (or reduce mistag rates at fixed efficiency) relative to classical taggers. The continuous output of such networks permits analyses to choose operating points (loose/medium/tight) corresponding to desired efficiencies or mistag targets. Calibration remains essential: data-driven scale factors derived from control samples (e.g. $t\bar t$, multijet, dilepton) are applied to correct simulation, and systematic uncertainties from the calibration, flavour composition, and kinematic extrapolation are propagated to physics results.

Examples in use are CMS DeepCSV / DeepJet and ATLAS MV2 / DL1~\parencite{CMS:2017wtu,ATLAS:2019bwq}, which illustrate the transition from expert-designed high-level variables to large-scale machine learning leveraging low-level detector information. Typical medium working points yield $b$-tag efficiencies of order 60–80\% with light-jet misidentification rates at or below the percent level; the precise choice of working point is tuned per analysis to optimise sensitivity while accounting for calibration and systematic uncertainties.


\subsection{The CMS Detector}

CMS is a general-purpose detector at the LHC~\parencite{CMS_2008}. With a length of 21.6~m, a diameter of 14.6~m, and a weight of 14,000 tonnes, its cylindrical geometry is divided into a central barrel section and two endcaps. This design provides hermetic coverage to accurately measure momentum and energy balance, which is crucial for identifying non-interacting particles like neutrinos through missing transverse energy.

The detector is constructed from concentric layers of sub-detectors, as illustrated in Figure~\ref{fig_cms}. The innermost component is the silicon tracker, comprising a pixel detector and silicon strip tracker. It reconstructs the trajectories of charged particles and measures their transverse momenta ($p_T$) with a resolution of $\approx 0.7\%$ for 10~GeV particles within a pseudorapidity range of $|\eta| < 2.5$.

\begin{center}
	\includegraphics[width=0.98\textwidth]{Images/CMS.png}
	\captionof{figure}{Diagram of the CMS detector showing its inner components (retrieved from~\parencite{CMS_2008}).}\label{fig_cms}
\end{center}

Surrounding the tracker is the calorimetric system. The electromagnetic calorimeter (ECAL) is made of lead-tungstate crystals. It is designed to measure electrons and photons with a high resolution of $\approx 0.6\%$ for 50~GeV electrons. The hadronic calorimeter (HCAL), located outside the ECAL, is a brass-scintillator sampling calorimeter that measures hadrons (e.g., charged pions, kaons, protons) with an energy resolution of $\approx 18\%$ for 50~GeV pions. Together, the ECAL and HCAL cover $|\eta| < 3$. The coverage is extended to $|\eta| < 5$ with steel and quartz-fiber hadron calorimeters in the forward regions.

A key feature of CMS is its large superconducting solenoid, which encloses the tracker and calorimeters. The solenoid is constructed from a niobium-titanium alloy and cooled to 4.2~K with liquid helium. It generates a uniform magnetic field of 3.8~T throughout the tracking volume, enabling precise momentum measurement from the curvature of charged particle tracks.

The outermost system is dedicated to muon identification and measurement. Gas-ionization detectors are embedded in the steel flux-return yoke that surrounds the solenoid. This system provides triggering and tracking capabilities for muons up to $|\eta| < 2.4$. The combination of the inner tracker and the muon system allows for a robust identification and momentum measurement of muons across a wide kinematic range.

The geometrical segmentation of the barrel and endcaps defines the detector's acceptance in terms of pseudorapidity. The central barrel provides optimal coverage for $|\eta| \lesssim 1.5$, while the endcaps extend the acceptance to $|\eta| \lesssim 2.5$ for the tracker and calorimeters, and to $|\eta| \lesssim 2.4$ for the muon system.

This segmentation impacts the detection efficiency. The silicon trackers are highly efficient in the barrel, where particles cross the layers perpendicularly. In the endcaps, the reduced hit multiplicity from shallow-angle traversals leads to a slight decrease in tracking efficiency and resolution. The calorimeters are also optimized to maintain performance across $\eta$, though the material budget and granularity vary.

Muon reconstruction performance exhibits regional differences. In the barrel, drift tubes (DTs) provide high spatial resolution, while in the endcaps, cathode strip chambers (CSCs) and resistive plate chambers (RPCs) are used to handle higher background rates and non-uniform magnetic fields. The assumed identification efficiency for muons (electrons) is 95\% (85\%), with a mis-identification rate of 0.3\% (0.6\%)~\parencite{CMS-PAS-FTR-13-014,CMS_MUON_17001,CMS_EGM_17001}.

For the identification of heavy-flavor jets, we adopt the DeepCSV algorithm~\parencite{CMS_BTV2016}. We use its ``medium'' working point, which provides a $b$-tagging efficiency of 70\% with a light-flavor jet misidentification rate of approximately 1\% across the entire $p_T$ spectrum. The ``loose'' (85\% efficiency, 10\% mis-id) and ``tight'' (45\% efficiency, 0.1\% mis-id) working points were also explored during the analysis optimization.

For hadronically decaying $\tau$ leptons ($\tau_h$), we use the DeepTau algorithm~\parencite{CMS_DeepTau}, which employs a deep neural network combining isolation and lifetime information to identify $\tau_h$ decay modes. The ``medium'' working point is chosen for this analysis, providing a $\tau_h$ identification efficiency of 70\% and a misidentification rate of 0.5\% for jets originating from light quarks and gluons. This working point was selected through an optimization process that maximized the discovery reach of the analysis.

\section{The Phenomenological Pipeline: From Theory to Observables}

The estimation of signal and background event yields is performed through a comprehensive Monte Carlo (MC) simulation pipeline. This approach, a cornerstone of high-energy physics research, enables robust studies of Beyond the Standard Model (BSM) scenarios by emulating the entire data collection and processing chain of a collider experiment. The key advantages of this methodology include:

\begin{itemize}
    \item The ability to perform automated calculations of theoretical quantities such as cross-sections and decay widths for complex processes.
    \item Conducting feasibility studies and optimizing analysis strategies prior to data acquisition.
    \item Estimating the efficiency of complex event selection criteria and the geometric acceptance of the detector.
    \item Predicting the rates and kinematical distributions of both irreducible and reducible background processes.
    \item Comparing and distinguishing between different theoretical hypotheses for a potential discovered signal.
\end{itemize}

The simulation workflow is modular, reflecting the logical progression from a theoretical Lagrangian to simulated detector-level observables. A schematic view of this pipeline is presented in Figure~\ref{fig:sim_workflow}. The process begins with the implementation of the theoretical model in \texttt{FeynRules} (v2.3.43)~\parencite{Christensen:2008py,Alloul:2013bka}. The Lagrangian of the new physics scenario, including all particle definitions, parameters, and interactions, is translated into a set of Feynman rules. The output is exported in the Universal FeynRules Output (UFO) format, a standard interoperable with modern matrix element generators.

This UFO module, accompanied by a parameter card defining the numerical values of all model parameters (masses, couplings, etc.), serves as input to the \texttt{MadGraph5\_aMC@NLO} (v3.5.7)~\parencite{Alwall:2014bza,Alwall:2014hca} framework. Within MadGraph, the hard scattering process is defined, and the corresponding matrix elements (ME) and Feynman diagrams are generated at leading order (LO) in QCD. For this analysis, proton-proton collisions are simulated at center-of-mass energies of $\sqrt{s}=13 \tev$ and $\sqrt{s}=13.6 \tev$, utilizing the NNPDF3.0 NLO~\parencite{NNPDF:2014otw} set of parton distribution functions (PDFs). This choice is motivated by its global fit accuracy and consistency for both LO and NLO simulations.

To accurately model processes featuring significant interference effects between the new physics signal (e.g., a $\zb'$ boson) and the Standard Model backgrounds, the full squared amplitude (often referred to as the Signal-Discriminated Events or SDE strategy) is employed for the phase-space integration. The \texttt{MadEvent} submodule then generates unweighted parton-level events, which are stored in the Les Houches Event (LHE) format, containing the four-momenta of all final-state particles. The generation is optimized through careful configuration of the \texttt{run\_card}, setting appropriate kinematic cuts on final-state partons to avoid wasting computational resources on events that would subsequently be rejected by the detector simulation.

Given the presence of additional jet radiation, the MLM matching scheme~\parencite{Alwall:2007fs} is applied to mitigate the double-counting of jet emission between the matrix element calculation and the subsequent parton shower. This ensures a smooth transition between the hard process and softer radiative effects.

The parton-level LHE events are then passed to \texttt{PYTHIA} (v8.2.44)~\parencite{Sjostrand:2014zea} for the modeling of QCD and QED radiation (parton showering), hadronization, and particle decays. This step translates the colored partons into stable, color-singlet hadrons and resonances that form the observable final state. The resulting events, which include a full list of generator-level particles, are saved in the HepMC2 format.

Detector effects are simulated using \texttt{DELPHES} (v3.4.2)~\parencite{deFavereau:2013fsa}, a fast parametric detector simulation framework. The \texttt{delphes\_card\_CMS.tcl} configuration card is used to emulate the response of the CMS detector, including the geometric acceptance, tracking efficiency, calorimeter energy resolution and segmentation, and magnetic field. Key reconstruction algorithms are applied within DELPHES:
\begin{itemize}
    \item Jets are clustered from calorimeter towers using the anti-$k_t$ algorithm~\parencite{Cacciari:2008gp} with a distance parameter of $R=0.4$, and $b$-tagging is simulated based on the efficiency and mis-tag rate of the CMS performance.
    \item Muons and electrons are identified with efficiency maps that are functions of $p_T$ and $\eta$.
    \item The missing transverse energy (MET) is calculated from the negative vector sum of all reconstructed particle momenta.
\end{itemize}
The final output, containing reconstructed physics objects (jets, leptons, MET), is stored in ROOT format~\parencite{Brun:1997pa}.

At this stage, the analysis of the simulated samples converges with the methodology applied to real collider data. The subsequent steps involve applying event selection criteria, calibrating and scaling the reconstructed objects (e.g., applying Jet Energy Corrections), and performing statistical interpretation. The reliability of the simulation is validated by comparing the modeling of well-known Standard Model processes (e.g., Drell-Yan, $t\bar{t}$ production) against published results and data-driven control regions. Dominant theoretical systematic uncertainties, such as those arising from the choice of factorization and renormalization scales, PDF variations, and parton shower modeling, are evaluated and propagated through the analysis.


\section{Measurement of the Power of an Analysis}
\label{sec:power_analysis}

In high-energy physics experiments, data is often discretized into bins (e.g., histograms of collision events versus energy or momentum) to test competing hypotheses. The fundamental framework compares two scenarios: the \textit{null hypothesis} ($H_0$), representing background-only processes ($b_i$ in each bin $i$), and the \textit{alternative hypothesis} ($H_1$), including both signal and background ($s_i + b_i$). Given the Poissonian nature of event counts $n_i$, the likelihood for observing the data under each hypothesis is:
\begin{equation}
    \mathcal{L}(n_i \mid \lambda_i) = \frac{e^{-\lambda_i} \lambda_i^{n_i}}{n_i!}, \quad \text{where } \lambda_i = 
    \begin{cases}
        b_i & \text{for } H_0, \\
        s_i + b_i & \text{for } H_1.
    \end{cases}
\end{equation}
The Neyman-Pearson lemma~\parencite{NeymanPearson1933} provides a rigorous framework for hypothesis testing by establishing that the \textit{likelihood ratio} $Q = \mathcal{L}(\text{data} \mid H_1)/\mathcal{L}(\text{data} \mid H_0)$ is the most powerful test statistic for distinguishing between two simple hypotheses, $H_0$ and $H_1$. This forms the basis for quantifying the evidence for new physics signals against known backgrounds. For binned analyses in particle physics, we define the likelihood ratio $Q_i$ for each bin $i$ as,
\begin{equation}
Q_i = \frac{\mathcal{L}(n_i \mid s_i + b_i)}{\mathcal{L}(n_i \mid b_i)} = e^{-s_i} \left( 1+\frac{s_i}{b_i} \right)^{n_i},
\end{equation}
where $n_i$ is the observed event count, $s_i$ the expected signal, and $b_i$ the expected background in bin $i$. 

The test for the full analysis is constructed as the product of individual bin likelihood ratios:
\begin{equation}
Q = \prod_{i=1}^{N} Q_i,
\end{equation}
where $N$ is the total number of bins. Under this formulation, each bin is treated as an independent experiment, allowing us to analyze the data in a modular way. This is particularly useful when combining results from multiple search channels or energy ranges. 

For convenience and in order to connect with asymptotic distributions, we take the logarithm of the likelihood ratio, turning the product into a sum:
\begin{equation}
-2\ln Q = 2\sum_{i=1}^{N}\left[s_i - n_i \ln\left(1 + \frac{s_i}{b_i}\right)\right],
\end{equation}
where $n_i$ is the observed number of events, $b_i$ the expected background, and $s_i$ the expected signal in bin $i$. By Wilks' theorem~\parencite{Wilks1938}, the asymptotic distribution of this test statistic under the background-only hypothesis ($H_0$) follows a $\chi^2$ distribution, facilitating $p$-value calculations and hypothesis testing.

In practice, the Neyman-Pearson lemma motivates the use of a test statistic $t$ that quantifies the evidence for a signal against the background-only hypothesis, which can be written as
\begin{equation}
t=-2\ln Q = \sum_{i=1}^{N} \left[2s_i - 2n_i w_i\right],
\end{equation}
with the optimal weight of each bin given by $w_i = \ln\!\left(1 + \frac{s_i}{b_i}\right)$.




The discovery significance $\kappa$ quantifies the statistical separation of $t$ if $n$ is distributed according to the background-only hypothesis ($H_0$) versus the signal-plus-background hypothesis ($H_1$), normalized by the standard deviation of the $H_1$ distribution,
\begin{equation}
\kappa = \frac{\braket{t}_{H_0} - \braket{t}_{H_1}}{\sigma_{H_1}}.
\end{equation}
The expected behavior differs under the signal-plus-background ($H_1$) and background-only ($H_0$) hypotheses:

\begin{itemize}
	\item \textbf{Under $H_1$ ($n_i \sim \text{Pois}(s_i + b_i)$)}:
	\begin{equation}
	\langle -2\ln Q \rangle_{s+b} = \sum_i \left[2s_i - 2(s_i + b_i)w_i\right]
	\implies \sigma^2_{s+b} = 4\sum_i (s_i + b_i) w_i^2.
	\end{equation}

	\item \textbf{Under $H_0$ ($n_i \sim \text{Pois}(b_i)$)}:
	\begin{equation}
	\langle -2\ln Q \rangle_{b} = \sum_i \left[2s_i - 2b_i w_i\right]
	\implies \sigma^2_{b} = 4\sum_i b_i w_i^2
	\end{equation}
\end{itemize}
Substituting in $\kappa$ gives a useful expression for the discovery significance,
\begin{align}
\kappa = \frac{\sum s_i w_i}{\sqrt{\sum (s_i + b_i) w_i^2}}
\end{align}
It quantifies the separation between the signal+background ($s+b$) and background-only hypotheses in units of standard deviations ($\sigma$), where $\kappa = 5$ corresponds to the traditional $5\sigma$ discovery threshold, $\kappa =3$ to a $3\sigma$ evidence to the traditional anomaly detection threshold, and $\kappa = 1.69$ to the $95\%$ confidence level (CL) exclusion limit.


This figure of merit automatically optimizes sensitivity through the logarithmic weights $w_i = \ln(1 + s_i/b_i)$, which naturally emphasize bins with either high signal-to-background ratios ($s_i/b_i$) or large absolute signal contributions ($s_i$). In asymptotic limits, $\kappa$ simplifies to intuitive forms: for dominant signals ($s_i \gg b_i$), it approaches $\sqrt{\sum s_i}$ (Poisson counting), while in background-dominated regimes ($s_i \ll b_i$), it reduces to an inverse-variance-weighted sum $\sum s_i / \sqrt{\sum b_i (s_i/b_i)^2}$. This dual behavior ensures optimal discrimination power across all signal regimes.

In practice, we must take into account systematic effects by incorporating nuisance parameters into the likelihood and profiling over uncertainty ranges. The power calculation can be extended to include systematic uncertainties by modifying the denominator as,
\begin{equation}
	\boxed{
	\kappa_{\text{sys}} = \frac{\sum_i s_i w_i}{\sqrt{\sum_i \left[(s_i + b_i) + \sigma^2_{\text{sys,signal},i} + \sigma^2_{\text{sys,bkg},i}\right] w_i^2}},
}
\label{eq:kappa_with_systematics}
\end{equation}
where $\sigma_{\text{sys}}$ terms represent the systematic uncertainties on signal and background predictions.
