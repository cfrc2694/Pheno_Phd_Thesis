\chapter{BSM signatures in di-tau final states}


\section{A Simplified Model for the $\tilde S_1$ Scalar Leptoquark}

Leptoquarks (LQs) are hypothetical bosonic particles that couple simultaneously to a quark and a lepton, thereby mediating lepton-quark interactions at a single vertex. They arise naturally in various extensions of SM, including grand unified theories and models with extended gauge symmetries, and have attracted renewed interest as a possible explanation for observed flavor anomalies and hints of lepton flavor non-universality.

To ensure Lorentz invariance, the interaction vertex involving a leptoquark, a quark, and a lepton must form a scalar or vector operator. Among these possibilities, scalar leptoquarks (sLQs) provide a minimal and renormalizable framework without requiring new gauge interactions, making them an ideal starting point for simplified model studies.

The classification of scalar leptoquarks is determined by their transformation properties under the SM gauge group $SU(3)_C \times SU(2)_L \times U(1)_Y$. For simplicity, we focus on sLQs that are singlets under $SU(2)_L$, so that the model contains a single new scalar degree of freedom. This constraint leads to interactions involving two fermions of the same chirality, which requires charge conjugation of one of the fields to build a Lorentz scalar.

There are two classes of gauge-invariant, renormalizable operators that couple such a singlet sLQ to SM fermions. These are:

\begin{enumerate}
    \item $I_1 = \overline{q}_L^C \vb*{\epsilon} \ell_L$ and $I_1' = \overline{u}_R^C e_R$, where $q_L$ and $\ell_L$ are left-handed quark and lepton doublets, respectively, and $u_R$, $e_R$ are right-handed singlets. Both $I_1$ and $I_1'$ are $SU(2)_L$ singlets, transform as color triplets under $SU(3)_C$, and carry hypercharge $Y = -2/3$.
    \item $\tilde I_1 = \overline{d}_R^C e_R$, a gauge-invariant operator involving two right-handed singlets. It is also a color triplet and an $SU(2)_L$ singlet, but with hypercharge $Y = -8/3$.
\end{enumerate}

The corresponding leptoquark representations are:
\begin{enumerate}
    \item $S_1 \dot= (\bar{\mathbf{3}}, \mathbf{1}, 2/3)$, consistent with the first set of operators.
    \item $\tilde{S}_1 \dot= (\bar{\mathbf{3}}, \mathbf{1}, 8/3)$, associated with the second invariant.
\end{enumerate}

In models with flavor alignment or preferential couplings to third-generation fermions, $S_1$ typically couples to $b\nu_\tau$ and $t\tau$ vertices. However, processes like $pp \to \tau^+\tau^-$ suffer suppression due to the low parton density of the top quark in the proton. 

In contrast, the $\tilde{S}_1$ leptoquark allows for direct couplings to $b$ quarks and $\tau$ leptons via a term of the form $\overline{d}_R^C \tilde{S}_1 e_R$. Despite the relatively small $b$-quark content in the proton, these channels remain accessible at the LHC.

For these reasons, this section focuses on the simplified model containing only the scalar leptoquark $\tilde{S}_1$. The most general renormalizable Lagrangian for the $\tilde{S}_1$ leptoquark extends the SM one by
\begin{equation}
    \mathcal{L}\supset |D_\mu\tilde{S}_1|^2 + \tilde{y}_{ij} \overline{d}_R^{C\,i} \tilde{S}_1 e_R^j + \tilde{z}_{ij} \overline{u}_R^{C\,i} \tilde{S}_1^* u_R^j + \text{h.c.} + V_{\text{ext}}(\tilde{S}_1,H)
\end{equation}
where $\vb*{\tilde{y}}, \vb*{\tilde{z}}$ are complex Yukawa matrices in the flavor space\marginpar{Note that $\vb*{\tilde{z}}$ must be antisymmetric (due to fermionic statistics) and is typically assumed to vanish or be highly suppressed to avoid large FCNCs.}, $D_\mu$ is the covariant derivative acting on the $\tilde{S}_1$ field as 
\begin{equation}
    D_\mu \tilde{S}_1 = \partial_\mu \tilde{S}_1 + i g_s T^a G_\mu^a \tilde{S}_1 + i \frac{4}{3} g' B_\mu \tilde{S}_1,
\end{equation}
and $V_{\text{ext}}$ is the extension of the SM Higgs potential to include the leptoquark field gived by
\begin{equation}
    V_{\text{ext}} = \mu_{\tilde{S}_1}^2 |\tilde{S}_1|^2 + \lambda_{\tilde{S}_1} |\tilde{S}_1|^4 + \lambda_{H\tilde{S}_1} |H|^2 |\tilde{S}_1|^2,
\end{equation}
such that the full scalar potential reads
\begin{equation}
    V = \underbrace{-\mu^2|H|^2 + \lambda|H|^4}_{\text{SM Higgs}} + \underbrace{\mu_{\tilde{S}_1}^2|\tilde{S}_1|^2 + \lambda_{\tilde{S}_1}|\tilde{S}_1|^4}_{\text{sLQ self-interactions}} + \underbrace{\lambda_{H\tilde{S}_1}|H|^2|\tilde{S}_1|^2}_{\text{Higgs portal}}.
\end{equation}

After EWSB, $H \to (0, (v+h)/\sqrt{2})^T$, the $H\tilde S_1$ interactions become in interactions between the physical Higgs boson $h$ and the leptoquark $\tilde{S}_1$ as
\begin{align}
    \mathcal{L}_{\text{int}} &\supset \lambda_{H\tilde{S}_1}v h|\tilde{S}_1|^2 + \frac{1}{2}\lambda_{H\tilde{S}_1}h^2|\tilde{S}_1|^2
\end{align}
and a mass shift for the leptoquark field
\begin{equation}
    \Delta m_{\tilde{S}_1}^2 = \frac{1}{2}\lambda_{H\tilde{S}_1}v^2 \implies m_{\tilde{S}_1}^2 = \mu_{\tilde{S}_1}^2 + \frac{1}{2}\lambda_{H\tilde{S}_1}v^2.
\end{equation}
Additionally, from the covariant derivative as $B_\mu = c_W A_\mu - s_W Z_\mu$, the $\tilde{S}_1$ interactions with the SM gauge bosons are given by
\begin{align}
    \mathcal{L} &\supset \frac{4}{3}e\left[(\partial_\mu\tilde{S}_1^*)\tilde{S}_1 - \tilde{S}_1^*(\partial_\mu\tilde{S}_1)\right]A^\mu + \left(\frac{4}{3}e\right)^2 A_\mu A^\mu |\tilde{S}_1|^2\\
    \mathcal{L} &\supset -\frac{4}{3}g_1 s_W\left[(\partial_\mu\tilde{S}_1^*)\tilde{S}_1 - \tilde{S}_1^*(\partial_\mu\tilde{S}_1)\right]Z^\mu + \left(\frac{4}{3}g_1 s_W\right)^2 Z_\mu Z^\mu |\tilde{S}_1|^2.
\end{align}

For preferential couplings to third-generation fermions in the lepton-quark vertices, we assume the texture of the Yukawa matrix $\tilde{y}_{33}$ is such that $\tilde{y}_{33} \gg \tilde{y}_{ij}$ for $i,j \neq 3$. This results in a preferential decay of the sLQ into $b\tau$ final states, with branching ratio ${\rm BR}(\tilde{S}_1 \to b\tau) \approx 1$ for large sLQ masses. 

{\huge ADD XS PLOT}

\section{A Simplified Model for the $U_1$ Vector Leptoquark}

In contrast to what is done for the sLQ, vector couplings between fermions of the same helicity are allowed without resorting to the charge conjugation matrix. In this way, the inclusion of a single vectorial LQ (vLQ) with signatures in the di-tau final state channel requires a vector that transforms as $U_1  \dot= (\bar{\mathbf{3}}, \mathbf{1}, 4/3)$ under the SM group.

Extending the SM with a massive $U_1$ vLQ is not straightforward, as one has to ensure the renormalizability of the model. Most of the theoretical community has focused on extensions of the Pati-Salam (PS) models which avoid proton decay, such as the scenario found in~\parencite{Assad:2017iib}. Other examples include PS models with vector-like fermions~\parencite{Calibbi:2017qbu,Blanke:2018sro,Iguro:2021kdw}, the so-called 4321 models~\parencite{DiLuzio:2017vat,Greljo:2018tuh,DiLuzio2018}, the twin PS$^2$ model~\parencite{King:2021jeo,FernandezNavarro:2022gst}, the three-site PS$^3$ model~\parencite{Bordone:2017bld,Bordone:2018nbg,Fuentes-Martin:2022xnb}, as well as composite PS models~\parencite{Gripaios:2009dq,Barbieri:2016las,Barbieri:2017tuq}.


However, in what follows, we shall restrict ourselves to a simplified non-renormalizable lagrangian for a gauge vLQ, understood to be embedded into a more complete model. The SM is thus extended by adding the following terms featuring the $U_1$ $\textrm{vLQ}$ \marginpar{The couplings in the second line of Eq.~(\ref{eq:BasicLagrangian}) can be found in the literature as $g_s\to g_s(1-\kappa_U)$ and $g'\to g'(1-\tilde\kappa_U)$, in order to take into account the possibility of an underlying strong interaction.}:
\begin{eqnarray}
\label{eq:BasicLagrangian}
  \mathcal{L}_{U_1}&=&-\frac{1}{2}U^\dagger_{\mu\nu}U^{\mu\nu}+M_U^2\, U_{1\mu}^\dagger U_1^\mu \nonumber \\
 &&  -ig_s\,U_{1\mu}^\dagger\, T^a\, U_{1\nu}\, G^{a\mu\nu}\!\!-i\frac{2}{3}g'\,U^\dagger_{1\mu}U_{1\nu}B^{\mu\nu} \nonumber \\
 && +\frac{g_U}{\sqrt 2}\sum_{ij}\big[U_{1\mu}\big(\beta_L^{ij} \bar Q_i\,\gamma^\mu L_i + \beta^{ij}_{R}\,\bar d_{R}^i\,\gamma^\mu e^j_{R}\big) +{\rm h.c.}\big] 
\end{eqnarray}
where $U_{\mu\nu}\equiv\mathcal{D}_\mu U_{1\nu}-\mathcal{D}_\nu U_{1\mu}$, and $\mathcal{D}_\mu\equiv\partial_\mu+ig_s T^a G_\mu^a+i\tfrac{2}{3}g'B_\mu$. As evidenced by the second line above, we assume that the $\textrm{vLQ}$ has a gauge origin.

The tight constraints from low-energy observables, such as $\Delta F=2$ amplitudes and lepton flavor violating processes, motivates the following choice of the $\vb*{\beta}_L$ and $\vb*{\beta}_R$ matrices, which are $3\times3$ matrices in flavor space:
\begin{equation}
    \vb*{\beta}_L = \begin{pmatrix}
        0 & 0 & \beta_L^{d\tau} \\
        0 & 0 & \beta_L^{s\tau} \\
        0 & \beta_L^{b\mu} & \beta_L^{b\tau}
    \end{pmatrix},\quad
    \vb*{\beta}_R = \begin{pmatrix}
        0 & 0 & 0 \\
        0 & 0 & 0 \\
        0 & 0 & \beta_R^{b\tau}
    \end{pmatrix}.
\end{equation}
Several works have studied the structure of the $\vb*{\beta}_L$ and $\vb*{\beta}_R$ matrices and the choice above is consistent with the breaking of $U(2)$ symmetry in the flavor space~\parencite{Cornella:2021sby,Assad:2017iib,Calibbi:2017qbu,Blanke:2018sro}. 

The third and fourth lines in in Eq.~(\ref{eq:BasicLagrangian}) shows the $\textrm{vLQ}$ interactions with SM fermions, with coupling $g_U$, which we have chosen as preferring the third generation%
%~\marginpar{Before the demise of the $R_{K^{(*)}}$ anomaly~\parencite{LHCb:2022qnv,LHCb:2022zom,Greljo:2022jac,Ciuchini:2022wbq}, a $3\times3$ $\beta_L$ matrix would be used instead, with values fitted to solve all $\Bm$ meson anomalies.}%
. These are particularly relevant for the $\textrm{vLQ}$ decay probabilities, as well as for the single-$\textrm{vLQ}$ production cross-section. The $\beta_L^{s\tau}$ parameter, which is the $\textrm{vLQ} \to s\tau$ coupling in the $\beta_L$ matrix (see marginpar), is chosen to be equal to $0.2$, following the fit done in~\parencite{Cornella:2021sby}, in order to simultaneously solve the $R_{D^{(*)}}$ anomaly and satisfy the $\mathrm{p}\,\mathrm{p}\to\tau^+\tau^-$ constraints. Although $\beta_L^{s\tau}$ technically alters the single-$\textrm{vLQ}$ production cross-section and $\textrm{vLQ}$ branching fractions, we have confirmed that a value of $\beta_L^{s\tau} = 0.2$ results in negligible impact on our collider results, and thus is ignored in our subsequent studies.


\begin{center}
    % \includegraphics[width=.8\linewidth]{images/VLQ_BranchingRatio.pdf}
    \captionof{figure}{Top: The $\textrm{vLQ}\to\textrm{b}\tau$ and $\textrm{vLQ}\to\textrm{t}\nu$ branching ratios for $\beta_R^{b\tau} = 0$ (solid lines) and $\beta_R^{b\tau} = -1$ (dashed lines). Bottom: Signal cross-section as a function of the $\textrm{vLQ}$ mass, for $\sqrt{ s}=13 \tev$, with $g_U=1.8$. We show single, pair, and non-resonant production, for $\beta_R^{b\tau}=-1,\,0$ in solid and dashed lines, respectively.}\label{fig:branching_ratios}
\end{center}

The $\textrm{vLQ}$ right-handed coupling is modulated with respect to the left-handed one by the $\beta_R^{b\tau}$ parameter. The choice of $\beta_R^{b\tau}$ is important phenomenologically, as it affects the $\textrm{vLQ}$ branching ratios \marginpar{Having $\beta_L^{s\tau}$ different from zero also opens new decay channels. These, however, are either suppressed by $\beta_L^{s\tau}$ and powers of $\lambda_{\rm CKM}$. In any case, this effect would decrease ${\rm BR}(\textrm{vLQ} \to \textrm{b}\,\tau)$ and ${\rm BR}(\textrm{vLQ} \to \tq\,\nu)$ by less than $3\%$.}, as well as the single-$\textrm{vLQ}$ production cross-section. To illustrate the former, Figure~\ref{fig:branching_ratios} (top) shows the $\textrm{vLQ}\to\textrm{b}\tau$ and $\textrm{vLQ}\to\textrm{t}\nu$ branching ratios as functions of the $\textrm{vLQ}$ mass, for two values of $\beta_R^{b\tau}$. For large $\textrm{vLQ}$ masses, we confirm that with $\beta_R^{b\tau} = 0$ then ${\rm BR}(\textrm{vLQ} \to \textrm{b}\,\tau) \approx {\rm BR}(\textrm{vLQ} \to \tq\,\nu)\approx \tfrac{1}{2}$. However, for $\beta_R^{b\tau} = -1$, as was chosen in~\parencite{Cornella:2019hct}, the additional coupling adds a new term to the total amplitude, leading to ${\rm BR}(\textrm{vLQ}\to \textrm{b}\,\tau) \approx \tfrac{2}{3}$. The increase in this branching ratio can thus weaken bounds from $\textrm{vLQ}$ searches targeting decays into $\tq\,\nu$ final states, which motivates exploring the sensitivity in b$\tau$ final states exclusively. Note that although a ${\rm BR}(\textrm{vLQ}\to \textrm{b}\,\tau) \approx 1$ scenario is possible by having the $\textrm{vLQ}$ couple exclusively to right-handed currents (i.e, $g_U\to0$, but $g_U\beta_R^{b\tau}\not=0$), it does not solve the observed anomalies in the $R_{D^{(*)}}$ ratios. Therefore, although some LHC searches assume ${\rm BR}(\textrm{vLQ}\to \textrm{b}\,\tau) = 1$, we stress that in our studies we assume values of the model parameters and branching ratios that solve the $R_{D^{(*)}}$ ratios.

Moreover, we also note that to solve the $R_{D^{(*)}}$ anomaly, the authors of~\parencite{Cornella:2021sby} point out that the wilson coefficient $C_U\equiv g^2_U\,v^2_{SM}/(4\,M^2_U)$ is constrained to a specific range of values, and this range depends on the value of the $\beta_R^{b\tau}$ parameter. Therefore, the allowed values of the coupling $g_{U}$ depend on $M_{U}$ and $\beta_R^{b\tau}$, and thus our studies are performed in this multi-dimensional phase space.

%To further understand the role of $\beta_R^{b\tau}$ at colliders, Figure~\ref{fig:branching_ratios} (bottom) shows the cross-section for single-$\textrm{vLQ}$ (s$\textrm{vLQ}$), double-$\textrm{vLQ}$ (d$\textrm{vLQ}$), and non-resonant (non-res) production, as a function of mass and for a fixed coupling $g_{U} = 1.8$, assuming $\mathrm{p}\,\mathrm{p}$ collisions at $\sqrt{s} = 13$ $\tev$. We note that this benchmark scenario with $g_{U}=1.8$ results in a $\textrm{vLQ}\to\textrm{b}\tau$ decay width that is $<$5\% of the $\textrm{vLQ}$ mass, for mass values from 250 $\gev$ to 2.5 $\tev$. In the Figure, we observe that, since d$\textrm{vLQ}$ production is mainly mediated by events from quantum chromodynamic processes, the choice of $\beta_R^{b\tau}$ does not affect the cross-section. However, for  s$\textrm{vLQ}$ production, a non-zero value for $\beta_R^{b\tau}$ increases the cross-section by about a factor of 2 and by almost one order of magnitude in the case of non-res production. These results shown in Figure~\ref{fig:branching_ratios} are easily understood by considering the diagrams shown in Figure~\ref{fig:feynmp-prod-channels}. The $\textrm{vLQ}$ mass value where the s$\textrm{vLQ}$ production cross-section exceeds the d$\textrm{vLQ}$ cross-section depends on the choice of $g_U$. 
 



% As noted in section~\ref{sec:intro}, we study the role of a $\zb'$ boson in $\mathrm{p}\,\mathrm{p}\to\tau\tau$ production. The presence of a $\zb'$ boson in $\textrm{vLQ}$ models has been justified in various papers, for example, in~\parencite{Baker:2019sli}. The argument is that minimal extensions of the SM which include a massive gauge $U_1$ LQ, uses the gauge group $SU(4)\times SU(3)^{\prime}\times SU(2)_L \times U(1)_{T_R^3}$. Such an extension implies the presence of an additional massive boson, $\zb^{\prime}$, and a color-octet vector, $G'$, arising from the spontaneous symmetry breaking into the SM \marginpar{Naively, the LQs are associated to the breaking of $SU(4)\to SU(3)_{[4]}\times U(1)_{B-L}$, the $G'$ arises from $SU(3)_{[4]}\times SU(3)'\to SU(3)_c$, and the $Z'$ comes from the breaking of $U(1)_{B-L}\times U(1)_{T_R^3}\to U(1)_Y$. Notice that the specific pattern of breaking, and the relations between the masses and couplings, are connected to the specific scalar potential used.}.  The $\zb'$ in particular can play an important role in the projected $\textrm{vLQ}$ discovery reach, as it can participate in $\mathrm{p}\,\mathrm{p}\to\tau\tau$ production by s-channel exchange, both resonantly and as a virtual mediator. To study the effect of a $\zb'$ on the $\mathrm{p}\,\mathrm{p}\to\tau\tau$ production cross-sections and kinematics, we extend our benchmark Lagrangian in Eq.~(\ref{eq:BasicLagrangian}) with further non-renormalizable terms involving the $\zb'$. Accordingly, we assume the $\zb'$ only couples to third-generation fermions. Our simplified model is thus extended by:
% \begin{eqnarray}
%     \label{eq:BasicLagrangianZp}
%         \mathcal{L}_{Z^{\prime}}&= & -\frac{1}{4} Z_{\mu \nu}^{\prime} Z^{\prime \mu \nu}+\frac{1}{2} M_{Z^{\prime}}^2 Z_\mu^{\prime} Z^{\prime \mu} \nonumber \\
%         && + \frac{g_{Z^{\prime}}}{2 \sqrt{6}} Z^{\prime \mu} (\zeta_q \bar{Q}_3 \gamma_\mu Q_3 \nonumber +\zeta_t \bar{t}_R \gamma_\mu t_R \\
%         &&  +\zeta_b \bar{b}_R \gamma_\mu b_R-3 \zeta_{\ell} \bar{L}_3 \gamma_\mu L_3-3 \zeta_\tau \bar{\tau}_R \gamma_\mu \tau_R)
% \end{eqnarray}
% where the constants $M_{\zb^{\prime}}$, $g_{Z^{\prime}}$, $\zeta_q $, $\zeta_t $, $\zeta_b$, $\zeta_{\ell}$, $\zeta_\tau$, are model dependent.

% We study two extreme cases for the $\zb'$ mass, following~\parencite{GINO_PhysRevD.102.115015}, namely $M_{\zb'} = \sqrt{\tfrac{1}{2}}M_U<M_U$ and $M_{\zb'} = \sqrt{\tfrac{3}{2}}M_U>M_U$. We also assume the $\textrm{vLQ}$ and $\zb'$ are uniquely coupled to left-handed currents, i.e. $\zeta_q=\zeta_\ell= 1$ and $\zeta_t=\zeta_b=\zeta_\tau=0$. With these definitions, Figure~\ref{fig:xsinterference} shows the effect of the $\zb'$ on the $\tau\tau$ production cross-section, considering $g_U = 1$, $\beta_R^{b\tau}=0$, and different $g_{\zb^{\prime}}$ couplings. On the top panel, the cross-sections corresponding to the cases where $M_{\zb'} = \sqrt{\tfrac{1}{2}}M_U$ are shown. As expected, the $\tau\tau$ production cross-section for the inclusive case (i.e., $g_{\zb'} \neq 0$) is larger than that for the $\textrm{vLQ}$-only non-res process ($g_{\zb'} = 0$, depicted in blue). This effect increases with $g_{\zb'}$ and, within the evaluated values, can exceed the $\textrm{vLQ}$-only cross-section by up to two orders of magnitude. In contrast, a more intricate behaviour can be seen in the bottom panel of Figure~\ref{fig:xsinterference}, which corresponds to $M_{\zb'} = \sqrt{\tfrac{3}{2}}M_U$. Here, for low values of $M_U$, a similar increase in the cross-section is observed. However, for higher values of $M_U$, the inclusive $\mathrm{p}\,\mathrm{p}\to\tau\tau$ cross-section is smaller than the $\textrm{vLQ}$-only $\tau\tau$ cross-section. This behaviour suggests the presence of a dominant destructive interference at high masses, leaving its imprint on the results.
% \begin{figure}[]
% \centering
%     \begin{subfigure}[b]{.94\linewidth}
%     % \includegraphics[width=.8\linewidth]{images/XS_gu_gzp_lower_limit_woRHC.pdf}
%     \end{subfigure}
%     \begin{subfigure}[b]{.94\linewidth}
%     % \includegraphics[width=.8\linewidth]{images/XS_gu_gzp_upper_limit_woRHC.pdf}
%     \end{subfigure}
%     \caption{$\tau \tau$ cross-section as a function of the $\textrm{vLQ}$ mass for different values of $g_U$ and $g_{\zb^{\prime}}$. The estimates are performed at $\sqrt s=13 \tev$, $\beta_R^{b\tau}=0$,  $M_{\zb^{\prime}} = \sqrt{1/2} M_{U}$ (top), and $M_{\zb^{\prime}} = \sqrt{3/2} M_{U}$ (bottom).}
% \label{fig:xsinterference}
% \end{figure}

% \begin{figure}[]
% \centering
%     \begin{subfigure}[b]{.94\linewidth}
%     % \includegraphics[height=6cm, width=6.0cm]{images/Kinematic_Interference_gu_1.0_gzp_1.0_zp_lower_limit_woRHC.pdf}
%     \end{subfigure}
%     \begin{subfigure}[b]{.94\linewidth}
%     % \includegraphics[height=6cm, width=6cm]{images/Kinematic_Interference_gu_1.0_gzp_1.0_zp_upper_limit_woRHC.pdf}
%     \end{subfigure}
%     \caption{The relative kinematic interference (RKI), as a function of the reconstructed mass of two taus, for different $\textrm{vLQ}$ masses. The studies are performed assuming $\sqrt s=13 \tev$, $\beta_R^{b\tau}=0$, $g_U = 1.0$, $g_{\zb^{\prime}} =1.0$, $M_{\zb^{\prime}} = \sqrt{1/2} M_{U}$ (top), and $M_{\zb^{\prime}} = \sqrt{3/2} M_{U}$ (bottom).
%     }    
% \label{fig:interference}
% \end{figure}
% In order to further illustrate the effect, Figure~\ref{fig:interference} shows the relative kinematic interference ($\mathrm{RKI}$) as a function of the reconstructed invariant mass $m_{\tau\tau}$, for $g_{\zb^{\prime}} = 1$ and varying values of $M_U$. The RKI parameter is defined as
% \begin{equation}
%     \mathrm{RKI}(m_{\tau\tau})=\frac{1}{\sigma_{\textrm{vLQ}+\zb'}}\left[\frac{d\sigma_{\textrm{vLQ}+\zb'}}{dm_{\tau\tau}}-\left(\frac{d\sigma_{\textrm{vLQ}}}{dm_{\tau\tau}}+\frac{d\sigma_{\zb'}}{dm_{\tau\tau}}\right)\right],
% \end{equation}
% where $\sigma_{X}$ is the production cross-section arising due to contributions from $X$ particles. For example, $\sigma_{\textrm{vLQ}+\zb'}$ represents the inclusive cross-section where both virtual $\textrm{vLQ}$ and s-channel $\zb'$ exchange contribute. For both cases, we can observe the presence of deep valleys in the RKI curves when $m_{\tau\tau}\to0$, indicating destructive interference between the $\textrm{vLQ}$ and the $\zb'$ contributions. This interference generates a suppression of the differential cross-section for lower values of $m_{\tau\tau}$ and, therefore, in the integrated cross-section. 
 
% The observed interference effects are consistent with detailed studies on resonant and non-res $\mathrm{p}\,\mathrm{p}\to\tq \bar{\tq}$ production, performed in reference~\parencite{Djouadi:2019cbm}.

% \section{The THDM Model type II}
% \lipsum

% \section{The Minimal $U(1)_{T^3_R}$ Model}
% \lipsum